% Load after hyperref
\usepackage[acronym,toc,acronymlists={acronym,notation},hyperfirst=false]{glossaries}
\usepackage{glossaries-extra}

% Expand on first use
% \setabbreviationstyle[acronym]{long-short}
% \setabbreviationstyle{short-long}

% Use three glossaries:
% Terms (default)
% Acronyms (added by the 'acronym' option)
% Notation (created manually here)
\newglossary[nlg]{notation}{not}{ntn}{Notation}

\makeglossaries

% Normal glossary entries
\newglossaryentry{gather}
{
name={gather},
description={A collection of traces sorted (or gathered) according to some
criteria such as ``shot gather'' or ``RP'' gather.  SIOSEIS frequently refers
to ``a gather'' meaning rp sorted, whereas a ``shot'' means a shot sorted.}
}
 
\newglossaryentry{shot}
{
    name={shot},
    description={The collection of traces associated with the field shot.  SEG-Y
trace header word 4 (shot record number) is used for the ``shot number''.  SEG-Y
word 7 (rp trace number) must be 0 to be considered shot sorted.}
}

\newglossaryentry{FNO_LNO_List}
{
name={FNO-LNO list},
description={}
}

\newglossaryentry{preset}
{
    name={preset},
    description={A parameter value which stays the same from FNO/LNO list to
    list until specified or given again.}
}

\newglossaryentry{default}
{
    name={default},
    description={A parameter value which reverts back to its original value
    after every FNO-LNO list.}
}

% Abbreviations
\newacronym{ascii}{ASCII}{American Standard Code for Information Interchange}%, a text encoding format standardized in 1966. Constrasted with EBCDIC
\newacronym{ebcdic}{EBCDIC}{Extended Binary Coded Decimal Interchange Code}%, a text encoding format typically used on IBM mainframe systems instead of ASCII}
\newacronym{nmea}{NMEA}{National Marine Electronics Association}
\newacronym{usgs}{USGS}{United States Geological Survey}

% Acronyms
\newabbreviation{cdp}{CDP}{common depth point}
\newabbreviation{cmp}{CMP}{common mid point}
\newabbreviation{cwp}{CWP}{Colorado School of Mines Center for Wave Phenomena}
\newabbreviation{dfls}{DFLS}{distance from the last shot}
\newabbreviation{esp}{ESP}{expanding spread profiles}
\newabbreviation{fno}{FNO}{first number}%, referring to a shot or RP depending on how the data are sorted}
\newabbreviation{fft}{FFT}{fast Fourier transform}
\newabbreviation{gmt}{GMT}{Greenwich Mean Time}
\newabbreviation{gps}{GPS}{Global Positioning System}
\newabbreviation{gpr}{GPR}{ground penetrating radar}
\newabbreviation{ldeo}{LDEO}{Lamont-Doherty Earth Observatory}
\newabbreviation{lno}{LNO}{last number}%, referring to a shot or RP depending on how the data are sorted}
\newabbreviation{ltz}{LTZ}{local trace zeroing}
\newabbreviation{nan}{NaN}{IEEE floating-point exception value `not-a-number'}
%\newabbreviation{nmea}{NMEA}{National Marine Electronics Association}
\newabbreviation{nsf}{NSF}{National Science Foundation}
\newabbreviation{obs}{OBS}{ocean bottom seismometers}
\newabbreviation{rp}{RP}{reflection point}
\newabbreviation{rms}{RMS}{root mean square}
\newabbreviation{seg}{SEG}{Society of Exploration Geophysicists}
\newabbreviation{sio}{SIO}{Scripps Institution of Oceanography}
\newabbreviation{ukooa}{UKOOA}{United Kingdom Offshore Operators Association}
\newabbreviation{wap}{WAP}{wide angle profiles}
% Symbols (Greek letters and such)

\newglossaryentry{not:p}{% This entry goes in the `notation' glossary:
  name={\ensuremath{p}},
  description={p},
  type=notation,
  sort={p}
}

\newglossaryentry{not:delta}{% This entry goes in the `notation' glossary:
  name={\ensuremath{\delta}},
  description={Delta},
  type=notation,
  sort={delta}
}

\newglossaryentry{not:tau}{% This entry goes in the `notation' glossary:
  name={\ensuremath{\tau}},
  description={Tau},
  type=notation,
  sort={tau}
}

\newglossaryentry{not:t0}{% This entry goes in the `notation' glossary:
  name={\ensuremath{t_{0}}},
  description={the normal incidence two way travel time},
  type=notation,
  sort={t0}
}

\newglossaryentry{not:x}{% This entry goes in the `notation' glossary:
  name={\ensuremath{x}},
  description={the shot to receiver distance of the trace; see Section~\ref{cmd_geom}},
  type=notation,
  sort={x}
}
