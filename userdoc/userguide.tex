\part{Users Guide}
\chapter{Introduction}

SIOSEIS is a software package for enhancing and manipulating marine
seismic reflection and refraction data, sponsored by the \gls{nsf}
and the Scripps Industrial Associates.
The system currently runs on SUN, SGI, PC-Linux, PC-CYGWIN, and Mac OSX.
X11 and an Unix shell are also required.

SIOSEIS features the use of the geophysical industry standard
demultiplexed SEG-Y data format for tape and disk files
(see Chapter \ref{c_segy_header}.) Seismic record
sections may be produced directly to a raster plotter without creating
intermediate files or making additional computer runs.  The processing
variables are given by the user via free field named parameters and are
edited for possible errors before any data are processed.  The system was
designed for use by people with some training in applied geophysics.
Users must know how to use a text editor and use a Unix shell.

SIOSEIS has three features that are particularly useful for processing
marine data:

\begin{enumerate}
\item Most academic seismic lines are collected over uneven bathymetry.
      SIOSEIS has the unique ability to define the water bottom, then
      having a single set of parameters for \texttt{NMO}, \texttt{MUTE}, \texttt{DECON}, \textit{etc.}
      reference the water bottom time automatically.  This results in an
      enormous time saving over having to define a new parameter each
      time there is a major variation in water depth.

\item Marine geophysicists often collect data in very deep water, using a
      delay to reduce the quantity of data actually recorded.  This
      results in the time of the  first data sample being a time other
      than zero.  In SIOSEIS each process that uses time as a variable
      honors the deep water delay.  SIOSEIS processes do not remove the
      delay by padding with zeroes to force the first time to be zero and
      adding to the amount of data processed.  The use of the deep water
      delay greatly increases the speed of processing.

\item SIOSEIS has a unique user interface.  All user variables are
      described via free field, order independent, named parameters.  The
      processing order is controlled by a single statement regardless of
      the order of the processing variables.  Very long multichannel lines
      may be described in full, yet can be processed in small pieces.  The
      program never generates large disk files with processing tables,
      rather it generates the appropriate information as the data are
      processed.
\end{enumerate}

The use of a shell script (parameter file) enables complex jobs to
be changed and rerun with minimal typing.  Parameter files also
allow easy modification so that \textbf{all} seismic lines within a project
are similar.

SIOSEIS has been under development at \gls{sio}
%Scripps Institution of Oceanography (SIO)
since 1978 and has been used by scientists,
students, and staff to process multi-channel and one and two
channel reflection and refraction data.  The system has been used
for production processing as well as experimental work with
expanding spreads, \gls{obs}, %OBS (ocean bottom seismometers),
sonobuoys, \gls{esp},
% ESP (expanding spread profiles),
and
% WAP (wide angle profiles)
\gls{wap}; with sample intervals ranging from 8 milliseconds to 0.1 millisecond and
``unusual'' sample rates of 128 samples per second.

See Chapter~\ref{c_seismic_processes} for the list of seismic processes currently supported by SIOSEIS.

Some differences between SIOSEIS and some other seismic packages are:
\begin{itemize}
\item Use of the deep water delay.
\item Allowing the trace length, sample interval, and deep water delay to vary from trace to trace.
\item SIOSEIS keeps the data in SEG-Y and does not use an internal data format.
\item SIOSEIS runs on all Unix workstations.
\item SIOSEIS is issued as a site license, permitting all machines to do seismic processing.
\item There are drivers for all raster plotters, thus additional plotting packages are not necessary.  SIOSEIS can make ``seismic wallpaper'' directly.
\item SIOSEIS can read SEG-D tape and disk files.
\item Most SIOSEIS processes allow time parameters to be ``hung'' from the
      water bottom time in the SEG-Y trace header. \textit{e.g.} processes gain,
      mute, decon, and nmo reference the water bottom time.
\item SIOSEIS has some special techniques for handling refraction data.
\item SIOSEIS's geometry is done completely in floating point which allows the subsurface binning to be defined with non-integers.
\item Realtime or near realtime processing is possible if the velocity function can be described as a function of water depth.
\item SIOSEIS can process data without building any external database files.  This allows quick access to any \gls{shot}.
\end{itemize}

\chapter{License}

% SIOSEIS is released under a 3-Clause BSD license; see \texttt{LICENSE}.
\VerbatimInput{LICENSE}

\chapter{SIOSEIS Processing Flow}

SIOSEIS processes seismic data in the order the user describes on the
\texttt{PROCS} statement. After each trace is entered into the system, it is
processed in order as far as it can before another trace is entered. This
makes SIOSEIS into a `trace processor'. There are 2 major advantages of a
trace processor over a `\gls{shot} processor'. First, each trace is left in memory
as long as possible, thus minimizing I/O time to disk. Secondly, there are no
limits as to the number of traces in a \gls{shot} or \gls{gather}.

Some seismic processes require more than one trace before they are finished,
such as stack. This type of process is called a `multi-input' process.
Processing control (or order) can not be advanced until enough traces are
entered into the system to satisfy the multi-input process requirements.
Processing order is retraced (backed-up) until an input process is found, and
then the processing order goes in a forward direction again.

Some seismic processes may output more than one trace from a single input
trace, such as constant velocity \glspl{gather}. This type of process is called a
multi-output process. In this case, processing proceeds after the multi-output
process until a trace is eliminated, then the processing order backs up until
it gets to the multi-output process, which then inputs the next trace if one is
available.

SIOSEIS has a table that defines each process as an input process, or an output
process, or neither. Each process sets the number of traces to be passed to
the next process. If the number of traces to be passed is zero, then SIOSEIS
searches the user defined processing list backwards from that process until an
input process is found. Processes \texttt{INPUT}, \texttt{SYN},
\texttt{STACK}, and \texttt{GATHER} are input processes.

Example 1:
\begin{verbatim}
PROCS INPUT FILTER OUTPUT END
\end{verbatim}
trace 1 is input, then filtered, then output, then trace 2 is input, \textit{etc.}

Example 2:
\begin{verbatim}
PROCS INPUT STACK FILTER OUTPUT EXEC END
\end{verbatim}
the processing order will be input stack until all of an \gls{rp} is finished, then
the stacked trace will be filtered and output. The processing order then goes
back to input.

\chapter{SIOSEIS Language and Language Syntax}
\label{c_syntax}

SIOSEIS applies seismic processes in the order specified via the \texttt{PROCS}
command (see the \texttt{PROCS} document for the list of processes available).
The list of processes must start with an ``input'' process and terminate with
an ``output'' process.  As with all SIOSEIS parameters, each process name in
the \texttt{PROCS} list must be separated by at least one blank and must be
terminated with the word \texttt{END}.  The \texttt{PROCS} list should be the
first thing given to SIOSEIS mainly because it clearly documents what the job
is doing.  \textit{e.g.} \texttt{PROCS INPUT FILTER AGC PLOT END}

It is usually simplest to run SIOSEIS out of a command file or shell script,
where the first line or lines are the operating system commands to execute the
program.  So, the command file might look like:
\begin{verbatim}
SIOSEIS << eof
PROCS INPUT FILTER AGC PLOT END
\end{verbatim}
which would execute the program SIOSEIS and process the data in the order
specified.

SIOSEIS must be told to stop reading from the command file by terminating
the command file with the word \texttt{END}.
\begin{verbatim}
SIOSEIS << eof
PROCS INPUT FILTER AGC PLOT END
END
\end{verbatim}

\begin{itemize}
\item SIOSEIS is case insensitive, \textit{i.e.} either upper or lower or both
      cases may be used.
\item SIOSEIS uses a line length of 200 characters.
\item SIOSEIS uses a blank or tab as delimiters to separate variables.
\item Multiple consecutive blanks are treated the same as a single blank.
\end{itemize}

The variables or parameters for each process described in the \texttt{PROCS}
list must be specified before the final \texttt{END}.  The parameters for each
process must start with the name of the process and must terminate with the
word \texttt{END}.  For clarity, it is wise to start and end with separate
lines.

\textit{e.g.}
\begin{verbatim}
SIOSEIS << eof
PROCS INPUT FILTER AGC PLOT END
   INPUT

   END
   FILTER

   END
   AGC

   END
   PLOT

   END
END
\end{verbatim}

The order in which the process parameters appear doesn't matter, \textit{e.g.}

\begin{verbatim}
SIOSEIS << eof
PROCS INPUT FILTER AGC PLOT END
   INPUT

   END
   PLOT

   END
   AGC

   END
   FILTER

   END
END
\end{verbatim}

The variables for each process are given in parameter lists governing a set of
\glspl{shot} (or \glspl{rp}).  Each parameter list must be terminated by the word
\texttt{END}.  A parameter list is similar to a \texttt{FOR} loop in many other
computer languages, but SIOSEIS has preset values for the initial loop value
(SIOSEIS parameter name \texttt{FNO}) and the terminal loop vaule
(\texttt{LNO}).

\textit{e.g.}
\begin{verbatim}
SIOSEIS << eof
PROCS INPUT FILTER AGC PLOT END
   INPUT
      END
   END
   FILTER
     END
   END
   AGC
      END
   END
   PLOT
      END
   END
END
\end{verbatim}

The individual parameters are always a parameter name followed by a value,
where the value may be a number or a group of letters.  \textit{e.g.}
\texttt{FIS 1   ADDWB YES}

The parameters within each parameter list apply to all \glspl{shot}/\glspl{rp} within
a first/last set.  \textit{e.g} \texttt{FNO 101 LNO 200 PASS 10 100 END}
means \gls{shot} 101 through \gls{shot} 200 will have that same filter passband.

SIOSEIS assumes that \gls{shot} numbers are monotonically increasing.
\textit{i.e.} each \gls{shot} number must be larger that the previous \gls{shot} number.

Multiple parameter lists may given in order to change parameter values on
different \glspl{shot}.  Using the \texttt{FOR} loop mentioned above, SIOSEIS permits
many loops within each process's parameters (nested loops are \textbf{NOT}
permitted).

\textit{e.g.}
\begin{verbatim}
NMO
    FNO 100 VTP 2000 5 2200 6.0  2600 6.7 END
    FNO 110 VTP 1950 5.1 2250 6.1 2620 6.8 END
END
\end{verbatim}
describes different velocity functions for \gls{shot} 100 and 110.

\begin{itemize}
\item The order in which parameters within each list is not important.
      \textit{e.g.} \texttt{FIS 1 LIS 5} is the same as \texttt{LIS 5 FIS 1}
\item Parameters and values must be separated by at least one blank or tab.
\item A single parameter list may be on more than one line.
\item Every parameter list must be terminated with the word \texttt{END}.
\item A single parameter line may not contain more than 200 characters.
\item Most parameters have default or preset values and do not have to be given
      unless the value needs to be changed.  A default value is reset to the
      program value on each parameter list, whereas a preset value does not
      change from list to list unless reset by the user.
\item SIOSEIS converts parameter values to the proper type so that decimal
      points on whole numbers are irrelevant.
      \textit{e.g.}   values 1 = 1. = 1.0 = 1.00
\item A duplicate parameter (mistakenly given twice in the same list) is not an
      error;  the value associated with the last one is used.
\item In line comments are useful to document the SIOSEIS job.  A parameter
      line is terminated by the normal carriage return or by the characters
      \texttt{!}, \texttt{\{}, or \texttt{\#}.  Anything after these
      characters will be ignored.
\end{itemize}

\textit{e.g}  In order to skip a \gls{shot} on tape, multiple \texttt{INPUT}
parameter lists are used.

\textit{i.e.} Use multiple \texttt{FOR} loops, with each loop terminated with
\texttt{END}.

\begin{verbatim}
SIOSEIS << eof
PROCS INPUT FILTER AGC PLOT END
   INPUT
      FIS 1 LIS 10 IUNIT 1 END
      FIS 12 LIS 20 END
   END
   FILTER
      PASS 20 80 END
   END
   AGC           # this is a comment
      WINLEN .5 END   ! anything after !, {, # is ignored
   END
   PLOT
      END
   END
END
\end{verbatim}

It is often convenient to spatially vary a parameter between \glspl{shot}/\glspl{rp}.  The
\glspl{shot}/\glspl{rp} on which the user defines the parameter may be called a ``control
point''  Shots/\glspl{rp} between control points will receive parameters that are
interpolated values of the control points, based on the number of \glspl{shot}/\glspl{rp}
between the control points and the current \gls{shot}.  Shots/\glspl{rp} outside
\textbf{ALL} controls points will receive parameters that are held constant
from the end control points.

\textit{e.g.} In order to vary the velocity function along the seismic line,
the velocity function (parameter \texttt{VTP}) is given at defined \glspl{rp}.
\begin{verbatim}
NMO
    FNO 10 VTP 1450 4.5 2200 9.0 END
    FNO 20 VTP 1400 4.0 2200 9.0 END
END
\end{verbatim}

In this case, \glspl{rp} 1-10 will receive the vtp associated with \gls{rp} 10.
Rps 11-19 receive interpolated velocity functions based on the distance
the \gls{rp} is from \gls{rp} 10 and \gls{rp} 20.  Rps 20 and larger get the velocity
defined at \gls{rp} 20.

\chapter{Definitions}

%\newglossaryentry{gather}
%{
%name=gather,
%description={A collection of traces sorted (or gathered) according to some
%criteria such as ``shot gather'' or ``RP'' gather.  SIOSEIS frequently refers
%to ``a gather'' meaning rp sorted, whereas a ``shot'' means a shot sorted.}
%}
%
%\newglossaryentry{shot}
%{
%    name={shot},
%    description={The collection of traces associated with the field shot.  SEG-Y
%trace header word 4 (shot record number) is used for the ``shot number''.  SEG-Y
%word 7 (rp trace number) must be 0 to be considered shot sorted.}
%}
%
%\newglossaryentry{FNO_LNO_List}
%{
%name={FNO-LNO list},
%description={}
%}
%
%\newglossaryentry{preset}
%{
%    name={preset},
%    description={A parameter value which stays the same from FNO/LNO list to
%    list until specified or given again.}
%}
%
%\newglossaryentry{default}
%{
%    name={default},
%    description={A parameter value which reverts back to its original value
%    after every FNO-LNO list.}
%}
%
%\newacronym{cdp}{CDP}{Common Depth Point}
%\newacronym{cmp}{CMP}{Common Mid Point}
%\newacronym{rp}{RP}{Reflection Point}
%\newacronym{fno}{FNO}{First number, referring to a shot or RP depending on how the data are sorted}
%\newacronym{lno}{LNO}{Last number, referring to a shot or RP depending on how the data are sorted}
%
%\printglossary[type=\acronymtype]

%\printglossary

\textbf{TODO: Convert to glossary and push to back of document}

\begin{verbatim}
GATHER - A collection of traces sorted (or gathered) according to some
         criteria such as ``shot gather'' or ``RP'' gather.  SIOSEIS
         frequently refers to ``a gather'' meaning \gls{rp} sorted, whereas
         a ``shot'' means a shot sorted.

RP  - Reflection Point or CDP (common depth point) or CMP (common
CMP   mid point) gather.  SIOSEIS uses SEG-Y trace header word 7 as
CDP   an indicator that the data are sorted by RP (reflection point).
       An SIOSEIS \gls{rp} sorted gather is terminated by using a special
      SEG-Y trace header flag (long integer word 51 set to -1), or a
      change in the RP number (header word 6) when using parameter
      FORGAT (foreign gather).

SHOT - The collection of traces associated with the field shot.  SEG-Y
       trace header word 4 (shot record number) is used for the "shot
       number".  SEG-Y word 7 (\gls{rp} trace number) must be 0 to be
       considered shot sorted.

FNO/LNO - First/Last number, where number is either a shot or \gls{rp}, depending
       on how the data are sorted.

FNO-LNO List - Many seismic processes require parameter values to change
       on different \glspl{shot}/\glspl{rp} or traces.  Each fno/lno list must be
       terminated by the word END (case insensitive).  e.g.
            fno 39987 twp 17 0 22 0 23 0 24 0  end
            fno 39989 twp 1 0 17 0  end
            fno 39990 twp 17 0 18 0  end

PRESET - The parameter value stays the same from fno/lno list to list
         until specified or given again.

DEFAULT - The parameter value reverts back to the original value after
          every fno/lno list.
\end{verbatim}

\chapter{Spatial Variation}
\label{c_spatial_variation}

It is often convenient to vary parameters between \glspl{shot}/\glspl{rp} because the
seismic data varies with the geology (\textit{e.g.} dip).  Think of the data as
a seismic line with time going down ($-y$ direction) and \gls{shot} distance in
the horizontal ($x$) direction.  Each \gls{shot} or \gls{rp} defines a unique place
horizontally.

\begin{verbatim}
         <---------    distance  (shots)  ------------>
shots        1  2  3  4  5  6 ........
        |
        |
        |
        |
      time
        |
        |
        |
        |
\end{verbatim}

The \glspl{shot}/\glspl{rp} on which the user defines the parameter are called control
points.  Shots/\glspl{rp} between control points receive parameters that are
interpolated values of the control points, based on the number of
\glspl{shot}/\glspl{rp} between the control points and the current \gls{shot}.  Shots/\glspl{rp}
outside ALL controls points will receive parameters that are held
constant from the end control point.

\textit{e.g.} In order to vary the velocity function along the seismic line, the
velocity function (parameter \texttt{VTP}) is given at defined \glspl{rp}.
\begin{verbatim}
nmo
    FNO 10 VTP 1450 4.5 2200 9.0 END
    FNO 20 VTP 1400 4.0 2200 9.0 END
end
\end{verbatim}
In this case, \glspl{rp} 1-10 will receive the \texttt{VTP} associated with \gls{rp} 10.
Rps 11-19 receive interpolated velocity functions based on the distance
the \gls{rp} is from \gls{rp} 10 and \gls{rp} 20.  Rps 20 and larger get the velocity
defined at \gls{rp} 20.

The parameter \texttt{LNO} (last \gls{shot}/\gls{rp} number) DEFAULTS to \texttt{FNO} in
the SIOSEIS processes that utilize spatial variation so that \texttt{LNO} does
not have to be given on every control point.  Default means that the parameter
value is reset on every parameter list (terminated by \texttt{END}).  Refer to
the SYNTAX documention in Chapter~\ref{c_syntax} for a description of parameter
list).

\chapter{Seismic Processes}
\label{c_seismic_processes}

\section{Overview}
\texttt{PROCS} is the method of describing the order of processing. The
process names must be listed \textbf{in the order of desired processing}. The
list must start with \texttt{PROCS} and is terminated with \texttt{END}. The
process names must be separated by blanks or tabs. The process names may be
upper or lower case.


The available processess as of August 2009 are:

\begin{itemize}
    \item \texttt{ACORR}: Autocorrelation. See Section~\ref{cmd_acorr}.
    \item \texttt{AGC}: Automatic Gain Control. See Section~\ref{cmd_agc}.
    \item \texttt{AVENOR}: Average amplitude normalize. See Section~\ref{cmd_avenor}.
    \item \texttt{CAT}: Concatenate traces or \glspl{shot}. See Section~\ref{cmd_cat}.
    \item \texttt{COFILT}: Dan Lizarralde's Coherency filter. See Section~\ref{cmd_cofilt}.
    \item \texttt{DEBIAS}: Bias removal. See Section~\ref{cmd_debias}.
    \item \texttt{DECON}: Linear predictive deconvolution. See Section~\ref{cmd_decon}.
    \item \texttt{DESPIKE}: Spike removal. See Section~\ref{cmd_despike}.
    \item \texttt{DISKIN}: Input of SEG-Y traces from disk. See Section~\ref{cmd_diskin}.
    \item \texttt{DISKOX}: Output of traces to an SEG-Y formatted disk file. x=a,b,c,{\ldots}j See Section~\ref{cmd_diskox}.
    \item \texttt{DMO}: Dip MoveOut. See Section~\ref{cmd_dmo}.
    \item \texttt{FDFMOD}: Finite-Difference Forward Modelling. See Section~\ref{cmd_fdfmod}.
    \item \texttt{FDMIGR}: Finite-Difference Migration. See Section~\ref{cmd_fdmigr}.
    \item \texttt{FILTER}: Time varying zero phase bandpass filter, Butterworth filter. See Section~\ref{cmd_filter}.
    \item \texttt{FKFILT}: Frequency-wavenumber filter. See Section~\ref{cmd_fkfilt}.
    \item \texttt{FKMIGR}: Frequency-wavenumber migration. See Section~\ref{cmd_fkmigr}.
    \item \texttt{FKSHIFT}: Phase shift in the frequency-wavenumber domain (depth migration) See Section~\ref{cmd_fkshift}.
    \item \texttt{FK2TX}: Frequency-wavenumber to time distance domain transformation. See Section~\ref{cmd_fk2tx}.
    \item \texttt{FLATEN}: Flatten the seismic section to a reference time. See Section~\ref{cmd_flaten}.
    \item \texttt{F2T}: Frequency to time domain transformation. See Section~\ref{cmd_f2t}.
    \item \texttt{GAINS}: Apply various time dependent gains. See Section~\ref{cmd_gains}.
    \item \texttt{GATHER}: Trace collection according to geometry. See Section~\ref{cmd_gather}.
    \item \texttt{GEOM}: Field geometry description. See Section~\ref{cmd_geom}.
    \item \texttt{GRDOUT}: Write a GMT format 1 grdfile. See Section~\ref{cmd_grdout}.
    \item \texttt{HEADER}: Manipulate the SEG-Y headers. See Section~\ref{cmd_header}.
    \item \texttt{HISTORY}: Append a history file. See Section~\ref{cmd_history}.
    \item \texttt{LOGSTX}: Log stretch and destretch. See Section~\ref{cmd_logstx}.
    \item \texttt{MIX}: Running or record sum of adjacent traces. See Section~\ref{cmd_mix}.
    \item \texttt{MUTE}: Zeroing of data. See Section~\ref{cmd_mute}.
    \item \texttt{NMO}: Normal moveout, movein, and slant moveout corrections. See Section~\ref{cmd_nmo}.
    \item \texttt{PLOT}: Record section plot. Sun rasterfiles, most Versatec and Calcomp electrostatic plotters, various thermal plotters, HP large format DesignJets. See Section~\ref{cmd_plot}.
    \item \texttt{PROUT}: Printer dump of traces. See Section~\ref{cmd_prout}.
    \item \texttt{PSMIGR}: Dan Lizarralde's Phase Shift Depth Migration (Split-Step Migration). See Section~\ref{cmd_psmigr}.
    \item \texttt{RESAMP}: Resample time domain data. See Section~\ref{cmd_resamp}.
    \item \texttt{SADD}: Scalar add. See Section~\ref{cmd_sadd}.
    \item \texttt{SEGDDIN}: SEG-D disk file input. See Section~\ref{cmd_segddin}.
    \item \texttt{SEG2IN}: SEG-2 disk file input. See Section~\ref{cmd_seg2in}.
    \item \texttt{SHIFT}: Time shift. See Section~\ref{cmd_shift}.
    \item \texttt{SMUTE}: Surgical mute. See Section~\ref{cmd_smute}.
    \item \texttt{SORT}: Generates tables for sorting SEG-Y disk files through diskin. See Section~\ref{cmd_sort}.
    \item \texttt{STACK}: Sum traces of a \gls{gather}. See Section~\ref{cmd_stack}.
    \item \texttt{STK}: Median stack, range varying stack See Section~\ref{cmd_stk}.
    \item \texttt{SWELL}: Swell removal. See Section~\ref{cmd_swell}.
    \item \texttt{SYN}: Spike trace generation. See Section~\ref{cmd_syn}.
    \item \texttt{TP2TX}: \gls{not:tau}-p to time-distance domain transformation. See Section~\ref{cmd_tp2tx}.
    \item \texttt{TREDIT}: Trace edit and spike removal. See Section~\ref{cmd_despike}.
    \item \texttt{TX2FK}: Time-distance to frequency-wavenumber domain transformation. See Section~\ref{cmd_tx2fk}.
    \item \texttt{TX2TP}: Time-distance to \gls{not:tau}-p domain transformation. See Section~\ref{cmd_tx2tp}.
    \item \texttt{T2D}: Time to depth conversion. See Section~\ref{cmd_t2d}.
    \item \texttt{T2F}: Time to frequency transformation. See Section~\ref{cmd_t2f}.
    \item \texttt{UADD}: User controlled addition to the seismic trace. See Section~\ref{cmd_uadd}.
    \item \texttt{UDECON}: User controlled Weiner filter design. See Section~\ref{cmd_udecon}.
    \item \texttt{UFILTR}: User given impulse response filter. See Section~\ref{cmd_ufiltr}.
    \item \texttt{UMULT}: User controlled multiplication with the seismic trace. See Section~\ref{cmd_umult}.
    \item \texttt{VELAN}: Constant velocity or semblance velocity analysis See Section~\ref{cmd_velan}.
    \item \texttt{WBT}: Water bottom time description. See Section~\ref{cmd_wbt}.
    \item \texttt{WEIGHT}: Trace weighting. See Section~\ref{cmd_weight}.
    \item \texttt{XCORR}: Cross-correlation. See Section~\ref{cmd_xcorr}.
    \item \texttt{XSTAR}: Convert EdgeTech XSTAR data to SEG-Y amplitude data. See Section~\ref{cmd_xstar}.
\end{itemize}

Several pre-\texttt{PROCS} directives may be given. The directive \textbf{must}
be given \textbf{before} the \texttt{PROCS} directive.
\begin{itemize}
    \item \texttt{DEBUG} will cause the name of each process to be printed just
          before it is entered. See Section~\ref{cmd_debug}
    \item \texttt{ECHO} indicates that \gls{ascii} output (stdout) will be enabled.
          See Section~\ref{cmd_echo}
    \item \texttt{NOECHO} indicates that most of the \gls{ascii} output (stdout) will
          be supressed. See Section~\ref{cmd_echo}
    \item \texttt{EDIT} indicates that the SIOSEIS parameters will be checked
          for syntax and validity without any data being processed.
    \item \texttt{LPRINT} Print internal information.
          See Section~\ref{cmd_lprint}
    \item \texttt{OVERRIDE} Overrides severe warnings.
          See Section~\ref{cmd_override}
    \item \texttt{REALTIME} indicates that process \texttt{DISKIN} will
          wait for additional trace; see Section~\ref{cmd_realtime}.
          \texttt{PLOT} and \texttt{SEGDDIN} could be made realtime - contact
          the author for further information
\end{itemize}

Example:

\begin{verbatim}
PROCS INPUT GEOM WBT GATHER NMO STACK OUTPUT DECON FILTER PLOT END
\end{verbatim}

Note:
\begin{enumerate}
\item The process' parameters do not have to be in the same order as  they are in the \texttt{PROCS} list.
\item The process' parameters may be given even if the process is not applied or in the \texttt{PROCS} list.
\item A process may be applied to the data more than once, however the process' parameters should be given only once.
\item Duplicate process' parameters cause the last set to be used.
\item The \texttt{PROCS} list should be the first set of parameters.
\end{enumerate}

\section{ACORR: Autocorrelation}
\label{cmd_acorr}

Process \texttt{ACORR} computes the one-sided autocorrelation function of a
trace. Only positive lags of the autocorrelation are computed and output. If
more than one autocorrelation window is specified on each trace, each output
trace is the concatenation of all the autocorrelations for that  trace.
Thus, the output length is the sum of the lengths of the individual windows.

More than one autocorrelation window may be given for each trace. The windows
may be spatially varied by \gls{shot} or \gls{rp} or by hanging the windows on the water
bottom.

All parameters that remain constant for a set of \glspl{shot} (\glspl{rp}) may be described
in a parameter set \texttt{FNO} to \texttt{LNO}. Windows between two parameter
sets are calculated by linearly interpolating between \texttt{LNO} of one set
and \texttt{FNO} of the next set.

Each parameter list must be terminated with the word \texttt{END}. The entire
set of \texttt{ACORR} parameters must be terminated by the word \texttt{END}.

\subsection{Parameter Dictionary}

\begin{description}
    \item[\texttt{SETS}] Start-end time pairs defining the autocorrelation windows.
        Times are in seconds and may be negative when hanging the
        windows from the water bottom. A maximum of 5 windows
        may be given.
        Required.    \textit{e.g.} \texttt{SETS 0 1}

    \item[\texttt{OLENS}] The output autocorrelation lengths in seconds. This
        corresponds to  the normal use of the number of lags to
        compute, but is in units of seconds. Each autocorrelation
        has it's own length. A maximum of 5 lengths may be given.
        Required.    \textit{e.g.} \texttt{OLENS .5}

    \item[\texttt{ADDWB}] When given a value of \texttt{YES}, the windows given via sets will
        be added to the water bottom time of the trace.
        (Water bottom times may be entered via process wbt).
        \Gls{preset}=\texttt{NO}     \textit{e.g.} \texttt{ADDWB YES}

    \item[\texttt{FNO}] The first \gls{shot} (or \gls{rp}) to autocorrelate. Shot (\gls{rp}) numbers must
        increase monotonically.
        \Gls{preset}=1

    \item[\texttt{LNO}] The last \gls{shot} (\gls{rp}) number to autocorrelate. \texttt{LNO} must be larger
        than \texttt{FNO} in each list and must increase from list to list.
        Default=\texttt{FNO}

    \item[\texttt{END}] Terminates each parameter list.
\end{description}

\section{AGC: Automatic Gain Control}
\label{cmd_agc}

Process \texttt{AGC} applies automatic gain control to every trace.
\texttt{AGC} is a type of amplitude normalization (modification) that results
in the amplitudes being more uniform, especially when the window length
decreases. \texttt{AGC} starts by finding the first non-zero sample and then
calculates the average absolute value of the window.  Successive windows are
calculated by shifting the window down one sample.  Each average absolute value
is then turned into a multiplier by dividing the average by an output level.

All parameters that remain constant for a set of \glspl{shot} (\glspl{rp}) may be described
in a parameter set \texttt{FNO} to \texttt{LNO}.  Windows between two parameter
sets are calculated by linearly interpolating between \texttt{LNO} of one set
and \texttt{FNO} of the next set.  Each parameter list must be terminated with
the word \texttt{END}.  The entire set of \texttt{AGC} parameters must be
terminated by the word \texttt{END}.

A null set of \texttt{AGC} parameters must be given even if all the parameters
are the presets.  e.g.  \texttt{AGC END END}

\texttt{AGC} honors the mute time in the SEG-Y header by starting the first
\texttt{AGC} window at the mute time rather than the first sample.

\subsection{Parameter Dictionary}

\begin{description}
\item[\texttt{FNO}] The first \gls{shot} (or \gls{rp}) to apply the \texttt{AGC} to.  Shot
    (\gls{rp}) numbers must increase monotonically.  \Gls{preset} = 1

\item[\texttt{LNO}] The last \gls{shot} (\gls{rp}) number to apply the \texttt{AGC} to.
    \texttt{LNO} must be larger than \texttt{FNO} in each list and must
    increase list to list.  Default = \texttt{FNO}

\item[\texttt{WINLEN}] The \texttt{AGC} window length in seconds.  \Gls{preset} = 0.100

\item[\texttt{PCTAGC}] Percent \texttt{AGC}.  The percentage of the computed
    multiplier to use in each \texttt{AGC} window. e.g. \texttt{PCTAGC} < 100.0
    ``softens'' the effect the \texttt{AGC}.  \Gls{preset} = 100.0  \textit{e.g.}
    \texttt{PCTAGC 50}

\item[\texttt{CENTER}] The center point, in seconds, of the \texttt{AGC} window
    that receives the multiplier of the window.  \Gls{preset} = \texttt{WINLEN} / 2

\item[\texttt{END}] Terminates each parameter list.
\end{description}

\section{AVENOR: Average Amplitude Normalize}
\label{cmd_avenor}

Process \texttt{AVENOR} normalizes every trace window to a user described
window level by calculating and appling a multiplier so that the average
amplitude within the window is at a certain level.  The resulting traces will
be more uniform in amplitude.

\texttt{AVENOR} finds a window multiplier by dividing the user's window level
by the average absolute value of the window.  The multiplier is held constant
for all data before the center of the first window, is linearly interpolated
between window centers and held constant for all data after the center of the
last window.  Thus, defining only one window results in a constant multiplier
for each trace.

Up to 4 windows may be given, each with a different window level, and may be
spatially varied by \gls{shot} or \gls{rp} or by hanging the windows on the water bottom.

All parameters that remain constant for a set of \glspl{shot} (\glspl{rp}) may be described
in a parameter set \texttt{FNO} to \texttt{LNO}.  Windows between two parameter
sets are calculated by linearly interpolating between lno of one set and
\texttt{FNO} of the next set.

Each parameter list must be terminated with the word \texttt{END}.  The entire
set of normalize parameters must be terminated by the word \texttt{END}.

A null set of \texttt{AVENOR} parameters must be given if all parameters to be
used are the presets.  \textit{e.g.} \texttt{AVENOR END END}

\subsection{Parameter Dictionary}

\begin{description}
\item[\texttt{SETS}] Start-end time pairs defining the windows.  Times are in
    seconds and may be negative when hanging the windows from the water bottom.
    A maximum of 4 windows may be given.  The windows may not overlap.
    \Gls{preset}= delay to last time.   \texttt{SETS 0 3 3 6}

\item[\texttt{LEVS}] The amplitude level of each window described by the sets.
    Each window may have a different level.  A negative level reverses the
    polarity.   Up to 4 levels may be given.  \Gls{preset} = 10000.0 10000.0
    10000.0 10000.0

\item[\texttt{FNO}] The first \gls{shot} (or \gls{rp}) to apply normalization to.  Shot
    (\gls{rp}) numbers must increase monotonically.  \Gls{preset} = 1

\item[\texttt{LNO}] The last \gls{shot} (\gls{rp}) number to apply normalization to.
    \texttt{LNO} must be larger than \texttt{FNO} in each list and must
    increase from list to list.  Default = \texttt{FNO}

\item[\texttt{VEL}] The velocity to use to `move-in' each window time.  Move-in
    is useful for describing window times that need to vary according to the
    \gls{shot}-receiver distance, as in following a reflector on a record before
        \texttt{NMO}.  Each window time is determined from the equation
        $t = \sqrt{t_{0}^{2} + \frac{x^{2}}{\texttt{VEL}^{2}}}$, where \gls{not:t0}%$t_{0}$
        is the two-way travel time, and \gls{not:x} %$x$
        is the \gls{shot} to receiver distance of the trace described via
        process \texttt{GEOM}.  \Gls{preset} = 0.

\item[\texttt{ADDWB}] When given a value of \texttt{YES}, the water bottom time
    will be added to all window times.  (Water bottom times may be entered via
    process \texttt{WBT}).  \Gls{preset} = \texttt{NO}

\item[\texttt{HOLD}] New multipliers are calculated on the first \texttt{HOLD}
    traces.  The multiplier from the last of these traces in then used on all
    successive traces.  \Gls{preset} = 0

\item[\texttt{LPRINT}] The normal debug parameter for values of 1 and 2.  When
    set to 4, the average absolute value for each window is printed.

\item[\texttt{END}] Terminates each parameter list.
\end{description}

\section{CAT: Concatenate Traces or Shots}
\label{cmd_cat}

Process \texttt{CAT} concatenates consecutive traces or \glspl{shot}.  The
concatenation is done without regard for the deep water delay.  \textit{i.e.}
The last sample of the trace being appended is always adjacent to the first
sample the next trace.

The output SEG-Y header is the header of the first trace of the series being
concatenated.  \Gls{shot}/\gls{rp} concatenation means that like traces are
concatenated.
\textit{e.g.}
\begin{verbatim}
TYPE SHOT N 2
\end{verbatim}
      shot 2 trace 1 is appended to shot 1 trace 1 and
      shot 2 trace 2 is appended to shot 1 trace 2.

Trace concatenation means \texttt{N} consecutive traces are appended.
\textit{e.g.} \texttt{FOR N 2}

\textit{e.g.}
\begin{verbatim}
TYPE TRACE N 2
\end{verbatim}
shot 1 trace 2 is appended to shot 1 trace 1.
shot 1 trace 4 is appended to shot 1 trace 3.

There is no spatial interpolation.

Each parameter list must be terminated with the word \texttt{END}.

A null set of parameters must be given if all the parameters are
presets. \textit{e.g.} \texttt{CAT END END}

\begin{description}
\item[\texttt{TYPE}] The type of concatenation, either \texttt{TRACE} or \texttt{SHOT}, \Gls{preset} = \texttt{SHOT} \textit{e.g.} \texttt{TYPE TRACE}

\item[\texttt{N}] The number of consecutive traces or \glspl{shot} to concatenate in each output record.  A value of 0 or 1 means that no concatenation should take place. \Gls{preset} = 2            \textit{e.g.} \texttt{N 3}

\item[\texttt{FNO}] The first \gls{shot}/\gls{rp} number the parameter list applies to.  \Gls{preset} = the first \gls{shot}/\gls{rp} received.    \textit{e.g.}   \texttt{FNO 101}

\item[\texttt{LNO}] The last \gls{shot}/\gls{rp} number the parameter list applies to.  \Gls{preset} = the last \gls{shot}/\gls{rp} received.    \textit{e.g.}   \texttt{LNO 101}

\item[\texttt{END}] Terminates the parameter list.
\end{description}

\section{COFILT: Dan Lizarralde's Coherency Filter}
\label{cmd_cofilt}

Process \texttt{COFILT} is a coherence filter based on semblance over a sweep of
linear trends about a point.  A sweep of velocities is performed across
a set of traces centered about each sample of each trace.  The velocity
scan with the maximum semblance is then selected and the amplitudes
along this velocity trajectory are weighted, summed, and optionally
scaled before being output.

\subsection{Parameter Dictionary}

\subsubsection{Required Parameters}
\begin{description}
\item[\texttt{DX}] Trace separation distance.  This is the distance between traces.  \textbf{REQUIRED}.  range 1.0 to 500.0 \textit{e.g.}  \texttt{DX 25}
\item[\texttt{NXWIN}] The number of traces to use in each velocity scan.  The velocity sweep is performed across \texttt{NXWIN} traces centered about each sample of each trace.  \texttt{NXWIN} must be an odd number.  \textbf{REQUIRED}.
\item[\texttt{VMMI}] Velocity minimum-maximum-increment to scan in every semblance window.  Either \texttt{VMMI} or \texttt{UMMI} is \textbf{REQUIRED}.  \textit{e.g.} \texttt{VMMI 2000 4000 100}
\item[\texttt{UMMI}] Slowness minimum-maximum-increment to scan in every semblance window.  Either \texttt{VMMI} or \texttt{UMMI} is \textbf{REQUIRED}.
\item[\texttt{WEIGHT}] A list of trace multipliers to use in weighting each trace before summing.  When the number of weights is less than \texttt{NXWIN}, the center \texttt{WEIGHT} is applied to the center trace of the sum.  \textit{e.g.}  \texttt{NXWIN 5 WEIGHT 1 2 1} and the samples along the maximum semblance path are: 1 0.8 0.9 0.8 0.9, then the output value for this point is: $(1 \times 0 + 0.8 \times 1 + 0.9 \times 2 + 0.8 \times 1 + 0.9 \times 0)$ \textbf{REQUIRED}
\end{description}

\subsubsection{Optional Parameters}

\begin{description}
    \item[\texttt{VSIGN}] Velocity sign.  \Gls{preset} = 0
\begin{description}
        \item[0] both positive and negative velocities between the minimum
         and maximum are used and the velocity increment is doubled.
         Stacked data should use \texttt{VSIGN 0}
     \item[1] the positive direction is defined as the direction of
         increasing offset values.  The sign of the velocities is
         determined from the sign of the range in the SEG-Y header.
         Only positive velocities are considered.
\end{description}

\item[\texttt{XSIGN}] Swap the sign of the range in the SEG-Y prior to using.  \Gls{preset} = 0

\item[\texttt{TYPE}] The type of additional weighting to use when summing the data of the \texttt{NXWIN} traces along the ``best velocity''.  \Gls{preset} = 0.
\begin{description}
    \item[1] the average of \texttt{WINLEN} semblance values of center trace is
         raised to power \texttt{SPOWER} is used as the weight.
         to weight the data.
         \textit{e.g.}  \texttt{NXWIN 5 WEIGHT 1 2 1  TYPE 1 SPOWER .5},
         and the samples along the maximum semblance path are:
         $(1, 0.8, 0.9, 0.8, 0.9)$ and the average semblance value is $0.5$,
         then the output value for this point is:
         \begin{equation}
             0.5^{\texttt{SPOWER}} \dfrac{\left((1 \times 0) + (0.8 \times 1) + (0.9 \times 2) + (0.8 \times 1) + (0.9 \times 0)\right)}{5}
         \end{equation}
     \item[2] the average of \texttt{WINLEN} semblance values of center trace is
         divided by the maximum semblance value and is then used to
         evaluate a sigmoidal thresholding function ranging from 0 to 1.
         This option produces something akin to a line drawing, but it
         normalizes the maximum semblance found on each trace, so be
         careful.
\end{description}

\item[\texttt{WINLEN}] The window length, in seconds, of the window to use in \texttt{TYPE 1} and \texttt{TYPE 2} weighting.  \Gls{preset}=.100.  Example: \texttt{WINLEN .080}

\item[\texttt{SPOWER}] The power to raise the average semblance when using \texttt{TYPE 1} weighting.

\item[\texttt{NXPAD}]  The number of dummy traces to insert at both ends of the set of traces being filtered.  \Gls{preset} = $\frac{\texttt{NXWIN}}{2}$.  Either \texttt{VMMI} or \texttt{UMMI} must be given.
\end{description}

\subsection{Discussion}
     If \texttt{VSIGN} = 1, then the positive direction is defined as the direction
     of increasing trace number, \textit{i.e.} traces are assume to have increasing
     offset values.  The sign of the velocities is determined from the
     sign of the header offset.

      If the traces are ordered in the correct  sense and the header
       offset sign is also correct, then choose \texttt{ixsign\_ch} = 1.
\begin{verbatim}
       ex.   trace # in file:  1  2  3  4  5  6
            offset in header: -5 -4 -2  0  1  4
\end{verbatim}

      If the traces are ordered in the opposite sense and the header
       offset sign is correct, then choose \texttt{ixsign\_ch} = -1.
\begin{verbatim}
       ex.   trace # in file:  1  2  3  4  5  6
            offset in header:  4  1  0 -2 -4 -5
\end{verbatim}

      If the traces are ordered in the correct sense and the header
       offset sign is incorrect, then choose \texttt{ixsign\_ch} = -1.
\begin{verbatim}
       ex.   trace # in file:  1  2  3  4  5  6
            offset in header:  5  4  2  0 -1 -4
\end{verbatim}

      If the traces are ordered in the opposite sense and the header
       offset sign is incorrect, then choose \texttt{ixsign\_ch} = 1.
\begin{verbatim}
       ex.   trace # in file:  1  2  3  4  5  6
            offset in header: -4 -1  0  2  4  5
\end{verbatim}

     If \texttt{VSIGN} = 0, then both positive and negative velocities between
     $v_{min}$ and $v_{max}$ will be used.  In this case, however, only $\frac{N_{vel}}{2}$
     velocities for each sign are used. Use this option for coherency
     filtering stacked data, for instance.

     All of the samples within a given velocity trajectory are used to
     calculate a semblance value for that velocity at each point.
     The maximum semblance value of each point is used to weight the
     point if \texttt{i\_wt\_type} = 1 or 2.
\begin{itemize}
    \item For \texttt{i\_wt\_type=1}, the maximum semblance value to the power
        of \texttt{SPOWER} is used to weight the data.
    \item For \texttt{i\_wt\_type=2}, the maximum semblance value is used to
        evaluate a sigmoidal thresholding function ranging from 0 to 1.  This
        option produces something akin to a line drawing, but it normalizes the
        max semblance found on each trace, so be careful.
    \item For \texttt{i\_wt\_type=0}, the maximum semblance value is used to
        evaluate
\end{itemize}

     In the following, ``the point'' means the center point of the velocity
     scan that is about to be scaled.

     A subset of the samples along the maximum semblance trajectory
     of a given point may also summed.  This subset
     is taken to be centered about the point.  The length of the subset
     is defined by \texttt{NWTS}, and \texttt{NWTS} values must be supplied.  Some but not
     all of these values may be zero.

\begin{verbatim}
     ex. for i_wt_type=1, nwts=3, wts= 1 2 1, nwin=5, assume the
             samples along the max semblance path are: 1 .8 .9 .8 .9
             and that the max. semblance value is .5, then the output
             value for this point is: (.5**spower)*(.8*1+.9*2+.8*1)/5,
             where the 5=1+2+1.
\end{verbatim}

\section{DEBIAS: Bias Removal}
\label{cmd_debias}

\texttt{DEBIAS} removes the bias of each and every trace.  Bias is the DC shift
or zero-frequency component.

There are no parameters necessary to run process \texttt{DEBIAS}.

\section{DECON: Linear Predictive Deconvolution}
\label{cmd_decon}

Process \texttt{DECON} designs and applies a least squares prediction error
filter.  Linear prediction deconvolution reduces periodic events such as
bubble pulses, ring, or even long period multiples.

Procedurally, an autocorrelation of the design window is taken and an
inverse filter is designed so that the autocorrelation of the same
window after decon results is a spike followed by zeroes.  \texttt{DECON} uses
the classic ``Weiner-Levinson'' method.  Les Hatton's "Seismic Data
Processing", Blackwell Scientific Publications, has an excellent section
on Weiner filtering.

Time varying decon is performed by applying different filters to
different parts of the trace.  The different parts of the trace are
called windows.  The portion of the trace between windows are merged by
ramping (linear).  The merge zone thus contains data that has been
filtered by different filters and then added together after being ramped.
\textit{e.g.}
\begin{verbatim}
               F1            F2            F3
          ..........     ..........     ..........
                    .   .          .   .
                     . .            . .
                      .              .
                     . .            . .
                    .   .          .   .
\end{verbatim}

Up to 5 windows may be given, and may be spatially varied by \gls{shot} or
by hanging the windows on the water bottom.

All parameters that remain constant for a set of \glspl{shot} (\glspl{rp}) may be
described in a parameter set \texttt{FNO} to \texttt{LNO}.  Windows between two parameter
sets are calculated by linearly interpolating between the \texttt{LNO} of one set
and the \texttt{FNO} of the next set.

Each parameter list must be terminated with the word \texttt{END}.  The entire
set of decon parameters must be terminated by the word \texttt{END}.

\subsection{Parameter Dictionary}

\begin{description}
\item[\texttt{SEDTS}] Start-end time pairs defining the design windows.  Times are in seconds and may be negative when hanging the windows from the water bottom.  A maximum of 5 windows may be given.  The window length should be many times the length of the period being removed; the period must be on the autocor- relation.  Parameter \texttt{SEATS} must be given when doing multi- window decon.  Required.

     \item[\texttt{VEL}] The velocity to use to `move-in' each design window time.  Move-in is useful for describing window times that need to vary according to the shot-receiver distance, as in following a reflector on a record before \texttt{NMO}.  Each design window time will be determined from the equation:
         \begin{equation}
             t = \sqrt{t_{0}^{2} + \dfrac{x^{2}}{\texttt{VEL}^{2}}}
         \end{equation}
         where $t_{0}$ is the normal incidence two way travel time, and $x$ is the \gls{shot} to receiver distance of the trace described via process \texttt{GEOM}.  \Gls{preset}=0.

     \item[\texttt{FILLEN}] The length of each filter in seconds.  Up to 5 filter lengths may be given.  The filter length must be sufficient to include the period being removed.  \Gls{preset} = $(0.160, 0.160, 0.160, 0.160 0.160)$

     \item[\texttt{PREWHI}] The percentage prewhitening to add before filter design.  A high level of prewhitening reduces the effectiveness of the filter.  Some level of prewhitening is needed in order for the filter to be stable.  Prewhitening is like performing a bandpass filter before decon.  \Gls{preset}=25.

     \item[\texttt{ADDWB}] When given a value of \texttt{YES}, the windows given via \texttt{SEDTS} will be added to the water bottom time of the trace.  (Water bottom times may be entered via process \texttt{WBT}).  \Gls{preset}=\texttt{NO}

     \item[\texttt{SEATS}] Start-end time pairs defining the application windows.  Times are in seconds and may be negative when hanging the windows from the water bottom.  A maximum of 5 windows may be given.  There must be the same number of design windows (\texttt{SEDTS}) as application windows (\texttt{SEATS}).  \Gls{preset} = whole trace

     \item[\texttt{PDIST}] The prediction distance, in seconds.  The prediction distance is the time delay of the event to be removed.  \Gls{preset} = $3 \times \mbox{sample interval}$.  \textit{e.g.} \texttt{PDIST} .15 (For water bottom multiple)

     \item[\texttt{PADDWB}] When given a value of yes, the water bottom time will be added to the value of \texttt{PDIST} on each trace.  \Gls{preset}=\texttt{NO}.  \textit{e.g.} \texttt{PADDWB} YES

     \item[\texttt{DOUBLE}] When given a value of yes, the correlations and convolution are performed in \texttt{DOUBLE PRECISION}.  \textbf{The use of this parameter will increase the CPU time considerably}, but might make the decon work better especially if long windows are used.  \Gls{preset} = \texttt{NO}.    \textit{e.g.}  \texttt{YES}

     \item[\texttt{FNO}] The first \gls{shot} (or \gls{rp}) to apply the decon to.  Shot (\gls{rp}) numbers must increase monotonically.  \Gls{preset}=1

     \item[\texttt{LNO}] The last \gls{shot} (\gls{rp}) number to apply the decon to.  \texttt{LNO} must be larger than \texttt{FNO} in each list and must increase list to list.  Default=\texttt{FNO}

     \item[\texttt{LPRINT}] Print switch.
\begin{description}
    \item[2] The edit phase parameters are printed.
    \item[4] The exectute phase parameters are printed.
    \item[8] The computed decon filter points are printed.
\end{description}

   \item[\texttt{END}] Terminates each parameter list.
\end{description}

\subsubsection{Obsolete Parameters}

\begin{description}
\item[\texttt{PREDIC}] The prediction distance, in samples.  Same as \texttt{PDIST} but
         in samples rather than time.  If both \texttt{PREDIC} and \texttt{PDIST} are
         given, \texttt{PDIST} is used.
\end{description}

\section{DESPIKE / TREDIT: Trace Edit and Spike Removal}
\label{cmd_despike}

   Processes \texttt{DESPIKE} and \texttt{TREDIT} are identical, yet both may be in
the same \texttt{PROCS LIST}.   Processes \texttt{DESPIKE} and \texttt{TREDIT} are trace
editing processes that remove spikes or kill traces that are ``bad''.
   Every sample is checked for being \texttt{NaN} (not-a-number) and is replaced
with a zero if it is.

There are several different detection algorithms:

\begin{enumerate}
\item Spike replacement.  Amplitudes less than \texttt{THRES(1)} or greater than \texttt{THRES(2)} are replaced by linear interpolation of the adjacent ``good'' amplitudes unless \texttt{KILL YES} is given, in which case the entire trace is killed.  Both \texttt{THRES} values must be given.  Parameters \texttt{ALPHA}, \texttt{SETS}, \texttt{VEL}, \texttt{ADDWB}, and \texttt{KILL} may be used in conjunction with this method.

\item The Trehu/Sutton method described in Marine Gephysical Researches
    16: 91-103, 1994. \cite{Trehu1994}  The algorithm is based on a five point moving
    window that compares the difference of the outer points with the
    inner points.
    \begin{equation}
        \texttt{SPIKE} = \abs{a_{3}-a_{2}} + \abs{a_{3}-a_{4}} > \texttt{FAC} \times \abs{a_{1}-a_{2}} + \abs{a_{4}-a_{5}}
    \end{equation}
    Spikes are replaced by linearly interpolating the adjacent
    ``good'' amplitudes or the entire trace may be killed if parameter
    \texttt{KILL YES} is given.  Parameter \texttt{FAC} must be given.  Parameters \texttt{SETS},
    \texttt{VEL}, \texttt{KILL}, \texttt{ALPHA}, and \texttt{ADDWB} are honored.

\item  A method that detects when amplitudes exceed the given quartile.
    Amplitudes that exceed \texttt{QUART} are clipped (replaced by the \texttt{QUART}
    amplitude).  Parameter \texttt{QUART} must be given.  Parameters \texttt{SETS}, \texttt{VEL},
    \texttt{ADDWB}, \texttt{ALPHA}, and \texttt{KILL} are honored.

\item  A window ratio algorithm where the trace is ``bad'' if the average
    amplitude of a ``short'' window over a ``long'' window exceeds the
    user given factor.  Parameters \texttt{SES}, \texttt{SEL}, and \texttt{FAC} must be given.
    Parameters \texttt{SETS}, \texttt{VEL}, \texttt{KILL}, \texttt{ALPHA}, and \texttt{ADDWB} are honored.

\item  The trace is ``bad'' and killed if the average absolute value of
    the windows described by \texttt{SETS} exceeds the given threshold.
    Parameters \texttt{SETS} and \texttt{THRES} must be given; Two windows may be defined,
    but only 1 threshold may be given.  Parameters \texttt{ADDWB}, \texttt{VEL}, \texttt{KILL},
    and \texttt{ALPHA} are also honored.

\item  The \texttt{MEDIAN} value is calculated for each time point within a \gls{gather}.
    The trace set must be indicated by the ``end-of-gather'' flag of -1
    (set by process gather or by diskin parameter ntrgat).  This is NOT
    a median stack; see process \texttt{STK} for a median stack.  The
    \texttt{DESPIKE}/\texttt{TREDIT} \texttt{MEDIAN} parameter simply returns the median value of
    all traces within a \gls{gather} for each time point.  While a gather
    goes into computing the median, only a single trace is output for
    each \gls{gather} set.

\item \texttt{MAXTYPE/MINTYPE} detection:
\begin{enumerate}
\item  \texttt{MINTYPE ABSVAL} with parameter \texttt{MINVAL} kills a trace if no absolute
    trace value exceeds \texttt{MINVAL}.  Useful for detecting ``dead'' traces
    created during acquisition or padding in FK and DMO processes.
\item  \texttt{MAXTYPE ABSVAL} with parameter \texttt{MAXVAL} kills a trace when any
    absolute trace value exceeds \texttt{MAXVAL}.
\item  \texttt{MINTYPE SDEV} with parameter \texttt{MINVAL} kills a trace if the standard
    deviation of the trace does not exceed \texttt{MINVAL}.
    Process \texttt{PROUT} parameter \texttt{INFO 5} prints the statistics of the trace.
\item  \texttt{MAXTYPE SDEV} with parameter \texttt{MAXVAL} kills a trace when any the
    stardard deviation of the trace exceeds \texttt{MAXVAL}.
\item  \texttt{MINTYPE VAR} with parameter \texttt{MINVAL} kills a trace if the variance
    of the trace does not exceed \texttt{MINVAL}.
\item  \texttt{MAXTYPE VAR} with parameter \texttt{MAXVAL} kills a trace when the
    variance of the trace exceeds \texttt{MAXVAL}.
\item  \texttt{MINTYPE SKEW} with parameter \texttt{MINVAL} kills a trace if the skew
    of the trace does not exceed \texttt{MINVAL}.
\item  \texttt{MAXTYPE SKEW} with parameter \texttt{MAXVAL} kills a trace when the
    skew of the trace exceeds \texttt{MAXVAL}.
\item  \texttt{MINTYPE KURT} with parameter \texttt{MINVAL} kills a trace if the kurtosis
    of the trace does not exceed \texttt{MINVAL}.
\item  \texttt{MAXTYPE KURT} with parameter \texttt{MAXVAL} kills a trace when the
    kurtosis of the trace exceeds \texttt{MAXVAL}.
\end{enumerate}

\item  Traces are killed when SEG-Y header values lie within user given
    \texttt{LIMITS} when \texttt{KILL INSIDE} is used or when the header values are
    outside \texttt{LIMITS} when \texttt{KILL OUTSIDE} is used.  This is an easy way
    to eliminate traces based on a SEG-Y header value such as the
    streamer depth.  The shallow or floating portion of the streamer
    can be excluded.

\item \gls{ltz} zeroes a portion of the trace if there are no zero crossing in
    a specified length of time.  This method was described by Stanghellini and
    Bonazzi in Geophysics Vol 67, No.1 (Jan/Feb 2002), pg 188-196
    \cite{Stanghellini2002}.  Parameter \texttt{WINLEN} is a sliding window for
    the sum of amplitudes used in determining when zero crossings occur.
    Parameter \texttt{HCYCLE} (half cycle) is the maximum time permitted
    between zero crossings.  If the time between zero crossings exceeds
    \texttt{HCYCLE}, the trace between the zero crossings is zeroed.

        Marine data \gls{shot} is rough seas often has long period streamer
    noise that overwhelms the signal.  The noise is not periodic
    and can not be removed by a frequency filter.  \gls{ltz} zeroes the
    portion of the trace when the low frequency noise is greater
    than the higher frequency signal.

    Only one \texttt{FNO}/\texttt{LNO} parameter list may be given.
\end{enumerate}

\subsection{Parameter Dictionary}

\begin{description}
\item[\texttt{PASS}] The passband of a 55 point time domain convolutional zero phase bandpass filter to apply before picking.  The filtered data are NOT passed to the next seismic process, thus the filter is applied ``offline''.  The low and high ``corners'' of the filter must be given.  \Gls{preset} = none      \textit{e.g.}  \texttt{PASS 100 200}

\item[\texttt{MINTYPE}] Allowable values are: \texttt{ABSVAL}, \texttt{SDEV}, \texttt{VAR}, \texttt{SKEW}, \texttt{KURT}
\item[\texttt{MAXTYPE}] Allowable values are: \texttt{ABSVAL}, \texttt{SDEV}, \texttt{VAR}, \texttt{SKEW}, \texttt{KURT}
         Use process \texttt{PROUT} parameter \texttt{INFO 5} to print the statistics.
\item[\texttt{MINVAL}] The minimum value of type \texttt{MINTYPE} each trace must have to be
         considered a good trace.  When there is no value of type \texttt{MINTYPE}
         that exceeds \texttt{MINVAL} within the \texttt{SET} window, the trace is killed
         by setting all amplitudes to zero and setting the SEG-Y trace id
         to 2, indicating a dead trace to other SIOSEIS processes.
\item[\texttt{MAXVAL}] The maximum value of type \texttt{MAXTYPE} allowable to be considered
         a good trace.  If any value of \texttt{MAXTYPE} exceeds \texttt{MAXVAL}, the trace
         will be set to zero and the SEG-Y dead trace flag will be set.

\item[\texttt{MEDIAN}] When set to \texttt{YES}, the median sample for each time sample is
         computed.  The input trace set (\gls{gather}) is replaced by a
         single trace which is the so called ``median trace''.
         \Gls{preset} = \texttt{NO}

\item[\texttt{THRES}] Threshold for algorithms 1 and 5 described above.  Algorithm
         1 requires two threshold values, a minimum and a maximum,
         so that any amplitude less than \texttt{THRES(1)} or greater than
         \texttt{THRES(2)} is considered a spike.  If only one \texttt{THRES} is given,
         then any trace with an average amplitude greater than the
         threshold is considered a spikey or wild trace and is
         killed if \texttt{KILL YES} is given.
         \Gls{preset} = 0 0    \textit{e.g.}   \texttt{THRES -1.e6 1.e6}

\item[\texttt{FAC}] The tolerance factor used in:
    \begin{description}
    \item[1]  The Trehu/Sutton method described in Marine Gephysical Researches 16: 91-103, 1994 \cite{Trehu1994}. The algorithm is based on a five point moving window.
    \item[2]  The \texttt{SES}/\texttt{SEL} method described below.  \Gls{preset} = 0.   \textit{e.g.} \texttt{FAC 5}.
    \end{description}

\item[\texttt{QUART}] Quartile, amplitudes above this quartile are replaced by
         the signed quartile value. (\texttt{QUART} must be between 1 and
         100).  This algorithm came from CWP/SU.
         \Gls{preset} = 0   \textit{e.g.} \texttt{QUART 99}.

\item[\texttt{SES}] Start and End times of the Short window when using the
         ``short over long average'' method.  \texttt{SEL} and \texttt{FAC} must be
         given also when using this method.  The trace is auto-
         matically killed (zeroed) when the ratio of the average
         absolute value of the samples in the \texttt{SES} window exceeds
         the average absolute value of the sample in the \texttt{SEL}
         window by a factor of \texttt{FAC}.  The average absolute value
         may be printed by using \texttt{LPRINT 4}.
         \Gls{preset} = none.
         \textit{e.g.}  \texttt{SES 0 .5 SEL 1 2 FAC .5}     will cause the trace to be
         killed if \texttt{SES} / \texttt{SEL} exceeds .5

\item[\texttt{SEL}] Start and end times of the Long window when using the
         ``short over long average'' method.  \texttt{SES} and \texttt{FAC} must be
         given also when using this method.
         \Gls{preset} = none.

\item[\texttt{ALPHA}] The trace is raised to the \texttt{ALPHA} power before detection is
    performed.  \textit{i.e.} $t_{i} = t_{i}^{\texttt{alpha}}$ is done first.
         \Gls{preset} = 1.       \textit{e.g.}   \texttt{ALPHA 2}.

\item[\texttt{SETS}] Start and End Times for the spike detection.  Only the data
         within \texttt{SETS} is examined.  Used in all methods except the
         \texttt{SES}/\texttt{SEL} and \texttt{MEDIAN} methods.  Only type 5 detection (average
         absolute value) uses two windows.
         \Gls{preset} = The entire trace         \textit{e.g.} 2.2 3.0 5 6

\item[\texttt{VEL}] The velocity to use to `move-in' each design window time.
         Move-in is useful for describing window times that need
         to vary according to the shot-receiver distance, as in
         following a reflector on a record before nmo.  Each
         design window time will be determined from the equation:
         $t = \sqrt{t_{0}^{2}+\frac{x^{2}}{\texttt{VEL}^{2}}}$, where $t_{0}$ is the normal
         incidence two way travel time, and $x$ is the \gls{shot} to
         receiver distance of the trace described via process \texttt{GEOM}.
         \Gls{preset}=0.        \textit{e.g.}   \texttt{VEL 1500}

\item[\texttt{ADDWB}] When given a value of \texttt{YES}, the water bottom time will be
         added to all window times.  (Water bottom times may be
         entered via process \texttt{WBT}).
         \Gls{preset}=no

\item[\texttt{KILL}] A switch when set to \texttt{YES} indicates that the trace be
         killed rather than just the spike be replaced.  It is also
         used with parameter \texttt{LIMITS} to define whether the trace
         header value governing the kill is \texttt{INSIDE} or \texttt{OUTSIDE} the
         \texttt{LIMITS}.  \Gls{preset} = \texttt{NO} \textit{e.g.}   \texttt{KILL YES}
\begin{description}
\item[\texttt{YES}] Indicates that the entire trace should be killed rather
         than just the detected spike samples replaced.  Honored
         by \texttt{THRES} (type 1), \texttt{SES}/\texttt{FAC} (Trehu - type 2), and \texttt{QUART}.
\item[\texttt{FLAGONLY}] sets the SEG-Y trace id to ``dead trace'' (word
         15 is set to 2), without zeroing the trace itself.  Honored
         whenever \texttt{KILL} YES is honored (\texttt{THRES} (type1), SES/\texttt{SEL}
         (type 2) and \texttt{QUART}).
\item[\texttt{NO}] Indicates that just the ``spikes'' should be replaced.
         Honored by \texttt{THRES} (type 1), \texttt{SES}/\texttt{FAC} (Trehu - type 2), and
         \texttt{QUART}.
\item[\texttt{INSIDE}] The trace is killed when the SEG-Y value is within
         or equal to the \texttt{LIMITS}.
\item[\texttt{OUTSIDE}] The trace is killed when the SEG-Y header value
         is outside the \texttt{LIMITS}.
\end{description}

\item[\texttt{LIMITS}] The minimum and maximum values, or limits, of the SEG-Y
         header value indicated by parameters \texttt{IHDR}, \texttt{LHDR}, or \texttt{HDR}.
         The trace is killed (zeroed and tagged as dead) when the
         SEG-Y header value is out of the \texttt{LIMITS} range when parameter
         \texttt{KILL} OUTSIDE is given.  The trace is killed when the SEG-Y
         header value is within \texttt{LIMITS} when \texttt{KILL} INSIDE is given.
         \Gls{preset} = 0 0    \textit{e.g.}    limits 0 -50 kill outside lhdr 11
         Kills traces where SEG-Y long word 11 is greater than 0 or
         less than -50.  (see document segy.header.  The streamer
         depth was in cm (word 11 scaled by word 35.  Depth is
         negative in this case.)

\item[\texttt{HDR}] Indicates the index of the floating point SEG-Y header word to
         use with the \texttt{LIMITS} parameter.

\item[\texttt{LHDR}] Indicates the index of the 32 bit integer SEG-Y header word to
         use with the \texttt{LIMITS} parameter.

\item[\texttt{IHDR}] Indicates the index of the 16 bit integer SEG-Y header word to
         use with the \texttt{LIMITS} parameter.

\item[\texttt{WINLEN}] The window length, in seconds, used in type 9 or \gls{ltz} (Local
         Trace Zeroing) where a portion of the trace is zeroed if no
         zero crossing are found with \texttt{HCYCLE} time.  \texttt{WINLEN} is the
         length of the window used in determining when zero crossing
         occur.

\item[\texttt{HCYCLE}] The length of time, in seconds, of a ``half cycle'' of the
         noise train to be zeroed in the \gls{ltz} method.   The data
         are zeroed if there isn't a zero crossing within
         \texttt{HCYCLE} seconds.
         \Gls{preset} = $\frac{\texttt{WINLEN}}{2}$.

\item[\texttt{ENDMUTE}] The mute start time relative to the detected spike rather
         than killing the entire trace.  Method 1, min/max spike
         detection only (thres neg pos).
         \Gls{preset} = not given.   \textit{e.g.}   \texttt{ENDMUTE -.01}   will mute from
         10 mils before the detected spike.

\item[\texttt{FNO}] The first \gls{shot}/\gls{rp} number the parameter list applies to.
    \textbf{ONLY ONE \texttt{FNO}/\texttt{LNO} LIST IS HONORED}.
         \Gls{preset} = the first \gls{shot}/\gls{rp} received.    \textit{e.g.}   \texttt{FNO 101}

\item[\texttt{LNO}] The last \gls{shot}/\gls{rp} number the parameter list applies to.
    \textbf{ONLY ONE \texttt{FNO}/\texttt{LNO} LIST IS HONORED}.
         \Gls{preset} = the last \gls{shot}/\gls{rp} received.    \textit{e.g.}   \texttt{LNO 101}

\item[\texttt{FTR}] The first trace number the parameter list applies to.
         \Gls{preset} = the first trace of each \gls{shot}/\gls{rp}.    \textit{e.g.}   \texttt{FTR 10}

\item[\texttt{LTR}] The last trace number the parameter list applies to.
         \Gls{preset} = the last trace of each \gls{shot}/\gls{rp}.    \textit{e.g.}   \texttt{LTR 10}

\item[\texttt{LPRINT}] The secret debug switch.
\begin{description}
    \item[4] The computed window arithmetic value is printed.  \textit{e.g.}
        the average absolute value of the window is printed when type 5
        depsiking is used.
     \item[8] Each trace that is killed is identified.
\end{description}

\item[\texttt{END}] Terminates the parameter list.
\end{description}

\section{DISKIN: Input of SEG-Y Traces from Disk}
\label{cmd_diskin}

\texttt{DISKIN} reads SEG-Y disk files.  The file may be in any SEG-Y data
format ( IEEE floating point, IBM floating point, 32 bit integer,
16 bit integer ).

Data in multiple files may be read by using multiple ipath lists,
each list ending with the word  end.  \textit{e.g.}
\begin{verbatim}
INPUT
   IPATH file1 END
   IPATH file2 END
END
\end{verbatim}

\texttt{DISKIN} ``corrects'' several sample intervals used in academia that are
not integers because there are thought of as a sample rate.  All other
processes in SIOSEIS use this floating point representation created by
\texttt{DISKIN}.  7812ns becomes 1./128., 488ns becomes 1./2048.,
67ns becomes 1/15000., 63ns becomes 1./16000., 31ns becomes 1./32000.

\subsection{Parameter Dictionary}

\begin{description}
\item[\texttt{IPATH}] The input SEG-Y pathname (filename).  200 characters maximum.
         The special Unix filename {/dev/null} may be used to create
         null (dead) \glspl{shot}.  \texttt{FNO}, \texttt{LNO}, \texttt{FTR}, \texttt{LTR} are honored.  \texttt{/dev/null}
         may not be the first file read.
         Required. \textit{e.g.} \texttt{IPATH /seis/vel.123}
\end{description}

\subsubsection{SHOT/RP Parameters}

\begin{description}
\item[\texttt{FNO}] The first \gls{shot}/\gls{rp} number the parameter list applies to.  \texttt{NO} is a \gls{shot} if SEG-Y word 7 is zero and an RP if non-zero.  Default = the first \gls{shot}/\gls{rp} in file \texttt{IPATH}.  \textit{e.g.}   \texttt{FNO 101}

\item[\texttt{LNO}] The last \glspl{shot}/\gls{rp} number the parameter list applies to.  Default = the last \gls{shot}/\gls{rp} in file \texttt{IPATH}.   \textit{e.g.} \texttt{LNO 101}

\item[\texttt{NOINC}] The increment between \texttt{FNO} and \texttt{LNO}.  Only honored when \texttt{FNO} and \texttt{LNO} are used.  A \texttt{NOINC} of 99999 indicates that any and all \glspl{shot}/\glspl{rp} will be used in any order as long as it is between \texttt{FNO} and \texttt{LNO}.  \texttt{NOINC} 99999 is useful when there are missing \glspl{shot}/\glspl{rp}.  \Gls{preset} = 99999 \textit{e.g.} \texttt{NOINC 2}

\item[\texttt{FTR}] The first trace of each \gls{shot}/\gls{rp} to read from disk file \texttt{IPATH}.  Traces less than \texttt{FTR} will not be read.  \Gls{preset} = the first trace of every \gls{shot}/\gls{rp}   \textit{e.g.}  \texttt{FTR 11}

\item[\texttt{LTR}] The last trace of each \gls{shot}/\gls{rp} to read from disk file \texttt{IPATH}.  Traces greater than \texttt{LTR} will not be read.  \Gls{preset} = the last trace of each \gls{shot}/\gls{rp}    \textit{e.g.} \texttt{LTR 11}

\item[\texttt{TRINC}] The trace increment between \texttt{FTR} and \texttt{LTR}.  \texttt{FTR} and \texttt{LTR} must be given when \texttt{TRINC} is used.  A \texttt{TRINC} of 99999 indicates that any and all traces will be used in any order as long as it is between \texttt{FTR} and \texttt{LTR}.  \texttt{TRINC} 99999 is useful when there are missing traces or out of order traces.  \Gls{preset} = 99999 \textit{e.g.} \texttt{TRINC 1}

\item[\texttt{FNTODO}] First number of traces TO DO.  A useful parameter when doing quality control and only the first \texttt{FNTODO} traces should be read.  Similar to the Unix \texttt{head} command.  \Gls{preset} = 0    \textit{e.g.} \texttt{FNTODO 300}

\item[\texttt{LNTODO}] Last number of traces TO DO.  Takes the last \texttt{LNTODO} traces in the input file.  This will only work on files where all traces are a fixed length.  Similar to the Unix \texttt{tail} command.  This was intended for realtime processing for display the last stacked traces.  \Gls{preset} = 0    \textit{e.g.} \texttt{LNTODO 300}
\end{description}

\subsubsection{Time parameters}

\begin{description}
\item[\texttt{FDAY}] The Julian day of the first data to read from disk file \texttt{IPATH}.  Data before \texttt{FDAY} will not be read.  \Gls{preset} = 0      \textit{e.g.}  \texttt{FDAY 123}

\item[\texttt{LDAY}] The day of year of the last data to read from disk file \texttt{IPATH}.  Data in \texttt{IPATH} after \texttt{LDAY} will be ignored.  \Gls{preset} = 366     \textit{e.g.}  \texttt{LDAY 123}

\item[\texttt{FGMT}] The \texttt{GMT} (in \texttt{HHMM}, or hours and minutes based on a 24 hour clock) of the first data in file \texttt{IPATH} to read.  \texttt{FGMT} is set to 0 after the first data are found (default).  Default = 0    \textit{e.g.}  \texttt{FGMT 0800}

\item[\texttt{LGMT}] The \texttt{GMT} of the last data to read from disk file \texttt{IPATH}.
         Data in file \texttt{IPATH} after \texttt{LDAY} will be ignored.
         Default = 2500     \textit{e.g.}   \texttt{LGMT 1300}

\item[\texttt{GMTINC}] The increment between \texttt{FGMT} and \texttt{LGMT}.
         \Gls{preset} = 1        \textit{e.g.}  \texttt{GMTINC 2}

\item[\texttt{FSEC}]   The second of the minute of \texttt{FGMT} of the first \gls{shot}/\gls{rp} in
         \texttt{IPATH} to read.  Data before \texttt{FSEC} of \texttt{FGMT} will be ignored.
         \Gls{preset} = 0       \textit{e.g.}   \texttt{FSEC 20}

\item[\texttt{LSEC}]   The second of the minute of \texttt{LGMT} of the last data to read
         from disk file \texttt{IPATH}.  Data after \texttt{LSEC} of \texttt{LGMT} will be
         ignored.
         \Gls{preset} = 60     \textit{e.g.}    \texttt{LSEC 30}

\item[\texttt{SECINC}] The increment in seconds between \glspl{shot}.  After \texttt{FSEC} is
         found, successive \glspl{shot} must be in exact increments of
         \texttt{SECINC}.  \textit{e.g.} the first \gls{shot} is at \texttt{FSEC}, the next is at
         \texttt{FSEC + SECINC}.  (Modulo 60).
         \Gls{preset} = 0      \textit{e.g.}     \texttt{SECINC 15}

\item[\texttt{FTR}]    The first trace of each \gls{shot}/\gls{rp} to read from disk file
         \texttt{IPATH}.  Traces less than \texttt{FTR} will not be read.
         \Gls{preset} = the first trace of every \gls{shot}/\gls{rp}   \textit{e.g.}   \texttt{FTR  11}

\item[\texttt{LTR}]    The last trace of each \gls{shot}/\gls{rp} to read from disk file
         \texttt{IPATH}.  Traces greater than \texttt{LTR} will not be read.
         \Gls{preset} = the last trace of each \gls{shot}/\gls{rp} \textit{e.g.}   \texttt{LTR 11}

\item[\texttt{TRINC}]  The trace increment between \texttt{FTR} and \texttt{LTR}.  \texttt{FTR} and \texttt{LTR}
         must be given when \texttt{TRINC} is used.
         \Gls{preset} = 1     \textit{e.g.}    \texttt{TRINC 2}
\end{description}

\subsubsection{Other Useful Parameters}

\begin{description}
\item[\texttt{RANDOM}] Use disk positioning (lseek) rather than sequential reads to
         find the first \gls{shot} (\texttt{FNO}).  This is only available for disk
         files conforming to SEG-Y Rev 1 standards and has the SEG-Y
         ``fixed trace length'' flag set.  In addition, \gls{shot} and/or RP
         numbers must be monotonically increasing by 1.
         \Gls{preset} = 1    \textit{e.g.}    \texttt{RANDOM 0}

\item[\texttt{RENUM}]  Renumber the \glspl{shot}/\glspl{rp} consecutively, starting with the
         number given.  Useful when there are duplicate \gls{shot}/\gls{rp}
         numbers in the job.  \texttt{RENUM} increments the \gls{shot} number
         after the \gls{shot} trace number is equal to the number of
         traces per \gls{shot}, which comes from the SEG-Y tape header
         unless it is specified via \texttt{DISKIN} parameter \texttt{NTRCS}.
         Honored on the first \texttt{FNO}/\texttt{LNO}/\texttt{END} list only.
         \Gls{preset} = none. \textit{e.g.}    \texttt{RENUM 101}

\item[\texttt{RETRAC}] Renumber the trace numbers within each \gls{shot}/\gls{rp} so that the
         first trace of each \gls{shot}/\gls{rp} is 1.
         Honored on the first fno/lno/end list only.
         \Gls{preset} = none. \textit{e.g.}     \texttt{RETRAC 1}

\item[\texttt{SECS}]   The number of seconds of data to process in file \texttt{IPATH}.
         Data later than delay+secs will be omitted.
         \Gls{preset} = none.

\item[\texttt{DECIMF}] The decimation factor used to resample the data while reading from disk.  No anti-alias filter is applied before decimation.
         \Gls{preset} = 1 \textit{e.g.} \texttt{DECIMF 2} (every other sample is discarded)

\item[\texttt{IFMT}]   The data format of the input disk file.  \texttt{IFMT} should be
    used only when it is necessary to override the value in the
    SEG-Y header of the data file \texttt{IPATH}.  \Gls{preset} = taken from the disk file.
    \begin{description}
        \item[1]  IBM floating point.
        \item[2]  32 bit 2's complement integer.
        \item[3]  16 bit integer.
        \item[4]  16 bit UTIG floating point (not IEEE)
        \item[5]  IEEE floating point (host computer)
    \end{description}

\item[\texttt{SI}]    The sample interval, in seconds, of the data in file
         \texttt{IPATH}.  \texttt{SI} overrides the value in the SEG-Y header.
         \Gls{preset} = disk  \textit{e.g.}   \texttt{SI .004}

\item[\texttt{DELAY}] The deep water delay, in seconds, of the data in \texttt{IPATH}.
         \texttt{DELAY} overrides the value in the SEG-Y trace header read from disk.
         \Gls{preset} = none  \textit{e.g.}  \texttt{DELAY 3.0}

\item[\texttt{NTRCS}] The number of traces in each \gls{shot} in disk file \texttt{IPATH}.  \texttt{NTRCS}
         overrides the value in the SEG-Y header in file \texttt{IPATH}.
         \Gls{preset} = none. \textit{e.g.}   \texttt{NTRCS 1}

\item[\texttt{NSAMPS}] The number of samples in each trace,  \texttt{NSAMPS} overrides the
         value in the SEG-Y trace header.
         \Gls{preset} = none. \textit{e.g.}   \texttt{NSAMPS 2560}

\item[\texttt{NTRGAT}] The number of traces in every \gls{rp}.  Every \texttt{NTRGAT} trace
         will contain an ``end of gather'' flag in header word 51.
         Used to convert \gls{shot} sorted files into \gls{rp} sorted files
         without going through processes geom or \gls{gather}.  Shot/\gls{rp}
         boundaries are ignored when counting traces for ntrgat.
         \texttt{NTRGAT} 0 sets the flag to 0 on every trace.
         \Gls{preset} = -1       \textit{e.g.}    \texttt{NTRGAT 24}

\item[\texttt{FORGAT}] Foreign \gls{gather} switch.  The use of \texttt{FORGAT} indicates that
         the input \glspl{gather} were not generated by SIOSEIS and do not
         have the end-of-gather convention used by SIOSEIS (a -1 in
         SEG-Y header word 51).  \texttt{FORGAT} is similar to \texttt{NTRGAT} but
         allows each \gls{gather} to have a different number of traces.
         The end-of-gather is detected when the next trace has a
         different \gls{rp} number or is equal to \texttt{LTR} (when ltr is given).
         The value of \texttt{FORGAT} indicates the number of \glspl{rp} to concatenate
         into a single \gls{gather} which is terminated by the -1
         end-of-gather flag.  \texttt{LDGO} \glspl{gather} start with the largest
         trace number first, which breaks the SIOSEIS
         monotonically increasing assumption; \texttt{LDGO} \glspl{gather} may be
         read using forgat 1 and ftr 99999, in which case SIOSEIS
         will use all traces within the \gls{gather}.
         \Gls{preset} = 0     \textit{e.g.} \texttt{FORGAT 1}

\item[\texttt{MINTRS}] The minimum number of traces each gather must have.  If
         an input \gls{rp} does not have the specified minimum number of
         traces, \texttt{DISKIN} will create dead traces so that there are
         \texttt{MINTRS} traces.
         \Gls{preset} = 0     \textit{e.g.}  \texttt{MINTRS 24}

\item[\texttt{SET}]    The start and end times of the data to read from disk.
         \texttt{SET} is a pair of times in seconds. The use of \texttt{SET} causes
         the deep water delay and the number of samples to be
         changed.  If either \texttt{SET} is outside of the data, the data
         is padded with zeroes.  The data will always be
         \texttt{SET}(2) - \texttt{SET}(1) long.
         \Gls{preset} = none  \textit{e.g.}   \texttt{SET 2.0 3.0}

\item[\texttt{SORT}]   The use of this parameter overrides the automatic
         determination of the data sort for \texttt{DISKIN}.  \texttt{SORT} does not
         change any part of the SEG-Y header, so other SIOSEIS
         processes are not affected by this parameter (\textit{i.e.}
         successive processes in the process list will receive the
         data with the original sort).   The choices are \texttt{SHOT} or \texttt{CDP} or \texttt{STACK}
         \Gls{preset} = none  \textit{e.g.}  \texttt{SORT SHOT}

\item[\texttt{REWIND}] A \texttt{YES}/\texttt{NO} switch indicating whether the input file should be
         rewound before reading the first \gls{shot}, \texttt{FNO}, of the fno-lno
         list.  The data on disk may not be accessed in reverse order,
         so the only way to get a \gls{shot}/\gls{rp} that has already been passed
         is to start searching from the beginning of the disk file.
         \texttt{REWIND} is reset to \texttt{NO} immediately after use; rewind only
         occurs at \texttt{FNO} and only occurs on the list specified.
         Default = \texttt{YES}    \textit{e.g.}   \texttt{REWIND NO}

\item[\texttt{FORMAT}] The type of seismic format of the file \texttt{IPATH}.  \Gls{preset} = \texttt{SEGY}           \textit{e.g.} \texttt{FORMAT EDGETECH}
    \begin{description}
        \item[\texttt{SEGY}]   An SEG-Y disk file with all headers.
        \item[\texttt{KNUDSEN}]   Does a 16 bit byte swap on trace header bytes 180-240 rather than the default 32 bit byte swap.
        \item[\texttt{SSC}] The Seimographic Services Corporation' Phoenix format.
        \item[\texttt{IRIS}]  The IRIS PASSCAL format is like SEG-Y but omits both file headers and the data format code is in the trace header.  SIOSEIS can only handle 64k data samples.  There is only one trace in each PASSCAL file.
        \item[\texttt{SU}]  the Colorado School of Mines Center for Wave Phenomena's Seismic Unix file format; no file headers and the data are host floating point.
        \item[\texttt{NOHEAD}]  No SEG-Y file headers.  Parameter \texttt{IFMT} must be given.
        \item[\texttt{SWAPPED}]  SEG-Y files written on Intel or DEC machines that are still in little-endian byte order (low byte first). SIOSEIS written SEG-Y files are in the correct order (big-endian).
        \item[\texttt{ODEC}]  The ODEC 3.5 format which is byte swapped and has a 320 byte trace trailer.
        \item[\texttt{EDGETECH}]   Edgetech uses a pseudo-SEG-Y format.  Edgetech puts the \gls{shot} number and trace number in different locations.  It also uses the cpu time instead of the \gls{gps} time for the time of \gls{shot}.  This format precludes the use of parameters \texttt{FNO}, \texttt{LNO}, \texttt{NOINC}, \texttt{FTR}, \texttt{LTR}, and \texttt{TRINC}.  Process \texttt{XSTAR} should be used on ALL Edgetech data.  GeoSTAR data must be converted to \texttt{XSTAR} format prior to use by sioseis since GeoSTAR data do not have the first two SEG-Y headers and is in SEG-Y byte order.  Course and speed are saved in SEG-Y words 63 and 64.
        \item[\texttt{UTIG-OBS}]   UTIG uses a 32 bit integer for the delay.
        \item[\texttt{WAV}]   A WAV file (44 byte header and 16 or 32 bit little endian int)
        \item[\texttt{ASCII}]   Each \texttt{IPATH} file should contain an \gls{ascii} list of trace amplitudes.  The length of the trace is the number of \gls{ascii} values read.  The \gls{shot} number will be 1.  The first \texttt{IPATH} will be trace 1 and successive \texttt{IPATH} will increment the trace number.  Parameter \texttt{SI} is honored.
        \item[\texttt{BINARY}]   Each \texttt{IPATH} file should contain a binary string of trace amplitudes.  The binary file must be IEEE floating point words in the byte order of the host computer.  The length of the trace is the number of values read.  The \gls{shot} number will be 1.  The first \texttt{IPATH} will be trace 1 and successive \texttt{IPATH} will increment the trace number.  Parameter \texttt{SI} is honored.
        \item[\texttt{NIUST}]   \texttt{NIUST} puts the delay in ms in bytes 91 and 92 with us in 181 and 182.  \texttt{DISKIN} converts the delay to SEG-Y Rev. 1
    \end{description}

\item[\texttt{IPATH2}] The pathname of a second data file that will be merged with
         the data of \texttt{IPATH}.  The merging is done by alternate reads
         from \texttt{IPATH} and \texttt{IPATH2}.  See parameter \texttt{ALT} for whether the
         merge is by trace or record (\gls{shot}/\gls{rp}).  \textit{e.g.} When \texttt{ALT} is 2,
         the first trace comes from \texttt{IPATH}, the second comes from
         \texttt{IPATH2}, the third from \texttt{IPATH}, the fourth from \texttt{IPATH2}, \textit{etc}
         \Gls{preset} = none    \textit{e.g.}  \texttt{IPATH2 datafile2}

\item[\texttt{ALT}] The type of interleaving when using \texttt{IPATH2}. \Gls{preset} = 2
\begin{description}
\item[1] Alternate records (\glspl{shot} or \glspl{rp}) are read from \texttt{IPATH} and
         \texttt{IPATH2}.  Both files MUST contain the same \gls{shot}/\gls{rp} numbers.
         \textit{e.g.} \texttt{IPATH SHOT 1, IPATH2 SHOT 1, IPATH SHOT 2, IPATH2 SHOT 2}.
\item[2] Alternate traces are read from \texttt{IPATH} and \texttt{IPATH2}.  \textit{e.g.}
         \texttt{IPATH SHOT 1 TRACE 1, IPATH2 SHOT 1 TRACE 1, IPATH SHOT 2 TRACE 2, IPATH2 SHOT 2 TRACE 2}
\end{description}

\item[\texttt{SPATH}] The pathname of the ``sort'' file.  When \texttt{SPATH} is given, process \texttt{DISKIN} reads the seismic traces in the order specified by \texttt{SPATH} rather than the order actually in \texttt{IPATH}.  \Gls{preset} = none  \textit{e.g.} \texttt{/usr/users/joe/data/sort.line1}

\item[\texttt{NO  xn}] The word type and the index of the SEG-Y trace header to use for searching rather than the \gls{shot} or \gls{rp} number.  \texttt{FNO} and \texttt{LNO} no longer refer to \gls{shot} or \gls{rp} number, but to the new SEG-Y word.
     \texttt{x} = \texttt{I}, means short integer (16 bit integer trace header).
     \texttt{x} = \texttt{L}, means long integer (32 bit integer trace header).
     \texttt{x} = \texttt{R}, means real word (host floating point).
     \texttt{n} = the index with the SEG-Y trace header.
         Example:  \texttt{NO L5} will use the SEG-Y ``Energy source point number'' rather than the \gls{shot} number.  \Gls{preset} = none

\item[\texttt{TR  xn}] The word type and the index of the SEG-Y trace header to use for searching rather than the \gls{shot}/\gls{rp} trace number.
     \texttt{x} = \texttt{I}, means short integer (16 bit integer trace header).
     \texttt{x} = \texttt{L}, means long integer (32 bit integer trace header).
     \texttt{x} = \texttt{R}, means real word (host floating point).
     \texttt{n} = the index with the SEG-Y trace header.
         Example:  \texttt{TR L10  FTR -500 LTR 500} will read all traces with ranges between -500 and +500.  \Gls{preset} = none

     \item[\texttt{ALLNO}] A \texttt{YES}/\texttt{NO} switch that indicates that the entire file will be searched for additional \glspl{shot}/\glspl{rp} that might be in the \texttt{FNO-LNO} range.  \texttt{ALLNO NO} will cause the sioseis job to end faster when \texttt{LNO} is not at the end of the file.  \texttt{ALLNO} governs \texttt{LNO} only.  It is impossible to skip data before \texttt{FNO}.  \texttt{DISKOX} parameter \texttt{FLINC} may be used to create small files for velocity analysis.  \Gls{preset} = \texttt{YES}

\item[\texttt{END}] Terminates each parameter list.
\end{description}

Example:
\begin{verbatim}
PROCS DISKIN PROUT END
    DISKIN
        RENUM 1
        FNO 1 LNO 10 IPATH dataset1 END
        FNO 1 LNO 10 IPATH dataset2 END
    END
    PROUT
        FNO 1 LNO 9999 FTR 1 LTR 999 END
    END
END
\end{verbatim}

\subsubsection{Obsolete and Hidden Parameters}
\begin{description}
\item[\texttt{ASCII}] Some processing packages (\textit{e.g.} LDGO) violate the SEG-Y
         standard by writing the first SEG-Y header in \gls{ascii}
         rather than \gls{ebcdic}.   The use of this parameter
         indicates that the header is in \gls{ascii}.   An \gls{ascii} header
         usually manifests itself in SIOSEIS in the plot header as
         bad characters ( usually \texttt{?????}'s).
         \Gls{preset} = 0     \textit{e.g.}  \texttt{ASCII 1}

     \item[\texttt{LPRINT}] A debugging parameter.
\end{description}

\section{DISKOX: Output of traces to an SEG-Y Formatted Disk File}
\label{cmd_diskox}

Process \texttt{DISKOX} is a set of 10 processes which write SEG-Y disk files.
The process names are actually \texttt{DISKOA}, \texttt{DISKOB}, \texttt{DISKOC}, \ldots \texttt{DISKOJ}, but
are listed here as a single process since they are virtually identical.
SIOSEIS permits any process to appear only once in the \texttt{PROCS} list, yet
the user may wish to save the data at multiple stages.  Processes
\texttt{DISKOA}, \texttt{DISKOB} \textit{etc.} permit up to 4 distinct disko output stages.  \textit{e.g.}
\texttt{PROCESS INPUT GEOM GATHER DISKOA NMO STACK DISKOB FILTER PLOT END}

Each process \texttt{DISKOX} is totally independent of any other process \texttt{DISKOX}.
The order A, B, C, D does not matter since this is just a way of allowing
four unique processes.  \textit{e.g.} \texttt{PROCESS DISKIN STACK DISKOD PLOT END} is
valid.

The \gls{shot}/\gls{rp} numbers and trace numbers of the output are the same as the
input except when \texttt{FON}, \texttt{NOINC}, \texttt{FTR}, or \texttt{TRINC} are given.  The output
shot/\glspl{rp} always start with trace one and output trace numbers are always
incremented by 1.

\subsection{Parameter Dictionary}

\begin{description}
\item[\texttt{OPATH}] The output pathname (filename).  Required. \textit{e.g.} \texttt{OPATH /seis/vel.123}
\begin{description}
\item[\texttt{DATE}] or date, the filename will come the SEG-Y trace header in the form: \texttt{day\textit{DDD}-\textit{HHMM}z.segy}, where \texttt{DDD} is the day of year and \texttt{HH}  is the hour, \texttt{MM} is the minute of the SEG-Y time of \gls{shot}.
\item[\texttt{SHOTNO}]  or shotno, the filename will come the SEG-Y trace header in the form: \texttt{shotSSSSSS.segy}, where \texttt{SSSSSS} is the \gls{shot} number (word 3) in the SEG-Y trace header.
\end{description}

\item[\texttt{OFMT}] The data word format of the output disk file.  \Gls{preset} = 5 (IEEE floating point)
\begin{description}
\item[0] Same as the input format.
\item[1] IBM floating point - \texttt{OBSOLETE}
\item[2] 32 bit 2's complement integer.
\item[3] 16 bit integer.
\item[4] UTIG 16 bit floating point - \textbf{NOT STANDARD}
\item[5] IEEE floating point.  (non standard in SEG-Y rev 0, but standard in SEG-Y rev. 1.)
\end{description}

\item[\texttt{FON}] The first output \gls{shot}/\gls{rp} number.
\item[\texttt{RENUM}] \Gls{preset} = 0     \textit{e.g.}   \texttt{FON 101}
\begin{description}
    \item[<0]       In conjunction with \texttt{POSAFT} means the last \gls{shot}/\gls{rp} + 1.
    \item[=0]       Means that the output number will be the same as the input.
    \item[>0]       The first \gls{shot}/\gls{rp} will be \texttt{FON}.  (\texttt{RENUM} is the same as \texttt{FON} - added for compatibility with \texttt{DISKIN}),
\end{description}

\item[\texttt{RETRAC}] Renumber the trace numbers within each \gls{shot}/\gls{rp} so that the first trace of each \gls{shot}/\gls{rp} is \texttt{RETRAC}.  \Gls{preset} = none. \textit{e.g.}     \texttt{RETRAC} 1

\item[\texttt{POSAFT}] Position after \gls{shot}/\gls{rp} or after the last trace in the file.  \Gls{preset} = 0.    \textit{e.g.}  \texttt{POSAFT} 1234
\begin{description}
    \item[<0] Position after the last trace of the last \gls{shot}/\gls{rp} (append).
    \item[=0] No positioning is done.
    \item[>0] Position after the \gls{shot}/\gls{rp} specified (after the last trace of the \gls{shot}/\gls{rp}).
\end{description}  

\item[\texttt{SET}]   Start and end times, in seconds, of the data to be written to disk file \texttt{OPATH}.  Data outside of \texttt{SET} will not be included in the output file.  \texttt{DISKOX SET} may not be used to pad with zeroes; use \texttt{DISKIN SET} for zero padding.  If \texttt{SET(1)} is before the deep water delay, the output will be from the delay, not \texttt{SET(1)}.  \Gls{preset} = none  \textit{e.g.}  \texttt{SET 2 3}

\item[\texttt{SECS}]  The number of seconds of data to write to file \texttt{OPATH}.  Data after delay+secs will be omitted.  \Gls{preset} = none

\item[\texttt{DECIMF}]The decimation factor to use in writing the output disk file \texttt{OPATH}.  No anti-aliasing filter is applied.  The data passed to any process after \texttt{DISKOX} will NOT be decimated.  \Gls{preset} = 1  \textit{e.g.} \texttt{DECIMF 2}  (every other sample is discarded)

\item[\texttt{SPATH}] The pathname of the ``sort'' file.  When \texttt{SPATH} is given, process \texttt{DISKOX} outputs the seismic traces in the order specified by \texttt{SPATH} rather than the order actually received. This ``desorts'' a dataset sorted by process \texttt{SORT} and \texttt{DISKIN}.  \Gls{preset} = none  \textit{e.g.} \texttt{/usr/users/joe/data/sort.line1}

\item[\texttt{FNO}]   The first \gls{shot}/\gls{rp} number the parameter list applies to.  Default = the first \gls{shot}/\gls{rp} in file \texttt{OPATH}.   \textit{e.g.}   \texttt{FNO 101}

\item[\texttt{LNO}]   The last \glspl{shot}/\gls{rp} number the parameter list applies to.  Default = the last \gls{shot}/\gls{rp} in file \texttt{OPATH}.    e.g \texttt{LNO 101}

\item[\texttt{NOINC}] The \gls{shot}/\gls{rp} increment between \texttt{FNO} and \texttt{LNO}.  \Gls{preset} = 1.    \textit{e.g.} \texttt{NOINC 10}

\item[\texttt{FLINC}] The increment between groups of \texttt{FNO}-\texttt{LNO} \glspl{shot}/\glspl{rp} output.  Both \texttt{FNO} and \texttt{LNO} are incremented by \texttt{FLINC} after \texttt{LNO} has been output.  This feature is useful when groups of \glspl{rp} are to be saved.  \textit{e.g.}  If you want to save 10 consecutive \glspl{rp} out of every 100 \glspl{rp}, then use       \texttt{FNO} 1000 \texttt{LNO 1009 FLINC 100}.  \glspl{rp} 1000-1009, 1100-1109, 1200-1209, \ldots   will be written.

\item[\texttt{FTR}]   The first trace of each \gls{shot}/\gls{rp} to write to disk file \texttt{OPATH}.  Traces less than \texttt{FTR} will be omitted.  \Gls{preset} = 0     \textit{e.g.}  \texttt{FTR  11}

\item[\texttt{LTR}]   The last trace of each \gls{shot}/\gls{rp} to write to disk file \texttt{OPATH}.  Traces greater than \texttt{LTR} will be omitted.  \Gls{preset} = the last trace of each \gls{shot}/\gls{rp} \textit{e.g.}  \texttt{LTR 11}

\item[\texttt{TRINC}] The trace increment between \texttt{FTR} and \texttt{LTR}.  \Gls{preset} = 1     \textit{e.g.} \texttt{TRINC 2}

\item[\texttt{FRANGE}]  The first range (the absolute value of the shot-receiver distance) to be included in the output disk file.  Traces with ranges less than \texttt{FRANGE} will not be written to disk.  The use of \texttt{FRANGE} and \texttt{LRANGE} is useful when CDP \glspl{gather} are being written to disk.  \Gls{preset} = 0     \textit{e.g.} \texttt{FRANGE 2000}

\item[\texttt{LRANGE}]  The last range (the absolute value of the shot-receiver distance) to be included in the output disk file.  Traces with ranges greater than \texttt{LRANGE} will not be written to disk. The use of \texttt{FRANGE} and \texttt{LRANGE} is useful when CDP \glspl{gather} are being written to disk.  \Gls{preset} = 999999     \textit{e.g.} \texttt{LRANGE 2000}

\item[\texttt{FORMAT}]  The file format of the output file.  \Gls{preset} = none      \textit{e.g.} \texttt{FORMAT BINARY}
\begin{description}
\item[\texttt{SU}]  the Colorado School of Mines Center for Wave Phenomena's Seismic Unix file format is written.  The \texttt{SU} format does not contain the SEG-Y \gls{ebcdic} or binary headers, but does have the SEG-Y trace headers.
\item[\texttt{BINARY}]  no SEG-Y headers at all are written.  Remember that the first and last time samples are written (\textit{e.g.} 1 second of 1 mil data has 1001 samples).  The data are written in word type \texttt{OFMT} and in big endian (SEG-Y) byte order.
\end{description}

\item[\texttt{ONTRCS}]  The number of traces per output \gls{shot}/\gls{rp}.  If \texttt{FON} is not given, \texttt{FON} 1 will be used.  If \texttt{RETRAC} is not given, \texttt{RETRAC} 1 will be used.  \Gls{preset} = 0 (not given)  \textit{e.g.}  \texttt{ONTRCS 1}

\item[\texttt{REWIND}]  A \texttt{YES}/\texttt{NO} switch indicating that the disk file is rewound for EVERY SHOT written.  This is an attempt to create a ``circular'' \gls{shot} file useful during realtime processing.  \Gls{preset} = \texttt{NO},      \textit{e.g.}  \texttt{REWIND YES    \# rewind the file}
\begin{description}
\item[\texttt{YES}]   Create a new file on every trace 1.
\item[\texttt{NO}]  No rewind - a normal file.
\end{description}

\item[\texttt{TRACE0}]  A \texttt{YES}/\texttt{NO} switch indicating whether the SEG-D external header should be written as a trace numbered 0 and SEG-Y trace id (short word 15) 28.  \Gls{preset} = \texttt{NO}    \textit{e.g.}    \texttt{TRACE0 YES   \# a trace 0 will be written}
\begin{description}
\item[\texttt{YES}]  trace 0 will be written.
\item[\texttt{NO}]  trace 0 will not be written.
\end{description}

\item[\texttt{EXTHDR}]  A \texttt{YES}/\texttt{NO} switch indicating whether to write the SEG-Y Rev. 1 Textual Extension Records, if present, or not.  \Gls{preset} = \texttt{YES} (to write the records).    \textit{e.g.}  \texttt{EXTHDR} NO
\begin{description}
\item[\texttt{YES}]  the records will be written.
\item[\texttt{NO}]  the records will not be output.
\end{description}

\item[\texttt{BIG}]  No longer needed.
         A \texttt{YES}/\texttt{NO} switch when set to \texttt{YES} indicates that the output
         file will be larger than 2GB.  SIOSEIS can not tell in
         advance how large the output file will be and some operating
         systems need to know if the file will be > 2GB so that
         64 bit pointers are used.  Sun and HP need this parameter.
         SGI does not need this parameter.
         \Gls{preset} = \texttt{YES}.    \textit{e.g.}  \texttt{BIG YES}
\end{description}

\section{DMO: Dip MoveOut}
\label{cmd_dmo}

Process \texttt{DMO}, Dip Move Out, applies a phase filter in Omega-K domain to
correct for move-out when dip is present.  Typically, \texttt{NMO} is used to
produce a zero-offset section which is then migrated to produce a final
image.  However, if dips exist, then the mapping of seismic data using
the \texttt{NMO} equation is dip dependent, and causes subsurface smear updip
away from the midpoint.  To mitigate this problem, dip moveout algorithms
have been developed to allow all dips to stacked simultaneously without
up-dip smear.  The algorithm used in process \texttt{DMO} is the EXACT LOG DIP
MOVEOUT formulation by \texttt{LINER} and assumes constant velocity.

The traces input to \texttt{DMO} must be sorted by offset distance (range), which
may be accomplished with process \texttt{SORT} with the \texttt{SORT} parameters \texttt{LKEY1 10 FLAG51 -1}, the ``end-of-sort'' flag.

This \texttt{DMO} algorithm has ``stretch'' problems which may be a alleviated in
the time domain using process \texttt{LOGST1} in the time domain before process
tx2fk and ``unstretching'' it in the time domain after \texttt{DMO} and \texttt{FK2TX LOGST2}.

Process \texttt{DMO} requires the data to be transformed into the FK (frequency-
wavenumber) domain using process \texttt{TX2FK}.  The data may be converted back to
the time domain after \texttt{DMO} using \texttt{FK2TX}.

A typical \texttt{DMO} processing sequence is:
\texttt{PROCS SORT DISKIN NMO LOGST1 TX2FK DMO FK2TX LOGST2 DISKOA END}

\subsection{Parameter Dictionary}

\begin{description}
\item[\texttt{DELTAX}] The distance between traces, also called the group spacing.  Required

\item[\texttt{WINDOW}] The type of window to apply before computing the \gls{fft}s.
         \Gls{preset} = \texttt{RECT}
\begin{description}
\item[\texttt{HANN}] Hanning window.
\item[\texttt{RECT}] Rectangular or box car window (no window).
\end{description}

\item[\texttt{OFFSET}] The source-receiver offset (range) which if invoked will
    override header value.  This is only useful when only one offset is input
    to \texttt{DMO} and the header value must be overridden.  \Gls{preset}
    - none
\end{description}

\section{FDFMOD: Finite-Difference Forward Modelling}
\label{cmd_fdfmod}

%     Finite-Difference Forward Modelling using the 45-degree algorithm

The finite-difference migration technique is an effective way to handle
many types of migration problems.  Often it is necessary to do the
reverse problem -- given a subsurface structure and velocity field,
calculate its response on a zero-offset section, or unmigrated section
(a zero-offset section is a good approximation to a stacked section
in regions of small dip).   This technique is very similar to migration
except that the receivers are upward continued to the surface $P(x,z=0,t)$,
whereas the migration process downward continues the receivers - via a
finite-difference approximation to the scalar wave equation - into the
subsurface and collecting terms at $P(x,z,t=0)$.   Input to \texttt{FDFMOD} can be
generated by process \texttt{SYN} and \texttt{FILTER}, and should represent the structure
you care to model (or unmigrate).  Therefore, point sources should become
diffractors, interfaces should increase both dip and length, \textit{etc.}\ldots

This process will most likely be used in conjunction with processes \texttt{SYN}
and \texttt{FILTER} which generate seismic traces for modelling.

\subsection{Limitations}

See process \texttt{FDMIGR} (Section~\ref{cmd_fdmigr})

\subsection{Some Important Parameters}

The parameter Rho is inserted into the expression for the discretization
of the time derivative. This serves to counteract any potential growing
waves from the expression for migration, as an explicit damping with time.
It can be thought of as a ``numerical viscosity''  A value of Rho less
than 1 reinforces stability.  However, any deviation of Rho away from 1,
by at most 1 percent,  results in some loss of signal as well as noise.

In the discretization of depth, the parameter Theta is introduced, with
the most natural value being .500.  If Theta = 0 is used, there is a
tendency to overshoot on variations, whereas   Theta = 1 will produce an
overdamping of change.

To discretize the horizontal distance component, an approximation to the
second derivative is found by an iterative method.  When the iteration is
truncated, the parameter Gamma is introduced, which is allowed to vary
between .08 and .17, based primarily on the look of migrated sections.
If Gamma is allowed to increase too much more, spurious noise results.

In the ideal case, Tau would equal the sample rate of the data, meaning
that the entire section would be migrated exactly one sample rate step
at each pass through the section. While this scheme reduces the errors,
it is impractical due to the huge run-time needed. In practice, Tau should
be chosen in the range of 20 to 200 ms. (.02 to .2 secs), with the smaller
Tau values producing greater accuracy.  It is possible to vary Tau
vertically (not recommended), and should be done in order to save run-time.
Generally, the value of Tau should decrease from shallow to deep data
times. This is because greater accuracy is needed in the migration of the
deeper events where the greatest movement is taking place.

More detailed explanation of the origin of these parameters, and some
results of allowing them to vary, may be found in the paper published
by H. Brysk (Geophysics: May 1983) \cite{Brysk1983}.

\subsection{Parameter Dictionary}

\begin{description}
\item[\texttt{DX}] Trace separation distance.  This is the distance between
         reflection points.  \texttt{DX} is a constant for the entire seismic line.
         REQUIRED.  range 1.0 to 500.0 \textit{e.g.}  \texttt{DX 25}

\item[\texttt{FNO}] The first \gls{shot}/\gls{rp} number the parameter list applies to.
         \Gls{preset} = the first \gls{shot}/\gls{rp} received.    \textit{e.g.}   \texttt{FNO 101}

\item[\texttt{LNO}] The last \gls{shot}/\gls{rp} number the parameter list applies to.
         \Gls{preset} = the last \gls{shot}/\gls{rp} received.     \textit{e.g.}   \texttt{LNO 101}

\item[\texttt{VTP}] The rms velocity to use in migration.  The rms velocity function
         is the same as the velocity function used to moveout the data.
         Given as velocity-time pairs.  Velocities not specified are
         calculated through interpolation and ``straight-lining'' from the
         ends.  Times must be given in seconds.
         \Gls{preset} = none    velocity range 350 to 32000

\item[\texttt{VDIX}] The interval velocities to use in migrating, given as interval
         velocity-time pairs.  Time must be in seconds.
         \Gls{preset} = none    range 350 to 32000

\item[\texttt{BPAD}] The number of zero amplitude traces to insert prior to the first
         trace.
         \Gls{preset} = 1   range 1 to 500   \textit{e.g.} \texttt{BPAD 10}

\item[\texttt{EPAD}] The number of zero amplitude traces to append after the last trace.
    \Gls{preset} = 1   range 1 to 500   \textit{e.g.} \texttt{EPAD 10}

\item[\texttt{OPAD}] A switch indicating that the pad traces (both \texttt{BPAD} and \texttt{EPAD})
         should be output in addition to the migrated input.
         \Gls{preset} = \texttt{NO}   range \texttt{YES}/\texttt{NO}    \textit{e.g.}   \texttt{OPAD YES}

\item[\texttt{NRHO}] A parameter used to control the Tau step interpolation.
         \Gls{preset} = 2.0   range 0. to 10000

\item[\texttt{FCRHO}]   - A parameter used to control the Tau step interpolation.
         \Gls{preset} = .99   range .0001 to 1.

\item[\texttt{RHO}]     - A ``hidden'' migration parameter discussed above.
         \Gls{preset} = .9990   range  0 to .9999

\item[\texttt{THETA}]   - A ``hidden'' migration parameter discussed above.
         \Gls{preset} = .501  range  0 to 1.0

\item[\texttt{GAMMA}]   - A ``hidden'' migration parameter discussed above.
         \Gls{preset} = .125   range  .08 to .17

\item[\texttt{TSTEPS}]  - A set of time-delta-\gls{not:tau} pairs governing the \gls{not:tau} step size
         (delta-\gls{not:tau}) in the time interval terminating with the time given.
         Up to 7 pairs of time and delta-\gls{not:tau} may be given.  The user must
         give the max time modelled in last pair. \textit{e.g.} 8.0 0.10 with 8.0
         secs being last sample modelled. It is HIGHLY suggested that the
         user use only one time-delta-\gls{not:tau} pair and vary the size of the
         step to conserve CPU time.  Units are in seconds.
         \Gls{preset} = REQUIRED

\item[\texttt{NX}]      - The total number of traces, including pads, to migrate.  The
         entire seismic line must be transformed from TX (time-space) to
         XT (space-time).  \texttt{FDDIFF} requires much extra disk I/O if the
         entire seismic line (nx*maxsam) is larger than the computer
         memory allocated for the transformation.  \texttt{NX} does not have to
         be a power of 2.
         \Gls{preset} = 4096 \textit{e.g.} \texttt{NX 500}

\item[\texttt{MAXSAM}]  - The maximum number of samples per trace, including the deep
         water delay, to migrate. A trace exceeding \texttt{MAXSAM} will be
         truncated.
         \Gls{preset} = the number of samples plus delay of the first trace.

\item[\texttt{PATH}]    - The pathname (filename) of a scratch file \texttt{FDDIFF} should use for
         the intermediate transposed data.  The purpose of this parameter
         is to allow the user to specify the exact disk partition to use
         in case the ``current'' partition does not have enough space.
         \Gls{preset} = a scratch file in the current directory
         \textit{e.g.}    \texttt{PATH /user/scratch/moreroom}
\end{description}

\subsection{Examples}

\subsubsection{Generate a constant velocity hyperbola}
(Script file \texttt{examples/c\_hypcvel})
\begin{verbatim}
SIOSEIS << eof
PROCS SYN FILTER DISKOA FDFMOD DISKOB END
SYN
   FNO  1 LNO 49  NTRCS 1 SECS 3.0 TVA 3.1 2500 1 END
   FNO 50 LNO 50
     TVA .3 2500 1 .7 2500 1  1.0 2500 1 1.3 2500 1 1.8 2500 1 END
   FNO 51 LNO 100  TVA 3.1 2500 1 END
END
FILTER
     PASS 10 20 END
END
DISKOA
     OPATH impulsecvel.segy FON 1 END
END
FDFMOD
     NX 102 BPAD 1 EPAD 1 OPAD NO MAXSAM 751 DX 25 MAXDIP .001
     PATH scratch VTP 2500 0.0 2500 3.0 TSTEPS 3.00 .1 END
END
DISKOB
     OPATH impulsecvel.fddiff FON 1 END
END
END
\end{verbatim}

\subsubsection{Generate a hyperbola with laterally varying velocity}
     See script file \texttt{examples/c\_hypvlat}

\subsubsection{Generate a hyperbola with RMS velocity}
     See script file \texttt{examples/c\_rms.fddiff}

\subsubsection{Generate a hyperbola with dip}
     See script file \texttt{examples/c\_mod.dip.refl}

\section{FDMIGR: Finite Difference Migration}
\label{cmd_fdmigr}

%                   Finite-difference Migration using the
%                            45-degree algorithm

The finite-difference migration technique is an effective way to handle
many types of migration problems.  It was developed and made popular by
J. F. Claerbout at Stanford University.  For most stack sections, finite-
difference migration gives results comparable to other schemes; however
there are assumptions and stability limitations which must be considered.
For certain conditions, frequency domain (process FKMIGR) migration is
more effective in resolving typical imaging and positioning problems.

\subsection{Limitations}


\subsubsection{Steep Dips}

It is possible to add more terms to the finite-difference equation to
obtain successively more accurate equations to deal with the steep dip
problem.  However, these schemes quickly become impractical to implement
due to their cost.  Further limitations on dip angle are imposed since
the finite-difference method itself introduces errors.  The equation
used in \texttt{FDMIGR} is known as the 45-degree equation, and is capable of
handling dips up to angles of 45 degrees with sufficient accuracy.

A certain confusion exists regarding the meaning the meaning of the dips
referred to in the 45 degree equation. This is not simply the dip of
continuous reflectors.  These are the dips included in all events of
interest as seen in the F-K domain.  A sharp fault, for example,
contains dips up to 90 degrees, and the 45 degree algorithm will only
properly migrate certain components, with increasing distortion at
higher dips.  The parameters in the algorithm are set to suppress those
dips which are poorly imaged.


\subsubsection{Velocity}

Within the finite-difference equation there is no term to describe
differences in velocity. Hence, a major assumption of the scheme is that
velocity is constant throughout the section.  In practice, it is
sufficient for the velocity to vary slowly enough that it looks roughly
constant within the effective ``aperture'' of the algorithm.  This
aperture can be thought of as a box whose time length equals one
Tau-step size and whose spatial length equals the effective width of a
point diffraction pattern.


\subsubsection{Boundary Effects}

Ideally, we would like to perform migration on all of space.  But in the
real situation, we can only migrate a finite section of the earth, so we
must consider the effects of the imposed boundaries.  The main
consideration is for the sides of the section, where we normally think
of the earth as simply ceasing to exist, and the events stopping.  This
view induces the mathematical equivalent of a vertical reflection
coefficient, and events which are migrated towards it will be partly
reflected back into the section.  In order to suppress, or at least
attenuate these undesirable events, a buffer zone, or pad, consisting of
a number of traces, is inserted at both sides of the section.  The traces
are set to zero before migration, and the velocity is the same as the
attached traces in the section.  Studying the padded traces after
migration can sometimes yield valuable information about events close to
the edge of the section, especially if other data in the area is
available.

\subsection{Comparison with FKMIGR}

\subsubsection{Run Time}
One of the most practical considerations when deciding which
migration scheme to use is the difference in cost.  Depending on the
values of certain parameters used, \texttt{FDMIGR} can run 3 to 4 times as long
as \texttt{FKMIGR}.  Clearly, if there is no advantage in data quality to be
obtained, \texttt{FDMIGR} should not be used.

\texttt{FDMIGR} must migrate from time zero, so it replaces the deep
water delay with sufficient zeroes.  The inserted zeroes are removed
after migration so that the output traces will have the same delay as
the input traces.  Deep water delays do not affect the \texttt{FDMIGR} run time.

\subsubsection{Steep Dips}
Use of \texttt{FDMIGR} will produce inaccuracies if events are dipping
by more than 45 degrees.  \texttt{FKMIGR}, the frequency domain approach,
migrates all dips with equal accuracy.

\subsubsection{Velocity}
In general, \texttt{FDMIGR} will perform better in the presence of
velocity variations, although both methods assume that velocity is
slowly varying.

\subsubsection{Stability}
While \texttt{FKMIGR} is very stable in almost all conditions, \texttt{FDMIGR}
uses parameters which, if mis-used, can cause the migration equation to
become unstable. It is also possible to set values for particular data
sets in order to control noise on the output section, but you should
have a good understanding of finite-difference migration first. In
general, the default values will produce stable  results.

\subsubsection{Noise Suppression}
All migration algorithms tend to suppress random noise and
enhance coherent events.  The result is that the output section will
look more ``mixed'' than the input.  The effect will be more prominent as
the accuracy of the algorithm increases.  For this reason, \texttt{FKMIGR} \texttt{FKMIGR}
will generally look more mixed than \texttt{FDMIGR}, which will, in turn, look
more mixed than a 15 degree algorithm.

\subsection{Some Important Parameters}

The parameter Rho is inserted into the expression for the discretization
of the time derivative. This serves to counteract any potential growing
waves from the expression for migration, as an explicit damping with
time. It can be thought of as a ``numerical  viscosity''.  A value of Rho
less than 1 reinforces stability.  However, any deviation of Rho away
from 1, by at most 1 percent, results in some loss of signal as well as
noise.

In the discretization of depth, the parameter Theta is introduced, with
the most natural value being .500.  If Theta = 0 is used, there is a
tendency to overshoot on variations, whereas Theta = 1 will produce an
overdamping of change.

To discretize the horizontal distance component, an approximation to the
second derivative is found by an iterative method.  When the iteration
is truncated, the parameter Gamma is introduced, which is allowed to
vary between .08 and .17, based primarily on the look of migrated
sections.  If Gamma is allowed to increase too much more, spurious noise
results.

In the ideal case, Tau would equal the sample rate of the data, meaning
that the entire section would be migrated exactly one sample rate step
at each pass through the section. While this scheme  reduces the errors,
it is impractical due to the huge run-time  needed. In practice, Tau
should be chosen in the range of 20 to 200 ms. (.02 to .2 secs), with
the smaller Tau values producing greater accuracy.  It is possible to
vary Tau vertically, and this should be done in order to save run-time.
Generally, the value of Tau should decrease from shallow to deep data
times. This is because greater accuracy is needed in the migration of
the deeper events where the greatest movement is taking place.

More detailed explanation of the origin of these parameters, and some
results of allowing them to vary, may be found in the paper published by
H. Brysk (Geophysics: May 1983) \cite{Brysk1983}.

\subsection{Parameter Dictionary}

\begin{description}
\item[\texttt{DX}] Trace separation distance.  This is the distance between
         reflection points.  \texttt{DX} is a constant for the entire seismic
         line.
         REQUIRED.  range 1.0 to 500.0 \textit{e.g.}  \texttt{DX} 25

\item[\texttt{FNO}] The first \gls{shot}/\gls{rp} number the parameter list applies to.
         \Gls{preset} = the first \gls{shot}/\gls{rp} received.    \textit{e.g.}   \texttt{FNO 101}

\item[\texttt{LNO}] The last \gls{shot}/\gls{rp} number the parameter list applies to.
         \Gls{preset} = the last \gls{shot}/\gls{rp} received.     \textit{e.g.}   \texttt{LNO 101}

\item[\texttt{VTP}] The RMS velocity to use in migration.  The rms velocity
         function is the same as the velocity function used to moveout
         the data.  Given as velocity-time pairs.  Velocities not
         specified are calculated through interpolation and "straight-
         lining" from the ends.  Times must be given in seconds.
         \Gls{preset} = none    velocity range 350 to 32000

\item[\texttt{BPAD}] The number of zero amplitude traces to insert prior to the
         first trace.
         \Gls{preset} = 1   range 1 to 500   \textit{e.g.} \texttt{BPAD 10}

\item[\texttt{EPAD}] The number of zero amplitude traces to append after the last
         trace.
         \Gls{preset} = 1   range 1 to 500   \textit{e.g.} \texttt{EPAD 10}

\item[\texttt{OPAD}] A switch indicating that the pad traces (both \texttt{BPAD} and \texttt{EPAD})
         should be output in addition to the migrated input.
         \Gls{preset} = \texttt{NO}   range \texttt{YES}/\texttt{NO}    \textit{e.g.}   \texttt{OPAD YES}

\item[\texttt{NRHO}] A parameter used to control the Tau step interpolation.
         \Gls{preset} = 2.0   range 0. to 10000

\item[\texttt{FCRHO}] A parameter used to control the Tau step interpolation.
         \Gls{preset} = .99   range .0001 to 1.

\item[\texttt{RHO}] A ``hidden'' migration parameter discussed above.
         \Gls{preset} = .9990   range  0 to .9999

\item[\texttt{THETA}] A ``hidden'' migration parameter discussed above.
         \Gls{preset} = .501  range  0 to 1.0

\item[\texttt{GAMMA}] A ``hidden'' migration parameter discussed above.
         \Gls{preset} = .125   range  .08 to .17

\item[\texttt{TSTEPS}] A set of time-delta-\gls{not:tau} pairs governing the \gls{not:tau} step size
         (delta-\gls{not:tau}) in the time interval terminating with the time
         given.  Up to seven (time, delta-\gls{not:tau}) pairs may be given.  The
         delta-\gls{not:tau} values will be interpolated between the specified
         times and will be ``straight-lined'' at the trace ends.  The
         units of time and delta-\gls{not:tau} are seconds.
         \Gls{preset} = REQUIRED   \textit{e.g.} \texttt{TSTEPS .1 .1 1.0 .2}

\item[\texttt{NX}] The total number of traces, including pads, to migrate.  The
         entire seismic line must be transformed from TX (time-space)
         to XT (space-time).  \texttt{FDMIGR} requires much extra disk I/O if the
         entire seismic line (nx*maxsam) is larger than the computer
         memory allocated for the transformation (the Cray does not have
         a virtual memory).  \texttt{NX} does not need to be a power of 2.
         \Gls{preset} = 4096  \textit{e.g.}   \texttt{NX 500}

\item[\texttt{MAXSAM}] The maximum number of samples per trace, including the deep
         water delay, to migrate.  A trace exceeding \texttt{MAXSAM} will be
         truncated.
         \Gls{preset} = the number of samples plus delay of the first trace.

\item[\texttt{PATH}] The pathname (filename) of a scratch file \texttt{FDMIGR} should use
         for the intermediate transposed data.  The purpose of this
         parameter is to allow the user to specify the exact disk
         partition to use in case the ``current'' partition does not have
         enough space.
         \Gls{preset} = a scratch file in the current directory
               \textit{e.g.} \texttt{PATH /user/scratch/moreroom}
\end{description}

\section{FILTER: Time Varying Zero Phase Bandpass Filter, Butterworth Filter}
\label{cmd_filter}

Process \texttt{FILTER} applies a frequency filter to every trace.  Filters
available are: (see parameter \texttt{FTYPE})
\begin{itemize}
\item Time domain (convolutional) zero phase time varying bandpass
\item Frequency domain zero phase bandpass
\item Frequency domain minimum phase
\item Low pass Butterworth
\item Frequency domain notch
\end{itemize}

Time varying filtering is performed by applying different time domain
filters to different parts of the trace.  The different parts of the
trace are called windows.  The portion of the trace between windows
are merged by ramping (linear).  The merge zone thus contains data that
has been filtered by different filters and then added together after
being ramped.  The weights of the windows can be different, however then
the merge zone will contain more of one type of filter than the other.
\textit{e.g.}
\begin{verbatim}
              F1            F2            F3
                        ..........     ..........
                       .          .   .
          ..........  .            . .
                    ..              .
                   .  .            . .
                  .    .          .   .
\end{verbatim}

Up to 5 windows may be given, each with a different window level, and
may be spatially varied by \gls{shot} or \gls{rp} or by hanging the windows on the
water bottom.  Time varying filters are available only on time domain
filters.

All parameters that remain constant for a set of \glspl{shot} (\glspl{rp}) may be
described in a parameter set \texttt{FNO} to \texttt{LNO}.  Windows between two parameter
sets are calculated by linearly interpolating between \texttt{LNO} of one set
and \texttt{FNO} of the next set.  Only the time windows (sets) are spatially
varied.  The filter (pass) remains constant even though the application
window (sets) vary.  Parameter \texttt{INTERP} may be used to turn spatial
interpolation off.

Each parameter list must be terminated with the word end.  The entire
set of filter parameters must be terminated by the word \texttt{END}.

\subsection{Parameter Dictionary}

\begin{description}
\item[\texttt{FNO}] The first \gls{shot} (or \gls{rp}) to apply the filter(s) to.  Shot (\gls{rp})
         numbers must increase monotonically.
         \Gls{preset}=1

\item[\texttt{LNO}] The last \gls{shot} (\gls{rp}) number to apply the filter(s) to.  \texttt{LNO} must
         be larger than \texttt{FNO} in each list and must increase list to list.
         Default=\texttt{FNO}

\item[\texttt{SETS}] Start-end time pairs defining the windows of a time varying filter.
         Times are in seconds and may be negative when hanging the
         windows from the water bottom.   A maximum of 5 windows may be
         given.  Only available on time domain filters.
         \Gls{preset}= delay to last time. \textit{e.g.} \texttt{SETS 0 3.0 3.3 6.0}

\item[\texttt{PASS}] For time domain (convolution filtering or \texttt{FTYPE 99}): A list of
         passbands. A passband is a set of two frequencies between which
         the data will be passed.  Frequencies outside the passband will
         be cut.  Pass is an approximate number but very sharp sloped
         filters can be obtained by increasing the filter length.  Up to
         5 passbands may be given.
         Required. \textit{e.g.} \texttt{PASS 10 70}
        \begin{itemize}
            \item For frequency domain bandpass filters (\texttt{ftype 0} and \texttt{20})
         The two corner frequencies of the passband.  The slopes of the
         filter are given via \texttt{DBDROP}.
     \item For low pass or high pass filters (\texttt{ftype 1} and \texttt{2}):
         The cutoff frequency.
     \item For notch filters (\texttt{ftype 3} and \texttt{23}):
         The two corner frequencies between which the frequencies
         will be cut out (filtered out).
        \end{itemize}

\item[\texttt{FILLEN}] The length of each time domain filter.  The number of points
         to use in the convolution.  Up to 5 filter lengths may be
         given.  An odd number of points should be used since the
         filters are symmetrical.  Short filters (25 points) may run
         fast, but the filter shape becomes poor.  Short filters do not
         filter low frequencies well.
         \Gls{preset} = 25 25 25 25 25  \textit{e.g.}  \texttt{FILLEN 39}

\item[\texttt{LEVS}] The amplitude level of each window described by sets. Each
         window may have a different level.  Each level must be >0.
         Up to 5 levels may be given.
         Preset= 1. 1. 1. 1.

\item[\texttt{ADDWB}] When given a value of yes, the windows given via sets will be
         added to the water bottom time of the trace.  (Water bottom
         times may be entered via process \texttt{WBT}).
         \Gls{preset}=no

\item[\texttt{FTYPE}] Filter type.  Time varying filter is permitted with time domain
         filtering only.  Add parameter \texttt{MINPHA} for minimum phase filters.
         \Gls{preset} = 99    \textit{e.g.}   \texttt{FTYPE 0}
\begin{description}
       \item[0] John Shay's frequency domain zero phase bandpass.
       \item[1] John Shay's frequency domain low pass. The corner frequency is the first value of parameter \texttt{PASS}.
       \item[2] John Shay's frequency domain high pass. The corner frequency is the first value of parameter \texttt{PASS}.
       \item[3] John Shay's frequency domain notch.
       \item[10] Low pass 3 pole Butterworth filter.
       \item[20] Warren Wood's frequency domain zero phase bandpass.
       \item[23] Warren Wood's frequency domain notch.
       \item[99] Time domain (convolutional) zero phase time varying.  These are VERY fast filters.
\end{description}

\item[\texttt{DBDROP}] Decibel drop per octave.  The slope of the filter response in
         db/octave.  Valid with \texttt{FTYPE} 0, 1, and 3
         \Gls{preset} = 48.   \textit{e.g.} \texttt{DBDROP 6}

\item[\texttt{WINDOW}] The type of window to apply before computing the \gls{fft}.
    \Gls{preset}=\texttt{HANN}  \textit{e.g.} \texttt{WINDOW RECT}
\begin{description}
       \item[\texttt{HAMM}] Hamming
       \item[\texttt{HANN}] Hanning
       \item[\texttt{BART}] Bartlett (triangular)
       \item[\texttt{RECT}] rectangular (box car - no window)
       \item[\texttt{BLAC}] Blackman
       \item[\texttt{EBLA}] exact Blackman
       \item[\texttt{BLHA}] Blackman-Harris
\end{description}

\item[\texttt{WINLEN}] The window length, in seconds.  A window length of zero causes
         the entire time domain gate to be windowed.  A non zero length
         indicates that winlen data will be modified at both ends of each
         data gate.
         \Gls{preset} = 0.  \textit{e.g.}  \texttt{WINLEN .2}

\item[\texttt{MINPHA}] A yes/no switch indicating that the filter should be a minimum
         phase filter rather than zero phase.  Valid for ftype 0, 1, 2.
         The switch is set by any no-zero value.
         \Gls{preset} = no     \textit{e.g.} \texttt{MINPHA YES}

\item[\texttt{INTERP}] A YES/NO switch indicating whether spatial interpolation should
         be done or not.  Normally all \glspl{shot}/\glspl{rp} and traces are filtered,
         but this can be overridden with \texttt{INTERP NO}.  Traces not specified
         in an fno/lno list will not be filtered with \texttt{INTERP OFF}.
         \Gls{preset} = \texttt{YES}    \textit{e.g.}   \texttt{INTERP NO}

\item[\texttt{END}] Terminates each parameter list.
\end{description}

\section{FKFILT: Frequency-Wavenumber Filter}
\label{cmd_fkfilt}

FKFILT calculates and applies a filter in the frequency-wavenumber (FK)
domain.

Currently the only type of filter that is implemented is a fan filter or
pie slice filter.  This type of filter is useful for removing or
retaining signals travelling across the seismic line at certain phase
velocities.  The filter is defined in terms of a series of lines from
the origin which delimit pass and cut slices of the filter.  In between
a cut and pass region the filter response is tapered according to a
chosen window function.

To define a fan filter, the filter lines may be given either in terms of
velocity or in terms of dip. The cut and pass lines may be input in any
order and will be sorted and checked for consistency.   For velocity the
filter region runs
                    v :  0- -> -inf / +inf -> 0+
While for dips it runs
                  dip :  -inf -> 0 -> +inf.
(Remember horizontal events have 0 dip and infinite velocity.  Steeply
dipping events have small velocity.)

It is a mistake to define a filter that, when sorted, consists of 3 or
more lines of the same type within the body of the filter or 2 or more
lines of the same type at either end.

Refer to two articles in the January 1983 ``First Break'' for more details
on both the FK domain and FK filtering.  The SIOSEIS document fk.forum
contains some discussions about the fk domain and has FKFILT examples.

Prestack FKFILT may be done by using the process TX2FK parameter PRESTK.


\subsection{Parameter Dictionary}

\begin{description}
\item[\texttt{DIPCUT}] The dip of the lines defining the FK region(s) to be removed.
         Dip is measured in ms per trace.

\item[\texttt{DIPPAS}] The dip of the lines defining the FK region(s) to be retained.
         Dip is measured in ms per trace.
         \texttt{e.g.}  \texttt{DIPPAS -1 1 DIPCUT -2 2} retains events with small dip,
              removing dips greater than 2mils trace to trace.

\item[\texttt{VELCUT}] The velocity of the lines defining the FK region(s) to be
         removed. The units for velocity must be consistent with those
         used for \texttt{DELTAX}.

\item[\texttt{VELPAS}] The velocity of the lines defining the FK region(s) to be retained. \textit{e.g.} 
\begin{verbatim}
VELCUT -100 -900 900 100
VELPAS -250 -500 500 250
\end{verbatim}
will retain only arrivals with apparent velocities between +/- 900 \& 500.

\item[\texttt{DELTAX}] If the filter is defined using \texttt{VELPAS}/\texttt{VELCUT }the \texttt{DELTAX }must be given.

\item[\texttt{WINDOW}] The type of window to use when tapering.  \Gls{preset} = \texttt{HANN}, \textit{e.g.} \texttt{WINDOW RECT}
\begin{description}
       \item[\texttt{HAMM}] Hamming
       \item[\texttt{HANN}] Hanning
       \item[\texttt{BART}] Bartlett (triangular)
       \item[\texttt{RECT}] Rectangular (box car - no window).
       \item[\texttt{BLAC}] Blackman
       \item[\texttt{EBLA}] Exact Blackman
       \item[\texttt{BLHA}] Blackman-Harris
\end{description}

\item[\texttt{WINOPT}] The windowing may be done as a function of angle, wavenumber
         or frequency.  However a window that spans infinite $v$ cannot be
         tapered as a function of frequency.
         \Gls{preset} = \texttt{BYA}
\begin{description}
\item[\texttt{BYA}] As a function of angle
\item[\texttt{BYK}] As a function of wavenumber
\item[\texttt{BYW}] As a function of frequency
\end{description}

\item[\texttt{END}] Terminates each parameter list.
\end{description}

\section{FKMIGR: Frequency-Wavenumber Migration}
\label{cmd_fkmigr}

Process \texttt{FKMIGR} performs F-K migration on data that is in the frequency-
wavenumber domain.  The data must have been through process \texttt{TX2FK} prior
to \texttt{FKMIGR}.  The output from \texttt{FKMIGR} is also in the F-K domain, thus, the
data may be transformed to the time-space domain via process \texttt{FK2TX} after
process \texttt{FKMIGR}.

F-K migration assumes a constant velocity for the entire section. This
type of migration also assumes that the data is ``zero-offset'' data.
Single channel data with fairly small shot-receiver distance are zero-
offset.  Moved out data are zero-offset.  The zero-offset diffraction
hyperbola that are collapsed by F-K migration  have the formula
\texttt{tx=2*sqrt(t0**2/4+x**2/v**2)}.

At least 60 traces should added to the beginning and the end of the
section to be migrated in process \texttt{TX2FK} (parameter \texttt{NXPAD}).  This padding
should be sufficient to prevent ``edge'' or boundary effects.

\texttt{FKMIGR} uses Stolt's algorithm to perform migration in the F-K domain and
may be found in ``Imaging the Earth's Interior'' by Jon Claerbout \cite{Claerbout1985}.

Prestack \texttt{FKMIGR} may be done by using the process \texttt{TX2FK} parameter \texttt{PRESTK}.

The deep water delay is honored by making t0 the delay time.  All traces
to be migrated must have the same delay however.

\texttt{FKMIGR} can not handle more than 32768 frequencies (which corresponds to
16834 sample in the time domain).

\subsection{Parameter Dictionary}

\begin{description}
\item[\texttt{VEL}]    The constant velocity to use to migrate the data.   
    Required       \textit{e.g.} \texttt{VEL 1500.}                       
                                                                          
\item[\texttt{NFINT}]  The number of adjacent frequencies to use in interpolation.
    \Gls{preset} =10     \textit{e.g.} \texttt{NFINT 2}                   
                                                                          
\item[\texttt{DELTAX}] The distance between traces.                       
    \Gls{preset} 1.      \textit{e.g.} \texttt{DELTAX 6.25}               
                                                                          
\item[\texttt{DELTAT}] The time sample interval, in seconds.              
    \Gls{preset} = trace header.   \textit{e.g.}  \texttt{DELTAT .001}    
                                                                          
\item[\texttt{END}] Terminates each parameter list.
\end{description}

\section{FKSHIFT: Phase Shift in the Frequency-Wavenumber Domain (Depth Migration)}
\label{cmd_fkshift}

Process \texttt{FKSHIFT} performs an extrapolation via a phase shift in the F-K
domain.  \texttt{FKSHIFT} is depth migration of a horizontally layered media
whose velocity is always increasing.

The data must have been transformed into the F-K domain prior to process
\texttt{FKSHIFT} and it is left in the F-K domain.  Use process \texttt{TX2FK} prior to
\texttt{FKSHIFT} and \texttt{FK2TX} after \texttt{FKSHIFT}.

\texttt{FKSHIFT} can be used to forward extrapolate \gls{shot} \glspl{gather} by specifying
the true velocity and extrapolation height (thickness).

\texttt{FKSHIFT} can be used to backwards extrapolate stacked data to the
seafloor by halving the specified velocity and specifying the output
time delay.

Only one parameter list may be given.

\subsection{Parameter Dictionary}

\begin{description}
\item[\texttt{VEL}] The constant velocity.
          \Gls{preset} = 0     \textit{e.g.} \texttt{VEL 1500}

\item[\texttt{DELTAX}] The distance between traces.
          \Gls{preset} 1. \textit{e.g.} \texttt{DELTAX 100.}

\item[\texttt{ZEXTRAP}] The extrapolation height.
          \Gls{preset} = 0.

\item[\texttt{ODELAY}] The time of the first output sample, in seconds.
          \Gls{preset} = 0.    \textit{e.g.}  \texttt{ODELAY 1.}

\item[\texttt{DELTAT}] The sample interval of the data in the time domain, in seconds.
          \Gls{preset} = SEG-Y header.   \textit{e.g.}  \texttt{DELTAT .004}

\item[\texttt{END}] Terminates each parameter list.
\end{description}

\section{FK2TX: Frequency-Wavenumber to Time Distance Domain Transformation}
\label{cmd_fk2tx}

Process \texttt{FK2TX} transforms from the FK (frequency-wavenumber) domain into
the TX (time-space) domain.  The data MUST have been transformed into
the FK domain using process \texttt{TX2FK}.

Most \texttt{FK2TX} jobs will not require any parameters since the basic FK
information is stored in the SEG-Y headers.  A null set of parameters
may be given by:
\begin{verbatim}
FK2TX
      END
END
\end{verbatim}

\subsection{Parameter Dictionary}

\begin{description}
\item[\texttt{PATH1}] This is where input FK traces are accumulated prior to back back
           transformation. It is deleted after back transformation, and before
           the output tx data are written to disk.  Thus it is usually safe to
           put this file in the same directory as the final output.  If
           procedure \texttt{TX2FK} is present in the procs list then this scratch file
           is the same as the second scratch file of \texttt{TX2FK}. Care should be taken
           not to specify this file twice. Only the first filename will be used
           by SIOSEIS.
           Default: Implementation dependent.

\item[\texttt{PATH2}] The name of the second scratch file to be used by SIOSEIS.
           Default: Implementation dependent.

\item[\texttt{IHDRPATH}] Filename containing a set of original TX trace headers that were
           written by process \texttt{TX2FK}. These headers will be added to the TX
           traces after transformation. On the assumption that a 2-D process was
           performed in F-K all traces will be marked live and all mute entries
           will be zeroed in the header.  This parameter is useful when
           SEG-Y header information, such as the trace range, need to be
           retained.  (The FK domain contains half the number of traces
           of the TX domain, thus the SEG-Y traces headers are omitted).

\item[\texttt{OPAD}] \texttt{YES}. All padding both in time and range added to data prior to 2-D
           \gls{fft} will be output.
           \Gls{preset} \texttt{NO}.

\item[\texttt{END}] Terminates each parameter list
\end{description}

\section{FLATEN: Flatten the Seismic Section to a Reference Time}
\label{cmd_flaten}

Process \texttt{FLATEN} flattens the seismic line to user given time.  Each trace
is shifted from the water bottom (depth or time) to the user given output
time.  Any SEG-Y trace header word may be used as the water bottom time.
Water bottom depths may be converted to travel times by giving a velocity.

The original concept of flatten started with the SIO Sea Beam center beam
depth being used as the depth.  That depth is the depth directly under
the center of the ship (SEG-Y header word ihdr(16)), not at all what
seismic really sees.  Next, the Sea Beam closest beam depth was put into
SEG-Y header word \texttt{IHDR(107)}.  Finally, process \texttt{WBT} was modified to not
only use the closest Sea Beam depth, but to look forward and aft for the
shallowest depth.

The SIO single channel system started recording the SeaBeam depths in
spring 1987 (Crossgrain 1).

\subsection{Parameter Dictionary}

\begin{description}
\item[\texttt{OTIME}] The time, in seconds, of the of the water bottom after process \texttt{FLATEN}.
         Required. \textit{e.g.} \texttt{OTIME} 5.6

\item[\texttt{VEL}] The velocity of the water column used to convert the water
         depth to the water bottom time.  A zero velocity indicates that
         the header word is a time.  time = header / vel.  SIO Sea Beam
         uses 1500 m/s
         \Gls{preset} = 0     \textit{e.g.} \texttt{VEL 1500}

\item[\texttt{HDR}] The index of the water bottom depth/time within the REAL SEG-Y
         header.  Process \texttt{WBT} puts the water bottom time in \texttt{HDR}(50).
         \Gls{preset} = 50

\item[\texttt{IHDR}] The index of the water bottom within the 16 bit SEG-Y trace
         header. Use only if the water bottom depth/time is not in
         word 16.
         \Gls{preset} = 0     \textit{e.g.}   \texttt{IHDR 66}

\item[\texttt{LHDR}] The index of the water bottom within the 32 bit SEG-Y trace
         header. Use only if the water bottom depth/time is not in
         word 16.
         \Gls{preset} = 0     \textit{e.g.}   \texttt{LHDR} 63

\item[\texttt{NAVE}] The number of trace depths to average across.  The depth for a
         given trace will be the average the current trace and the
         previous \texttt{NAVE-1} traces.
         \Gls{preset} = 1    \textit{e.g.}   \texttt{NAVE} 5

\item[\texttt{FNO}] The first \gls{shot} (or \gls{rp}) to \texttt{FLATEN}.  Shot (\gls{rp}) numbers must
         increase  monotonically.

\item[\texttt{LNO}] The last \gls{shot} (\gls{rp}) to \texttt{FLATTEN}.  \texttt{LNO} must be larger than \texttt{FNO} in
         each successive parameter list.

\item[\texttt{END}] Terminates each parameter list.
\end{description}

\section{F2T: Frequency to Time Domain Transformation}
\label{cmd_f2t}

Process \texttt{F2T} transforms frequency domain data to the time domain.
Process \texttt{T2F} must have been used to create the frequency domain.  The
frequency domain data may be in rectangular or polar coordinates
(parameter \texttt{COORDS} in process \texttt{T2F}).

A quadrature trace (or 90 degree phase shifted trace) may be obtained
through a Hilbert transform and parameter \texttt{TYPE HILBERT}.

An analytic trace may be formed by using parameter \texttt{TYPE ANALYTIC}.  In
this case, the trace samples are an interleaving of the input trace
and the Hilbert transformed trace, so that there are twice as many
output samples as input.  The instantaneous amplitude may be formed
by using process \texttt{GAINS TYPE 7} (modulus of a complex trace).

\subsection{Parameter Dictionary}

\begin{description}
\item[\texttt{TYPE}] The type of data output by process \texttt{F2T}.  Default = \texttt{TIME}.
\begin{description}
\item[\texttt{TIME}]  The real time domain.
\item[\texttt{COMPLEX}]  The complex time domain.
\item[\texttt{HILBERT}]  The real time domain with Hilbert transform (phase shifted by 90 degrees).
\item[\texttt{ANALYTIC}]  The analytic time domain trace. (c(t) = a(t)+ib(t), where c(t) is a complex trace of a(t) the input trace and b(t) is the phase shifted trace).
\end{description}
\end{description}

\section{GAINS: Apply Various Time Dependent Gains}
\label{cmd_gains}

Process \texttt{GAINS} applies a gain function.  Chapter 4 of Claerbout's ``Imaging the
Earth's Interior'' \cite{Claerbout1985} mentions several of the gain functions implemented.

At least one parameter list must be given, even if no parameters are
specified, in order that the parameter presets be set.  \textit{e.g.}
\begin{verbatim}
GAINS
     END
END
\end{verbatim}

\subsection{Parameter Dictionary}

\begin{description}
\item[\texttt{TYPE}] The type of gain to apply.  \Gls{preset} = none     \textit{e.g.} \texttt{TYPE 4 ALPHA 2}
\begin{description}
\item[1] $a_{i} = a_{i} (1000 t)^{\alpha}$ ( \gls{usgs} gain ) where $a_{i}$
         is the trace and $t$ is the time of the trace sample in seconds.
         Restrictions:  All traces must have the same start time.
         \Gls{preset} = 1
     \item[2] $a_{i} = a_{i} \times (\texttt{ABS(range)/SIGN(rscale,range))}^{\alpha}$ when
         \texttt{ABS(range) .GE. rscale};  where \texttt{range} is the range in the
             SEG-Y header, \texttt{rscale} and $\alpha$ are given by the user, \texttt{SIGN}
             is the Fortran \texttt{SIGN} function which means that \texttt{ABS(rscale)}
             is used when range is positive and \texttt{-ABS(rscale)} is used
             when range is negative.
\item[3] $a_{i} = a_{i} t ^{\alpha}$
\item[4] $a_{i} = a_{i}^{\alpha}$
\item[5] $a_{i} = a_{i} e^{\alpha t}$
\item[6] $a_{i} = \texttt{SIGN}(a_{i}) \times \texttt{ABS}(a_{i}) ^{\alpha}$
\item[7] $a_{i} = \sqrt{a_{2i-1}^{2} + a_{2i}^{2}}$ (modulus of complex trace)
             The only SEG-Y header value modified is the number of samples
             (since there are half as many samples after doing the modulus).
             The SEG-Y header value for the number of samples is divided
             by 2, as is the sample inetrval.
             The deep water delay is NOT modified.
         \item[8] $a_{i} = a_{i} \times \texttt{ABS(range/rscale)} ^{\alpha}$ when
             \texttt{ABS(range) .GE. rscale};  where \texttt{range} is the range in the
             SEG-Y header, \texttt{rscale} and $\alpha$ are given by the user.
\item[9] Time-Gain-Pairs.  Automatically set to type 9 when
             parameter TGP (time-gain-pairs) is given.
         \item[10]$20\log{t \times v}$.  Spherical spreading in water decreases amplitudes
             by $20 \log{R}$, where $R = t \times v$ and $t$ is the two-way travel time.
             Parameter \texttt{ETIME} is honored.
\end{description}
\end{description}

\subsection{Additional Parameters}

\begin{description}
\item[\texttt{ETIME}] The end time of the gain function types 1 and 10.  Data after the end
         time will receive the gain of the end time.  Types 1 and 10 ONLY.
         \Gls{preset} = the last time of the first trace.   \textit{e.g.}  \texttt{ETIME 4}.

\item[\texttt{ALPHA}] The exponent used in \texttt{TYPE}s 1 - 6 gain.
         \Gls{preset} = 1.   \textit{e.g.}  \texttt{TYPE 3 ALPHA 1.5}

\item[\texttt{RSCALE}] The range scalar used in \texttt{TYPE} 2 gain.
         \Gls{preset} = 1.

\item[\texttt{SUBWB}] Subtract water bottom time switch.  Type 3, 5 and 9 ONLY.
       = \texttt{YES}, The water bottom time is subtracted from the data time in
         the gain types that use time as a variable.  \textit{e.g.}
\begin{equation}
a_{i} = a_{i} t^{\alpha}
\end{equation}
               becomes
\begin{equation}
a_{i} = a_{i} \left(t - \texttt{WBT}\right)^{\alpha}
\end{equation}
        for $t \geq \texttt{WBT}$
         \Gls{preset} = \texttt{NO}

\item[\texttt{TGP}] Time-Gain-Pairs.  A list of gains or multipliers to apply
         to each data trace.  Time/gain not specified in \texttt{TGP} are
         obtained through extrapolation and interpolation of \texttt{TGP}.
         The gain for data times between \texttt{TGP} times will be linearly
         interpolated using the adjacent time and gain pairs.
         Times prior to the first time of \texttt{TGP} will use the first gain.
         Times after the last time of \texttt{TGP} will use the last gain.
         \Gls{preset} = none     \textit{e.g.} \texttt{TGP 0 1 1 10 2 100}

\item[\texttt{ADDWB}] When given a value of \texttt{YES}, the times given via \texttt{TGP}
         will be added to the water bottom time of the trace.
         (Water bottom times may be entered via process \texttt{WBT}).
         Valid with \texttt{TYPE 9} or \texttt{TGP} gains only.
         \Gls{preset}=\texttt{NO} \textit{e.g.}   \texttt{ADDWB YES}

\item[\texttt{TMULT}] Time multiplier used in gains type 3 and 5.  \Gls{preset} = 1.
\item[\texttt{TADD}] Time additive used in gains type 3 and 5.  \Gls{preset} = 0.
         When using $a_{i} = a_{i} t^{\alpha}$, first $t$ is:
         $t$ = \texttt{DELAY}; if \texttt{SUBWB == YES} then $t$ = \texttt{delay - water\_bottom};
         $t = t \times \texttt{TMULT} + \texttt{TADD}$
         (note that arithmetic is done left to right)

\item[\texttt{WINLEN}] The window length, in seconds, of an amplitude running average.
    The averaging is independent of gains 1-9 and is done \textbf{after}
         the other gains.  \textit{e.g.} \texttt{ALPHA 2 TYPE 4 WINLEN .02} is also
         called the trace ENVELOPE.  Every amplitude is squared and
         then a running average over 20 mils. is done.  The window
         average is placed at the center of the window.
         \Gls{preset} = none.

\item[\texttt{V}] The velocity in converting time to depth for spherical spreading
         in \texttt{TYPE 10}.  $TL = 20 \log{R}$, where $R = t \times v$.
         \Gls{preset} 1500.

\item[\texttt{FNO}] The first \gls{shot}/\gls{rp} number the parameter list applies to.  Data
         (\glspl{shot}/\glspl{rp}) before \texttt{FNO} WILL NOT HAVE GAINS APPLIED.
         \Gls{preset} = the first \gls{shot}/\gls{rp} received.    \textit{e.g.}   \texttt{FNO 101}

\item[\texttt{LNO}] The last \gls{shot}/\gls{rp} number the parameter list applies to.  Data
         (\glspl{shot}/\glspl{rp}) AFTER \texttt{LNO} WILL NOT HAVE GAINS APPLIED.
         \Gls{preset} = the last \gls{shot}/\gls{rp} received.     \textit{e.g.}   \texttt{LNO 101}
\end{description}

\section{GATHER: Trace Collection According to Geometry}
\label{cmd_gather}

A \gls{gather} is a collection or rearrangement of traces according to some
criteria.  Process \texttt{GATHER} collects or sorts the input traces according
to the reflection point (\gls{rp}) number calculated by process \texttt{GEOM}.  The
\texttt{GEOM} parameters may be manipulated by the user to gather the input
traces according to any criteria by fudging the \texttt{GEOM} parameters.
A constant offset \gls{gather} of a uniform marine line may be made by
omitting traces via process input.

Process \texttt{GATHER} sorts each \gls{gather} by the absolute value of the shot-
receiver distance (SEG-Y bytes 37-40), so that the shortest range trace
is first within the \gls{gather}.  Each \gls{gather} is is terminated by setting a
special flag in the trace header.  A \gls{gather} record is the collection of
all these traces.

See process \texttt{GEOM} for the method of calculating \gls{rp} numbers.

\texttt{GATHER} creates a temporary disk file to store the partial \glspl{gather} while
the data are being read.  \texttt{GATHER} assumes that the geometry of the data
does not skip around very much. \textit{i.e.} the geometry doesn't go backwards
nor does it skip more than a cable length forward.  The temporary disk
file can hold \texttt{MAXRPS} \glspl{rp} (preset to 20 plus the number of traces per \gls{shot}
from the SEG-Y binary header), with each \gls{rp} able to hold a maximum of
\texttt{MAXTRS} traces (also preset to the number of traces per \gls{shot} in the SEG-Y
binary header), with each trace having a maximum of \texttt{NWRDS} samples, with each
sample being 4 bytes long.  The temporary disk file size will be:
   \texttt{MAXTRS * MAXRPS * (NWRDS+60) * 4}
The temporary file must be less than 16GB.

A null set of \texttt{GATHER} parameters must be given even if all the parameters
are presets.    \textit{e.g.}  \texttt{GATHER END END}

An input process such as \texttt{INPUT}, \texttt{DISKIN}, or \texttt{SEGDIN}, must precede the
\texttt{GATHER} parameters if \texttt{GATHER} parameter presets are used since the
presets use the number of input traces per \gls{shot}.

\subsection{Parameter Dictionary}

\begin{description}
\item[\texttt{MAXRPS}] The maximum number of bins (or \gls{rp}'s) that are needed on the
         disc at any one time.  In marine work the number of traces per
         \gls{shot} plus a few (+10) will usually suffice since no two
         ungathered traces with the same \gls{rp} number are more than a cable
         length away.
         \Gls{preset}=the number of traces from the segy binary header plus 20

\item[\texttt{MAXTRS}] The maximum number of traces any one \gls{gather} can have.   In \gls{rp}
         \glspl{gather} this is the maximum cdp allowed.  An example of
         estimating \texttt{MAXTRS} is:  A 468 channel streamer with group spacing
         of 12.5m (6km streamer) and a \gls{shot} spacing of 37.5m would have
         78 traces in each \gls{rp} \gls{gather}. (468/6, 6 because each \gls{shot}
         advances 6 \glspl{rp}).  The largest value allowed is 200 -
         \textit{i.e.} The maximum number of traces in a \gls{gather} is 200.
         \Gls{preset} = 100

\item[\texttt{NWRDS}] The largest number of samples per trace in the job.  This
         should be trace length plus the trace header length.  All
         traces output from \texttt{GATHER} will contain \texttt{NWRDS}-60 samples.
         Traces input to gather that are longer than \texttt{NWRDS}-60 samples
         will be truncated.  Traces input to gather that have less
         than \texttt{NWRDS}-60 samples will have unpredicable contents as fill.
         \Gls{preset} = from first input trace.

\item[\texttt{MINTRS}] The minimum number of traces each \gls{gather} can have.  If mintrs=0
         and no input traces contribute to a given \gls{gather}, that \gls{gather}
         will not be output.
         \Gls{preset} = 1  \textit{e.g.} \texttt{MINTRS 24}

\item[\texttt{FRP}] The first \gls{rp} number to gather.  Traces with \gls{rp} numbers less
         than frp are not gathered. \gls{rp} numbers are calculated by process \texttt{GEOM}.
         \Gls{preset} = \gls{rp} number of the first trace.

         Note: \texttt{FRP} is needed if the first trace read into process \texttt{GATHER} is
         not the first trace in the subsurface (the preset).

\textbf{\texttt{FRP} is needed on marine geometry when trace 1 is closest to
to the ship - a reverse streamer.  Some \gls{ukooa} processed files
may have this.  The subsurface point associated with trace 1
is NOT the first subsurface point.  The subsurface point
associated with the furthest from the ship is first.}

         For instance, say the \gls{rp} numbers of the first traces are:
         5, 4, 3, 2, 1, 6, 7, 8.  The preset value of \texttt{FRP} is 5, which
         would cause \glspl{rp} 4, 3, 2, and 1 to be omitted.  \texttt{FRP 1} makes the
         the \gls{rp} 1 be the first \gls{rp}.

\item[\texttt{RPINC}] The increment of \gls{rp} numbers between the \glspl{rp} to gather.  The
         only traces gathered will have bin numbers \texttt{FRP}, \texttt{FRP+RPINC},
         \texttt{FRP+2*RPINC}, \texttt{FRP+3*RPINC}, \ldots \textit{etc}.
         \Gls{preset} = 1.
\end{description}

\section{GEOM: Field Geometry Description}
\label{cmd_geom}

Process \texttt{GEOM} is used to describe the \gls{shot} and cable geometries and to
calculate the reflection point (\gls{rp}) numbers used to gather the seismic
line.  Process \texttt{GEOM} sets the shot-receiver distance into the trace
header of every trace.  Likewise, every trace is assigned an \gls{rp} number.
Thus, process \texttt{GEOM} must precede processes \texttt{NMO} and \texttt{GATHER}, which assume
that the shot-receiver range and \gls{rp} numbers are in the trace headers.

\texttt{GEOM} assumes the seismic line is \gls{shot} in a straight line; there are
no crooked line adjustments.

\texttt{GEOM} utilizes many different methods of calculating the shooting
geometry.  In marine shooting a source is usually fired every few
seconds (rep rate or repetion rate) according to either a set time
interval or distance interval.

Most surveys since \gls{gps} provide a navigation file that includes the
shot time and \gls{shot} position.  Process \texttt{GEOM} uses it's own format
(see below) and perl script \texttt{ts2sio} exists on the
web to convert \gls{ldeo}'s TS files to SIOSEIS NAVFIL files.

Older seismic surveys often did not include a \gls{shot} position for EVERY
shot, so it was easier to describe the shooting ``rule'' and the
exceptions to the rule.  The biggest reason for an exception was
when a \gls{shot} was missed and the ship kept going.

The first two \texttt{TYPES} of describing the shooting pattern differ only in
the method of handling missing \gls{shot} point numbers.  The first method
allows adjacent \glspl{shot} to have different \gls{shot} point numbers (the \gls{shot}
number incremented even though there was not \gls{shot}) whereas the second
method assumes adjacent \glspl{shot} have consecutive numbers.  \texttt{TYPE}s 3, 4, 5
use an LDGO navigation file.  \texttt{TYPE} 6 needs a navigation file in
SIOSEIS format.  \texttt{TYPE} 8 needs a \gls{ukooa} navigation file.  \texttt{TYPE} 9 uses
the SEG-Y header $x$/$y$ location (on Ewing \texttt{SEGDIN} uses the realtime
lat/long).

The \gls{shot} is assigned an $x$-coordinate by adding (type 1) or multiplying
(type 2) \texttt{DFLS} (distance from the last \gls{shot}) to the $x$-coordinate of the
shot.  Each receiver is assigned an $x$-coordinate by adding the
shot-receiver distance to the \gls{shot} $x$-coordinate.  The  RP $x$-coordinate
is calculated by assuming the RP is halfway between the \gls{shot} and
receiver.  The RP number is the RP $x$-coordinate divided by DBRPS
(distance between RPs) and truncating to an integer.  The coordinate of
the first \gls{shot} of the job is the \gls{shot} number (from the header) times the
distance from the previous \gls{shot}.  \textit{i.e.}
\lstset{language=[77]Fortran}
\begin{lstlisting}
      xs = FLOAT(lhead(3)) * dfls
      rx = FLOAT(lhead(10))
      xr = xs + rx
      xrp = (xr + xs) / 2.
      lhead(6) = NINT( xrp/dbrps )
\end{lstlisting}

Older versions of GEOM (prior to Oct 2000) wrote the \gls{shot} (source)
$x$ coordinate in SEG-Y header word 19 and the receiver $x$ coordinate
in word 21, which clobbered the existing navigation in those positions.

\subsection{Parameter Dictionary}

\begin{description}
\item[\texttt{FS}] The first \gls{shot} to apply the parameters of this parameter list.  \Gls{preset}=1.

\item[\texttt{LS}] The last \gls{shot} to apply the parameters of this parameter list.  \Gls{preset}=\texttt{FS}.

\item[\texttt{GXP}] Group-range-pairs.  A list of cable group numbers and
         shot-receiver distances.  Ranges not specified are calculated
         by interpolation or extrapolation using ggx.  Group numbers must
         be strictly increasing. A maximum of 100 pairs may be given.  Ranges
         are normally negative in marine shooting since the ship always
         goes forward, in the positive direction.
         Not honored on \gls{ukooa} nav files.
         \Gls{preset}=required.    \textit{e.g.} \texttt{GXP 18 -1350 24 -450  GGX 300}

         If the first value passed to \texttt{GXP} is 0 and \texttt{GGX} is given, If the first group number is 0, then the number
         of traces per \gls{shot} from the SEG-Y trace header is used as the
         instead of the 0.  This is useful when the streamer length may
         change between deployments but the streamer leader (distance from
         the guns to the first group) is the same.
         \textit{e.g.}  \texttt{GXP 0 -142 GGX -12.5}  can be used with the 630 channel streamer
         (same as giving \texttt{GXP 630 -142}) or the 480 channel streamer (same as
         giving \texttt{GXP 480 -142}).

\item[\texttt{GGX}] The constant distance between groups.  Used for calculating
         ranges outside of those given via \texttt{GXP}.  The sign of \texttt{GGX} implies
         the direction of the unknown ranges relative to the closest
         group given.  Not honored on \gls{ukooa} nav files.
         \texttt{((Group given - group wanted) * GGX + (range of group given))}.
         \Gls{preset}=-300.
         \textit{e.g.}
\begin{verbatim}
GXP 480 -250 GGX 12.5
GXP 1 -250 GGX 12.5
\end{verbatim}

\item[\texttt{DFLS}] The distance from the last \gls{shot}.  The sign of \texttt{DFLS} implies the
         direction the \gls{shot} moved relative to the last \gls{shot}.
         Not honored on \gls{ukooa} nav files.
         \Gls{preset} = 150 for \texttt{TYPES 1, 2, 6}.
         \Gls{preset} = 1 for \texttt{TYPE 9}.

\item[\texttt{MINDFLS}] The minimum and maximum allowable distances between \glspl{shot}
\item[\texttt{MAXDFLS}] when using \texttt{TYPE} 9 (dfls computed from the fixes in the SEG-Y header).
         If the absolute value of computed distance (ABS(dfls)) is less than
         \texttt{MINDFLS} or greater than \texttt{MAXDFLS}, then the fix is considered bad.
         As of 2016.2.3, geom computes delta\_lat and delta\_long between good
         fixes and applies them to the last good fix when dfls is bad.
         \Gls{preset}s:   \texttt{MINDFLS} = 0, \texttt{MAXDFLS} = not given.
         \textit{e.g.}  \texttt{MINDFLS .1 MAXDFLS 500}

\item[\texttt{CKNAV}] Similar to \texttt{MAXDFLS}, but good for ALL TYPES.  If the distance
         between \glspl{shot} exceeds \texttt{CKNAV}, the last good fix replaces the
         bad fix in the SEG-Y header.
         \Gls{preset} = 99999.

\item[\texttt{SETBACK}] The distance between the \gls{gps} antenna and the source.  This is
         always a positive number.  Often, this is the sum of the
         distance of the \gls{gps} to the stern and the distance of the guns
         from the stern.  The source coordinates in the SEG-Y header
         (words 19 and 20) are modified by \texttt{SETBACK} and the receiver
         coordinates (words 21 and 22) are computed using the source
         coordinates and the range (word 10) determined by \texttt{GXP}.  Valid
         with \texttt{TYPE} 9 only since the ship's course must be known.
         \Gls{preset} = 0   \textit{e.g.}  \texttt{SETBACK 140}

\item[\texttt{DBRPS}] The distance between \glspl{rp}.
         \Gls{preset}:   ABS(GGX)/2

\item[\texttt{SMEAR}] The subsurface smear factor.  The distance from a \gls{rp} in which
         to look for a trace.  The smear is centered about the \gls{rp}.
         Not honored on \gls{ukooa} nav files.
         \Gls{preset}=\texttt{DBRPS}

\item[\texttt{RPADD}] A scalar to add to every \gls{rp} number.  Sometimes \texttt{GEOM} calculates
         a neagtive \gls{rp} number, which might cause other sioseis processes
         problems.  Sometimes it might be useful to identify different
         seismic lines by having different \gls{rp} number on each line.
         \Gls{preset} = 0        \textit{e.g.}   \texttt{RPADD 1000}

\item[\texttt{YOFFA}] The y-offset (perpendicular) of the \gls{shot} from the seismic line
         (the x-axis).  The y-offset results in the shot-receiver
         distance being the hypotenuse of the triangle of the in-line
         shot-receiver distance and the y-offset.  The y-offset is
         applied to the range after the \gls{rp} computation is performed.
         \Gls{preset} = 0        \textit{e.g.}   \texttt{YOFFA 100}

\item[\texttt{TYPE}] The type of missing \gls{shot} geometry.  Also see a discussion of \texttt{TYPE}
         above in the description of the algorithm.
         \Gls{preset} = 2

\begin{description}
    \item[1] Missing \glspl{shot} must be explicitly described by using multiple
         \texttt{GEOM} lists.  \textit{e.g.}
\begin{verbatim}
FS 1 LS 1 DFLS 50 TYPE 1 END
FS 2 LS 2 DFLS 100 END
FS 3 LS 99999 DFLS 50 END
\end{verbatim}
         describes a situation where there is a missing \gls{shot} between \gls{shot}
         point numbers 1 and 2.  The x-coordinate of the \gls{shot} is obtained
         by ADDING \texttt{DLFS} to the previous \gls{shot} x-coordinate.

\item[2] Missing \gls{shot} points are assumed to occur whenever a \gls{shot}
         point number is missing.  \textit{e.g.} fs 1 ls 999999 dfls 50 type 2 end
         will cause the geometry to jump ahead when a \gls{shot} is missing.
         The x-coordinate of the \gls{shot} is obtained by MULTIPLYING the \gls{shot}
         number by \texttt{DLFS}.

\item[3, 4, 5] LDGO navigation method. \gls{cdp} = 3, \gls{wap} = 4 and \gls{esp} = 5,
         When LDGO method is used, \texttt{NAVFIL} and \texttt{NTRCS} must be given. \texttt{GXP} is
         used to find the range from the guns to the closest receiver
         (assumed to be the highest channel number), with the rest of the
         cable to be defined in the navigation file.
             LDGO binary nav files may come from Lamont or  may be
         generated by Graham Kent's 'navcmp' program which reads the
         Ewing \texttt{ts.n*} file.
             This method creates both the \gls{shot} and receiver $x$ and $y$
         coordinates in SEG-Y header words 19-22.

\item[6] SIOSEIS navigation file method.  \texttt{GEOM} will compute a \texttt{DLFS}
         on each \gls{shot} based of the \gls{shot} time in the SEG-Y trace header
         and the navigation in an \gls{ascii} file named via the parameter
         \texttt{NAVFIL}.  The format of the SIOSEIS navigation file is:
         year day hour minute second lat/deg lat/min long/deg long/min \textit{e.g.}     
\begin{verbatim}
1997 67 12 0 0 -69 43.2954 170 23.646
\end{verbatim}
         Each quantity must be separated by a space or tab.  The year
         is ignored by the program, so it may be any number of digits.
         South latitude and west longitude must be negative.  If the
         \gls{shot} is not at exactly the same time as a \gls{shot} in SIOFIL, it's
         position is computed by interpolation between adjacent points.
         The lat and long arc seconds are written into the source and
         receiver spots (19, 20, 21, 22) of the SEG-Y header.
         Parameter \texttt{NAVFIL} must be given.

\item[7] Elevations are inserted into the SEG-Y trace header with no
         other geometry done.

\item[8, 10] \gls{ukooa} file input (implied when parameter \texttt{NAVFIL} is a \gls{ukooa}
         file).  The shot-receiver range and \gls{rp} bin number are computed
         and stored in the SEG-Y trace header.  The receiver (streamer)
         depth and the water depth at the receiver are also transferred
         to the SEG-Y trace header.  SEG-Y word 45 (Coordinate units)
         is set to 1 (Length).  Parameter \texttt{NAVFIL} must be given.  The
         X and Y coordinates of the first \gls{shot} are used as the origin.
\begin{description}
\item[8]\texttt{TYPE} 8 writes the distance from the origin into SEG-Y long
           words 19, 20, 21, 22 (Source and Group coordinates).
\item[10]\texttt{TYPE} 10 writes the \gls{ukooa} eastings and northings into SEG-Y long
           words 19, 20, 21, 22 (Source and Group coordinates).
\end{description}

\item[9] Parameter \texttt{DFLS} is calculated from the SEG-Y trace header
         longitude and latitude (SEG-Y bytes 73-76 and 77-80 respectively)
         and the coordinates scalar in bytes 17-18.  Bowditch's formula
         for determining the number of meters per degree of latitude
         and longitude is used, so it is imperative that the longitude
         be in SEG-Y order and precede the latitude ($x,y$ vs lat/long).
         The distance computed is a simple distance; earth curvature is
         not considered.  This method should be very useful when the
         ship's navigation is in differential mode.  The computed \texttt{DFLS}
         will always be positive, so set \texttt{GXP} appropriately.  This will
         NOT work on data that has already been through process \texttt{GEOM} since
         \texttt{GEOM} writes in word 19.

\item[11] The trace range (word 10) and \gls{rp} number (word 6) are computed
         from SEG-Y long words 19, 20, 21, 22 (Source and Group
         coordinates).  The coordinate of the receiver of the first
         trace is used as the origin.  \texttt{TYPE} 11 works ONLY on ``normal''
         marine geometry where trace 1 is furthest from the source
         and the streamer is pulled.

\item[13] Marine feathered streamer geometry using \gls{ukooa} navigation file.
         The \gls{cmp} bin number, feathering angle, and cross-line offset
         are written into the SEG-Y header.  The angle is in tenths of
         a degree in short word 48.  The offset is in units of meters
         in short word 49.

\item[14] Healy05 where \texttt{NAVFIL} is for the  Geometrics Log file which had
         the only \gls{shot} number (trigger count).  It also contains the
         ``true time'' of the \gls{shot} and the Ashtech navigation.  \texttt{EPATH}
         is use for the Healy SeaBeam centerbeam water depth.  The
         Geometrics FFID is moved to word 5 (SEG-Y energy source number)
         and the log file \gls{shot} number is place in word 3, the "original
         field record number".  \emph{No longer available}

\item[17] The range (SEG-Y word 10) is the distance of the current \gls{shot}
         along the ship's track using the first trace as the origin.
         \texttt{SEG}-Y words 19 \& 20, source x and y, must be in arcseconds.
         When \texttt{DBRPS} is given, the \texttt{SIOSEIS} ``gather'' convention is
         implemented so that traces within \texttt{DBRPS} of each other may be
         stacked with process stack (this is similar to process gather).
         Traces that are more than a 90 degree angle from the previous
         trace is ``killed'' by setting the SEG-Y trace id (word 15) to 2.
         This represents a ``back and ram''.  Trace angles less than 90
         are considered a turn.  The \gls{rp} number (word 6) is
         range / \texttt{DBRPS} + 1.

         Type 17 was designed to stack Knudsen chirp data where the ping rate (distance between pings) is variable.

\item[18] Read a \gls{sio} NAVFIL and insert the lat/long into the SEG-Y trace
         headers.  The \gls{ascii} navigation is associated with the SEG-Y
         using the timestamps.  If the navigation file timestamps are
         zero, then the ``record'' field is used.  The record number is
         assumed to be the SEG-Y \gls{shot} number (word 3) unless the SEG-Y
         the \gls{rp} trace number (SEG-Y word 7) is non-zero, in which case
         the record number is assumed to be the SEG-Y \gls{rp} number (word 6).

\item[19] Get the source lat/long from the \gls{ukooa} file and insert into
         SEG-Y words 20/19 as real numbers.

\item[20] Determine the streamer geometry (\texttt{GXP}, \texttt{GGX}, \texttt{DBRPS}) from the first
         \gls{shot} in the \gls{ukooa} file given in \texttt{NAVFIL} and get the navigation
         (\texttt{DFLS}) from the \gls{ldeo} trace header.  The intent is provide a
         method of automatically detecting the number of traces in the
         streamer, yet not depend on the \gls{ukooa} file for the actual
         ship navigation.  In real-time processing, the \gls{ukooa} file may
         not be available as quickly as the \gls{shot} data.  The \gls{ukooa} H0900
         (Offset from ship positioning) is NOT used, thus the S (source)
         and R (receiver) positions are used directly/unmodified.
\end{description}

\item[\texttt{BGP}] Bird-group-pairs.  A list of bird numbers (ids) and streamer
         group numbers so that the bird (which have the depth sensors)
         location can be associated with the group range (\texttt{GXP}).  The
         depth is placed in SEG-Y trace header long word 11 as a
         negative number since it is an elevation relative to sea level.
         \Gls{preset} = none    \textit{e.g.} bgp 1 5 3 9 4 17 5 25 6 33 7 41 9 45 10 53
          11 65 12 73 13 85 14 98 15 106 16 118 17 125 18 133 8 141 2 149

\item[\texttt{DECLIN}] The magnetic declination to add to the compasses to convert
         the readings to true north.
         \Gls{preset} = 0.   \textit{e.g.}   \texttt{DECLIN -14.5}

\item[\texttt{BIN\_H}] Bin height.  When using type 13 geometry (streamer feathering
         from \gls{ukooa} files), traces with a crossline midpoint offset
         larger than \texttt{BIN\_H} are flagged as dead (and thus dropped by
         process gather).

\item[\texttt{EPATH}] The pathname of a file containing the elevations that \texttt{GEOM}
         will insert into the SEG-Y header locations for \gls{shot} and
         receiver elevations.  Process \texttt{SHIFT} parameters \texttt{DATUME} and
         \texttt{DATUMV} may be used do elevation shifts.  The format of the
         elevation file is a surface location and elevation pair on
         a single line.  Each pair must be on a separate line and
         the location values must increase from line to line.  The
         surface location values may be obtained using processes \texttt{GEOM}
         and \texttt{PROUT}.  \textit{e.g.}
\begin{verbatim}
PROCS GEOM PROUT END
GEOM  WRITEXY YES ........
PROUT   FNO 0 LNO 999999 FTR 1 LTR 99999
        INDICES L3 L4 L19 L21 END
\end{verbatim}
         Locations not specified in the file are obtained through
         interpolation or extrapolation.
         \Gls{preset} = none    \textit{e.g.}  \texttt{/data/vol3/henkart/cats/elevations}

\item[\texttt{WRITEXY}] A yes/no switch indicating whether geom should write the
         the calculated source x and y coordinates into SEG-Y trace
         words 19 and 21.
         \Gls{preset} = no        \textit{e.g.} \texttt{WRITEXY YES}
\end{description}

\subsection{Additional Type 3 Parameters}

\begin{description}
\item[\texttt{NAVFIL}] Navigation filename.  LDGO binary files and \gls{ukooa} P1 files
         may be used (SIOSEIS determines it's a \gls{ukooa} file if the
         first byte of the file is an \gls{ascii} letter H).
         Required for \texttt{TYPE}s 3, 4, 5, 6, 8, 16 geometry.
\item[\texttt{NAVFIL2}] A second navigation file for type 16 geom.  Needed when the
         navigation is in multiple files (there may be a new nav file
         at midnight every night).

\item[\texttt{NTRCS}] The number of traces per \gls{shot}.  Required for LDGO navigation
         since the number of traces per \gls{shot} in the SEG-Y file includes
         the auxiliary channels.

\item[\texttt{OFFSET}] The offset between the guns and the navigation antenna.  (On
         the recording ship).  Used with LDGO navigation only.
         Required when \texttt{TYPE 3} geometry is used.

\item[\texttt{OFFSET2}] The offset between the guns and the navigation antenna on the
          ``shooting'' ship on 2 ship experiments.  Used with LDGO
          navigation only.
\end{description}

\subsection{Cross-Line Offset}

\subsubsection{Definitions}
\begin{description}
    \item[Shot-line] The straight line between the first \gls{shot} (\texttt{FS}) and the last \gls{shot} (\texttt{LS}).  The $x,y$ coordinates for the shot-line endpoints are taken from the \gls{ukooa} file for \glspl{shot} \texttt{FS} and \texttt{LS}.
    \item[Midpoint] The $(x,y)$ point halfway between the \gls{shot} and the receiver.
    \item[Bin] A rectangular box with height \texttt{BIN\_H} and width \texttt{DBRPS}.
    \item[Bin center] The center of the box.  A point around which the bin is constructed.
    \item[Bin center-line] The straight line between \texttt{(FBINX,FBINY)} and \texttt{(LBINX,LBINY)}.  The center of the first bin is \texttt{(FBINX,FBINY)} and successive bincenters are \texttt{DBRPS} away along the line.
    \item[CMP] Common midpoint. The bin that is common (the same) for multiple traces.
    \item[Feathering angle] The angle between the streamer (a line between the \gls{shot} and the receiver) and the shot-line.
    \item[Processing-line] Meaningful only when defined by the user with parameters \texttt{FBINX,FBINY} and \texttt{LBINX,LBINY} so that the bin cross-line offset can be calculated.  Without user definition, the processing line lies along the shot-line.
    \item[Cross-line offset] The perpendicular distance between the receiver and the shot-line.
\end{description}

\subsection{Parameters}

\begin{description}
\item[\texttt{BIN\_OFF}] The offset of the bin center-line from the shot-line.  This assumes the bin center-line is parallel to the shot-line and starts and ends with the bins computed from the \gls{shot} and streamer geometry.  \Gls{preset} = 0.
\item[\texttt{FBINX}] The $x$ coordinate of the first bin.  Used when the bin center-line is not parallel to the shot-line.  \Gls{preset} = 0.
\item[\texttt{FBINY}] The $y$ coordinate of the first bin.  Used when the bin center-line is not parallel to the shot-line.  \Gls{preset} = 0.
\item[\texttt{LBINX}] The $x$ coordinate of the last bin.  Used when the bin center-line is not parallel to the shot-line.  \Gls{preset} = 0.
\item[\texttt{LBINY}] The $y$ coordinate of the last bin.  Used when the bin center-line is not parallel to the shot-line.  \Gls{preset} = 0.
\end{description}

\subsection{Hidden Print Parameter}

\begin{description}
\item[\texttt{LPRINT}] Print switch.  Intended as programmer debug information.
         Process \texttt{PROUT} parameter \texttt{TRLIST} is also a convenient method of printing
         the geometry (\textit{e.g.} lat/long ) information within the SEG-Y trace
         headers.  \textit{e.g.} \texttt{LPRINT 1026}    does \texttt{LPRINT 2} and \texttt{LPRINT 1024}
\begin{description}
     \item[2] Every trace will print a line containing various geometry variables such as the CDP number.  USE CAUTION when using this parameter since the print output may be LARGE.
\item[4] The Ewing/Digicon trace 0 is printed.
\item[8] The streamer depth and compasses are printed.
\item[16] More Ewing/Digicon trace 0 navigation.
\item[32] Print the source and receiver location and elevations.
\item[128] Print the information needed for \texttt{TYPE 13} (crossdip analysis).
\item[256] Print the \gls{shot} number, date/time, lat/long, depth        \textit{e.g.} 133709 2005+254:07:05:24 N 88 59 26.052 W 179 00 47.516 2179 This is suitable for the LGL Marine Mammal Observer's Report.
\item[512] Print the \gls{shot} number, \texttt{DFLS}, lat/long          \textit{e.g.} Shot:  133709 dfls:     1.00 lat:   88 59 26.052 long: -179  0 47.516 
\item[1024] Print \texttt{DFLS} and the heading.
\end{description}

\item[\texttt{CGP}] Compass-group-pairs.  A list of compass numbers (ids) and
         streamer group numbers so that the compass location can be
         associated with the group range (\texttt{GXP}).  The compass data are
         not saved in the SEG-Y header nor used by any SIOSEIS process.
         The compasses may be printed by using \texttt{LPRINT 8}.
         \Gls{preset} = none    \textit{e.g.} \texttt{CGP 1 4 6 44 11 84 16 140}
\end{description}

Alistair Harding's version of ts2sio (in Perl) is shown in
Listing~\ref{lts2sio}.

\lstset{language=Perl}
\begin{lstlisting}[caption={ts2sio.pl},label=lts2sio]
# usage:   ts2sio  tsfile
# \textit{e.g.} ts2sio  /data/processed/0008/ts.n255 > sioseis_nav_file
# 2000+255:01:08:36.112 017382 N 32 13.2870 W 075 29.1645 test
#
while (<>)
{
  my ($timestamp,$shotno,$ns,$latdeg,$latmin,$ew,$longdeg,$longmin)
                          = split /\s+/;
  my ($year,$jday,$hour,$minute,$second) = split /\+|-|:/, $timestamp;

  $latdeg  = -$latdeg  if ($ns eq "S");
  $longdeg = -$longdeg if ($ew eq "W");

  $, = " ";  # separate output with a space
  $\ = "\n"; # append newline
  print($year,$jday,$hour,$minute,$second,
         $latdeg,$latmin,$longdeg,$longmin,$shotno);
}
\end{lstlisting}

\section{GRDOUT: Write a GMT Format 1 Grdfile}
\label{cmd_grdout}

     Process GRDOUT writes a GMT grid file (grdfile) format 1.
GRDOUT collects all the traces on disk and then transposes them
so that all like times are in a row.

     GRDOUT is an ``offline process'' or ``fork process''; it does
not write its output for the next process in the procs list.
The input trace is not modified and is passed to the next
process in the PROCS list.

     The data are time reversed because the GMT Y-axis and the
seismic reflection time axis are reversed.


\subsection{Parameter Dictionary}

\begin{verbatim}
OPATH  - The output grdfile.  When using this filename for GMT,
         append ``=1'' to it since it is in GMT format 1.  \textit{e.g.}
         grdimage grdfil=1
         REQUIRED.           \textit{e.g.} grdfil

SET    - Start and End Time to output.
         \Gls{preset} = delay to end of first trace.

XMIN   - The x_min value to insert into the grdfile header.
         \Gls{preset} = 0

XMAX   - The x_max value to insert into the grdfile header.
         \Gls{preset} = the number of traces - 1

XINC   - The x_inc value to insert into the grdfile header.
         \Gls{preset} = 1

YMIN   - The y_min value to insert into the grdfile header.
         \Gls{preset} = 0

YMAX   - The y_max value to insert into the grdfile header.
         \Gls{preset} = the number of time samples - 1

YINC   - The y_inc value to insert into the grdfile header.
         \Gls{preset} = 1

ZMIN   - The z_min value to insert into the grdfile header.
         \Gls{preset} = most negative trace amplitude of all traces.

ZMAX   - The z_max value to insert into the grdfile header.
         \Gls{preset} = most positive trace amplitude of all traces.

ZSCALE - The z_scale value to insert into the grdfile header.
         \Gls{preset} = 1

XUNITS - The \gls{ascii} x_units to insert into the grdfile header.
         \Gls{preset} = km

YUNITS - The \gls{ascii} y_units to insert into the grdfile header.
         \Gls{preset} = secs

ZUNITS - The \gls{ascii} x_units to insert into the grdfile header.
         \Gls{preset} = amplitude

TITLE  - The \gls{ascii} title to insert into the grdfile header.
         \Gls{preset} =

COMMAND - The \gls{ascii} command to insert into the grdfile header.
         \Gls{preset} =

COMMENT - The \gls{ascii} comment to insert into the grdfile header.
          \Gls{preset} = Processed by SIOSEIS

HDRPAD - A YES/NO swicth indicating whether to pad the GMT header
         with an extra 4 bytes or not.  SUN/SGI/HP need the pad.
         OSX and PC do not.  The need of the pad depends on how GMT
         was compiled and is due to the byte alignment of the C
         DOUBLE.  Must be set to NO on Mac OSX.
         \Gls{preset} = yes, except on PC            \textit{e.g.}   hdrpad no

LPRINT - SIOSEIS debug print switch.
       = 4, Print the grdfile header, similar to grdinfo
         \Gls{preset} 4
\end{verbatim}

\lstset{language=[ANSI]C}
\begin{lstlisting}[caption={gmt\_grd.h}]
/*--------------------------------------------------------------------
 *    The GMT-system:	@(#)gmt_grd.h	3.17  02/06/99
 *
 *	Copyright (c) 1991-1999 by P. Wessel and W. H. F. Smith
 *	See COPYING file for copying and redistribution conditions.
 *
 *	This program is free software; you can redistribute it and/or modify
 *	it under the terms of the GNU General Public License as published by
 *	the Free Software Foundation; version 2 of the License.
 *
 *	This program is distributed in the hope that it will be useful,
 *	but WITHOUT ANY WARRANTY; without even the implied warranty of
 *	MERCHANTABILITY or FITNESS FOR A PARTICULAR PURPOSE.  See the
 *	GNU General Public License for more details.
 *
 *	Contact info: www.soest.hawaii.edu/gmt
 *--------------------------------------------------------------------*/
/*
 * grd.h contains the definition for a GMT-SYSTEM Version >= 2 grd file
 *
 * grd is stored in rows going from west (xmin) to east (xmax)
 * first row in file has yvalue = north (ymax).
 * This is SCANLINE orientation.
 *
 * Author:	Paul Wessel
 * Date:	26-MAY-1990
 * Revised:	21-OCT-1998
 */

#include "netcdf.h"

/* Nodes that are unconstrained are assumed to be set to NaN */

#define GRD_COMMAND_LEN	320
#define GRD_REMARK_LEN	160
#define GRD_TITLE_LEN	 80
#define GRD_UNIT_LEN	 80

struct GRD_HEADER {
	int nx;				/* Number of columns */
	int ny;				/* Number of rows */
	int node_offset;		/* 0 for node grids, 1 for pixel grids */
	double x_min;			/* Minimum x coordinate */
	double x_max;			/* Maximum x coordinate */
	double y_min;			/* Minimum y coordinate */
	double y_max;			/* Maximum y coordinate */
	double z_min;			/* Minimum z value */
	double z_max;			/* Maximum z value */
	double x_inc;			/* x increment */
	double y_inc;			/* y increment */
	double z_scale_factor;		/* grd values must be multiplied by this */
	double z_add_offset;		/* After scaling, add this */
	char x_units[GRD_UNIT_LEN];	/* units in x-direction */
	char y_units[GRD_UNIT_LEN];	/* units in y-direction */
	char z_units[GRD_UNIT_LEN];	/* grid value units */
	char title[GRD_TITLE_LEN];	/* name of data set */
	char command[GRD_COMMAND_LEN];	/* name of generating command */
	char remark[GRD_REMARK_LEN];	/* comments re this data set */
};

/*-----------------------------------------------------------------------------------------
 *	Notes on node_offset:

	Assume x_min = y_min = 0 and x_max = y_max = 10 and x_inc = y_inc = 1.
	For a normal node grid we have:
		(1) nx = (x_max - x_min) / x_inc + 1 = 11
		    ny = (y_max - y_min) / y_inc + 1 = 1
		(2) node # 0 is at (x,y) = (x_min, y_max) = (0,10) and represents the surface
		    value in a box with dimensions (1,1) centered on the node.
	For a pixel grid we have:
		(1) nx = (x_max - x_min) / x_inc = 10
		    ny = (y_max - y_min) / y_inc = 10
		(2) node # 0 is at (x,y) = (x_min + 0.5*x_inc, y_max - 0.5*y_inc) = (0.5, 9.5)
		    and represents the surface value in a box with dimensions (1,1)
		    centered on the node.
-------------------------------------------------------------------------------------------*/
\end{lstlisting}

\section{HEADER / HEADER2 / HEADER3: Manipulate the SEG-Y Headers}
\label{cmd_header}

Process HEADER modifies the SEG-Y trace headers by specifying the header
word number (index) and the value.  The SEG-Y \gls{ebcdic} header may be
modified also.  There are two methods for indexing the SEG-Y header, one
by using the name of the variable within the header and then by knowing
the SEG-Y header indices.

Processes HEADER2 and HEADER3 are identical to HEADER and enable three
unique HEADER processes to be given in a single SIOSEIS job.

When using the index to the SEG-Y trace header structure, consult the
SIOSEIS document segy.header, since the trace header is composed of 16
bit integers, 32 bit integers, and floating point words.  The SEG-Y trace
header values are changed by specifying the index of the value and the
new value.  16 bit header words are specified using the IHDR parameter,
32 bit words are specified via the LHDR parameter, and floating point
words are specified via HDR.  These parameters are lists of indices and
values.  HDR, LHDR, and IHDR may be given on each control point.

Spatial variation between control points is available.   The SIOSEIS
document SYNTAX discusses control point usage, however a control point
is an fno/lno list of parameters.  Spatial interpolation is different
from the interpolation/extrapolation of values on traces within a
shot/\gls{rp}.  Interpolation of values between traces is NOT available.

Consult the document segy.header for a partial list of header word
meanings and a list of header words used by SIOSEIS.  Various SIOSEIS
processes use the unassigned traces header words; process diskin
uses long words 46 and 49 to store the floating point values of the
deep water delay and the sample interval.

The order the HEADER parameters are done is: IHDR, LHDR, HDR, NX, CLEAN

Example 1:     Change the range (header word 10) on two \glspl{shot}:
using the SEG-Y variable name scheme:
\begin{verbatim}
HEADER FNO 796 LNO 796 HEADER RANGE 415 INTERP NO  END
           FNO 797 LNO 797 HEADER RANGE 120 END
END
\end{verbatim}
or using the SEG-Y index approach:
\begin{verbatim}
HEADER FNO 796 LNO 796 LHDR 10 415 INTERP NO  END
           FNO 797 LNO 797 LHDR 10 120 END
END
\end{verbatim}

Example 2:  show the effects of spatial variation:
\begin{verbatim}
HEADER
LHDR 10 10 16 16 FNO 4 LNO 5 END
LHDR 10 12 16 20 FNO 7 LNO 8 END
END
\end{verbatim}

With spatial variation (the preset);

\begin{itemize}
\item shots 1 - 5 32 bit header 10 = 10, and 16 = 16,
\item shot 6 32 bit header word 10 = 11, and 16 = 18,
\item shots 7 - end 32 bit header word 10 = 12, and 16 = 20.
\end{itemize}

Without spatial variation (\texttt{INTERP NO});

\begin{itemize}
\item shots 1 - 3, no change
\item shots 4 - 5 32 bit header 10 = 10, and 16 = 16,
\item shot 6 no change
\item shots 7 - 8 32 bit header word 10 = 12, and 16 = 20.
\item shots 9 - end no change
\end{itemize}

Example 1, suppose the \gls{rp} number (32 bit word 6) for \gls{rp} 1-10 should be
changed to 1001-1010 and the year the data was recorded (16 bit word 79)
should be 1991.
\begin{verbatim}
HEADER
     FNO   1 LHDR 6 1001 IHDR 79 1991 END
     FNO 10 LHDR 6 1010 IHDR 79 1991 END
END
\end{verbatim}

\subsection{Parameter Dictionary}

\begin{description}
\item[\texttt{BHDR}] Binary header modification.  Pairs of numbers, the first of
         the pair is the index of the SEG-Y binary header word to be
         replaced by the value of the second number.  There is no
         limit to the number of pairs given.
         \Gls{preset} = none.  \textit{e.g.} \texttt{BHDR 5 3}   indicates that word 5 will become 3.

\item[\texttt{Cxx}] Comment card images to replace in the SEG-Y \gls{ebcdic} header.  \texttt{xx}
         is the 2 digit number of the card image to replace.   The
         comment itself must be enclosed in quotes and must be less than
         76 characters long.
         \Gls{preset} = none   \textit{e.g.} \texttt{C09 \# 'This comment replaces card image 9.'}

\item[\texttt{CLEAN}] Sets the SEG-Y trace header and/or binary header to zero.
         \Gls{preset} = 0  \textit{e.g.} \texttt{CLEAN 3}
\begin{description}
       \item[1] Cleans the unused (by SIOSEIS) portion of the SEG-Y binary
           header.  See the document segy.header for the list of
           words used by sioseis.
       \item[2] Mildly cleans the unused (by SIOSEIS) portion of each
           trace header.
       \item[3] Cleans the binary header and mildly cleans every trace
           header to zero (1+2 = 3)
       \item[4] Harshly cleans every trace header.  Only the \gls{shot} number,
           \gls{shot} trace number, sample interval and number of samples
           are spared.
       \item[5] Cleans the binary header and harshly cleans the trace
           headers (1+4 = 5).
\end{description}

\item[\texttt{FNO}] The first \gls{shot}/\gls{rp} number defining a control point.  Shot/\gls{rp}
         numbers \textbf{MUST} be strictly monotonically increasing.  If all
         \glspl{shot} are to receive the same parameters, use:
         fno 0 lno 9999999
         \Gls{preset} = 1

\item[\texttt{LNO}] The last \gls{shot}/\gls{rp} number defining a control point.  Shot/\gls{rp}
         numbers \textbf{MUST} be strictly monotonically increasing.  \texttt{LNO} is
         reset (default) to \texttt{FNO} on every \texttt{FNO}/\texttt{LNO}/\texttt{END} list.
         Default = \texttt{FNO}

\item[\texttt{NOINC}] The increment between \texttt{FNO} and \texttt{LNO}.  Only \glspl{shot}/\glspl{rp} which
         match  DO  \texttt{FNO}, \texttt{LNO}, \texttt{NOINC}  are modified by process \texttt{HEADER}.
         \Gls{preset} = \texttt{NOINC IGNORED}

\item[\texttt{FTR}] The first trace of \texttt{FNO}/\texttt{LNO} to modify.  Traces numbers smaller
         than \texttt{FTR} will not be modified.
         \Gls{preset} = 1

\item[\texttt{LTR}] The last trace of \texttt{FNO}/\texttt{LNO} to modify.  Traces numbers larger
         than \texttt{LTR} will not be modified.
         \Gls{preset} = 9999987

\item[\texttt{TRINC}] The increment between \texttt{FTR} and \texttt{LTR}.  Only traces which
         match  DO  \texttt{FTR}, \texttt{LTR}, \texttt{TRINC}  are modified by process HEADER.
         \Gls{preset} = \texttt{TRINC IGNORED}

\item[\texttt{REV1}] The contents of the file specified will be ``appended'' to any
         existing SEG-Y Rev 1 Textual Extension Records.  Process
         \texttt{HEADER} inserts the contents before the ``((EndText))'' record,
         so it should not be included in the file.  The contents
         are written in sets of 40 \gls{ascii} lines.  \texttt{HEADER} will blank
         fill each line to be 80 characters and \texttt{HEADER} will write
         the required CR/LF in columns 79 and 80.
         If the specified file does not exist, \texttt{HEADER} will create
         a ``(( SIOSEIS SEG-Y ))'' along with the ``((EndText))'' stanza.
         \Gls{preset} = none,    \textit{e.g.}   \texttt{REV1 JUNK}
\end{description}

\subsection{Trace Header Modification with Spatial Variation}

\begin{description}
\item[\texttt{INTERP}] A \texttt{YES}/\texttt{NO} switch to indicate spatial variation between control
         points.
         \Gls{preset} = \texttt{YES}, except with \texttt{XN}
\begin{description}
\item[\texttt{YES}] Spatial variation will be done.
\item[\texttt{NO}] Spatial variation will NOT be done.
\end{description}

\item[\texttt{HEADER}] A list of header variable names and values.  The permissible
         header variable names are:
         \Gls{preset} = none.   \textit{e.g.} \texttt{HEADER DELAY 1.   \# delay in seconds}
\begin{description}
\item[\texttt{SHOTNO}]  The \gls{shot} number is used. (third long integer)
\item[\texttt{SHOTTR}]  The \gls{shot} trace number is used. (fourth long integer)
\item[\texttt{RPNO}]    The \gls{rp} number is used. (sixth long integer)
\item[\texttt{RPTR}]    The \gls{rp} trace number is used. (seventh long integer)
\item[\texttt{RANGE}]   The range or shot-receiver distance. (tenth long integer)
\item[\texttt{FOLD}]    The CDP fold or CDP coverage.
\item[\texttt{DELAY}]   The deep water delay time in seconds.
\item[\texttt{SI}]      The sample interval in seconds.
\item[\texttt{WBT}]     The water bottom time in mils.
\item[\texttt{WBD}]     The water bottom depth in meters.
\end{description}

\item[\texttt{IHDR}] A list of indices and values for the 16 bit integer SEG-Y trace
         header.  Up to 60 index-value pairs may be given.  Used with \texttt{ITYPE}.
         An index may be repeated, but only the last one will count.
         \textit{e.g.} \texttt{IHDR 15 1} sets the SEG-Y header word for trace id to 1
         Default = none

\item[\texttt{LHDR}] A list of indices and values for the 32 bit integer SEG-Y trace
         header.  Up to 60 index-value pairs may be given. Used with \texttt{LTYPE}.
         An index may be repeated, but only the last one will count.
         \textit{e.g.}   \texttt{LTYPE MULTIPLY LHDR 10 -1}
         \textit{e.g.}  \texttt{LHDR 7 0 51 0}    sets long integer words 7 and 51 to 0
         Default = none

\item[\texttt{HDR}] A list of indices and values for the floating point SEG-Y trace
         header.  Up to 60 index-value pairs may be given.  Used with \texttt{TYPE}.
         An index may be repeated, but only the last one will count.
         Default = none

\item[\texttt{ITYPE}] The type of 16 bit integer trace header modifications.
         Used with parameter \texttt{IHDR}.
         \Gls{preset} = \texttt{REPLACE}      \textit{e.g.}  \texttt{ITYPE MULTIPLY}
\begin{description}
\item[\texttt{REPLACE}] The user given values replace the SEG-Y header values.
\item[\texttt{ADD}] User given values are added to the SEG-Y header values.
\item[\texttt{MULTIPLY}] User given values multiply the SEG-Y header values.
\end{description}

\item[\texttt{LTYPE}] The type of 32 bit integer trace header modifications.
         Used with parameter \texttt{LHDR}.
         \Gls{preset} = \texttt{REPLACE}      \textit{e.g.}  \texttt{LTYPE MULTIPLY}
\begin{description}
\item[\texttt{REPLACE}] The user given values replace the SEG-Y header values.
\item[\texttt{ADD}] User given values are added to the SEG-Y header values.
\item[\texttt{MULTIPLY}] User given values multiply the SEG-Y header values.
\end{description}

\item[\texttt{TYPE}] The type of floating point trace header modifications.
         Used with parameter \texttt{HDR}.
         \Gls{preset} = \texttt{REPLACE}      \textit{e.g.}  \texttt{TYPE MULTIPLY}
\begin{description}
\item[\texttt{REPLACE}] The user given values replace the SEG-Y header values.
\item[\texttt{ADD}] User given values are added to the SEG-Y header values.
\item[\texttt{MULTIPLY}] User given values multiply the SEG-Y header values.
\end{description}
\end{description}

\subsection{Trace Header Modification by Equation, without spatial variation}
All indices start with 1.

\begin{verbatim}
XN  =  XN/C op XN/C, or
XN     XN/C op XN/C, where;
       X = I, means short integer (16 bit integer trace header)
         = L, means long integer (32 bit integer trace header)
         = R, means real word (host floating point)
         = B, means byte
         = D, means double precision real (64 bit floating point)
       N = the index with the SEG-Y trace header.
       C = a constant. (Either XN or C may be given).
       OP is an operator of +, -, *, /, **

       A maximum of 10 XNs may be given in a parameter list.  HEADER
       does the operations in the same order in which they are given.

       Example 1: l1 0, means long integer word 1 is 0.
       Example 2: i59  i59 * 1000   Means multiply short word by 1000
       Example 3: r49  i48 / 100000.  means real word 49 = short
                                    integer word 48 divided by 100000.
       Example 4: save the \gls{shot} and shottrace number in header
       words 1 and 2.
             header
                   fno 0 lno 99999 ftr 1 ltr 9999
                   l1 = l3 l2 = l4   # same as l1 l3 l2 l4
             end
       Example 5: Do multiple operations on the same header word.
                   l19 = -1900 l21 = 2100 l57 = 1000
                   l19 = l19 * -2
                   L19 = L19 - L57
                   L21 = L21 + L57
                   l57 = 0
       results in word 19 conatins 2800,  word 21 contains 3100,
                  word 57 contains 0

       Default = none

SWAP XN - A list of up to 50 trace header entries to byte swap.  Byte
         swapping is done before any other HEADER operations.
       X = I, means 16 bit word (2 bytes).
         = L, means 32 bit word (4 bytes - long integer or real).
       N = the index with the SEG-Y trace header.
         \Gls{preset} = none.    \textit{e.g.}  swap i45 l19 l20

\end{verbatim}

\section{HISTORY: Append a History File}
\label{cmd_history}

Process \texttt{HISTORY} keeps a running history or log of the SIOSEIS
processing steps.  The log is an \gls{ascii} file that may be edited
with a text editor.  The log file is appended each time it is opened
so the log can contain information about multiple SIOSEIS runs.

Two lines are always appended to the file stating the start
time of the job and the list of processes.  A terminating line is
also written when the job completes normally.  \textit{e.g.}
\begin{verbatim}
Job      1 started on Thu Feb 13 11:41:57 1997, SIOSEIS ver 97.2 (13 Feb. 1997)
PROCS SYN HISTORY PROUT
Job      1 finished on Thu Feb 13 11:41:57 1997
\end{verbatim}

\texttt{HISTORY} may be placed anywhere in the \texttt{PROCS} list, but will
operate only on the first and last trace of the job.

The history file may be plotted prior to plots on the HP DesignJet
plotters by using paramter \texttt{SLPATH} (side label) in program \texttt{SIO2HP}.

\subsection{Parameter Dictionary}

\begin{description}
\item[\texttt{HPATH}] The filename of the \texttt{HISTORY} file.
         REQUIRED.      \textit{e.g.}  \texttt{HPATH ew9607.line1.hist}

\item[\texttt{ALL}] When set to \texttt{YES}, all user given parameters are
         logged.  The entire SIOSEIS parameter script is
         copied to the history file.
         \Gls{preset} = \texttt{NO}       \textit{e.g.}  \texttt{ALL YES}

\item[\texttt{A}] Append mode.  A method of entering any text into the log
         file.  This is similar to the append mode of many text
         editors.  The append mode is terminated by a placing
         a period or dot in the first character of a line, as
         in the ``ed'' editor,\textit{e.g.}
\begin{verbatim}
A
This line is inserted into the log file.
Remember to terminate with a period in column 1.
.
\end{verbatim}

\item[\texttt{END}] Terminates each parameter list.
\end{description}

\section{LOGSTX: Log Stretch and Destretch}
\label{cmd_logstx}

Process \texttt{LOGSTX} applies a log function to ``stretch'' each trace in time.
The most common use of process \texttt{LOGSTX} is prior to, and subsequent to,
process \texttt{DMO}, exact log Dip Move Out.  Log stretching allows \texttt{DMO} to work
in F-K space while preserving the \texttt{DMO} ellipse.

Prior to dip move out in quasi-FK space (Fourier Transform of log
stretched trace does not give F, but a log frequency (OMEGA) - see Liner,
Geophysics May 1990 \cite{Liner1990}) use log stretch \texttt{TYPE 1} (stretch), after \texttt{DMO} use log
stretch \texttt{TYPE 2} (compress).

To preserve dips in OMEGA-K space the data which is resampled via a
cubic spline must be resampled at an adequate rate to prevent temporal
and spatial aliasing of frequency content and seismic dips, respectively.

\subsection{Parameter Dictionary}

\begin{description}
\item[\texttt{TYPE}] The type of log stretch to be applied 1 or 2.
         \Gls{preset} = 1 ``compress''
\begin{description}
    \item[\texttt{1}] ``Stretch'', $\tau = \ln{\frac{t}{t_{c}}}$ where $\tau$ is the new log time, $t$
        is time of trace, and $t_{c}$ is the cutoff time which prevents the
         log of zero to be taken.
     \item[\texttt{2}] ``Compress'' = 2, $t = t_{c} e^{\tau}$ where the variables ($t$, $t_{c}$,
         and $\tau$) are defined as above.
\end{description}

\item[\texttt{TSAMP1}] The new sample rate (*1000) for the log stretched trace.  May
         be different than time sample rate.   Used in Process \texttt{LOGST1}
         \Gls{preset} = 0.004  (250 samples/sec)

\item[\texttt{TSAMP2}] The new sample rate (*1000) subsequent to log compression.  May
         be different than initial sampling or log sampling.  Used in
         process \texttt{LOGST2}.
         \Gls{preset} = 0.004  (250 samples/sec)

\item[\texttt{TCUT}] Time cut for log stretch/destretch.  Must be same for processes
         \texttt{LOGST1} and \texttt{LOGST2}.  Times prior to \texttt{TCUT} will be nulled.
         \Gls{preset} = 0.1 seconds

\item[\texttt{SLTIME}] Start time of trace subsequent to \texttt{DMO} and ``unstretch''.
         \Gls{preset} = 0.0

\item[\texttt{ELTIME}] End time of trace subsequent to \texttt{DMO} and ``unstretch''.
         \Gls{preset} = 6.0

\item[\texttt{LOGHZ}] Highest frequency (prestretch) used.
\end{description}

\section{MIX: Running or Record Sum of Adjacent Traces}
\label{cmd_mix}

Process MIX performs a running, weighted, dip mix.  Mix is defined to be
the sum or addition of adjacent traces (in a numerical sense).   There
are three types of mix:

\subsection{A running roll-along mix}
The mix crosses record boundaries.  If the data are \glspl{shot}, the last
trace of a \gls{shot} is mixed with the first trace of the next \gls{shot}.
\textit{e.g.} A three trace equally weighted zero dip unweighted running mix
will output the following traces:
\begin{verbatim}
     trace 1 = input traces 1
     trace 2 = input traces 1+2
     trace 3 = input traces 1+2+3
     trace 4 = input traces 2+3+4
     trace 5 = input traces 3+4+5
\end{verbatim}
\textit{etc.}

\subsection{A running record mix}
The mix stops at the end of each \gls{shot} and starts over on the next
\gls{shot}.  A three trace equally weighted zero dip running record mix of 24
trace \glspl{shot} will output the following:
\begin{verbatim}
     trace 1 = input traces 1
     trace 2 = input traces 1+2
     trace 3 = input traces 1+2+3
     trace 4 = input traces 2+3+4
          .     .
          .     .
     trace 24 = input traces 22+23+24
\end{verbatim}

\subsection{A record mix}
This mix starts over after every mix set and does not output the tapered traces.  \textit{e.g.} A three trace record mix of 24 trace \glspl{shot} will output the following:
\begin{verbatim}
     trace 1 = input traces 1+2+3
     trace 2 = input traces 4+5+6
     trace 3 = input traces 7+8+9
          .        .
          .        .
     trace 8 = input traces 22+23+24
\end{verbatim}


A weighted or tapered mix allows each trace to be independently scaled
prior to the mix.  Referring to example 1 above, a 1 2 1 weighted mix
will have output trace 3 containing \texttt{(trace1)*1+(trace2)*2+(trace3)*1,
output trace 4 = (trace2)*1.+(Trace3)*2.+(Trace4)*1}.

A dip mix is a mix with each trace shifted in time prior to the mix.
The time shift is relative to the first trace within the mix so that the
first input trace is not shifted, the second is shifted by dip seconds,
the third by dip*2 seconds.

Each parameter list must be terminated with the word \texttt{END}.  The entire
set of mix parameters must be terminated by the word \texttt{END}.

\subsection{Header Mix}

A simple ship heave removal scheme required the averaging of the automatic
water bottom picks which were placed in the SEG-Y trace header by process
WBT.  Averaging is a mix, so trace header mixing may be done in process
MIX by using parameters HDR or LDR or IHDR.  When using these parameters
the normal trace mix is turned off by setting parameter TYPE to 4.  The
user may override TYPE 4 by specifying it to 1-3.  Up to 100 weights may
be given when doing TYPE 4 (header) mix.

\subsection{Parameter Dictionary}

\begin{description}
\item[\texttt{WEIGHT}] A list of weights to use in the mix.  The total number of
         weights given indicates the number of traces in the mix;
         \textit{e.g.} a 3 trace equally weighted mix will be done by giving
         \texttt{WEIGHT 1 1 1}.  At least 2 weights must be given and no more
         than 10 trace weights or 100 header weights may be given.
         Required.

\item[\texttt{TYPE}] The type of mix to be performed.
         \Gls{preset} = 1     \textit{e.g.} \texttt{TYPE} 3
\begin{description}
\item[\texttt{1}] Running mix.
\item[\texttt{2}] Running record mix.
\item[\texttt{3}] Record mix.
\item[\texttt{4}] Header mix without a data trace mix.
\end{description}

\item[\texttt{DIP}] The amount of shift, in seconds, to apply to successive input
         traces within each mixed trace.
         \Gls{preset} = 0.

\item[\texttt{MAXLEN}] The maximum length, in seconds, of the largest input trace
         (needed for allocating memory).
         \Gls{preset} = the length of the first input trace.

\item[\texttt{FNO}] The first \gls{shot} (or \gls{rp}) to apply the mix to.  Shot (\gls{rp}) numbers
         must increase monotonically.
         \Gls{preset}=1

\item[\texttt{LNO}] The last \gls{shot} (\gls{rp}) number to apply the mix to.  \texttt{LNO} must be
         larger than \texttt{FNO} in each list and must increase list to list.
         Default=\texttt{FNO}

\item[\texttt{IHDR}] The index of the 16 bit integer SEG-Y trace header value to mix.
         See the discussion of \texttt{HEADER MIX} above.  Remember that parameter
         \texttt{TYPE} is set to 4 by use of this parameter.
         \textit{e.g.} \texttt{IHDR 100}
         Default = none

\item[\texttt{LHDR}] The index of the 32 bit integer SEG-Y trace header value to mix.
         See the discussion of \texttt{HEADER MIX} above.  Remember that parameter
         \texttt{TYPE} is set to 4 by use of this parameter.
         \textit{e.g.} \texttt{LHDR 50}
         Default = none

\item[\texttt{HDR}] The index of the 32 bit floating point SEG-Y trace header value
         to mix.  See the discussion of \texttt{HEADER MIX} above.  Remember that
         parameter \texttt{TYPE} is set to 4 by use of this parameter.
         \textit{e.g.} \texttt{HDR 100}
         Default = none

\item[\texttt{END}] Terminates each parameter list.
\end{description}

\subsection{Notes}
\begin{enumerate}
\item At least 2, but not more than 10 trace weights or 100 header weights, may be given.
\item Neither weight nor dip may be changed within a job.
\end{enumerate}

\section{MUTE: Zeroing of Data}
\label{cmd_mute}

Process \texttt{MUTE} zeroes the beginning of selected traces.  A mute is used to
silence the noise preceding the first arrivals.  Mutes are applied to
marine data to eliminate the water column.  The beginning of the mute is
time zero or the deep water delay.  The user must specify the end time
of the mute.  The mute times may be varied spatially by \gls{shot} or by \gls{rp}.
The mutes may be given by trace number or by range (distance from the
shot). The mutes may be given relative to time zero or relative to the
water bottom times.  A 5 sample ramp is applied to the mute window in
order to avoid discontinuities.

Use process \texttt{SMUTE} for surgical mutes (mutes that do not start at 0).
Inner or tail mutes are considered surgical mutes and are done by
process \texttt{SMUTE}.

All parameters that remain constant for a set of \glspl{shot} (\glspl{rp}) may be
described in a parameter set \texttt{FNO} to \texttt{LNO}.  Mute times between two
parameter sets are calculated by linearly interpolating between \texttt{LNO} of
one set and \texttt{FNO} of the next set.

Process \texttt{MUTE} uses the SIOSEIS control point scheme.  See that
documentation for further information.

Each parameter list must be terminated with the word \texttt{END}.  The entire
set of mute parameters must be terminated by the word \texttt{END}.

Example 1: Mute single channel data to the water bottom described via
process \texttt{WBT}.
\begin{verbatim}
MUTE
   FNO 1 TTP 1 0 ADDWB YES END    # This does all shots!
END
\end{verbatim}

Example 2: Traces with ranges 500 or less are muted to the water bottom.
Traces with ranges 1000 or greater are muted to the water bottom plus
.5.  Traces with ranges between 500 and 1000 are muted according to
linear interpolation between 0 and .5 plus the water bottom.
\begin{verbatim}
MUTE
   FNO 1 XTP 500 0 1000 .5 ADDWB YES END
END
\end{verbatim}


\subsection{Parameter Dictionary}

\begin{description}
\item[\texttt{XTP}] Range-time-pairs.  A list of range and mute time pairs.  Mute
         times for ranges not specified are obtained through linear
         interpolation if the range is between two ranges specified.
         Traces with a range less than the smallest given range will be
         muted to the mute time of the smallest given range.  Likewise,
         ranges larger than the largest given range will be muted to the
         mute time of the largest given range.  \texttt{XTP} must be given with
         increasing ranges.  The program computes the absolute value of
         both user ranges and data ranges.  \textit{e.g.}
         \texttt{XTP 1000 3.0 2000 4.0}  - Traces with ranges less than 1000 will
         be muted to 3 seconds, traces with ranges greater than 2000
         will be muted to 4 seconds, and traces with ranges between
         1000 and 2000 will be muted proportionately between 3 and 4
         seconds.
         \Gls{preset}=none

\item[\texttt{TTP}] Trace number-time-pairs.  A list of trace numbers (of a \gls{shot} or
         \gls{rp}) and mute times (listed in pairs).  The mute time for a
         trace between two traces that are specified is obtained through
         linear interpolation of the mute times of the specified traces.
         Traces with a trace number less than the smallest given will be
         muted to the mute time of the smallest trace number.  Likewise,
         traces with a trace number larger than the largest given will
         be muted to the mute time of the largest given.  \texttt{TTP} must be
         given in increasing trace numbers.  \textit{e.g.}
         \texttt{TTP 4 2. 20 5}. - Traces 1 thru 4 will be muted to 2 seconds,
         traces 20 and up will be muted to 5 seconds, and traces 5 thru
         19 will be muted proportionately between 2. and 5 seconds.
         \Gls{preset}=none

\item[\texttt{ADDWB}] When given a value of \texttt{YES}, the mute times given via \texttt{XTP} or \texttt{TTP}
         will be added to the water bottom time of the trace.  (Water
         bottom times may be entered via process wbt).
         \Gls{preset}=no

\item[\texttt{WBDVEL}] Water bottom depth to time conversion velocity.  \texttt{WBDVEL}
         indicates that the water bottom depth should be used rather
         than the water bottom time WHEN \texttt{ADDWB} IS GIVEN also.  The
         velocity given should be the two way velocity (\textit{e.g.} 1500) to
         convert depth to time.  Water depths of zero are considered
         bad or missed and the previous water depth is used.
         \texttt{WBDVEL} is the same as using process \texttt{WBT} and \texttt{WBT} parameter \texttt{VEL}.
         \Gls{preset} = 0     \textit{e.g.}  \texttt{ADDWB YES WBDVEL 1500}

\item[\texttt{FNO}] The first \gls{shot} (or \gls{rp}) to apply the mutes to.  Shot (\gls{rp})
         numbers must increase monotonically.
         \Gls{preset}=1        \textit{e.g.} \texttt{FNO 12345}

\item[\texttt{LNO}] The last \gls{shot} (\gls{rp}) number to apply the mutes.  \texttt{LNO} must be
         larger than \texttt{FNO} in each list and must increase list to list.
         Default=\texttt{FNO}

\item[\texttt{END}] Terminates each parameter list.
\end{description}

\subsection{Notes}
\begin{itemize}
\item Either \texttt{XTP} or \texttt{TTP} must be given.
\item All times are in seconds.
\end{itemize}

\section{NMO / NMO2 / NMO3: Normal Moveout, Movein, and Slant Moveout Corrections}
\label{cmd_nmo}

Processes \texttt{NMO} apply a travel-time correction to each trace.
Process \texttt{NMO2} and \texttt{NMO3} are identical to process \texttt{NMO}, enabling \texttt{NMO} to be
given three times in a job.  The following corrections are available:

\begin{enumerate}
    \item  Normal MoveOut or NMO: \texttt{T(0) = SQRT(T(X)**2 - X**2/V(T)**2)}
    \item  MoveIn or deNMO: \texttt{T(X) = SQRT(T(0)**2 + X**2/V(T)**2)}
    \item  Slant MoveOut or SMO:  \texttt{T(0) = T(X) + X/V}
    \item  XMO: MoveOut followed by MoveIn: \texttt{T(0) = SQRT(T(X)**2 - X**2/V(T)**2)}
        followed by: \texttt{T(XEW) = SQRT(T(0)**2 + XEW**2/V(T)**2)}
\end{enumerate}

where $x$ is the range (shot-receiver distance) in the SEG-Y trace header
and $V$ is the velocity specified by the user.  It is usually thought that
all traces common to the same reflection point should have the same
velocity function.  Thus, the velocity function given by the user may be
specified at various points along the seismic line (control points).
Traces lying between the user specified control points are obtained
through interpolation of the adjacent specified velocity functions.  The
times associated with the velocity function may be relative to the water
bottom rather than time zero.

\subsection{Spatial Variation of Velocity Functions}
RPs that do not have a velocity function defined by the user are
assigned a velocity function generated by one of two spatial variation
schemes (parameter \texttt{VINTPL}).  See also Chapter~\ref{c_spatial_variation}.

\subsubsection{ISO-Velocity spatial variation (\texttt{VINTPL 1})}
The program calculates the time associated with each velocity when the
velocities are the same. \textit{e.g.}
\begin{verbatim}
NMO
   FNO 1 VTP 1500 0. 2000 .4 2500 1.0 END
   FNO 3 VTP 1500 0. 2200 .6 2400 1.1 END
END
\end{verbatim}
then \gls{rp} 2 will get \texttt{VTP 1500 0. 2000 .414 2450 1.050}

\subsubsection{Layer cake spatial variation (\texttt{VINTPL 2})}
The program interpolates both the velocity and the time of each
velocity-time pair.  Every velocity function must have the same number
of pairs.  \textit{e.g.}

\begin{verbatim}
NMO
   FNO 1 VTP 1500 .2 1600 .4 2000 1.0 END
   FNO 3 VTP 1450 0. 1800 .6 2100 1.1 END
END
\end{verbatim}
then \gls{rp} 2 will get \texttt{VTP 1475.0 0.100 1700.0 0.500 2050.0 1.050}

The velocity function (\texttt{VTP}) applies to all \glspl{shot}/\glspl{rp} with \texttt{FNO} to \texttt{LNO} of
the parameter list (recall that each \texttt{FNO}/\texttt{LNO} list is terminated with the
word \texttt{END}).  Spatial variation is accomplished by giving multiple
\texttt{FNO}/\texttt{LNO} lists, each with a unique \texttt{VTP} and terminated with \texttt{END}.  The
velocity function is constant for all \glspl{shot}/\glspl{rp} from \texttt{FNO} to \texttt{LNO}.

\textbf{Note: Spatial variation of velocities with velocity inversions has problems!!!!}

Each parameter list must be terminated with the word \texttt{END}.  The entire
set of \texttt{NMO} parameters must be terminated by the word \texttt{END}.

     Process \texttt{NMO2} is identical to process \texttt{NMO}, enabling \texttt{NMO} to be given
twice in a job.  \textit{e.g.}  \texttt{PROCS DISKIN NMO NMO2 DISKOA END}:'a

\subsection{Parameter Dictionary}

\begin{description}
\item[\texttt{VTP}] Velocity-time-pairs.  A list of stacking velocity and two-way
         travel time (in seconds) pairs.  \texttt{VTP} must be given in each \texttt{NMO}
         parameter list.  A maximum of 25 pairs may be given.  Data
         times before the first given time in \texttt{VTP} are held constant from
         the first given time.  Likewise, data times exceeding the last
         given time in \texttt{VTP} are held constant from the last given \texttt{VTP}
         time.  Velocities and times with \texttt{VTP} MUST INCREASE.
         Default=none.  \textit{e.g.} \texttt{VTP 1490 1.0 2000 2.0}

\item[\texttt{IVTP}] Interval velocity-time-pairs.  A list of interval velocity
         and two-way travel (thickness) time pairs.  Spatial
         interpolation is done before \texttt{ADDWB} is applied (before the
         interval velocities are converted to rms velocities).  When
         using \texttt{ADDWB} YES, the first interval should have 0 thickness and
         a velocity of water.  Each \texttt{IVTP} function must have the same
         number of intervals described.
         Default = none   \textit{e.g.}    \texttt{IVTP  1500 0 1600 .1 1700 .05}

\item[\texttt{FNO}] The first \gls{shot} (or \gls{rp}) to apply the velocities to.  Shot (\gls{rp})
         numbers must increase monotonically.
         \Gls{preset}=1

\item[\texttt{LNO}] The last \gls{shot} (\gls{rp}) number to apply the velocities to.  \texttt{LNO} must
         be larger than \texttt{FNO} in each list and must increase list to list.
         Default=\texttt{FNO}

\item[\texttt{VMUL}] Velocity multiplier.  Every velocity is modified so that:
         \texttt{vel = (vel - VADD) * VMUL + VADD}
         \Gls{preset} = 1.    \textit{e.g.}  \texttt{VMUL .9}

\item[\texttt{VADD}] Velocity additive.  Every velocity is modified so that:
         \texttt{vel = (vel - VADD) * VMUL + VADD}
         \Gls{preset} = 0.    \textit{e.g.}  \texttt{VADD -200}

\item[\texttt{DSTRETCH}] The maximum stretch allowed as a percentage of the ratio
         of the \texttt{NMO} over the zero offset time $t0$.  \textit{i.e.} $(tx-t0)/t0$.
         Move-out of long ranges or shallow data change the frequency
         of the waveform and will cause interference when stacked with
         short range traces.  $(tx-t0)/t0$ approximates the ratio $df/f$
         where $f$ is frequency and $df$ is the change in frequency due
         to \texttt{NMO}.
         \Gls{preset} = not given    \textit{e.g.} \texttt{DSTRETCH 50}

\item[\texttt{STRETC}] The maximum amount of \texttt{NMO} (delta t), in seconds.  Data
         exceeding stretch will be muted.  Valid with \texttt{NMO} only (\texttt{TYPE 1}).
         \Gls{preset} =1.

\item[\texttt{ADDWB}] A yes/no switch.  \texttt{YES} indicates that the two-way water bottom
         time from SEG-Y trace header floating point word 50 will
         be added to each time of the velocity function, after spatial
         variation has been done.
         \Gls{preset}=\texttt{NO}

\item[\texttt{TYPE}] The type of travel time correction to apply.  \Gls{preset} = 1     \textit{e.g.} \texttt{TYPE 2}
\begin{description}
\item[\texttt{1}]  Normal MoveOut or \texttt{NMO}: \texttt{T(0) = SQRT(T(X)**2 - X**2/V(T)**2)}
\item[\texttt{2}]  MoveIn or de\texttt{NMO}: \texttt{T(X) = SQRT(T(0)**2 + X**2/V(T)**2)}
\item[\texttt{3}]  Slant MoveOut or \texttt{SMO}:  \texttt{T(0) = T(X) + X/V}
\item[\texttt{4}]  \texttt{XMO} or Normal MoveOut followed by MoveIn.  Requires parameter \texttt{NEWX}.
\end{description}

\item[\texttt{VINTPL}] The type of velocity interpolation between successive velocity
         control points.
         \Gls{preset} = 1 when \texttt{VTP} is given, = 4 when \texttt{IVTP} is given.
\begin{description}
\item[\texttt{1}] Velocity spatial interpolation according to ``iso-velocity''
\item[\texttt{2}] Allows regions of constant velocity to be interpolated correctly, however \texttt{VTP} pairs must be equal across control points.
\item[\texttt{3}] ``Iso-velocity'' spatial variation except on the first velocity time pair, which is interpolated in both velocity and time, making the first velocity time pair (water bottom reflector usually) spatially smooth.
\item[\texttt{4}] Both the velocities and times are interpolated.  Each \texttt{VTP} function must contain the same number of points.
\end{description}

\item[\texttt{VTRKWB}] Velocity Tracking Waterbottom.  When given a positive value,
         velocity interpolation is cued from the water depth at the source
         (long SEG-Y header word 16). This is typically the centerbeam
         depth value which is placed in the header during acquistion
         for each individual \gls{shot}.

         The value of \texttt{VTRKWB} is the maximum depth change between adjacent
         cmp \glspl{gather}.  If the change in water depth exceeds \texttt{VTRKWB}, the
         new water depth is ``bad'' and the previous depth is used instead.

         Although each trace in a \gls{gather} may have a different centerbeam
         depth values, the depth value from first trace will be applied
         to the entire \gls{gather} to minimize static shift problems (process
         gather normally sorts so that the trace with the smallest range
         is first).

         This option is useful during realtime shipboard processing and allows
         a pre-defined 2-D velocity field to be used during processing,
         with fno and lno values representing water depth, vtp pairs
         defining the stacking velocity function for a given water depth.
         This scheme allows more flexibility for varying velocities along
         a reflection line (as opposed to hanging a single velocity
         function from seafloor).
         \Gls{preset} = -9999.0      \textit{e.g.} \texttt{VTRKWB 100.0}

\item[\texttt{OPATH}] The pathname of an output file containing the velocity used in
         move out after spatial and temporal interpolation have been
         done.  Each seismic sample has a velocity associated with it
         and is written to \texttt{OPATH}.  The velocities are written in \gls{ascii},
         one velocity per line, unless the last four characters of the
         pathname end with \texttt{`.segy'}.  When an SEG-Y file is written, the
         SEG-Y headers are the same as the input and the trace length
         is the same; the seismic samples are replaced with velocities.
         \Gls{preset} = none  \textit{e.g.}  \texttt{OPATH velocities.1234}

\item[\texttt{NEWX}] The constant range that all traces will be MovedIn when using
         \texttt{TYPE} 4.  The use of \texttt{NEWX} automatically changes parameter \texttt{TYPE}
         to be 4.  \texttt{NEWX} causes every trace to be MovedOut so that the
         trace is time corrected as if it was at normal incidence (x=0),
         and the MoveIn is applied so that the trace is time corrected
         as if it was at range \texttt{NEWX} (x=newx).  This could also be
         accomplished using processes \texttt{NMO1 HEADER NMO2}, where \texttt{NMO1} did
         \texttt{NMO}, \texttt{HEADER} changed the range, and \texttt{NMO2} did MoveIn to range
         \texttt{NEWX}.
         \Gls{preset} = none.   \textit{e.g.}   \texttt{NEWX 1000}

\item[\texttt{XFACTOR}] A ranger multiplier.  The SEG-Y specification for range is a
        thirty two bit integer.  If the range isi to be expressed
        with greater precision, such as centimeters, then the range
        can be multiplied by a factor of 100 in process geom and then
        removed in \texttt{NMO} by using xfactor .01.
        \Gls{preset} = 1.

\item[\texttt{HIRES}] A yes/no switch when set to \texttt{YES} uses a polinomial interpolation
        of trace values to obtain ``high resolution''.  Normally, nmo uses
        the closest sample of the TX trace.  This is like resampling the
        data before \texttt{NMO}.  Shallow data recorded at too coarse a sample
        interval will have repeated amplitudes that result is a "square
        wave" appearance.
        \Gls{preset} = \texttt{NO}             \textit{e.g.}  \texttt{HIRES YES}

\item[\texttt{END}] Terminates each parameter list.
\end{description}

\section{PLOT: Record Section Plot}
\label{cmd_plot}

Process plot creates a seismic section plot.  Each SIOSEIS computer
installation has a different plotter, so the actual mechanics of getting
the plot to the plotter is different.  Generally though, the entire sec-
tion is generated and stored in a disc file and another program must be
run in order to send the disc file to the plotter.  ***  Note  *** files
generated for one plotter can not be sent to a different plotter.

The HP DesignJet plotters must be converted from SIOSEIS format to
HP-RTL format using program SIO2HP.

Sun rasterfiles created by parameter SRPATH may be viewed by
program XLOADIMAGE or XV (XVIEW).  A common xloadimage statement
in a sioseis shell script is:
\begin{verbatim}
xloadimage -r 90 sunfil.ras &
\end{verbatim}

Sun program \texttt{ras2ps} may be used to convert a Sun RasterFile to a
PostScript file.

ImageMagick programs (\textit{e.g.} ``convert'' and ``display'') also recognize Sun
rasterfiles.

Mac OSX program GraphicConverter is another excellent program for
manipulating Sun rasterfiles.  GraphicConverter may be executed from
a shell with the following line:
open -a /Applications/GraphicConverter\ US/GraphicConverter.app sunfil.ras

Adobe PhotoShop and Illustrator read Sun rasterfiles.

Program SUNTOPS may be used to convert Sun rasterfiles to PostScript,
though it is very primitive.

SIOSEIS only supports black and white Sun rasterfiles.

Program SIOPLT is an X11 program to view all SIOSEIS rasterfiles,

Process PLOT is an offline process.  The data passed to the next process
is the same as input to plot.  (Process PLOT is similar to a T joint in
the pipeline).

\subsection{Parameter Dictionary}

\begin{verbatim}
STIME  - The start time of the data to plot. This is the actual record
         time and must include any delays included in the data, \textit{e.g.} if
         the data has a delay of 3 seconds and the first data to be
         plotted is at 3 seconds, an STIME of 3 will plot the data as it
         is on tape.  If STIME is not given, the data will always be
         plotted with a start time of the deep water delay (a gap in the
         plot will occur so that the delay changes are obvious).
         \Gls{preset}=delay

NSECS  - The number of seconds of data to plot.  The plot will contain
SECS     data times stime to stime+nsecs.
       = 0.  The plot will be the length of the first trtace.
         \Gls{preset} = Length of the first trace.

VSCALE - The vertical scale in inches per second.  SIOSEIS will plot the
         data at ``any'' vertical scale, but has ``preferred'' scales that
         for each plotter.  Using the preferred scale so that every
         seismic sample is exactly on a plotter point eliminates the
         need for sample interpolation.  The preferred scale is
         calculated as: vscale = sr / dpi, where sr is the sample rate,
         dpi is the number of dots per inch of the plotter.  \textit{e.g.} The
         HP DeisgnJet has 300 dpi; if the sample rate is 250 samples per
         second, then the preferred vscale is a multiple of 250/300 (.8333).
         Various preferred scales for sample rates of 125 and 128 are:
         nibs = 60, vscale 4.17
         nibs = 75, vscale 3.333
         nibs = 80, vscale 3.125
         nibs = 100, vscale 1.25 or 1.5625
         nibs = 120, vscale 1.041667 or 1.06667
         nibs = 160, vscale 1.5625, 3.125, 6.25, or 12.5
         nibs = 200, vscale .5, .64, or .625
         nibs = 201, vscale 1.25 or 1.5625
         nibs = 300, vscale .625 or .64
         nibs = 850, vscale 1.23 (~250./203.)
         nibs = 2124, vscale 2.5
         nibs = 2144, vscale 2.5
         nibs = 2368, vscale 2.46
         nibs = 2847, 2848, 2858, 2859 vscale .4166 or .42666 (128/300)
         nibs = 3436, vscale 5.
         nibs = 3444, vscale .5, .64, or .625
         nibs = 4160, vscale 1.5625
         nibs = 5732, vscale 1.25 or 1.5625
         nibs = 5845, vscale 1.25 or 1.5625
         nibs = 7222, vscale .625 or 1.5625
         nibs = 7224, vscale 1.25 or 1.5625
         nibs = 7225, vscale 1.25 or 1.5625
         nibs = 7424, vscale .64, .625 or 1.5625
         nibs = 7425, vscale 1.25 or 1.5625
         nibs = 7436, vscale 1.25 or 1.5625
         nibs = 7444, vscale .5, .625, .64 or 1.5625
         nibs = 8122, vscale 2.5 or 2.56
         nibs = 8222, vscale .625 or 1.5625
         nibs = 8224, vscale 1.25 or 1.5625
         nibs = 8242, vscale .625 or 1.5625
         nibs = 8424, vscale 1.25 or 1.5625
         nibs = 8625, vscale 1.25 or 1.5625
         nibs = 8936, vscale .5, .625, .64 or 1.5625
         nibs = 9242, vscale .625 or 1.5625
         nibs = 9315, vscale 1.23 (250./203.)
         nibs = 9800, vscale 2.46 (500./203.)

TRPIN  - The number of traces per inch or the horizontal scale.  Large
         values of trpin may not result in the exact scale requested due
         to the spacing of the rasters (nibs or dots).  The 160 nib
         plotter handles trpin 1-17, 20, 22, 26, 32, 40, 53, 80, 160
         exactly.  The 200 nib plotters handle trpin 1-16, 18, 20, 22,
         25, 28, 33, 40, 50, 66, 100, 200 exactly.  Other values of
         TRPIN will be rounded down to the next lower exact value of
         TRPIN for the appropriate plotter.
         \Gls{preset}=20.     \textit{e.g.} trpin 10

SCALAR - The constant multiplier for all traces used to convert trace
         amplitudes to inches of plot.  A positive SCALAR value
         indicates no other determination of plot size will be used
         (\textit{i.e.}  the parameter DEF is ignored).  SCALAR should be used
         when trace to trace amplitude relationships are important.
       <0,  SCALAR is determined from the first non-zero trace by
            finding the largest amplitude and setting it to be DEF
            inches big.  Subsequent traces will be scaled by the scalar
            determined on the first trace.  The program determined
            scalar value is printed and may be used in subsequent plots
            via the SCALAR parameter.
       =0,  A new scalar will be determined on each trace by setting the
            largest trace amplitude of each trace to be DEF inches big.
            This destroys the trace to trace amplitude relationship, but
            every trace will have a portion large enough to be visible.
       >0,  The SCALAR value will be used on all traces to convert trace
            amplitudes to plot inches, regardless of the DEF parameter.
            Incorrect values may cause extremely large or small trace
            excursions in the plot (see CLIP).
            \Gls{preset} = -1    \textit{e.g.}  SCALAR .3e-4

DEF    - The deflection, in inches, of each trace.  The distance between
         the maximum and minimum value of the trace.
         \Gls{preset}=.1 \textit{e.g.} def .08

CLIP   - The maximum size, in inches, of the plotted trace.  The
         deflection is the distance between the peak and the trough.
         The maximum trace deflection is 1.0 inches.
         \Gls{preset} = 2./TRPIN     \textit{e.g.}  clip .08

DIR    - The direction of plotting.  Normally the first trace lotted is
         in the right hand side of the section.  In order to plot the
         first trace on the left, dir must be 'ltr'.  The plot heading
         is on the right hand side of the section when plotting from
         rtl.  The plot heading is on the left hand side of the section
         when plotting ltr.
         \Gls{preset}= rtl

OPATH  - The pathname of the output disc plot file.  Program sio2sun
         converts color sioseis plotfiles to color Sun rasterfiles,
         which in turn may be displayed and manipulated by Imagemagick.
         Program sioplt displays sioseis plotfiles on an X11 root window.
         Program so2hp converts sioseis raster files to HP DesignJet
         RTL files (HP-RTL requires HPGL).
         The sio_plotfile_format document describes the sioseis raster file format.
         \textit{e.g.} opath LINE5.RP289
                             sio2sun LINE5.RP289 sunfil
                             ras2ps -C -w8.5 -h11 sunfil psfil
                             lpr -Pking -h psfil
         \Gls{preset} = pltfil, only when no plotfile is specified (\textit{i.e.}
                pltfil when neither SRPATH, SIPATH, nor opath is given.)

SRPATH - The pathname of the Sun raster file.  The file is created only
         when this parameter is given.  Black and white files only.
         Programs SUNTOPS and RAS2PS convert Sun rasterfiles to
         PostScript. The Sun program SCREENLOAD and the X11 programs
         XLOADIMAGE and XV display Sun rasterfiles.
         \Gls{preset} = ' '   \textit{e.g.} srpath sunfil nibs 200 vscale 2.5 nsec 3
                             suntops -w 8.5 -h 11 < sunfil > psfil
                             ras2ps -w8.5 -h11 sunfil psfil
                             lpr -Pking -h psfil

SIPATH - The pathname of the Sun ImageTool file.  Imageplot is available
         from the authors at the Illinois Supercomputer Center.  The
         file format is just a series of real numbers in trace and time
         order.  \textit{i.e.}  trace 1, trace 2, trace 3, \textit{etc.}   The file is
         created only when this parameter is given.
         \Gls{preset} = ' '        \textit{e.g.}  sipath imagefil

HPATH  - The pathname of the SEG-Y trace header file needed by program
         SIOPLT.  Each trace that is plotted also has it's trace header
         written.  The plot trace header may not be the same as the
         data trace.  \textit{e.g.} plot parameters stime changes the time of
         the first point.
         \Gls{preset} = ' '      \textit{e.g.}  hpath line1_headers

ANN    - The type of annotation.  ANN controls the annotation at the
ANN2     top of the plot while ANN2 controls a second annotation at
         the bottom of the plot.  Both annotations occur on traces that
         are tagged ( see parameters FTAG and TAGINC ).  The annotation
         contains up to 8 integer characters.
       = FANNO,  The annotation comes from plot parameter FANNO.
       = SHOTNO, The \gls{shot} number is used.
       = SHOTTR, The \gls{shot} trace number is used.
       = SH&TR,  The \gls{shot} number and the \gls{shot} trace number
       = RPNO,   The \gls{rp} number is used.
       = RPTR,   The \gls{rp} trace number is used.
       = RP&TR,  The \gls{rp} number and the \gls{rp} trace number
       = GMT,    The GMT as DDD HHMM associated with the trace is used.
       = GMTSEC, The GMT as HHMM SS associated with the trace is used.
       = GMTINT, Traces are annotated at GMT intervals.  The interval is
                 given in plot parameter ANNINC.  Plot parameters FTAG
                 and TAGINC are ignored.
       = RANGE,  The annotation is the shot-receiver distance (header
                 word 10).  When used in conjunction with HSCALE the
                 annotated range is adjusted for roundoff.  The
                 annotated range is exact for placing on the plot and is
                 not the exact range for the trace.
       = GMTRP,  The first four characters are the GMT and the second
                 four characters are the \gls{rp} number.
       = ESPN,   The energy source point number.
       = LAT,    The latitude as +/-DDD MM
       = LONG,   The longitude as +/-DDD MM
       = LAT100, The latitude as MM SS SS  (minutes, seconds, hundreths)
       = LONG100, The longitude as MM SS SS  (minutes, seconds, hundreths)
       = WBDEPTH@S, The water bottom depth at the source (segy long word 16)
       = WBTIME, The water bottom time in seconds (sioseis' segy real waord 50)
       = FOLD,   The number of traces used to stack the trace. (segy short word 17)
       = NONE,   No annotation is done.
       = HEADER, The annotation value comes from the SEG-Y trace header
                 word indicated by plot parameters HDR, LHDR, or IHDR.
         \Gls{preset} = GMT

FTAG   - The first output trace number to tag.  A tag is a extension of
         the data before and after the data.  Ignored with ANN GMTINT.
         \Gls{preset}=1

TAGINC - The trace increment between traces to tag (skip cycle).   A
         TAGINC of 0 prevents any tagging or annotation being done.
         Ignored with ANN GMTINT.
         \Gls{preset}=10

HDR    - Indicates the index of the floating point SEG-Y header word to
         use when annotating by SEG-Y header values.
         Used with ANN HEADER.

LHDR   - Indicates the index of the 32 bit integer SEG-Y header word to
         use when annotating by SEG-Y header values.
         Used with ANN HEADER.

IHDR   - Indicates the index of the 16 bit integer SEG-Y header word to
         use when annotating by SEG-Y header values.
         Used with ANN HEADER.

FANNO  - The annotation to precede the first tagged trace (ftag).   The
         annotation must be a number.  Each successive tagged trace is
         incremented by ANNINC. FANNO is used only with ANN FANNO.
         \Gls{preset}=1

ANNINC - The increment between successive annotation values.  Used when
         ANN FANNO or GMTINT is used.
         \Gls{preset}=1 except for ann gmtint, which has ANNINC 5.
         \textit{e.g.} anntyp 100

FSPACE - The first output trace number after which a space will occur.
         A space is def inches wide.
         \Gls{preset}=0  \textit{e.g.}  fspace 24  (useful for 24 trace \glspl{shot} or \glspl{rp})

NSPACE - The number of consecutive spaces to leave for each space, thus
         the spacing is nspace*def inches wide.
         \Gls{preset}=3  \textit{e.g.} nspace 2

SPACEI - The trace increment between spaces.  The number of traces to
         plot between spaces.
         \Gls{preset}=0  \textit{e.g.} spacei 24 (useful for 24 trace \glspl{shot} or \glspl{gather})

RECSP  - A YES/NO switch to indicate an automatic record space before
         every trace number 1.  When set YES, NSPACE traces are
         inserted after the shot/cmp record.  Frequently CMP \glspl{gather}
         do no contain the same number of traces, make parameter
         SPACEI useless.
         \Gls{preset} = NO       \textit{e.g.}  recsp yes

PLOTTER - The plotter model.  The SIOSEIS plot file is a raster file
         or bit map.  Each raster or bit or dot is also called a nib.
         The dot density (or raster spacing) varies among plotter
         manufacturers and plotter models.
         \Gls{preset} = 2859
NIBS   - The density of the plotter being used.  The number of nibs
         (dots) per inch.  Parameter NIBS is the same as PLOTTER and
         remains valid for historical compatibility purposes.
       = 60,   The Printronix 300.
       = 75,   A color or monochrome crt display.  Allows 40 inch plot.
       = 80,   The c.Itoh printer.
       = 120,  The Epson ``rugged writer 480''
       = 160,  the 18 inch Versatec (model 2160).
       = 200,  The 11 inch Versatec (model 1200).
       = 100,  The Printronix mvp printer.
       = 201,  The Versatec model ??  (24 inch, 200 dpi)
       = 300,  The NovaJet, 300 dpi, 36inch, Color Inkjet
       = 624,  The OYO GS-624 (23.68in., 400 dots per in.)
       = 624,  The Isys V-24 (23.68in., 400 dots per in., B&W)
       = 850,  The Raytheon TDU 850 (8.5 inches, 203 dpi, greyscale)
       = 2124, The 24 inch HP Z2100 (600dpi)
       = 2144, The 44 inch HP Z2100 (600dpi)
       = 2368, The iSys V12 (11.654 inches, 203 dpi, B&W)
       = 2847, The HP 2847A DesignJet 600 (300dpi, 24in., B&W )
       = 2848, The HP 2848A DesignJet 600 (300dpi, 36in., B&W )
       = 2858, The HP 2858 DesignJet 650c (300dpi, 24in., color )
       = 2859, The HP 2859 DesignJet 650c, 750, 755CM (300dpi, 36in., color )
       = 3436, The Versatec 3436 (400 dpi, 34 in., color)
       = 3444, The Versatec 3444 (color, 400 dpi, 43.04 in.)
       = 4160, The Printronix 4160
       = 5732, The Calcomp model 5732, 200 dots per in, 24 in wide)
       = 5845, The Calcomp model 5845, 400 dots per in, 44 in wide)
       = 7222, The Versatec model 7222 (21.12in., 200 dots per in.)
       = 7224, The Versatec model 7224 (23.04in., 200 dots per in.)
       = 7224, The Atlantek (23.04in., 200 dots per in.)
       = 7225, The Versatec model 7225 (23.52in., 200 dots per in.)
       = 7422, The Versatec model 7422 (21.12in., 400 dots per in.)
       = 7424, The Versatec model 7424 (23.04in., 400 dots per in.)
       = 7425, The Versatec model 7425 (23.52in., 400 dots per in.)
       = 7436, The Versatec model 7436 (35.20in., 400 dots per in.)
       = 7444, The Versatec model 7444 (43.04in., 400 dots per in.)
       = 7600, The HP-7600 electrostatic (35.5 in., 406 dots per in.)
       = 8122, The Versatec model 8122 (21.12in., 100 dots per in.)
       = 8222, The Versatec model 8222 (21.12in., 200 dots per in.)
       = 8224, The Versatec model 8224 (23.04in., 200 dots per in.)
       = 8242, The Versatec model 8242 ( B&W 40.96in., 200 dots per in.)
       = 8936, The Versatec model 8936 (Color 34.2in., 400 dpi)
       = 9242, The Versatec model 9242 (Color 40.0in., 200 dots per in.)
       = 9315, The Alden 9315 (8 bit Grey scale, 203 dpi, 10.08 in.)
       = 9800, The EPC 9800 (6 bit Grey scale, 203 dpi, 20.18 in.)
        \Gls{preset} = required

WIGGLE - Wiggle trace plot signal.  The wiggle trace is the graph of the
         seismic amplitudes as if a pen was dragged across the paper.
         The value of wiggle is a percentage of the deflection.
       <0., Trace amplitudes more negative than wiggle/100.*def will not
            be plotted.
       =0., Only the shaded portion of the trace is plotted.
       >0., Trace amplitudes larger than wiggle/100.*def will not be
            plotted.
         \Gls{preset} 100.    \textit{e.g.} wiggle -75.

PCTFIL - The percentage of the maximum deflection (DEF) that gets shaded.
       >0,  Positive values get shaded.
       =0,  No shading - wiggle trace plot only.
       <0,  Negative values get shaded.
          \Gls{preset}=100.    \textit{e.g.} pctfil 0

TLINES - Timing lines.  A set of up to 4 times to receive a marking line
         on the section.  Each successive given timing line will be
         darker and wider, \textit{e.g.} tlines .1 .5 1. results in a section
         with every .1 seconds marked with a solid line 1 dot wide,
         every .5 seconds marked with a solid line 2 dots wide, and
         every 1.0 seconds marked with a solid line 3 dots wide.  Timing
         lines may be omitted totally by giving tlines 0.0.
         \Gls{preset} = .1 .5 1.   \textit{e.g.} tlines .5 1.

TLANN  - Timing line annotation switch.  Process plot normally annotates
         the heavist time line.  This annotation may be suppressed by
         setting TLANN to 0.
         \Gls{preset} = 1     \textit{e.g.} tlann 0

COLORS - BLACK, RED, GREEN, YELLOW, BLUE, MAGENTA, CYAN, WHITE
         A list of deflection - color pairs to use on color plots.  A
         maximum of 8 colors are permitted.  Each amplitude receives a
         color based on it's size relative to the DEF (deflection)
         parameter.  Amplitudes exceeding the deflection in COLORS
         receive the associated color. \textit{e.g.}  colors 0 red means that all
         positive amplitudes will be red.  Negative amplitudes may be
         colorized by giving a negative deflection.
         \textit{e.g.} colors -.000001 blue 0 red
         indicates that negative amplitudes exceeding -.000001 will be
         blue and all positive amplitudes will be red.  In order to use
         three or more colors, the intermediate interval colors must be
         defined, too.  \textit{e.g.} colors -.1 blue 0 green .1 red
         means that amplitudes between -.1 and +.1 will be green.  The
         eight permissible colors are:
         BLACK, RED, GREEN, YELLOW, BLUE, MAGENTA, CYAN, WHITE
         Available on SIOSEIS rasterfiles only (OPATH only).  Sun
         rasterfiles may be made from SIOSEIS rasterfiles through
         program SIO2SUN.  (Sun rasterfiles may be viewed on the screen
         using program xloadimage or xv).
         Additional examples are located on the SIOSEIS web pages under
         examples.

COLORS = GRAY or GRAY0, GRAY1, GRAY2, GRAY3, GRAY4, GRAY5, GRAY6, GRAY7
         A list of deflection - grayscale pairs to use on grayscale plots.
         Available on SIOSEIS rasterfiles only (OPATH only).  Sun
         rasterfiles may be made from SIOSEIS rasterfiles through
         program SIO2SUN.  (Sun rasterfiles may be viewed on the screen
         using program xloadimage or xv).
         \textit{e.g.} colors  .001 gray3 .002 gray4 .003 gray5 .004 gray6 .005 gray7

DPTR   - Dots Per TRace.  When using color or grayscale plots, each
         plotter point may be use to express the trace amplitude
         rather than width of the trace.  DPTR specifies width, in
         plotter points, of all the trace samples.  This can be
         thought of as the pixel width.  The pixel height is controlled
         by the vertical scale.  DTPR = NIBS / TRPIN will fill all
         the dots between traces.  (see the sioseis examples also).
         \textit{e.g.}  nibs 2859 trpin 300 dptr 1  (the 2859 plotter is 300 dpi)

BCOLOR - The background color.  The color of the plot which is not
         ``in the seismic trace''.  The eight permissible colors are:
         BLACK, RED, GREEN, YELLOW, BLUE, MAGENTA, CYAN, WHITE
         \Gls{preset} = WHITE.    \textit{e.g.}    bcolor blue

DECIMF - Plot decimation factor.  If the data to be plotted needs to be
         decimated, set decimf to the increment between samples to
         retain.  For example if every other sample is to be plotted,
         decimf is 2.  This is particularly useful when plotting the
         real components of a complex valued trace (\textit{e.g.} F-K domain data
         in rectangular coordinates).  The data passed on to the next
         seismic process is not decimated.
         \Gls{preset} = 1     \textit{e.g.} decimf 2

FNO    - The first \gls{shot}/\gls{rp}/gmt the set of plot parameters applies to.
         Shots with \gls{shot} numbers smaller than the first FNO will not be
         plotted.  The set is terminated by the word END. FNO may be a
         \gls{shot} number, a \gls{rp} number, or an hour-minute of the 24 hour
         clock.  When FNO is a \gls{gmt} time, fday must be given.  FNO must
         be given if more the one parameter list is given (the plot
         parameters change).
         \Gls{preset} = 1.    \textit{e.g.} FNO 1230 fday 100

FTR    - The first trace number with FNO the set of parameters applies
         to. Traces with trace numbers smaller than ftr will not be
         plotted.
         \Gls{preset} = 1.  \textit{e.g.} ftr 10

FDAY   - The day-of-year associated with FNO when the plot parameters
         are given according to gmt (i.E. When FNO is a \gls{gmt} time).
         \Gls{preset} = 0     \textit{e.g.} fday 365

LNO    - The last \gls{shot}/\gls{rp}/gmt the set of plot parameters applies to.
         Shots with \gls{shot} numbers larger than the last LNO will not be
         plotted.  LNO is not necessary if all the data to be plotted
         get the same parameters.
         \Gls{preset} = 99999999  \textit{e.g.} LNO 1230

NINC   - The number increment between FNO and LNO.  Shot/\glspl{rp} that are
NOINC    not an increment between FNO and LNO are not plotted.  The only
         \glspl{shot}/\glspl{rp} plotted are fno, fno+ninc, fno+ninc*2, fno+3*ninc, \ldots
         LNO FNO and LNO must be given when using NINC.
         \Gls{preset} = 1.    \textit{e.g.}  ninc 6

LTR    - The last trace number within the \gls{shot}/\gls{rp}/gmt the set of
         parameters applies to.  Traces with trace number larger than
         LTR will not be plotted.
         \Gls{preset} = 99999  \textit{e.g.} ltr 23

LDAY   - The day-of-year associated with LNO when LNO is an hour-minute.
         \Gls{preset} = 0  \textit{e.g.} lday 366

WRAP   - When YES, wrap the trace around the end of line as many times
         as needed until the end of trace is plotted.  This is
         especially useful when plotting PASSCAL or passive source data.
         \Gls{preset} = NO        \textit{e.g.}   wrap  yes

RECTIFY - When ON or YES, the data are rectfied before plotting.
         Rectifying is taking the absolute value of each sample.
         Most echo sounders are rectified before plotting.
         \Gls{preset} = OFF,       \textit{e.g.}  rectify on

CHART  - Indicates that the plot should mimic a strip chart recorder
         similar to a continuous copy of a screen display.  The strip
         chart is NSECS wide.  STIME should not be used with CHART since
         CHART will determine, and adjust, STIME automatically.
         The water bottom is kept in a window within the plot, thus the
         plot is shifted up or down depending on the water depth.  CHART
         requires the water bottom time to be in the SEG-Y trace header
         (process wbt parameter VEL converts water depth to time).
         CHART requires two values to be given; the top and bottom of
         the window, as a percentage of NSECS.  \textit{e.g.}  chart 5 50
         will keep the water bottom in the top half of the plot.

TRIM   - Trims or crops the top and/or sides of the plot so that all the
         non-seismic white space is removed.   Do not use with DIR LTR.
       = ALL, the top and side margins are removed.
       = LEFT, RIGHT, TOP;  Removes the specified margin.
         \Gls{preset} = not given.    \textit{e.g.}   trim all

REFRACTION PLOT PARAMETERS
---------- ---- ----------

HSCALE - The horizontal scale when plotting refraction data (non-equal
         trace spacing).  The traces will be plotted according to the
         distance from the first trace, as determined from the SEG-Y
         trace header.  Ranges must be monotonically increasing,
         \textit{e.g.} -1000, -900.  Each trace is plotted
         IABS((range trace n - range trace 1) / hscale * nibs
         from the first trace plotted, where nibs is the dots-per-inch
         if the plotters selected via parameter nibs.  SEG-Y ranges are
         meters or feet, so hscale is in meters or feet per in. HSCALE
         must be positive.
         \Gls{preset} = none. \textit{e.g.} hscale 250

FRANGE - First range plotted when plotting by range.
         \Gls{preset} = the range of the first trace.

LRANGE - Last range plotted when plotting by range.  LRANGE must be
         given when FRANGE is given in order that PLOT knows whether the
         ranges are increasing or decreasing.  Besides excluding traces
         outside the FRANGE-LRANGE interval, ``null'' traces or spaces can
         be created.  If the first or last range is in the interval, but
         not on the interval boundary, spaces are created so that the
         plot extends to the interval boundary.
         \Gls{preset} = the range of the last trace.

RSTIME - Relative start time.  The time of the first data to be plotted
         relative to the delay contained in the trace header.  This
         parameter is necessary when process shift parameter reduce was
         used.  Normally process plot puts a gap in the plot whenever a
         delay change occurs, but the reduce parameter causes each trace
         to have a different delay.  RSTIME and STIME are mutually
         exclusive.
         Preset = none  e.g rstime 0.

Copyright (C), Paul Henkart, Scripps Institution of Oceanography
ALL RIGHTS RESERVED

HIDDEN AND OLD PARAMETERS:

LPRINT - debug parameter
       = 1, Prints the parameters in the edit stage (ploted).
         \Gls{preset} = 0

ABSVAL - When set non-zero and plotting using HSCALE, ABSVAL indicates
         that a special case exists whereby the ranges are the magnitude
         or absolute value of the range.  When this condition exists,
         assume that the ranges are monotonically decreasing so that any
         range that increases means that it is negative.
         \Gls{preset} = 0

ANNTYP - The type of annotation to be done. A maximum of eight
         characters are printed before the tag.
       = 1, The annotation is taken from the user specified FANNO.
       = 2, The annotation is the \gls{shot} number of the trace plotted.
       = 3, The annotation is the \gls{rp} number of the trace plotted.
       = 4, The annotation is the \gls{gmt} time in hours, minutes and seconds
            associated with the trace.
       = 5, The annotation is the \gls{gmt} time. Traces are annotated at even
            intervals of ANNINC minutes.  FTAG and TAGINC are ignored.
       = 6, The annotation is the shot-receiver distance (header word
            10).  When used in conjunction with hscale, the annotated
            range is adjusted for roundoff.  \textit{i.e.}  The annotated range
            is exact for placing on the plot and is not the exact range
            for the trace.
       = 7, The annotation is the trace number of the \gls{shot}/\gls{rp}.
       = 8, The annotation is the energy source point number (header(5))
       = 9, The first four characters are the GMT and the second four
            characters are the \gls{rp} number.
       = 10, No annotation is done.
         \Gls{preset}=5  \textit{e.g.} anntyp 2

ICDOTS - A switch for 'connecting the dots'.  High density dot plotters
         have more dots than the seismic data.  Normally process plot
         will connect the dots between seismic samples by interpolating
         between seismic samples.  This may cause the unfilled portion
         of the trace to look too dark.  Setting ICDOTS to zero prevents
         the interpolation or the ``connecting the dots''.
         \Gls{preset} = 1.    \textit{e.g.} icdots 0

BIAS   - The percentage of the maximum deflection (def) to add to all
         samples before plotting (a dc bias).
         \Gls{preset}=0. \textit{e.g.} bias 15.

PLOTSI - The override sample interval for plotting the data.  PLOTSI
         will be used rather than the sample interval contained in the
         data.  The use of PLOTSI only affects process PLOT.  Data in
         the FK and \gls{not:tau}-p domains are in different units and are not
         ``nice'' (.001, .002, .004) for process plot.
         \Gls{preset} = 0.    \textit{e.g.} plotsi .001

C      - Comment card images to replace in the plot header.  The comment
         card must start with the letter c and must be followed
         immediately by a 2 digit number, followed by a blank.  The
         number is the card number within the header to replace.  The
         comment itself must be enclosed in single quotes.  The comments
         are for the plot only.  The two digit number must include a
         leading zero if necessary.
         \Gls{preset}=none.  \textit{e.g.} C05 'this is an example of a comment'
\end{verbatim}

\section{PROUT: Printer Dump of Traces}
\label{cmd_prout}

Process PROUT prints various items of SEG-Y file, including up to five
sets of amplitude values or the SEG-Y trace header or the SEG-Y
tape/file headers.  See document segy.header for information about
the SEG-Y format.

Each trace to be printed must be explicitly specified by giving FNO,
FTR, and SETS.

Each parameter list must be terminated with the word END.  The entire
set of PROUT parameters must be terminated by the word END.

Example: In order to print the trace values between times 1.2 and 1.5 on trace 24
and \glspl{shot} 45 and 46, the following could be used:

\begin{verbatim}
PROCESS PROUT
   FNO 45 FTR 24 SETS 1.2 1.5 END
   FNO 46 SETS 1.2 1.5 END
END
\end{verbatim}

\subsection{Parameter Dictionary}

\begin{description}
\item[\texttt{FNO}] The first \gls{shot} (or \gls{rp}) to print.  Shot (\gls{rp}) numbers must
         increase monotonically.
         \Gls{preset}=1

\item[\texttt{LNO}] The last \gls{shot} (\gls{rp}) number to print.  \texttt{LNO} must be larger than
         \texttt{FNO} in each list and must increase list to list.
         Default=\texttt{FNO}

\item[\texttt{NOINC}] The increment between \texttt{FNO} and \texttt{LNO} for \glspl{shot}/\glspl{rp} to print.
         \Gls{preset} = 1

\item[\texttt{SETS}] Start-end time pairs defining the data to be printed.  Times
         are in seconds and may be negative when hanging the windows
         from the water bottom.  A maximum of 5 windows may be given.
         Required.

\item[\texttt{INC}] The increment between data values to print.  This is the
         Fortran do loop increment.  \textit{e.g.}  In order to print every other
         data value, set \texttt{INC} to 2.
         \Gls{preset} = 1.  \textit{e.g.} \texttt{INC 2}

\item[\texttt{FTR}] The first trace within the \gls{shot} (\gls{rp}) to print.
         \Gls{preset}=1

\item[\texttt{LTR}] The last trace within each \gls{shot} (\gls{rp}) to print.  Traces
         exceeding \texttt{LTR} will not be printed.
         \Gls{preset} = \texttt{FTR}

\item[\texttt{TRINC}] The trace increment between \texttt{FTR} and \texttt{LTR}. Also called the trace skip cycle.  \Gls{preset} = 1

\item[\texttt{HEADER}] When set nonzero, the entire SEG-Y trace header is printed.
         Specific header values may be printed by using parameter
         \texttt{INDICES} below.
         \Gls{preset} = 0

\item[\texttt{THEADS}] When set nonzero, the tape headers are printed.
         \Gls{preset} = 0

\item[\texttt{ADDWB}] When given a value of yes, the windows given via \texttt{SETS} will be
         added to the water bottom time of the trace.  (Water bottom
         times may be entered via process \texttt{WBT}).
         \Gls{preset} = no

\item[\texttt{FORMAT}] Indicates how the output should be formatted.
         Preset = ' '      \textit{e.g.}
         \texttt{FORMAT '(6H shot ,F6.0,12H depth (ms) ,F5.0)' INDICES L3 L60}
\begin{description}
     \item[\gls{ascii}] Each trace printed will be in a separate file.  The file
         name will be \texttt{sh} or \texttt{rp} followed by the \gls{shot}/\gls{rp} number followed
         by the trace number followed by \texttt{.txt}.  These files may be used
         with the MATLAB load command.
     \item[( Fortran format statement )]  The specified Fortran format
         will be used when enclosed in parenthesis.  The \texttt{I} format is not
         recommended (it may not work) since the variables are kept
         internally as \texttt{REAL} (\texttt{F10.0} is almost the same as \texttt{I10}).
\end{description}

\item[\texttt{INDICES}] A list of up to 10 indices of the SEG-Y trace header values
          to print rather than the all of the header.  The header
          values are converted to floating point before printing.
          \texttt{INDICES} must be given as \texttt{xn}, where;
\begin{description}
    \item[\texttt{x}] Variable type
\begin{description}
    \item[\texttt{I}] means short integer (16 bit integer trace header)
    \item[\texttt{L}] means long integer (32 bit integer trace header)
    \item[\texttt{R}] means real word (host floating point)
\end{description}
    \item[\texttt{n}] the index with the SEG-Y trace header.
\end{description}
          A maximum of 10 \texttt{xn}s may be given in a parameter list.
          Examples:
          \begin{itemize}
              \item \texttt{L1} means long integer word 1
               \item \texttt{I59} means short word
              \item \texttt{R49}  means real word 49
          \end{itemize}

\item[\texttt{INFO}] Print to STDOUT various information or statistics.
\begin{verbatim}
      = 1,  The file name, first and last \gls{shot} times and fix are
        display as well as the smallest data start time and the
        largest etime.  \textit{e.g.}
env-0005_1211_LF.sgy      Begins: day265 12:11:45, lat:   84 21 20.209 long:   42 55 56.364
env-0005_1211_LF.sgy        Ends: day266 07:39:46, lat:   82 35 44.586 long:   42 58  8.255 data times: 3.333 to  6.000 secs
        \textit{e.g.}   script > file
        grep day file > tmp
        sort +2 -5 tmp > info

      = 2,  Print the sum of the amplitudes within each SET window.  This
        is similar to the energy of the window, but the amplitudes are
        not squared before sumation.
        \textit{e.g.} INFO 2  sets  0 .01 .005 .015 .01 .02 .03 .04
  shot  12  trace  1  window sums:  145.00000   195.00000   245.00000  445.00000

      = 3,  Write the NGDC meta-data for appropriate for Knudsen correlates
        and envelopes.  OPATH is honored
      = 4, Print the energy (sum of the amplitudes squared) of each window.
        \textit{e.g.}   info 4 sets .1 .2 1.9 2.1
      = 5, Print various statistics:
           min - minimum trace amplitude.
           max - maximum trace amplitude.
           smallest - smallest non-zero absolute trace value.
           ave - average amplitude.
           adev - average deviation (mean absolute deviation) of the trace.
           sdev - standard deviation.
           var - variance.
           curt - kurtosis.
\end{verbatim}

\item[\texttt{TRLIST}] An order list of trace header words to print on each trace.
         Up to 10 items may be given.  The mnemonics available are:
\begin{verbatim}
   SHOTNO - Long word 3 - the \gls{shot} number.
   SHOTTR - Long word 4 - the trace number within the \gls{shot}.
   ESPN   - Long word 5 - the energy source point number.
   RPNO   - Long word 6 - the \gls{rp} number.
   RPTR   - Long word 7 - the trace number within the \gls{rp}.
   RANGE  - Long word 10 - the source to receiver distance.
   GMT    - Short words 79, 80, and 81 - the day of year, hour, and minute.
   GMTSEC - Short words 80, 81, and 82 - the hour, minute and second.
   WBDEPTH@S - Long word 16 - the water depth at the source.
   WBDEPTH@R - Long word 17 - the water depth at the receiver.
   SXD    - Long word 19 - the source x coordinate (longitude) expressed in
                           decimal degrees.
   SXDM   - Long word 19 - the source x coordinate (longitude) expressed in
                           integer degrees and decimal minutes.
   SXDMS  - Long word 19 - the source x coordinate (longitude) expressed in
                           integer degrees and minutes, and decimal seconds.
   SYD    - Long word 20 - the source y coordinate (latitude) expressed in
                           decimal degrees.
   SYDM   - Long word 20 - the source y coordinate (latitude) expressed in
                           integer degrees and decimal minutes.
   SYDMS  - Long word 20 - the source y coordinate (latitude) expressed in
                           integer degrees and minutes, and decimal seconds.
   RXD    - Long word 21 - the receiver x coordinate (longitude) expressed in
                           decimal degrees.
   RXDM   - Long word 21 - the receiver x coordinate (longitude) expressed in
                           integer degrees and decimal minutes.
   RXDMS  - Long word 21 - the receiver x coordinate (longitude) expressed in
                           integer degrees and minutes, and decimal seconds.
   RYD    - Long word 22 - the receiver y coordinate (latitude) expressed in
                           decimal degrees.
   RYDM   - Long word 22 - the receiver y coordinate (latitude) expressed in
                           integer degrees and decimal minutes.
   RYDMS  - Long word 22 - the receiver y coordinate (latitude) expressed in
                           integer degrees and minutes, and decimal seconds.
   DELAY  -              - the deep water delay in seconds.
   WBTIME -              - the water bottom time in seconds.
   FOLD   - Short word 17 - The number of traces used to stack the trace.
\end{verbatim}

         \Gls{preset} = none.  \textit{e.g.}  \texttt{TRLIST SHOTNO GMTSEC SYDM SXDM}
          produced the printout:
\begin{verbatim}
          5267 1318z 41s -71 43.647766  173 18.757324
          5268 1318z 53s -71 43.571320  173 19.011841
\end{verbatim}

\item[\texttt{OPATH}] The output pathname (filename) in order to print to a file rather
         than stdout.  Only used by \texttt{INFO 3}.

\item[\texttt{END}] Terminates each parameter list.
\end{description}

\section{PSMIGR: Dan Lizarralde's Phase Shift Depth Migration (Split-Step Migration)}
\label{cmd_psmigr}

%                        Phase Shift Migration

Reference: Stoffa et al., Split-step Fourier Migration,
                          Geophysics,55,p.410-421,1990. \cite{Stoffa1990}

This process performs a ``split-step'' Fourier migration on
stacked seismic data.  The method is designed to provide a
fast f-k approach to migration in laterally varying velocity
media.  The approach is straightforward.  The data are migrated
in small depth increments of $dz$.  For each $dz$ a loop over a
frequency range $f_{min}$ to $f_{max}$ is performed in which the data are:

\begin{enumerate}
\item transformed from f-x to f-k;
\item phase shifted by $exp(i*dz*K_{z})$ where $[K_{z}=csqrt(w^2*u_{0}^2-K_{x}^2)]$ using a reference slowness $u_{0}$;
\item transformed from $f-k$ to $f-x$;
\item phase shifted by $exp(i*w*(u(x,z)-u_{0}))$
\end{enumerate}

Imaging is then done at depth $Z_{j}$ by a sum over frequencies
from $f_{min}$ to $f_{max}$.

\subsection{Notes}

The complex square root calculation of $K_{z}$ results in an exponentially
damped response to inhomogeneous interface waves, a difficulty in
other implementations of this type of phase-shift-plus-correction
method.

Extreme lateral velocity variations, such as seen at some
continental margins, may be handled by breaking the migration
into several overlapping panels which are later spliced together.

\subsubsection{Big step, ZSKIP}
   You may migrate in one big step through an upper constant
   slowness region such as the water column.  No imaging is done
   for this region.  If there is a deep water delay in your data,
   you may want to remove delay to negate temporal wrap-around.
   This is most easily done through start-end-time pair in process
   \texttt{DISKIN} (\textit{i.e.} \texttt{SET 0.0 8.0})

\subsubsection{Migration bandwidth, FMIN, FMAX}
   Runtime increases in direct proportion to migration bandwidth
   and the number of frequencies migrated.  The number of frequencies
   is initially computed from the \gls{fft}.  The larger the \gls{fft}, the larger
   the number of frequencies (a 2048 point \gls{fft} has 2049 frequencies).
   It is worthwhile doing a couple of small tests to see what
   frequencies are actually useful and adjusting the parameters
   FMIN and FMAX accordingly.

\subsubsection{Depth step, DELTAZ}
   The migration depth step $dz$ need not be tiny, but in general small
   enough to provide unaliased sampling of the smallest vertical
   wavenumber of interest.

\subsubsection{Data tapers, BPAD, EPAD}
   The user is given the option of padding either side of the data
   panel being migrated. The padded region is filled with copies of
   the end traces which are tapered down within the pad region. This
   is highly recommended as it reduces Gibbs phenomena and Nyquist
   noise.  It should be kept in mind, however, that a power of 2 \gls{fft}
   is used and that the length of the \gls{fft} is determined as the next
   power of 2 above (data panel length)+BPAD+EPAD.  So if you are
   migrating 1900 traces and you pad with 100 on each side, you will
   be using a 4096 \gls{fft} for each depth step. If you had padded with 50
   on each you would be using a 2048 \gls{fft}, much faster.

\subsubsection{Migration taper, MTAP}
   In addition to the tapered padding of the first and last trace, an
   exponential taper can be applied to the ends of the spatial window
   at each migration step.  This inhibits ``wraparound'' of migration
   smiles from the data panel ends sides.  An exponental taper on the
   order of MTAP=25 traces is recommended.  The example below should
   clarify the issue.  It is important to note that the \gls{fft} length is
   not based on the length of the exponential taper, only on the
   (data panel length)+BPAD+EPAD, so that if care is not taken data
   may be affected by the exponental taper.

\begin{verbatim}
   MTAP=5
   |<5>|                            |<5>|
   eeeee1111111111111111111111111111eeeee  <-*1
   tttttttdddddddddddddddddttttttt0000000
   |< 7 >||               ||< 7 >|      |
   |BPAD=7|               |EPAD=7       |
   |      |               |             |
   |      ||             |
   |                                    |
   |<--- 2^n based on 17+(2*7) = 32 --->| <-*2


   *1) this vector is applied to each frequency at each depth
       step.  In f-k jargon this type of taper is called a sponge.
   *2) if EPAD were given as 9 in this case, then the code would
       use an n=6 (2^n=64) would be used for the \gls{fft}.
\end{verbatim}


Present limitations include 8192 traces, 5000 depth steps and 800
frequencies.  If the number of depths ($n_{z}$) times the number of traces
($n_{x}$) exceeds 500,000 use \texttt{SIOSEIS.BIG}, which allows $n_{z} \times n_{x} = 3,500,000$.
Padding should be including when estimating $n_{x}$.


\subsubsection{Example Migration}
  Shallow Sediments near ODP HOLE 504B
\begin{verbatim}
  DISKIN
      IPATH 504b.dmo.stack.224.624 SET 0.0 8.0 END
  END
  PSMIGR
     DELTAX 12.5 DELTAZ 10 VSKIP 1500 ZSKIP 3000 EZ 10000
     BPAD 50 EPAD 50 MTAP 25 TWINLEN 0.30
     REF 0 NVSMTH 3 VPATH v.scratch
     FMIN 5 FMAX 20 PATH ps.scratch SGYPATH vmodel.segy
     FNO  224 LNO  224 VDP 1500.0 3480.0 1850.0 3490.0 1850.0 3721.3 END
     FNO  424 LNO  424 VDP 1500.0 3442.5 1850.0 3452.5 1850.0 3785.5 END
     FNO  524 LNO  524 VDP 1500.0 3397.5 1850.0 3407.5 1850.0 3694.3 END
     FNO  624 LNO  624 VDP 1500.0 3390.0 1850.0 3400.0 1850.0 3686.8 END
 END
\end{verbatim}

\subsection{Parameter Dictionary}

\begin{description}
\item[\texttt{EZ}] End Depth, in meters.  The number of samples output will be
         \texttt{EDEPTH}  / \texttt{DELTAZ} + 1.  The first output sample is ALWAYS 0.
       - Required.  \textit{e.g.}  \texttt{EZ} 6000
\item[\texttt{DELTAX}] The distance between traces, in meters.
         Required.
\item[\texttt{DELTAZ}] The output sample interval in meters per sample.  \texttt{DELTAZ} may
         not exceed 32 since it is carried in the SEG-Y header in
         millimeters and as a 16 bit integer (thus 32767 millimeters
         is the maximum).  This is analogous to the sample interval
         in time which the SEG-Y format carries as nanoseconds
         (milliseconds / 1000).
         Required.
\item[\texttt{TWINLEN}] Time WINdow LENgth of temporal taper at end of trace. This
          taper is exponential in nature and is given in seconds. The
          time taper should reduce energy migrating from truncation
          of time trace.
          \Gls{preset} = 0.25
\item[\texttt{MTAP}] Migration TAPer.  The exponential spatial taper discussed above.
         preset = 25
\item[\texttt{FMIN}] Minimum frequency of interest.
         \Gls{preset} = 0     \textit{e.g.} \texttt{FMIN 5}
\item[\texttt{FMAX}] Maximum frequency of interest.  Run time may be reduced
         significantly by using \texttt{FMIN}/\texttt{FMAX}.  The Nyquist frequency is
         2/(sample interval) or (sample rate)/2, thus 4 mil data has
         has \texttt{FMAX} preset to 125/2 or 62.5.
       - preset = nyquist / 2  or   (4/si)    \textit{e.g.} \texttt{FMAX 50}
\item[\texttt{VDP}] The interval velocity to use in migration, given as a list of
         Velocity Depth Pairs.  The velocity should be in units of
         meters per second and the depth should be in units of
         meters.
         Required.    velocity range 350 to 32000
NVSMTH - This parameter specifies the number of Velocity SMooTHing
         operations desired before constructing velocity file.
         The \texttt{NVSMTH} parameter is useful for smoothing across velocity
         contrasts which otherwise can cause distortion in depth
         migration.  The running average of \texttt{NVSMTH} velocities is
         calculated after the \texttt{VDP} velocity function is expanded
         into an uniformly sampled function.  \texttt{NVSMTH} applies in
         depth, not spatially.
         \Gls{preset} = 3
\item[\texttt{FNO}] The first \gls{shot}/\gls{rp} number the parameter list applies to.
         \Gls{preset} = the first \gls{shot}/\gls{rp} received.    \textit{e.g.}   \texttt{FNO 101}
\item[\texttt{LNO}] The last \gls{shot}/\gls{rp} number the parameter list applies to.
         \Gls{preset} = the last \gls{shot}/\gls{rp} received.    \textit{e.g.}   \texttt{LNO 101}
\item[\texttt{BPAD}] The number of zero amplitude traces to insert prior to the
         first trace.
         \Gls{preset} = 1   range 1 to 500      \textit{e.g.} \texttt{BPAD 10}
\item[\texttt{EPAD}] The number of zero amplitude traces to append after the last
         trace.
         \Gls{preset} = 1   range 1 to 500      \textit{e.g.} \texttt{EPAD 10}
\item[\texttt{ZSKIP}] The first depth to compute.  The first output sample is
         ALWAYS 0.  This saves computer time!  The first velocity of
         \texttt{VDP} of the first \texttt{FNO}/\texttt{LNO} list is used as the velocity for
         time to depth conversion unless parameter \texttt{VSKIP} is given.
         \Gls{preset} = 0.
\item[\texttt{VSKIP}] Velocity used in extrapolation for parameter \texttt{ZSKIP}, usually
         the water velocity in marine work. If using \texttt{ZSKIP}, then \texttt{VSKIP}
         is required.
     \item[\texttt{REF}] Reference slowness, $u_{0}$, from a given depth.  The minimum
         slowness may provide a better reference than the average
         slowness for imaging features such as a rough basement
         surface beneath low velocity sediments.
\begin{description}
\item[\texttt{0}] minimum slowness.
\item[\texttt{1}] average slowness.
\item[\texttt{2}] maximum slowness.
\end{description}
         \Gls{preset} = 1
\item[\texttt{PATH}] The pathname (filename) of a scratch file \texttt{PSMIGR} should use
         for the intermediate transposed data.  The purpose of this
         parameter is to allow the user to specify the exact disk
         partition to use in case the ``current'' partition does not
         have enough space.
         preset = a scratch file in the current directory
         \textit{e.g.}    \texttt{PATH /user/scratch/moreroom}
\item[\texttt{VPATH}] The pathname (filename) of a file \texttt{PSMIGR} should use for the
         transposed velocity slices.  If the file exists, \texttt{PSMIGR} will
         not calculate a new velocity model; \texttt{VDP} will be ignored.  If
         the file does not exist, \texttt{PSMIGR} will write the velocity
         slices to \texttt{VPATH} so that it may be used in other runs of \texttt{PSMIGR}
         without having to recalculate the velocities.  The velocity
         slices may be quite large and \texttt{VPATH} allows the user to specify
         the exact disk partition to use in case the ``current'' partition
         does not have enough space.
         preset = a scratch file in the current directory
         \textit{e.g.}    \texttt{VPATH /user/scratch/vmoreroom}
\item[\texttt{SGYPATH}] The pathname (filename) of an additional SEG-Y compatible
         velocity file to be output for external purposes. Includes
         the smoothing operators.
         preset = none
         \textit{e.g.}    \texttt{SGYPATH /user/scratch/vmoreroom.segy}
\end{description}

\section{RESAMP: Resample Time Domain Data}
\label{cmd_resamp}

Process RESAMP resamples the seismic trace from one sample interval to
another by doing a polynomial interpolation in the time domain.

This process is useful for converting data that is recorded at "not
nice" sample rates such as data are recorded with 128 samples per
second which is a sample interval of .0078125 seconds per sample.

Many ``high res'' seismic systems oversample the data while digitizing
so that processing can be done in the frequency domain.  The data
are resampled in the frequency domain at a sample rate that is not
an integer sample interval in the time domain.  Most chirp systems
function this way so that the matched filter (correlation) is done
in the frequency domain.

The SEG-Y format specifies the sample spacing as a sample interval,
not a sample rate.  The SEG-Y sample interval must be a 16 bit integer
and is expressed in microseconds (thousandth of a millisecond.)

\subsection{Parameter Dictionary}

\begin{description}
\item[\texttt{NEWSI}] The output sample interval, in seconds.
         Required.            \textit{e.g.}   \texttt{NEWSI  .004}

\item[\texttt{ORDER}] The order of the interpolation when using type 2 interpolation.
         According to ``Numerical Recipes'' \cite{WHPress1989a} \enquote{We enthusiastically endorse
         interpolations with 3 or 4 points, we are perhaps tolerant of
         5 or 6; but we rarely go higher}.
         \Gls{preset} = 4   limits   1 < \texttt{ORDER} < 7        \textit{e.g.}   \texttt{ORDER 3}
\end{description}

\section{SADD: Scalar Add}
\label{cmd_sadd}

Process \texttt{SADD} performs a scalar addition to the seismic traces, \textit{i.e.} a
constant number specified by the parameter \texttt{SCALAR} is added to every
amplitude.  Only those \gls{shot}/\glspl{rp} and traces specifically given have the
scalar addition applied.  There is no spatial interpolation.


\subsection{Parameter Dictionary}

\begin{description}
\item[\texttt{SCALAR}] The scalar to add to the trace.  \Gls{preset} = 0. \textit{e.g.}   \texttt{SCALAR 10.e3}
\item[\texttt{FNO}] The first \gls{shot}/\gls{rp} number the parameter list applies to.  \Gls{preset} = the first \gls{shot}/\gls{rp} received.    \textit{e.g.}   \texttt{FNO 101}
\item[\texttt{LNO}] The last \gls{shot}/\gls{rp} number the parameter list applies to.  \Gls{preset} = the last \gls{shot}/\gls{rp} received.    \textit{e.g.}   \texttt{LNO 101}
\item[\texttt{FTR}] The first trace number the parameter list applies to.  \Gls{preset} = the first trace of each \gls{shot}/\gls{rp}.    \textit{e.g.}   \texttt{FTR 10}
\item[\texttt{LTR}] The last trace number the parameter list applies to.  \Gls{preset} = the last trace of each \gls{shot}/\gls{rp}.    \textit{e.g.}   \texttt{LTR 10}
\item[\texttt{END}] Terminates the parameter list.
\end{description}

\section{SEGDDIN: SEG-D Disk File Input}
\label{cmd_segddin}

Process \texttt{SEGDDIN} reads SEG-D formatted disk files.  The SEG-D
shot files may be listed in a separate ``list'' file or all the \glspl{shot}
may be in a single file or the \glspl{shot} may be in a ``realtime'' \gls{shot}
directory.
      The demultiplexed data formats available are:
\begin{itemize}
\item 8015 = 20 bit SEG-D floating point (4 bit hex exponent)
\item 8022 = 8 bit integer
\item 8024 = 16 bit IEEE floating point (not the same as UTIG F.P.)
\item 8036 = 24 bit integer (new in rev 2)
\item 8038 = 32 bit integer
\item 8048 = 32 bit IBM FP
\item 8058 = 32 bit IEEE FP
\end{itemize}

SIOSEIS considers the SEG-D file number to be a count of the
SEG-D files produced by the recording system.  This file number is
used as the SEG-Y trace header ``Original field record number'', word 3,
as well as the ``Energy source point number'', word 5, EXCEPT on \gls{ldeo}
systems.  The \gls{ldeo} system ``shot point number'' is put the SEG-Y word 3
and the file number in word 5.

Example:
\begin{verbatim}
SEGDDIN
    LISTPATH file_list END
END
\end{verbatim}

\subsection{Parameter Dictionary}

\begin{description}
\item[\texttt{IPATH}] The pathname of a single SEG-D file that contains one or
          more \glspl{shot}.  The SEG-D File number is used to control which
          \glspl{shot} are read though parameters \texttt{FFILEN} and \texttt{LFILEN}.
          \Gls{preset} = none   \textit{e.g.} \texttt{/export/home/bigfile.segd}

\item[\texttt{LISTPATH}] The pathname of a list of SEG-D files to be read and
         processed.  Only one SEG-D file name may be on each line in
         the list.  The list is read in order from the first line to
         the last line.  There are many ways to create a list of files.
         \texttt{ls -lt /export/home3/seisnet/BOL38/FFID\_2* | sort -r | awk '{print \$9}'}
         or something using the Unix \texttt{ls -1tr} command (one (\texttt{1}), not \texttt{l}).
         The shell script below generates a partial list of a directory:
\lstset{language=csh}
\begin{lstlisting}
#!/bin/csh -f
set first = 2746
set last = 4165
set i = $first
while ( $i < $last )
   ls -1 /export/home3/seisnet/BOL38/FFID_$i%*
    @ i = $i + 1
end
\end{lstlisting}
         \Gls{preset} = none    \textit{e.g.} \texttt{LISTPATH list}     where \texttt{list} contains:
\begin{verbatim}
/data/project_xyz/FFID0001
/data/project_xyz/FFID0002
/data/project_xyz/FFID0003
\end{verbatim}

\item[\texttt{\texttt{STACK}}] pathname [\texttt{ODD}/\texttt{EVEN}]
         The pathname of a file containing a push down-stack
         containing at least two SEG-D \glspl{shot}.  The file on the second
         line of the push-down stack is read and processed.  The file of
         SEG-D filenames (the stack) is reread and the \gls{shot} on the second
         line is read and processed if it is different from the previous
         \gls{shot}.  One way of creating a push down stack is to run a shell
         script with a loop with \texttt{ls -1t} in it.  \textit{e.g.}
\lstset{language=csh}
\begin{lstlisting}
#!/bin/csh -f
set forever = 1
while( $forever )
   ls -t /export/home/public/seisnet_tmp/* | head -n 2 > /tmp/latest
   sleep 5
end
end
\end{lstlisting}

         The second file from the top of the stack is used since the
         write of the top file may not be completed.  In this manner,
         new SEG-D \glspl{shot} may be read and processed continuously.

         \texttt{ls /pathname/*} includes the pathname in the listing.  Otherwise use:
\lstset{language=csh}
\begin{lstlisting}
sed '1,2 s%^%/export/home/public/seisnet_tmp/%;w latest' /tmp/latest
\end{lstlisting}

         SIOSEIS jobs using the \texttt{STACK} parameter may be stopped
         gracefully by creating a file ``in'' with a negative number
         when there are no more files to be added to the stack.

\item[\texttt{ODD/EVEN}] An additional part of the \texttt{STACK} parameter indicating that
         only \texttt{ODD} or \texttt{EVEN} \gls{shot} numbers are processed.

         \textit{e.g.}    \texttt{STACK /tmp/latest  EVEN}

\item[\texttt{LDEOLIST}] The pathname to a list of files used in conjunction with
         the Seisnet real-time system on the Ewing.  \texttt{SEGDDIN} reads
         and processes the seisnet files listed in the \texttt{LDEOLIST} file.
         When the list is exhausted the \texttt{LDEOLIST} file is deleted.
         Processing resumes when a new \texttt{LDEOLIST} file is detected.
         Process \texttt{SEGDDIN} is in a sleep loop until the new file is
         found, so the user should stop the sioseis job by using a
         negative number in file ``in''.

\item[\texttt{FCSET}] First channel set.  The SEG-D format allows all traces with
         similar characteristics to be grouped together.  If traces
         have different trace length, all traces with the same length
         are grouped together.  Likewise, auxiliary traces belong
         to a different set from seismic traces.
      If \texttt{FCSET} = 99, then all channel sets are processed.
         \Gls{preset} = 99

\item[\texttt{LCSET}] Last channel set.  Each channel starts with trace 1, however
         \texttt{SEGDDIN} will renumber the traces if more than 1 channel set
         is processed.
         \Gls{preset} = The last channel set

\item[\texttt{FFILEN}] The first file number to read from \texttt{IPATH}.  File numbers
         less than \texttt{FFILEN} will be omitted.  Only used with
         \texttt{FORMAT GEOMETRICS}
         \Gls{preset} = 99999

\item[\texttt{LFILEN}] The last file number to read from \texttt{IPATH}.  File numbers
         greater than \texttt{LFILEN} will be omitted.  Only used with
         \texttt{FORMAT GEOMETRICS}
         \Gls{preset} = 999999999

\item[\texttt{FILINC}] The increment between \texttt{FFILEN} and \texttt{LFILEN}.  \texttt{FILINC} 99999
         indicates that all \glspl{shot} between \texttt{FFILEN} and \texttt{LFILEN} will
         be used regardless of order.  Only valid when \texttt{FFILEN} is
         given.
         \Gls{preset} = 99999

\item[\texttt{FTR}] The first trace within each channel set to process.
         Trace numbers less than \texttt{FTR} will be omitted.
         \Gls{preset} = 1     \textit{e.g.}   \texttt{FTR 97}

\item[\texttt{LTR}] The last trace within each channel set to process.  Traces
         with numbers larger than \texttt{LTR} will be omitted.
         \Gls{preset} = The largest data trace number in the channel scan.

\item[\texttt{TRINC}] The increment between traces.  The trace skip increment.
         preset = 1      \textit{e.g.}  \texttt{TRINC 2 \#  every other trace is skipped}

\item[\texttt{SECS}] The maximum number of seconds of data to process.  Zero fill
         or padding will NOT be done if \texttt{SECS} exceeds the SEG-D
         record length.
         \Gls{preset} = length in the header.     \textit{e.g.}   \texttt{SECS 6}

\item[\texttt{STIME}] The time of the first data to output.  The delay of the
         trace after reformatting.  The data times will be from \texttt{STIME}
         to \texttt{STIME} + \texttt{SECS}.  The first portion of the data trace will
         be discarded whenever \texttt{STIME} is greater than the delay of the
         recorded data.  \texttt{STIME} less than the recorded delay will NOT
         result in zero padding.  \textit{e.g.} \texttt{STIME} must be greater than or
         equal to the deep water delay.
         \Gls{preset} - The recorded delay.   \textit{e.g.} \texttt{STIME 4.0}

\item[\texttt{DECIMF}] Sample interval decimation factor.  \Gls{preset} = 1,     \textit{e.g.}   \texttt{DECIMF 2} decimates by a factor of 2

\item[\texttt{NTRGAT}] The number of traces per \gls{gather}.  SIOSEIS requires RPs to be
         terminated with a -1 in word 51 of the SEG-Y trace header.
         Because this is unique to SIOSEIS, \glspl{gather} from other
         computers may be converted by setting \texttt{NTRGAT} to the proper
         value.  \texttt{SEGDIN} will set every \texttt{NTRGAT} trace to be a
         terminator.  \textit{e.g.} \texttt{FTR 91 LTR 96 NTRGAT 6} will read only
         traces 91 - 96 and will set every trace 96 (the sixth trace
         to be read) to be a \gls{gather} terminator.
         \Gls{preset} = 0,    \textit{e.g.}  \texttt{NTRGAT 12}

\item[\texttt{RENUM}] Renumbers the \glspl{shot} (file numbers) so that the first \gls{shot}
         read will be \texttt{RENUM} and successive \glspl{shot} will be incremented
         by 1.
         \Gls{preset} = 0,    \textit{e.g.}   \texttt{RENUM 1}

\item[\texttt{RETRAC}] Renumber the trace numbers within each \gls{shot}/\gls{rp}.  Most recording
         systems start each channel set with trace 1.
         \Gls{preset} = 0.
\begin{description}
     \item[<0] The trace number is the accumulated trace count within each
         file.  When there are multiple channel sets the trace numbers
         will always increase.  This is what is used by \gls{ukooa}.
     \item[=0] The trace number is the trace number within the SEG-D \gls{shot}.
     \item[>0] The first trace of each \gls{shot} starts wIth \texttt{RETRAC}.
\end{description}

\item[\texttt{FORMAT}] The type of SEG-D format.  \Gls{preset} \texttt{SEG-D}
\begin{description}
       \item[\texttt{SEG-D}] the preset.  The SEG-D external header is searched
       for \gls{nmea} strings \$GPGGA, \$DBT, and \$SDDBT
   \item[\texttt{SEISNET}]  The data from the PC used on the Ewing for splitting
       the seismic data from the Syntron.
   \item[\texttt{HTI}] The HydroScience SS format with a 32 byte PC header and
       24 bit integer (SEG-D 8036) data word.
   \item[\texttt{GEOMETRICS}] generates Geometrics style filenames (*.sgd) based
       on \texttt{FFILEN} and \texttt{LFILEN}.  \texttt{IPATH} must be the name of the directory
       containing the SEG-D \gls{shot} files.  \texttt{FFILEN} and \texttt{LFILEN} are required.
   \item[\texttt{LDEO}] The \gls{ldeo} Syntron external header is expected.  The
       Digiscan bird blocks, if present, are passed to process \texttt{GEOM}.
\end{description}

\item[\texttt{NSPFILE}] Number of Shots Per SEG-Y File.  \texttt{SEGDDIN} will set a flag for
          processes \texttt{OUTPUT} and \texttt{DISKOX} to start a new file after \texttt{NSPFILE}
          \glspl{shot} have been written.  The new output file includes the
          SEG-Y \gls{ascii} and binary headers.  The \texttt{DISKOX} opath parameter
          should be set to \texttt{DATE} or \texttt{SHOTNO} to automatically generate
          the output disk file names.
          \Gls{preset} = none      \textit{e.g.}    \texttt{NSPFILE} 100

\item[\texttt{LOGPATH}] The pathname of a file that will contain a ``log'' of each
          \gls{shot}.  The \gls{shot} number, file number, time and position of
          each \gls{shot} will be recorded.
          \Gls{preset} = none       \textit{e.g.}    \texttt{LOGPATH   log}

\item[\texttt{DESCALE}] A \texttt{YES}/\texttt{NO} switch indicating whether to apply the SEG-D
          ``descalar'' multiplier to every amplitude.  The descalar
          brings the data back to the recording system input
          level,  See the \gls{seg} standard (which? \cite{SEG_Y_r0}, \cite{SEG_Y_r1}, or \cite{SEG_Y_r2}) for further information.
          Preset = \texttt{NO}  (prior to rev. 2013.4)
          Preset = \texttt{YES} (after rev. 2013.4)
          \textit{e.g.}   \texttt{DESCALE} YES

\item[\texttt{LPRINT}] The bit oriented debugging print switch.
\begin{description}
       \item[32] The SEG-D External header is printed as an \gls{ascii} string.
       \item[255] Turns all bits on.  May cause a lot of printing.
\end{description}
\end{description}



\subsection{Parameters for Reading by GMT}

NOTE:  GMT as used by these parameters is the time contained
       in the SEG-D General Header.  On the Ewing, this is the
       Syntron clock, not the external \gls{gps} clock used as
       ``the Official Shot Time''.

\begin{description}
\item[\texttt{FDAY}] The Julian day of the first \gls{shot} to be read from \texttt{IPATH}.
         Preset = the day of the first \gls{shot}    \textit{e.g.} \texttt{FDAY} 365

\item[\texttt{LDAY}] The Julian day of the last \gls{shot} to be processed.
         Preset = \texttt{FDAY}.    \textit{e.g.} \texttt{LDAY 366}

\item[\texttt{FGMT}] The GMT of the first data to extract from \texttt{IPATH}/\texttt{FFILEN}.  Shots
         prior to \texttt{FGMT} will not be processed.
         Preset = the GMT of the first \gls{shot} (\texttt{FFILEN}).

\item[\texttt{LGMT}] The GMT of the last \gls{shot} to extract from \texttt{IPATH}.
         Preset = the last \gls{shot} (\texttt{LFILEN}).
\end{description}

\subsection{Deprecated Parameters}

\begin{description}
    \item[\texttt{TR0}] A \texttt{YES}/\texttt{NO} switch indicating that the SEG-D external header
         should be written and saved in a disk file similar to the
         old Digicon/LDGO trace 0.  The disk file will start with \texttt{'tr0'}
         and have the Syntron date of the first \gls{shot} appended to it.
         Not to be confused with process DISKOX or output parameter
         \texttt{TRACE0}.
         \Gls{preset} = \texttt{NO}      \textit{e.g.} \texttt{TR0 YES}
\end{description}

\subsection{Special Information for the LDEO R/V Maurice Ewing}

     \texttt{SEGDDIN} uses the \gls{ldeo} ``shot point'' number, ``shot time'',
navigation (lat/long), and Hydrosweep water depth, which are in the
SEG-D external headers.
     \gls{ldeo} supplies a daily navigation file that lists every \gls{shot}
point number, ``official shot time'', and position.  This file is
called a \texttt{ts.n} file.  \texttt{ts.n} files do not contain the SEG-D file
number.
     \texttt{SEGDDIN} optionally carries the old DIGICON ``trace 0'' forward.
Trace 0 contains the nonseismic information in the SEG-D General
and External Headers.  SIOSEIS process \texttt{OUTPUT} parameter
\texttt{TRACE0 YES} will write these headers as SEG-Y trace number 0 and
trace id 28 (see the SEG-Y documentation).

     The Hydrosweep water depth is inserted as a floating point
number into the 54th real word of the SEG-Y trace header and is
converted to time by dividing by 750. and placed in real word 50.
The new (2001) Spectra/Syntron system does not put the Digicourse
streamer depth in SEG-D data, instead it is now in an external
\gls{ukooa} navigation file.

     The \texttt{SEGDIN} code is:
(where ibuf is 16 bit integers, lbuf is 32bit integer, buf is \texttt{REAL})

\begin{verbatim}
          lbuf(3) = ldgo_shotno   ! or SEG-D shot number if not LDEO
          lbuf(4) = itrcno        ! shot trace number
          lbuf(5) = ifilen        ! SEG-D file number
          ibuf(15) = 1            ! SEG-Y trace id
          buf(46) = AMAX1(delay(icsn),stime)  ! REAL delay
          ibuf(55) = NINT(buf(46) * 1000.)    ! delay in mils
          ibuf(58) = (nsamps + decimf - 1) / decimf  ! number of samples
          ibuf(59) = micros * decimf  ! sample interval in micros
          buf(49) = si * decimf   ! REAL sample interval in seconds
          lbuf(16) = wdepth       ! integer water depth
          ibuf(79) = ldgo_yr      ! year of "official shot time"
          ibuf(80) = ldgo_day     ! day of "official shot time"
          ibuf(81) = ldgo_hr      ! hour of "official shot time"
          ibuf(82) = ldgo_min     ! minute of "official shot time"
          ibuf(83) = ldgo_sec     ! second of "official shot time"
          ibuf(84) = ldgo_mil     ! "millisecond of "official shot time"
             ***   This use of word 84 is an SEG-Y violation  ****
          ibuf(36) = -10          ! scalar for coordinates
          lbuf(19) = NINT(ship_long*360.*60.*10.)  ! shot x-coordinate
          lbuf(20) = NINT(ship_lat*360.*60.*10.)   ! shot y-coordinate
          lbuf(21) = NINT(ship_long*360.*60.*10.)  ! shot x-coordinate
          lbuf(22) = NINT(ship_lat*360.*60.*10.)   ! shot y-coordinate
          ibuf(45) = 2
          buf(50) = wdepth / 750.
          buf(54) = wdepth
\end{verbatim}

If process \texttt{GEOM} parameter \texttt{GXP} and \texttt{BGP} are used, the group offset and
streamer depth are added as:
\begin{verbatim}
          lbuf(10) = NINT(range)
          lbuf(11) = -NINT(depths(lbuf(4))*100.) | streamer depth in cm.
\end{verbatim}

\subsubsection{SIOSEIS \texttt{SEGDDIN} LDEO error and warning messages}

\begin{description}
    \item[\texttt{***  WARNING  ***  Bad LDEO clock.}]
    This message occurs when the \gls{ldeo} ``official shot time'' does not have a \texttt{+} or \texttt{-} quality flag between the year and day of year.

\item[\texttt{***  WARNING  ***  Bad LDEO block.}]
    This message occurs when the \gls{ldeo} nav block has garbage in place of the \gls{gps} fix.

\item[\texttt{***  WARNING  ***  Bad LDEO Hydrosweep of XXXXXX}]
    This message occurs when the Hydrosweep center beam depth is greater than 10000 or less than 100.  The previous depth is substituted.
\end{description}

\section{SEG2IN: SEG-2 Disk File Input}
\label{cmd_seg2in}

Process \texttt{SEG2IN} reads seismic disk files in the SEG-2 format as described in
the September 1990 Geophysics article ``Recommended standard for seismic
(/radar) data files in the personal computer environment'' \cite{SEG_2_1990}.
Each \gls{shot} is in a separate disk file.  Data files from
the Geometrics Strataview have been used.

\texttt{SEG2IN} will handle SEG-2 files that are in little or big endian
byte order.

\subsection{Parameter Dictionary}

\begin{description}
\item[\texttt{FFILEN}] The first file number (\gls{shot}) to read from disk.  File numbers
         less than \texttt{FFILEN} will be omitted.  File names are assumed to
         have Geometrics filename.  \textit{e.g.} \texttt{nnnn.DAT}  where \texttt{nnnn} is the
         \gls{shot} number.  The file names and \gls{shot} numbers must be
         monotonically increasing by 1 from \texttt{FFILEN} to \texttt{LFILEN}.  The
         files must be in the current (working) directory.
         \Gls{preset} = 99999 (none)

\item[\texttt{LFILEN}] The last file number (\gls{shot}) to read from disk.  File numbers
         greater than \texttt{LFILEN} will be omitted.
         \Gls{preset} = 99999 (none)

\item[\texttt{IPATH}] The input SEG-2 pathname (filename).  The SEG-Y \gls{shot} number
         will be the value specified by the SEG-2 string
         SHOT\_SEQUENCE\_NUMBER if it exists.  If SHOT\_SEQUENCE\_NUMBER
         does not exist, then the first four digits of \texttt{IPATH} are used,
         assuming the data were recorded on a Geometrics recorder.
         Not used with \texttt{FFILEN}/\texttt{LFILEN}.
         \Gls{preset} = none    \textit{e.g.} \texttt{/usr/people/henkart/data/1019.DAT}

\item[\texttt{FTR}] The first trace in the SEG-2 file to read.  Traces (channels)
         numbered less than \texttt{FTR} will be omitted.
         \Gls{preset} = 1

\item[\texttt{LTR}] The last trace in the SEG-2 file to read.  Traces (channels)
         numbered greater than \texttt{LTR} will be omitted.
         \Gls{preset} = last trace in the file.

\item[\texttt{LPRINT}] Debug print switch.
\begin{description}
\item[\texttt{4}] The SEG-2 File Descriptor and the Trace Descriptor strings
         are printed for EVERY trace between FTR/LTR.
\end{description}

\item[\texttt{END}] Terminates each parameter list.
\end{description}

\subsection{Example}

To read several files that are not strictly monotonically numbered.

\begin{verbatim}
SEG2IN
    IPATH /usr/people/henkart/data/1019.DAT END
    IPATH /usr/people/henkart/data/1022.DAT END
    IPATH /usr/people/henkart/data/1023.DAT END
    IPATH /usr/people/henkart/data/1025.DAT END
END
\end{verbatim}

\section{SHIFT: Time Shift}
\label{cmd_shift}

Process \texttt{SHIFT} applies a time shift to the seismic traces by the entire
line, or by individual \gls{shot} or \gls{rp}, or by the trace within the \gls{shot} or
\gls{rp}.  Any combination may be applied, with the resulting trace shift being
the sum of all the shifts for the trace.

Shifts are given in units of seconds.  A negative time shift will shift
the trace to the left, or the resulting trace will appear earlier.
\textit{e.g.} for a sample interval of .004, a shift of -.1 seconds, data between
input times 0.000 and .096 will be dropped, time 1.00 become time 0.000,
time .104 will become time .004, time .108 will become .008, \textit{etc},
and the last .100 will be zero filled.

Likewise, the positive shifted traces will have the front of the trace
zeroed.

Parameters that are defaulted are reset after being used, whereas
parameters that are preset remain the same until changed by giving the
parameter again.

Process \texttt{SHIFT} will shift only the \glspl{shot}/\glspl{rp} and traces specified within
each \texttt{FNO}/\texttt{LNO} list when parameter \texttt{INTERP} is \texttt{NO}, which is the preset.
Parameters \texttt{RSHIFT}, \texttt{TSP}, \texttt{XSP}, and \texttt{GSP} are defaulted to not given and must
be given for each \gls{shot}/\gls{rp} to be shifted.
            \textit{e.g.} \texttt{FNO} 1 rshift .008 tsp 10 .004 end
                 FNO 2 tsp 2 -.04 end
results in \gls{shot} 1 trace 10 to be shifted .008+.004 and all other traces
in \gls{shot} 1 to be shifted .008.  The only trace shifted in \gls{shot} 2 is trace
2, which will be shifted -.04.


\texttt{XSP}, \texttt{TSP}, and \texttt{GSP} allow a maximum of 1300 pairs to be given.

Each parameter list must be terminated with the word \texttt{END}.  The entire
set of shift parameters must be terminated by the word \texttt{END}.


\subsection{Parameter Dictionary}

\begin{description}
\item[\texttt{LSHIFT}] Line shift. The amount of time to shift every trace of every
         record.
         \Gls{preset} = 0.

\item[\texttt{RSHIFT}] Record shift.  The amount of time to shift the \gls{shot} (\gls{rp})
         defined by \texttt{FNO} - \texttt{LNO}.  Every trace within the \gls{shot} (\gls{rp}) is
         shifted by \texttt{RSHIFT}.  \texttt{RSHIFT} is reset to 0 after \texttt{LNO} is reached.
         Default = 0.

\item[\texttt{FNO}] The first \gls{shot} (or \gls{rp}) to apply \texttt{RSHIFT} and/or \texttt{XSP}/tsp to. Shot
         (\gls{rp}) numbers must increase monotonically.  Spatial
         interpolation and extrapolation are controlled by parameter
         \texttt{INTERP}
         \Gls{preset} = none

\item[\texttt{LNO}] The last \gls{shot} (\gls{rp}) number to apply and/or \texttt{XSP}/tsp to.  \texttt{LNO} must
         be larger than \texttt{FNO} in each list and must increase list to list.
         Default = \texttt{FNO}

\item[\texttt{XSP}] Range-shift-pairs.  A list of range and shift pairs.  \texttt{XSP} must
         be given with increasing ranges.  The program computes the
         absolute value of both user ranges and data ranges.  When
         parameter \texttt{INTERP} is \texttt{NO}, the only traces shifted are those
         with the exact range specified.
         \textit{e.g.} xsp 1000 3.0 2000 -1.1  - Traces with ranges exactly equal
         to 1000 will be shifted by 3 seconds, and traces with a range
         of exactly 2000 will be shifted by -1.1 seconds.  All other
         traces will not be shifted unless \texttt{LSHIFT} or \texttt{RSHIFT} was given.
         Default = all 0.

\item[\texttt{TSP}] Trace number-shift-pairs.  A list of trace numbers and shifts
         listed in pairs.  Only those traces specified will be shifted.
         Trace numbers must increase within each list.  Parameter \texttt{INTERP}
         controls the spatial variation of \texttt{TSP}.
         \textit{e.g.} tsp 4 -1. 20 .5 indicates that trace 4 will be shifted by
         -1.000 seconds and trace twenty will be shifted by .5 seconds.
         Default = all 0.

\item[\texttt{GSP}] Group-shift-pairs.  A list of group numbers and times shifts.
         Group numbers are the same as trace numbers.  Groups not
         specified within the \texttt{FNO}/\texttt{LNO} list receive a shift calculated
         through interpolation or extrapolation regardless of the
         setting of INTERP.  Parameter \texttt{\texttt{INTERP}} controls the spatial
         variation of \texttt{GSP}.
         \texttt{GSP 2 +10 4 -10} results in \texttt{trace 1 +10 2 +10 3 0 4 -10 5 -10 6
               -10 7 -10 \ldots}
         Default = all 0

\item[\texttt{APPVEL}] Apparent velocity.  Used to calculate the angle by which \glspl{shot}
         are projected onto the sea floor
         (\textit{i.e.} sin(angle) = \texttt{VELH2O} / \texttt{APPVEL}).
         Time shifts are calculated using:
\begin{verbatim}
TIMECORR = WATERDEPTH / (VELH2O*COS(ANGLE))
\end{verbatim}
         A shift for range is also applied using:
\begin{verbatim}
RANGECORR = WATERDEPTH * TAN(ANGLE)
\end{verbatim}
         THE RANGE IN THE SEG-Y TRACE HEADER IS ADJUSTED!

         Shifts from \texttt{APPVEL} are performed prior to the reduction
         velocity time shift determined by parameter \texttt{REDVEL}.
         Preset = 0. (No correction)    \textit{e.g.}   \texttt{APPVEL 9000}

\item[\texttt{VELH2O}] Velocity of water.  Used to calculate the time shift for
         projecting onto the sea floor (parameter \texttt{APPVEL}).
         Preset = 0.  (No correction)

\item[\texttt{REDVEL}] Reduction velocity.  Shifts every trace in time according to:
\begin{verbatim}
SHIFT =  RANGE / REDVEL
\end{verbatim}
         Preset = 0. (No shift)    \textit{e.g.} \texttt{REDVEL 8000}

\item[\texttt{REDUCE}] Reduce the amount of data in the trace by changing the SEG-Y
         header values of the delay and the number of samples as well as
         shifting the data.  Data reduction will take place when \texttt{REDUCE}
         is give a value of \texttt{YES}.
         Preset = no     \textit{e.g.}   \texttt{REDUCE YES}

\item[\texttt{LAGA}] When set to \texttt{YES}, each trace will be shifted by the negative
         of the amount contained in the SEG-Y trace header bytes 105-106,
         the lag time A in milliseconds.  The lag time is defined to be:
         the time in ms. between the end of the 240-byte trace ident-
         ification header and the time break.  Positive if the time
         break occurs after the end of the header, negative if the time
         break occurs before the end of the header. The time break is
         defined as the initiation pulse \ldots"
         Preset = 0              \textit{e.g.}   \texttt{LAGA YES}

\item[\texttt{LAGB}] When set to \texttt{YES}, each trace will be shifted by the negative
         of the amount contained in the SEG-Y trace header bytes
         107-108, the lag time B in milliseconds.
         Preset = 0              \textit{e.g.}   \texttt{LAGB YES}

\item[\texttt{DATUMV}] Datum velocity.
\item[\texttt{DATUME}] Datum elevation.
         When \texttt{DATUMV} and \texttt{DATUME} are given, a datum shift is computed
         using the ``source elevation'', ``source depth'', ``datum elevation at the source'', ``group elevation'', ``datum elevation at the receiver'', and ``scalar'' from each SEG-Y trace header words
         11 - 15 and 17.  The shift is (after the elevation scalar):
\begin{verbatim}
         ((group elevation) + (receiver datum) - datume) / datumv
         (((source elevation)+(source datum)-(source depth))-datume)/datumv
\end{verbatim}
         Elevations are relative to their datum (elevations may not
         be relative to 0).

     \item[\texttt{INTERP}] A \texttt{YES}/\texttt{NO} switch.  \texttt{INTERP} \texttt{YES} means \glspl{shot}/\glspl{rp} not specified
         will be calculated through normal sioseis spatial variation.
\item[\texttt{TSP}] , \texttt{GSP}, \texttt{XSP}.  \texttt{INTERP} \texttt{YES} indicates that the shift for traces not
         specified should be calculated by interpolation or ``extension''.
         Traces between specified traces will be shifted using times
         linearly interpolated.  Traces ``outside'' specified traces will
         be shifted using the ``closest'' trace (times held constant).
         \Gls{preset} = \texttt{NO}    \textit{e.g.}  \texttt{INTERP YES}

\item[\texttt{INDICES}] A list of up to 10 indices of the SEG-Y trace header to use
          as shift values.  Each shift is additive.  Parameter \texttt{SMULT}
          may be used to modify the header values before summing.  The
          shift values must be in seconds because the shift is divided
          by the sample interval in order to convert it to the number
          of samples.
          \texttt{INDICES} must be given as \texttt{xn}, where;
\begin{description}
\item[\texttt{x}] Data type
\begin{description}
\item[\texttt{I}] means short integer (16 bit integer trace header)
\item[\texttt{L}] means long integer (32 bit integer trace header)
\item[\texttt{R}] means real word (host floating point)
\end{description}
\item[\texttt{n}] the index with the SEG-Y trace header.
\end{description}
          A maximum of 10 \texttt{xn}s may be given in a parameter list.
          Examples:
          \begin{itemize}
              \item \texttt{L1} means long integer word 1
              \item \texttt{I59} means short word 59
              \item \texttt{R49}  means real word 49
          \end{itemize}

\item[\texttt{SMULT}] Scalar multiplier of the SEG-Y header values described via
         \texttt{INDICES}.  \textit{e.g.} smult -.001 converts milliseconds to seconds
         and makes it a negative shift.
         \Gls{preset} = 1.

\item[\texttt{LPRINT}] Programmer's debug switch.
\begin{description}
\item[\texttt{2}] The shift value is printed.
\item[\texttt{4}] The datum shifts are printed.
\end{description}

\item[\texttt{END}] Terminates each parameter list.
\end{description}

Note:
\textbf{\texttt{XSP}, \texttt{TSP} AND \texttt{GSP} ARE MUTUALLY EXCLUSIVE, ONLY ONE MAY BE GIVEN IN A
PARAMETER LIST.}

Example 1:  \texttt{XSP} and default \texttt{INTERP}  (INTERP NO)

\begin{verbatim}
 PROCS SYN SHIFT PROUT END
 SYN
    FNO 1 LNO 3 NTRCS 10 SECS 1 TVA .5 1000 1 END END
 SHIFT
    FNO 2 XSP 200 .2  500 .5 END END
\end{verbatim}

Results in \gls{shot} 2 trace 3, range 200, being shifted .2 seconds and
           \gls{shot} 2 trace 6, range 500, being shifted .5 seconds.
All other traces are not shifted.


Example 2:  \texttt{XSP} and \texttt{INTERP YES}

\begin{verbatim}
PROCS SYN SHIFT PROUT END
SYN
   FNO 1 LNO 3 NTRCS 10 X 0 XINC 100 SECS 1 TVA .5 1000 1 END END
SHIFT
   INTERP YES
   FNO 2 XSP 200 .2  500 .5 END END
\end{verbatim}

results in all \glspl{shot} (\glspl{shot} 1-3) receiving the same shifts.
\begin{itemize}
    \item Traces  1-3, ranges 0-200 are shifted .2 seconds.
\item Trace 4, range 300 is shifted .3 seconds.
\item Trace 5, range 400, is shifted .4 seconds.
\item Trace 6-10, ranges 500-900, are shifted .5 seconds.
\end{itemize}

\section{SMUTE: Surgical Mute}
\label{cmd_smute}

Process \texttt{SMUTE} performs either a surgical mute or a tail mute to the
seismic traces.  A surgical mute zeroes a portion of the trace according
to user given start and end times.  A tail mute zeroes the end or tail
of the trace starting at a user given time.  \texttt{SMUTE} is similar to process
\texttt{MUTE} except that \texttt{MUTE} always zeroes the front or beginning of the trace.

\texttt{SMUTE} was designed for zeroing areas that are occasionally bad, so
spatial interpolation is turned off except when using tail muting
(parameter \texttt{XTP}, which is similar to process \texttt{MUTE} parameter \texttt{XTP}).

Surgical mutes are used to zero out spike while tail mutes are often
applied to the traces closest to the source (and are then called
inner mutes).

As with process \texttt{MUTE}, a 5 sample linear ramp is applied to the mute zone
edges in order to reduce ``edge effects''.

Traces not specified are muted according to interpolation and
extrapolation of the mute times of adjacent traces and \glspl{shot}/\glspl{rp}, unless
the \texttt{INTERP} parameter is used.

Surgical muting and tail muting may not be perform simultaneously.  \textit{i.e.}
\texttt{XTP}, \texttt{TTP}, \texttt{XSETS}, \texttt{TSETS} are all mutually exclusive.

Each parameter list must be terminated with the word \texttt{END}.  Spatial
variation is obtained by giving multiple lists or control points.  (See
doc/syntax).  The spatial variation may be turned off by using the
\texttt{INTERP} parameter.


\subsection{Parameter Dictionary}

\begin{description}
\item[\texttt{FNO}] The first \gls{shot} (or RP) to apply the mutes to.  Shot (RP)
         numbers must increase monotonically.
         Preset=1

\item[\texttt{LNO}] The last \gls{shot} (RP) number to apply the mutes to.  \texttt{LNO} must be
         larger than \texttt{FNO} in each list and must increase list to list.
         Default=\texttt{FNO}

\item[\texttt{XTP}] RANGE-TIME-PAIRS.  A list of range and tail mute time pairs.
         Mute times for ranges not specified are obtained through linear
         interpolation.  If the range is between two ranges specified.
         traces with a range less than the smallest given range will be
         muted to the mute time of the smallest given range.  Likewise,
         ranges larger than the largest given range will be muted to the
         mute time of the largest given range.  \texttt{XTP} must be given with
         increasing ranges. The program computes the absolute value of
         both user ranges and data ranges.  \textit{e.g.} xtp 1000 3.0 2000 4.0
         traces with ranges less than 1000 will be muted to 3 seconds,
         traces with ranges greater than 2000 will be muted to 4 seconds,
         and traces with range between 1000 and 2000 will be muted
         proportionately between 3 and 4 seconds.
         Preset=NONE

\item[\texttt{TTP}] Trace number-time-pairs.  A list of trace numbers (of a \gls{shot} or
         RP) and tail mute times (listed in pairs).  The mute time for a
         trace between two traces that are specified is obtained through
         linear interpolation of the mute times of the specified traces.
         Traces with a trace number less than the smallest given will be
         muted to the mute time of the smallest trace number.  Likewise,
         traces with a trace number larger than the largest given will
         be muted to the mute time of the largest given.  \texttt{TTP} must be
         given in increasing trace numbers. \textit{e.g.} ttp 4 2. 20 5.
         traces 1 thru 4 will be muted to 2 seconds, traces 20 and up
         will be muted to 5 seconds, and traces 5 thru 19 will be muted
         proportionately between 2 and 5 seconds.
         Preset=NONE

\item[\texttt{XSETS}] Range-start time-end time triples.  A list of range and mute
         window times.  Only those traces actually specified are muted.
         Sets must be given so that the ranges increase.  The sign of the
         range is ignored since the program uses the absolute value of
         the trace ranges as well as the ranges specified by the user.
         \textit{e.g.}  xsets 1000 2.5 3.0 2000 3.990 4.000
         Traces with a range of 1000 will be muted from time 2.5 To 3.0
         and traces with a range of 2000 will be muted from time 3.99 to
         4.0 seconds.
         Preset=NONE

\item[\texttt{TSETS}] Trace number-start time-end time triples.  A list of trace
         number and mute window times.  Only those traces actually
         specified are muted.  \texttt{TSETS} must be given so that the trace
         numbers increase.  \textit{e.g.}  tsets 5 0.9 1.0 15 3.0 4.0
         Trace number 5 will be muted from time .9 to time 1.0 and trace
         15 will be muted from 3 to 4 seconds.
         Preset=NONE.

\item[\texttt{ADDWB}] Indicates that the mute times given via \texttt{XSETS} or \texttt{TSETS} will be
         added to the water bottom time of the trace. (Water bottom
         times may be entered via process \texttt{WBT}).
         Preset = \texttt{NO}     \textit{e.g.}  \texttt{ADDWB YES}
\begin{description}
\item[\texttt{YES}]  The water bottom time will be added.
\item[\texttt{2X}]  Two times.  The water bottom time will be added twice to
         \texttt{XSETS} and \texttt{TSETS}.  This is useful for describing ``inner mutes''
         where the mute starts at the water bottom multiple.
\end{description}

\item[\texttt{INTERP}] A yes/no switch indicating that the start and end
         times for traces not specified via \texttt{XSETS}/\texttt{TSETS} and \texttt{XTP}/\texttt{TTP}
         should be calculated by interpolation or ``extension''.  Traces
         between specified traces will be muted using times linearly
         interpolated.  Traces not between specified traces will be muted
         using the ``closest'' trace (times held constant).  Shots/\glspl{rp} not
         specified will be calculated through interpolation and
         ``extension'' in a similar manner.
\begin{itemize}
\item \Gls{preset} = \texttt{YES} for \texttt{XSETS}/\texttt{TSETS}   \textit{e.g.}   \texttt{INTERP NO}
\item \Gls{preset} = \texttt{NO} for \texttt{XTP}/\texttt{TTP}       \textit{e.g.}    \texttt{INTERP YES}
\end{itemize}

\item[\texttt{END}] Terminates each parameter list.
\end{description}

\subsection{Notes}

\begin{enumerate}
    \item   Either \texttt{XSETS}, \texttt{TSETS}, \texttt{XTP}, or \texttt{TTP} must be given.
\item   All times are in seconds.
\end{enumerate}

\section{SORT: Generates Tables for Sorting SEG-Y Disk Files Through Diskin}
\label{cmd_sort}

Process \texttt{SORT} creates a disk file containing a list of traces ordered
differently from the input SEG-Y disk file.  Process \texttt{DISKIN} can read the
traces from the SEG-Y file in the order contained in the output of this
process.  The sort is done so that the values being sorted are
increasing.  Process \texttt{DISKOX} can ``de-sort'' the sorted SEG-Y file using
this same ``sort'' file.

The following example sorts and desorts disk file data:
\begin{verbatim}
PROCS SORT DISKIN DISKOA END
SORT
   LKEY2 3 LIMIT1 2 3 IPATH data OPATH test LKEY1 4 END END
DISKIN
   SPATH test IPATH data END END
DISKOA
   OPATH junk SPATH test END END
END
\end{verbatim}

\subsection{Limitations}

\begin{enumerate}
\item Only 1 input diskin file may be used at a time.
\item A maximum of 500,000 traces may be sorted at a time.
\end{enumerate}

\subsection{Parameter Dictionary}

\begin{description}

\item[\texttt{IPATH}] The input SEG-Y pathname.
          REQUIRED.      \textit{e.g.}  \texttt{IPATH /usr/seismic/line1}

\item[\texttt{OPATH}] The output pathname of the file that will contain the ordered
          list of traces.
          REQUIRED.      \textit{e.g.}  \texttt{OPATH sort1.line1}

\item[\texttt{LKEY1}] The index to the SEG-Y trace header of the primary key to use
          in sorting.  The index is of the 32 bit integer or long
          integer variables of the header.
          REQUIRED       \textit{e.g.}  \texttt{LKEY1 4}

\item[\texttt{IKEY1}] The index to the SEG-Y trace header of the primary key to use
          in sorting.  The index is of the 16 bit integer or short
          integer variables of the header.
          Preset = NONE  \textit{e.g.}  \texttt{IKEY 83}

\item[\texttt{LKEY2}] The index to the SEG-Y trace header of the secondary key to
          use in sorting.  The index is of the 32 bit integer or long
          integer variables of the header.
          Preset = NONE  \textit{e.g.}   \texttt{LKEY2 3}

\item[\texttt{IKEY2}] The index to the SEG-Y trace header of the secondary key to
          use in sorting.  The index is of the 16 bit integer or long
          integer variables of the header.
          Preset = NONE

\item[\texttt{LIMIT1}] The limits of the primary sort.  Traces outside the limits are
          not read by process diskin.  The limit values MUST be
          increasing in value.
          Preset = NONE  \textit{e.g.} \texttt{LIMIT1 2 3}

\item[\texttt{LIMIT2}] The limits of the secondary sort.  Traces outside the limits
    are not read by process \texttt{DISKIN}.  The limit values MUST be
          increasing in value.
          Preset = NONE   \textit{e.g.} \texttt{LIMIT2 -500 -200}

\item[\texttt{FLAG51}] The value to place in 32 bit integer header word 51 whenever
          the value of the primary key changes.  Word 51 is the
          ``end-of-sort'' flag for many SIOSEIS process (\textit{e.g.} stack),
          where a -1 is used to indicate the end of ``gather''.  Word 51
          will be set to zero if it is not the ``end-of-sort''.  \texttt{SORT} will
          not modify word 51 unless \texttt{FLAG51} is given.
          Preset = NONE  \textit{e.g.}  \texttt{FLAG51 -1}

\item[\texttt{REV1}] A \texttt{YES}/\texttt{NO} switch indicating that the sort of the primary key
          is to be done in reverse or in decreasing order.  This is
          done by negating the header value associated with \texttt{KEY1}.
          Preset = \texttt{NO}  \textit{e.g.}     \texttt{REV1 YES}

\item[\texttt{REV2}] A \texttt{YES}/\texttt{NO} switch indicating that the sort of the secondary key
          is to be done in reverse or in decreasing order.  This is
          done by negating the header value associated with \texttt{KEY2}.
          Preset = \texttt{NO}  \textit{e.g.}     \texttt{REV2 YES}
\end{description}

\section{STACK: Sum Traces of a Gather}
\label{cmd_stack}

Process \texttt{STACK} adds consecutive traces, completing the sum when the
special SIOSEIS end-of-gather flag (header word 51 set to -1) is
detected.  The end-of-gather flag may be set by process \texttt{GATHER},
processes \texttt{INPUT} or \texttt{DISKIN} parameter \texttt{NTRGAT}, or process \texttt{HEADER}.

The summed trace is scaled or averaged by the number of live trace
samples contributing to each stacked trace.  \textit{i.e.} mute times are
accounted for, as well as the number of live traces in the summation.

Process \texttt{STACK} honors trace length changes within the \gls{gather} as well
as changes of the deep water delay with the \gls{gather}.

Some of the SEG-Y trace headers are modified by process \texttt{STACK}.

The \gls{rp} number is always set to the \gls{rp} number of the first trace of
the \gls{gather}.  The \gls{rp} trace number is always 1.

The SIOSEIS end-of-gather flag, SEG-Y long header word 51, is set to 1.

The number of stacked traces (cdp or fold) is set SEG-Y short word 17.

See process \texttt{STK} (Section~\ref{cmd_stack} for median stack.


\subsection{Parameter Dictionary}

\begin{description}
\item[\texttt{HEADER}] The type of header replacement. \Gls{preset} = \texttt{NORMAL}
\begin{description}
    \item[\texttt{FIRST}] The SEG-Y trace header of the first trace of the \gls{gather} is used for the stacked trace.
    \item[\texttt{NORMAL}] Same as \texttt{FIRST} but is addition the \gls{shot} number, the \gls{shot} trace number, and the trace range are set to 0.
    \item[\texttt{LAST}] The above header values are taken from the last trace of each \gls{gather}.
\end{description}

\item[\texttt{NEW}] A \texttt{YES}/\texttt{NO} switch.  Yes indicates that the
    ``new'' stacking algorithm is to be used.  Only non-zero values are
    included in the stacked trace average.  Process stack has always
    honored process \texttt{MUTE} and \texttt{SMUTE} zeroing, but
    \texttt{NMO} \texttt{DSTRETCH} and \texttt{DESPIKE} zeroes were not
    honored.  The new algorithm assumes that modern recording and process
    is done in floating point and a zero amplitude only happens when a
    sioseis process intentionally sets it to zero.
     Preset = \texttt{YES}        \textit{e.g.}   \texttt{NEW NO}

\item[\texttt{END}] Terminates the parameter list.
\end{description}

\section{STK: Median Stack, Range Varying Stack}
\label{cmd_stk}

Process \texttt{STK} contains several alternative algorithms for stacking
(summing) traces.  These stacking algorithms could not be options
of process \texttt{STACK} because they require the entire ``\gls{gather}'' before
the summation can start.  Recall that a ``\gls{gather}'' is a set of traces
where the last trace of the ``\gls{gather}'' has the special SIOSEIS
end-of-gather flag (header word 51 set to -1).  The end-of-gather
flag may be set by process \texttt{GATHER}, processes \texttt{INPUT} or \texttt{DISKIN}
parameter \texttt{NTRGAT}, or process \texttt{HEADER}.

A \texttt{TRIM} stack is a \texttt{MEDIAN} stack where a percentage of the trace values
relative to the median value are summed.  For each time sample, all
traces values within the \gls{gather} are examined and the median value
is determined by sorting the values.  See parameters \texttt{TRIMOUT} and
\texttt{TRIMIN} for further explanation.

A stack \texttt{PANEL} is a progressive stack; each trace within the \gls{gather}
is added to the stack and the partial stack is output.  There will
be as many stacked traces as there are input traces.  \textit{e.g.} Output
trace 1 is just one trace, trace 2 is the sum of trace 1 + 2,trace 3
is the sum of trace 1 + 2 + 3, \textit{etc}

\subsection{Parameter Dictionary}

\begin{description}
\item[\texttt{TRIM}/\texttt{TRIMOUT}] The \texttt{TRIM} percent of trace values furthest from the median
value are excluded (trimmed off) from the stack.  The percentage
         is of the total number of traces in the \gls{gather},  \textit{e.g.}
         \texttt{TRIM 30} means 30\% of the traces are trimmed off (excluded)
         each side of the median.  If there are 10 traces in the \gls{gather}
         only the middle 4 values are included in the stack: the outer
         3 values on each side of the median excluded from the stack.
         \textit{e.g.} With \texttt{TRIMOUT 40} and 10 traces, then only 2 values are stacked.
         Preset = 0.

\item[\texttt{TRIMIN}] The \texttt{TRIMIN} percent of trace values closet to the median
         value are excluded (trimmed off) in the stack.  The \texttt{TRIMIN}
         percent of trace values furthest from the median are
         included in the stack.  \texttt{TRIMIN} and \texttt{TRIM} are mutually exclusive.
         \textit{e.g.} With 10 traces in the \gls{gather} and \texttt{TRIMOUT 20}, then only the
         middle 4 trace values are included.
         Preset = 0.

\item[\texttt{PANEL}] When set non-zero, a progressive stack panel is produced.
    \Gls{preset} = 0.        \textit{e.g.}    \texttt{PANEL 1}

\item[\texttt{END}] Terminates the parameter list.
\end{description}

\section{SWELL: Swell Removal}
\label{cmd_swell}

     Process \texttt{SWELL} removes or dampens the effects ship heave due to
heavy seas or swells.  Sea swells can cause a periodic time shift in
seismic data especially when the seismic equipment is attached to the
hull of the ship.  The ship motion causes the two-way travel-time to
change.

     Process \texttt{SWELL} computes the average water bottom time of a group of
traces, then shifts the middle trace of the group by the difference of
the middle trace water bottom time and the average water bottom time.
For example, if the swell period is 15 traces, the time associated with
the seventh trace is subtracted from the average time of all 15 traces.
The time may be in any SEG-Y header location, but it must be in
seconds since the shift is converted to number of samples using the
SEG-Y sample interval.

     The water bottom time may be picked using process \texttt{WBT},  The SEG-Y
water bottom depth (long integer word 16) may be converted to time using
process \texttt{WBT} parameter \texttt{VEL}, or with process \texttt{HEADER}.


\subsection{Parameter Dictionary}

\begin{description}
    \item[\texttt{N}] The number of traces in the period of the swell to be removed.
        Either \texttt{N} or \texttt{WEIGHTS} is required.

\item[\texttt{HDR}] The index of the 32 bit floating point SEG-Y trace header word
         containing the water bottom time.  (The input value to average)
         \textit{e.g.} \texttt{HDR 60}
         Default = 50 (the SIOSEIS water bottom time)

\item[\texttt{INDEX}] The index of the 32 bit floating point SEG-Y trace header word
         containing the computed shift.  (How much the trace was shifted.)
         \textit{e.g.} \texttt{INDEX 60}
         Default = none

\item[\texttt{WEIGHTS}] A list of weights to use.  The total number of weights given
         indicates the number of traces in the swell period.
         Parameter \texttt{N} is ignored when \texttt{WEIGHTS} is given.
         Default = none

\item[\texttt{END}] Terminates each parameter list.
\end{description}

\section{SYN: Spike Trace Generation}
\label{cmd_syn}

Process \texttt{SYN} creates a ``spike'' synthetic trace in the form of a \gls{shot}
consisting of a group of traces containing either user given trace
values or seismic events.  The seismic events are spikes placed in time
and amplitude according to user specified parameters and may be either
reflected or refracted events.

Each trace is assigned a range and the spikes may then be altered from
trace to trace according to the inverse normal moveout.  In other words,
the user specifies a set of time-velocity-amplitudes for zero offset and
Process \texttt{SYN} then computes where the spikes are on each trace according
to the formula $tx=sqrt(t0*t0+(x/v)*(x/v))$ or $tx=abs(t0)+x/v$ where $t0$, $x$,
and $v$ are specified through user parameters.

Synthetic \glspl{shot} are commonly filtered by process filter before being
processed further.

The synthetic parameters may be spatially varied.  Each parameter list
must be terminated with the word \texttt{END}.  The entire set of syn parameters
must be terminated by the word \texttt{END}.

\subsection{Parameter Dictionary}

\begin{description}
    \item[\texttt{FNO}] The first \gls{shot} number to create with the current \texttt{SYN} parameters.
         Shot numbers must increase monotonically.
         \Gls{preset}=1

\item[\texttt{LNO}] The last \gls{shot} number to create with the current set of syn
         parameters.  \texttt{LNO} must greater than or equal to \texttt{FNO} in each list
         and must increase list to list.
         Default=\texttt{FNO}

\item[\texttt{NOINC}] The increment between \gls{shot} numbers.  \textit{e.g.} DO fno, lno, noinc
         \Gls{preset} = 1

\item[\texttt{TVA}] Time-velocity-amplitude tuples.  A set of triples describing
         the position of the spikes making up the 0 offset trace.  A
         negative time indicates that the event is a refracted event,
         whereas a positive time indicates a reflected event. A maximum
         of 30 triplets (90 numbers) may be given.
         \Gls{preset} = none

\item[\texttt{TTVA}] Type-time-velocity-amplitude tuples.  A set of triples
         describing the seismic event to create.
\begin{verbatim}
         Type = 1, a hyberbolic event:  tx = sqrt(t0**2 + x**2/v**2)
         Type = 2, a linear event: tx = t0 + x/v
         time = the two-way travel time of the event at x = 0.
\end{verbatim}
         \Gls{preset} = none
         \textit{e.g.} \texttt{TTVA 2 0 1500 1   1 .5 1500 1   1 .6 1550 1}

\item[\texttt{VALUES}] The list of trace values, each value separated from the other
         by a blank.  The values may be given on as many lines as
         necessary.  When the values are given in exponential format,
         positive exponents must have a + rather than a blank.  The
         first value given will be the first data value of the trace
         (\textit{i.e.} the first value will be at time delay).  A maximum of
         90 values may be given.  Use process \texttt{DISKIN} parameter
         \texttt{FORMAT ASCII} when more than 90 values are needed.
         \Gls{preset} = none        \textit{e.g.} \texttt{VALUES .5 1 .5 0 -.5 -.1E+1}

\item[\texttt{SI}] Sample interval in seconds.
         \Gls{preset}=.004

\item[\texttt{NTRCS}] Number of traces per \gls{shot}.
         \Gls{preset}=24

\item[\texttt{SECS}] Shot length in seconds.
         \Gls{preset}=6.

\item[\texttt{X}] The range of the first trace.  The units of x don't matter, so
         long as they are consistent with the velocities used.
         \Gls{preset}=0.

\item[\texttt{XINC}] The range increment from trace to trace.  Trace 1 will have a
    range of \texttt{x}, trace 2 will have a range of \texttt{x+xinc}, trace 3 will
         have a range of \texttt{x+2*xinc}, \ldots
         \Gls{preset}=100.

\item[\texttt{NTRGAT}] The number of traces per \gls{gather}.  When given non-zero, every
         \texttt{NTRGAT} trace will be flagged as the end of a \gls{gather} (the 51st
         \texttt{INTEGER*4} word in the trace header).  Other traces will have
         the flag cleared to zero.  The use of ntrgat also sets the
         binary tape header sort flag to 2 (sorted by \gls{rp}) and the \gls{rp}
         number (word 6) is set to the \gls{shot} number (word 3).  The \gls{shot}
         number and \gls{shot} trace number (words 3 and 4) are set to 0 and the
         \gls{rp} trace number (word 7) is set to the trace count in \texttt{NTRGAT}.
         \Gls{preset}=none

\item[\texttt{DELAY}] The deep water delay.  The value for delay is inserted into the
         SEG-Y trace header after the seismogram is created.  The times
         specified in \texttt{TVA} must reflect the delay.
         \textit{e.g.} \texttt{TVA .1 10000 1 DELAY 1}  puts a spike at time 1.1
         \Gls{preset}=0.

\item[\texttt{NOISE}] The level of white noise to add to the trace is made with
         \texttt{VALUES} or \texttt{TVA}.  The level given multiplies the unit variance
         trace.  The seed for each trace is the \gls{shot} number * 1000 plus
         the trace number.  See the ``Numerical Recipes (Fortran)'' function
         \texttt{GASDEV} for the precise algorithm used, ``Normally distributed random deviates'', Section~7.2 of \cite{WHPress1989a}.
         \Gls{preset} = 0.

\item[\texttt{EXTHDR}] The number of SEG-Y Rev 1 Textual Extension Records to create.
         All records will be blank, except the last one which has the
         ``((EndText))'' stanza.  Also see parameter \texttt{EXTHDR} in process
         \texttt{DISKOX}.
         \Gls{preset} = 0        \textit{e.g.}   \texttt{EXTHDR 1}

\item[\texttt{END}] Terminates each parameter list.
\end{description}

\section{TP2TX: Tau-p to Time-Distance Domain Transformation}
\label{cmd_tp2tx}

Process \texttt{TP2TX} transform data in the \gls{not:tau}-p space domain to the
time-distance space using the ``HOP'' method for slant stack (see
process \texttt{TX2TP}).  Process \texttt{TX2TP} must have been run prior to \texttt{TP2TX},
preferably in the same \texttt{PROCS} list since many \texttt{TP2TX} parameters are
set by process \texttt{TX2TP} and are lost in they are not in the same list.

\subsection{Parameter Dictionary}

\begin{description}
\item[\texttt{SETAU}] The start and end taus to input.
         \Gls{preset} = The start and end taus of the TAU-P data if \texttt{TX2TP} is
                  also a process, otherwise calculated from the data.

\item[\texttt{SET}] The start and end times of the output t-x data.
         \Gls{preset} = The start and end times of the TX data if \texttt{TX2TP} is
                  also a process.

\item[\texttt{SEX}] Start and end ranges (x) to calculate and output.
         \Gls{preset} = The start and end ranges of the TX data if \texttt{TX2TP} is
                  also a process.

\item[\texttt{NX}] The number of ranges to calculate.  \texttt{NX} is the increment
         between the start and end ranges (\texttt{SEX}).
         \Gls{preset} = the number of input time traces if \texttt{TX2TP} is
                  also a process.

\item[\texttt{IHPATH}] Input header pathname.  \texttt{TP2TX} will ouput the data with the
         SEG-Y trace headers from disk file \texttt{IPATH} rather than from the
         input \gls{not:tau}-p traces.
         \Gls{preset} =  none

\item[\texttt{PCNTO}] Percent taper applied to models before the inverse \gls{fft}.
         \Gls{preset} = 0.
\end{description}

\section{TREDIT: Trace Edit and Spike Removal}
\label{cmd_tredit}

Processes \texttt{DESPIKE} and \texttt{TREDIT} are identical see Section~\ref{cmd_despike}.

\section{TX2FK: Time-Distance to Frequency-Wavenumber Domain Transformation}
\label{cmd_tx2fk}

Process \texttt{TX2FK} transforms data from the TX (time-space) domain into the
F-K (frequency-wavenumber) domain by performing a 2-D \gls{fft}.  The input
is a set of normal (t-x) seismic traces and the output is a transformed
set of F-K traces. The transformed output traces are in rectangular
form unless polar form (amplitude and phase) is requested.

The frequencies within each output trace are ordered from -Nyq to 0 to
Nyq.  Each output trace contains a power of 2 number of samples. The
traces are ordered in wavenumber from 0 to K (Nyquist). Data in polar
form are ordered with the modulus followed by the argument.

The sample interval is in hundreds of microhertz in order to prevent
truncation problems.  \textit{e.g.} an SEG-Y trace header interval value of 610
is really 610/10,000. or .0610.  See doc/fk.forum for further discussion.

The input traces must be sorted by increasing range (SEG-Y header
word 10) and must be separated uniformly.  \texttt{DELTA X} must be positive
and must be constant.  The only exception is when ``super-\glspl{gather}''
are to be used (see parameter nprestk).

Any SIOSEIS process may follow \texttt{TX2FK}, but care should be taken that it
makes sense!  The imaginary part of data in rectangular coordinates may
be omitted from plot by decimating by a factor of 2.

The steps used in the F-K transformation are:
\begin{enumerate}
    \item  Each trace is windowed temporally in order to minimize edge effects
     along the time trace.
 \item  The data are transposed so that all the data of constant times are
     adjacent. This sorts the data by x rather than t (or f).
 \item  The 'range' traces are zero padded. Effectively adding dummy traces
     to the end of the dataset. The data must be padded to a power of 2,
     but additional padding may be desirable for migration \textit{etc.} A user
     minimum number of dead traces is specified by nxpad. The data are
     also windowed in x if desired.
 \item  The complex \gls{fft} is performed, transforming x to k, or from space to
     wavenumber.
 \item  The data are transposed back to ordering by time.
 \item  The forward \gls{fft} is performed, converting time to frequency.  The
     maximum number of samples in time is 8192 real samples.
 \item  The data are converted to polar coordinates if requested.  Data in
     the fk domain may be processed by any other seismic process in
     SIOSEIS.
\end{enumerate}

The run time of \texttt{TX2FK} is governed by the number of traces, including pads,
and the trace length, including pads.  Both dimensions are a power of 2,
so transforming 1500 points takes as long as 2000 points (2048 being the
closest larger power of 2).

The frequency-wavenumber domain is discussed in a paper "A Review of the
Two-Dimensional Transform and Its Use in Seismic Processing" by
D.W. March and A.D.Bailey in the ``First Break'', January 1983 \cite{March1983}.

Each parameter list must be terminated with the word \texttt{END}.  The entire set
of \texttt{TX2FK} parameters must be terminated by the word \texttt{END}.

\subsection{Parameter Dictionary}

\begin{description}
\item[\texttt{STIME}] The start time of the data for the entire data set.  Any trace
          that has an initial time (delay) greater than \texttt{STIME} will be zero
          padded so that the data starts at \texttt{STIME} .  Any trace that has a
          delay less than stime will be shortened.
          \Gls{preset} = the delay of the first trace.

\item[\texttt{ETIME}] The end time of the data for the entire data set. Data in excess
          of \texttt{ETIME} will be omitted from the transformation.
          \Gls{preset} = the last time of the first trace.

\item[\texttt{NXPAD}] The number of dummy traces to insert at both ends of the seismic
          line.  Process \texttt{FKMIGR} needs dummy traces in order to prevent
          ``wrap around''.
          \Gls{preset} = 10

\item[\texttt{TWINDOW}] The type of window to apply before computing the temporal \gls{fft}.
         \Gls{preset}=\texttt{HANN}  \textit{e.g.} window rect
\begin{description}
\item[\texttt{HAMM}] Hamming
\item[\texttt{HANN}] Hanning
\item[\texttt{GAUS}] Gaussian
\item[\texttt{BART}] Bartlett (triangular)
\item[\texttt{RECT}] rectangular (box car - no window)
\item[\texttt{BLAC}] Blackman
\item[\texttt{EBLA}] exact  Blackman
\item[\texttt{BLHA}] Blackman-Harris
\end{description}

\item[\texttt{TWINLEN}] The window length, in samples.  A window length of zero causes
          the entire time domain trace to be windowed.  A non zero length
          indicates that \texttt{WINLEN} data will be modified at both ends of
          each trace.
          \Gls{preset} = 25  \textit{e.g.} \texttt{WINLEN 50}

\item[\texttt{XWINDOW}] Same as \texttt{TWINDOW} but it windows the data by range.
          \Gls{preset} = \texttt{RECT} ( No windowing of ranges )

\item[\texttt{XWINLEN}] The window length in number of traces. A window length of zero
          causes the entire set of ranges to be windowed. A non-zero
          length causes that number of traces to be tapered at the
          binning and end.
          \Gls{preset} = 10

\item[\texttt{COORDS}] The coordinates of the output trace.
         \Gls{preset} = \texttt{RECT}   \textit{e.g.}  \texttt{COORDS POLAR}
\item[\texttt{RECT}] Rectangular coordinates.  The output trace will be complex.
         The trace values will consist of real and imaginary pairs.  The
         frequency dimension runs from 0 to Nyquist (pi) then the
         negative frequencies back to 0, as output from the \gls{fft}.  The
         number of samples in the output frequency trace is a power of
         two.  The k dimension is also ordered as it comes directly out
         of the \gls{fft}, 0 to Nyquist followed by the most negative k.
\item[\texttt{POLAR}] Polar coordinates.  The first half of the output trace
         will be the amplitude spectrum and the second half of the trace
         will be the phase spectrum, each ordered 0 to Nyquist back to 0.
         The k dimension is reordered so that the first trace is the
         negative Nyquist, then increasing to 0, then increasing to the
         positive Nyquist.  The total number of traces output is a power
         of two plus 1.
\item[\texttt{POLARU}] ``user friendly'' polar coordinates. This produces output
         data as a series of wavenumber traces running from -Nyq -> +Nyq
         and only positive frequencies in polar form (magnitude followed
         by phase).  This format is useful if it is desired to look at
         the FK data using for example a contouring program.  The data are
         not used internally in this form but is converted on input/output
         from SIOSEIS.

\item[\texttt{OHDRPATH}] If processes \texttt{TX2FK} and \texttt{FK2TX} are called in the same job then
         the original TX headers will be used as the trace headers of the
         the processed TX data. This will preserve all RP/Shot numbering
         as well as GMT information. However, if these two processes are
         done in separate jobs then the user may specify a permanent disk
         file to hold the original TX headers. This filename can then be
         given to \texttt{FK2TX} when the inverse transform is done.

\item[\texttt{PATH1 / PATH2}] Before \texttt{TX2FK} transforms the input traces it accumulates
         them them in a scratch datafile.  The size of this file is the
         number of input traces prior to padding in range * the sample
         length of each trace (etime - stime) prior to padding in time.
         The size of the data after transformation is equal to the number
         of traces * trace length after padding of each to the nearest
         power of 2. This is stored in a second scratch dataset after
         which the first scratch file is deleted. If there is a \texttt{FK2TX} in
         the procs list then it will use this second scratch file as its
         first scratch file.  The default location of these files is
         implementation dependent and may be of insufficient size. The
         \texttt{PATH} parameters allow the user to specify the location of the
         two scratch files.

\begin{description}
\item[\texttt{PATH1}] The location of the first scratch file.

\item[\texttt{PATH2}] The location of the second scratch file. Since this is the same as
         the first scratch file used by \texttt{FK2TX} (if present) it should be
         given only here or in \texttt{FK2TX}. If contradictory definitions are
         given the first on processed will be used.
\end{description}

\item[\texttt{NPRESTK}] The number of \glspl{rp} to use in each fk transformation.  Each \gls{rp} is
         terminated with a -1 in SEG-Y trace header word 51.  Processes
         \texttt{SORT} and \texttt{GATHER} set this ``end-of-sort'' flag.  Shot data may set
         this flag with process \texttt{INPUT} or \texttt{DISKIN} parameter \texttt{NTRGAT}.  The data
         are sorted by increasing range prior to transforming x to k.
         Process fk2tx will automatically ``unsort'' the data into the
         original order when nprestk is greather than 1.

\item[\texttt{END}] Terminates each parameter list.
\end{description}

\section{TX2TP: Time-Distance to Tau-p Domain Transformation}
\label{cmd_tx2tp}

Process \texttt{TX2TP} transforms data in the time-space domain to the \gls{not:tau}-p
space using the ``HOP'' method for slant stacking refraction data by
Henry, Orcutt, and Parker (GRL, Dec 1980) \cite{tau_p_1980}. The problem of slant-stacking
seismic records at a number of ranges to synthesize a \gls{not:tau}-p curve is
posed as a linear inverse problem for fixed frequency.  Using an inner
product weighted by
\begin{equation}
    k^{2} + b^{2}) k
\end{equation}
 (where $k$ is wavenumber and $b$ some real positive number), then the
 representers are Bessel functions of $k \times$range, scaled by
\begin{equation}
\dfrac{1}{(k^{2} + b^{2})},
\end{equation}
and the model is $U(k,w)$ (vert comp only).  The
inverse of the Gram matrix can be found analytically as sums and
differences of products of modified Bessel functions of $b \times$range.

Since the tx to \gls{not:tau}-p transformation changes the meaning of \glspl{shot} and
\glspl{rp}, \texttt{TX2TP} will output the data as if it is a \gls{shot} containing np traces
and \textbf{ALL THE TX DOMAIN TRACE INFORMATION IS LOST}.  The SEG-Y trace
headers will contain the following information:

\begin{itemize}
\item record (``gls{shot}'') number   (SEG-Y word 3)
\item trace number with the record  (SEG-Y word 4)
\item trace id flag (live trace flag)  (SEG-Y word 15)
\item $p \times 1000$. (in the tx domain position for range)  (SEG-Y word 10)
\item $\tau \times 1000$ of the first output sample (delay in ms) (SEG-Y word 55)
\item \gls{not:tau} of the first output sample (delay in seconds) (SEG-Y word 46)
\item number of data samples (SEG-Y word 58)
\item sample interval in millihertz (instead of microseconds) (SEG-Y word 59)
\end{itemize}

The input data must be order so that the ranges are monotonically
increasing.

Limitations :
\begin{itemize}
\item 300 input time traces
\item 1024 samples within a time trace
\item 300 output $p$ traces
\item 400 $\tau$'s within a $p$ trace
\end{itemize}


\subsection{Parameter Dictionary}

\begin{description}
\item[\texttt{SET}] The start and end times of the data to transformation.  Data
         prior to the start time or after the end time will be omitted.
         \Gls{preset} = the times of the first input trace.  \textit{e.g.}  \texttt{SET 3 7}

\item[\texttt{SEP}] Start and end p values to calculate.  The \texttt{SEP} range should only
    span the dip components of interest (Yilmaz pg 437; see also \cite{Yilmaz2001}).  The p
         values are in units of kilometers - \texttt{TX2TP} converts the SEG-Y
         ranges from meters to kilometers.
         \Gls{preset} = .2 .7  (1/5 1/1.4)   \textit{e.g.}  \texttt{SEP .05 .55}

\item[\texttt{SETAU}] The start and end tau's to calculate.  The first output sample
         will be at the start tau.
         \Gls{preset} = delay  2*(power of two > nsamps) + 1

\item[\texttt{NP}] The number of p values to calculate.  \texttt{NP} is the increment
         between the start and end p values.
         \Gls{preset} = the number of input time traces

\item[\texttt{FC}] Cutoff frequency for calculating models.
         \Gls{preset} = Nyquist frequency      \textit{e.g.} \texttt{FC 125}

\item[\texttt{PCNTI}] Percent taper applied to the input data before the forward \gls{fft}.
         \Gls{preset} = 0

\item[\texttt{PCNTO}] Percent taper applied to the models before the inverse \gls{fft}.
         \Gls{preset} = 0

\item[\texttt{OHPATH}] Output header pathname.  \texttt{TX2TP} saves all of the SEG-Y trace
         headers in a disk file so that the output \gls{not:tau}-p traces have the
         same headers as the input.  The disk file is temporary unless
         \texttt{OHPATH} is given.  This is particularly important if the data are
         converted back to tx in a separate SIOSEIS run.
         \Gls{preset} = temporary file.    \textit{e.g.}   \texttt{OHPATH txheaders.line1-2}

\item[\texttt{SHIFT}] The time shift to apply to successive traces.  The shift is
         accumulative, thus each trace is shifted by \texttt{SHIFT} relative to
         the previous trace.  The shift represents a constant reduction
         velocity, \textit{i.e.} start time for each trace must be delayed by
         same amount relative to previous trace.  The input traces must
         have a constant range increment for this to work properly.
         Required.              \textit{e.g.}  \texttt{SHIFT .05}

\item[\texttt{PRESTK}] When set nonzero, the transformation is done whenever the
         \texttt{PRESTK} number of ``\glspl{gather}'' have been collected.  A ``\gls{gather}'' is
         whenever the ``end-of-sort'' flag (SEG-Y header word 51) is -1.
         Processes \texttt{SORT} and \texttt{GATHER} set the ``end-of-sort'' flag.
         Other schemes using \texttt{DISKIN} and \texttt{HEADER} can be used also.
         \Gls{preset} = 0.
\end{description}

\section{T2D: Time to Depth Conversion}
\label{cmd_t2d}

\texttt{T2D} transforms time domain traces to depth domain traces using user
defined velocity functions.  The velocities may be varied temporally and
spatially.  The velocities may be average velocities or interval
velocities.  Interval velocities are converted to average velocities.
Average velocity depth conversion utilizes the formula
\begin{equation}
    d = \dfrac{t}{2v}
\end{equation}
whereas interval velocity depth conversion is

\begin{equation}
    d_{n+1} = \dfrac{(t_{n+1} - t_{n})}{2 v_{n+1}}
\end{equation}

The velocity function may be varied spatially between velocity control
points.  Velocity control points are defined as either \gls{shot} point numbers
or \gls{rp} numbers. Each parameter list has a start and end control point with
the parameters being constant for all points between the first and last
control point of the list.  For example if the first control point=100
and the last control point=110, the points 101 to 109 will also have the
same parameters applied.

\textit{e.g.}
\begin{verbatim}
FNO 1 VTP 2100 1.1 3100 2.1 END
FNO 3 VTP 2300 1.3 3300 2.3 END
FNO 4 VTP 2200 1.0 2500 1.5 3000 2.0 END
\end{verbatim}

Results in
\begin{verbatim}
FNO 2 vtp 2200 1.2 3200 2.2
\end{verbatim}

The average velocity function is evaluated for every time sample for the
control points.  Traces between velocity control points receive a
velocity function that is the linear interpolation of the velocities of
the corresponding times on the adjacent control points.

Spatial variation is by \gls{shot} if the data are sorted by \gls{shot} and is varied
by \gls{rp} if the data are sorted by \gls{rp}.

The output sample interval is changed by Process \texttt{T2D} to whatever parameter
\texttt{OSI} is set to.  \texttt{OSI} 1 means 1 millisecond (0.001 second), which works very
for sub-bottom profilers using a time domain sample interval around 50
microseconds. Thus, in order to plot depth data, use plot parameters for 1
mil time data.

The output units of \texttt{T2D} are meters.  Whenever the SEG-Y header uses
milliseconds, \texttt{T2D} converts it to meters. Whenever the SEG-Y header uses
seconds, \texttt{T2D} converts it to kilometers.

Each parameter list must be terminated with the word \texttt{END}.  The entire set
of t2d parameters must be terminated by the word \texttt{END}.

Example 1 - sub-bottom profiler:
\begin{verbatim}
T2D
    VTYPE AVE VTP 1475 0 END
END
\end{verbatim}

Example 2:
\begin{verbatim}
T2D
     SDEPTH 0 EDEPTH 20000 OSI 10 VTYPE AVE
     FNO 1 VTP 1500 2.2 2200 4.5 3500 6.7 4500 7.0 END
     FNO 223 VTP 1500 2.9 2100 4.2 3500 6.6 4500 7.0 END
     FNO 253 VTP 1500 3.0 2100 4.9 3500 6.5 4500 7.0 END
END
\end{verbatim}

\subsection{Parameter Dictionary}

\begin{description}
\item[\texttt{FNO}] The first \gls{shot} (or \gls{rp}) to apply the velocities to.  Shot (\gls{rp})
         numbers must increase monotonically.
         \Gls{preset}=1

\item[\texttt{LNO}] The last \gls{shot} (\gls{rp}) number to apply the velocities to.  \texttt{LNO} must
         be larger than \texttt{FNO} in each list and must increase list to list.
         Default=\texttt{FNO}
\item[\texttt{VTYPE}] The type of velocity function given in the \texttt{VTP} parameter.
         \Gls{preset} = \texttt{INT}.  \textit{e.g.}  \texttt{VTYPE AVE}
\begin{description}
\item[\texttt{AVE}]  average velocities.
\item[\texttt{INT}]  interval velocities.
\end{description}

\item[\texttt{VTP}] Velocity-Time-Pairs.  A list of velocity and two-way travel time
         (in seconds) pairs.  \texttt{VTP} must be given in each \texttt{T2D} parameter
         list.  A maximum of 25 pairs may be given.  Data times before
         the first given time in vtp will receive the first given
         velocity.  Likewise, data times exceeding the last given time in
         \texttt{VTP} will receive the last velocity given in \texttt{VTP}.
         Default=none. \textit{e.g.}  \texttt{VTP 1490 1.0 2000 2.0}

\item[\texttt{SDEPTH}] The start depth of the output depth section.  The SEG-Y header
         is changed so that the deep water delay = sdepth.  All traces will
         have the same delay when \texttt{SDEPTH} is used.  If \texttt{SDEPTH} is not given,
         the delay will be converted from time to depth on every trace.
         \Gls{preset} = none    \textit{e.g.}  \texttt{SDEPTH 1000.}

\item[\texttt{EDEPTH}] The end depth of the output depth section.  All output traces
         will have the same end depth when \texttt{EDEPTH} is given.  \texttt{SDEPTH} must
         be used when \texttt{EDEPTH} is given.  If \texttt{EDEPTH} is not given, the output
         trace will have the same number of samples as the input time trace.
         \Gls{preset} = none     \textit{e.g.}  \texttt{EDEPTH 8000}

\item[\texttt{OSI}] The output sample interval expressed as the distance between
         samples.  The output sample interval should be similar to time
         sample  intervals.  \textit{e.g.}  osi 4 gets written in the SEG-Y as 4
         meters whereas in time it would be 4 mils.
         \Gls{preset} = 1. (Meters).  \textit{e.g.}  \texttt{OSI 10.}

\item[\texttt{ADDWB}] When given a value of \texttt{YES}, the water bottom time is added to the
         velocity function after spatial variation has been done.
         \Gls{preset}=\texttt{NO}

\item[\texttt{END}] Terminates each parameter list.
\end{description}

\section{T2F: Time to Frequency Transformation}
\label{cmd_t2f}

Process \texttt{T2F} converts traces from the time domain to the frequency domain.
Each trace is windowed, padded with zeroes to a power of 2, transformed,
and optionally converted to polar coordinates.  Traces left in rect-
angular coordinates are in complex form such that the imaginary part of
each frequency immediately follows the real (real-imaginary pairs).

Traces in polar form have the amplitude spectrum in the first half of the
output trace while the wrapped phase spectrum is in the second half of
the trace.

Each output trace will contain a i((power of 2) + 2) number of samples and is
the number of samples used in the \gls{fft} (\texttt{fftlen+2}).  There are \texttt{fftlen/2+1}
frequencies in the output trace representing frequencies 0 (DC) to the
Nyquist frequency $\frac{1}{2 * \mbox{Sample interval}}$.

Any seismic process may be used on frequency data, but some post \texttt{T2F}
processes may not make geophysical sense.  Frequency domain plots may be
made using polar coordinates and plotting the first half of the trace.

Bandpass filter and deconvolution should be done in the time domain using
processes \texttt{FILTER} and \texttt{DECON} before the frequency domain processes.

A good review of the frequency domain is ``A Guided Tour of the Fast Fourier Transform'' by G.D.Bergland in ``IEEE Spectrum'', July 1969. \cite{fft_1969}

\subsection{Parameter Dictionary}

\begin{description}
\item[\texttt{STIME}] Start time, in seconds, of the data to transform.  The use of
         stime causes the the deep water delay to be set to 0. and the
         first time sample on an inverse transform (process \texttt{F2T}) to be 0.
         \Gls{preset} = the delay of the trace.  \textit{e.g.}  \texttt{STIME 3.3}

\item[\texttt{ETIME}] End time, in seconds, of the data to transform.  The data after
         etime will be padded with zeroes in order for the data length be
         a power of 2.
         \Gls{preset} = the entire data trace.    \textit{e.g.}  \texttt{ETIME 4.0}

\item[\texttt{ADDWB}] When given a value of \texttt{YES}, the windows given via sets will
        be added to the water bottom time of the trace.
        (Water bottom times may be entered via process wbt).
        \Gls{preset}=\texttt{NO}     \textit{e.g.} \texttt{ADDWB YES}

\item[\texttt{WINDOW}] The type of window to apply before computing the \gls{fft}.
        \Gls{preset}=\texttt{HAMM}  \textit{e.g.} \texttt{WINDOW RECT}
\begin{description}
\item[\texttt{hamm}] Hamming
\item[\texttt{hann}] Hanning
\item[\texttt{bart}] Bartlett (triangular)
\item[\texttt{rect}] rectangular (box car - no window)
\item[\texttt{blac}] Blackman
\item[\texttt{ebla}] exact Blackman
\item[\texttt{blha}] Blackman-Harris
\end{description}

\item[\texttt{WINLEN}] The window length, in seconds.  A window length of zero causes
         the entire time domain gate to be windowed.  A non zero length
         indicates that winlen data will be modified at both ends of each
         data gate.
         \Gls{preset}=0  \textit{e.g.} \texttt{WINLEN .2}

\item[\texttt{COORDS}] The coordinates of the output trace.
\item[\texttt{RECT}] rectangular coordinates.  The output trace will be complex.
         The trace values will consist of real and imaginary pairs.  The
         use of rectangular coordinates is slightly faster than polar
         coordinates, and is understood by frequency domain processes
         such as \texttt{F2T}.
\item[\texttt{POLR}] polar coordinates.  The first half of the output trace
         will be the full ($-\pi$ to $+\pi$) amplitude spectrum and the second
         half of the trace will be the phase spectrum.
\item[\texttt{AMPL}] amplitude spectrum.  Only the positive portion ( 0 to $\pi$ )
         of the amplitude spectrum is output.
         \Gls{preset} = \texttt{RECT}   \textit{e.g.}  \texttt{COORDS POLR}

\item[\texttt{FFTLEN}] The number of points to use in the \gls{fft}.  \texttt{T2F} will make this a
         power of 2 if it is not.
         \Gls{preset} = the number of points in the trace.  \textit{e.g.} \texttt{FFTLEN 1024}

\item[\texttt{END}] Terminates each parameter list.
\end{description}

\section{UADD: User Controlled Addition to the Seismic Trace}
\label{cmd_uadd}

Process \texttt{\texttt{UADD}} adds user given points to every specified seismic trace.
The user may specify the start time of the addition.

A common use of process uadd is to subtract the average trace from all
the data traces in \gls{gpr} (\gls{gpr}).

Spatial interpolation is not available.

Each parameter list must be terminated with the word \texttt{END}.  The entire
set of \texttt{UADD} parameters must be terminated by the word \texttt{END}.

\subsection{Parameter Dictionary}

\begin{description}
\item[\texttt{PTS}] A list of points to be added to each seismic trace.  The
        values must be separated by a blank, tab or new line.
        The points may be given on as many lines as necessary.
        When the values are given in an exponential format,
        positive exponents must have a \texttt{+} rather than a blank.
        Required. \textit{e.g.} \texttt{PTS 1 2 3 4 5 6 7 8 9 10 11}

\item[\texttt{FNO}] The first \gls{shot} (or \gls{rp}) to be added.  Shot (\gls{rp}) numbers
        must increase monotonically.
        \Gls{preset} = 0

\item[\texttt{LNO}] The last \gls{shot} (\gls{rp}) number to be added.  \texttt{LNO} must be
        larger than \texttt{FNO} in each list and must increase list to list.
        \Gls{preset} = 999999999

\item[\texttt{STIME}] The start time, in seconds, of the addition.  \textit{i.e.} the time of
        the first point given in \texttt{PTS}.
        \Gls{preset} = the delay of the trace . \textit{e.g.} \texttt{STIME 1}

\item[\texttt{END}] Terminates each parameter list.
\end{description}

\section{UDECON: User Controlled Weiner Filter Design}
\label{cmd_udecon}

\texttt{UDECON} calculates and applies a Wiener deconvolution filter given a
series of input and desired output values or the cross-correlation and
auto-correlation functions.  The output appears in the ``standard output''
(print file) and may be the input, output, cross-correlation, auto-
correlation, the Wiener filter coefficients, the Wiener-Levinson error,
the deconvolved input, or the rms error of the deconvolved and input
series.

No standard SIOSEIS input or output is available.  All input and output
is within this process.  \textit{i.e.} the procs list should be \texttt{PROCS UDECON END}

Each parameter list must be terminated with the word \texttt{END}.  The entire set
of udecon parameters must be terminated by the word \texttt{END}.

\subsection{Parameter Dictionary}

\begin{description}
\item[\texttt{INPUT}] A list of input values representing a seismic trace.  The input will be correlated with itself and with the desired output.

\item[\texttt{DESIRE}] A list of desired output values representing the desired output trace.

\item[\texttt{AUTO}] A list of auto-correlation values that will be used instead of the input and desired output.

\item[\texttt{CROSS}] A list of cross-correlation values.

\item[\texttt{OPRINT}] A bit switch indicating the printer output.  Bit $x$ is $2^{x}$.
         \Gls{preset} = 0     \textit{e.g.}
\begin{verbatim}
OPRINT 32    sets bit 5 only
OPRINT 96    sets bits 5 and 6
OPRINT 127   sets all 7 bits
\end{verbatim}
\begin{description}
    \item[bit 0]  The cross-correlation is printed.
    \item[bit 1]  The auto-correlation is printed.
    \item[bit 2]  The prediction filter is printed.
    \item[bit 3]  The Wiener-Levinson error is printed.
    \item[bit 4]  The deconvolved input is printed.
    \item[bit 5]  The \gls{rms} error between the deconvolved input and the inputs printed.
\end{description}

\item[\texttt{NFPTS}] The number of filter points to use.  \texttt{NFPTS} must be larger than
         the period to be removed.  \texttt{NFPTS} is also the number of lags in
         the correlations.  If \texttt{NFPTS} is omitted when the autocorrelation
         is given, \texttt{NFPTS} is set to the number of points in the autocorrelation.

\item[\texttt{PREWHI}] The percentage prewhitening to add before filter design.  A
         high level of prewhitening reduces the effectiveness of the
         filter.  Some level of prewhitening is needed in order for the
         filter to be stable.  Prewhitening is like performing a bandpass
         filter before \texttt{DECON}.
         \Gls{preset}=25.

\item[\texttt{END}] Terminates each parameter list.
\end{description}

\section{UFILTR: User Given Impulse Response Filter}
\label{cmd_ufiltr}

Process \texttt{UFILTR} applies user given filter points to every seismic trace.
The filtering is done as convolution in the time domain. The user may
specify the time origin of the filter by specifying the amount of shift
to apply after convolution.  Zeroes are inserted before and after the
data trace so that full convolution is performed without program buffer
edge effects.

One use of process \texttt{UFILTR} is in applying a single compensation filter
to all data.  Another use is in applying a Hilbert transform or first
derivative filter $(1,-1)$.

Each parameter list must be terminated with the word \texttt{END}.  The entire
set of \texttt{UFILTR} parameters must be terminated by the word \texttt{END}.

\subsection{Parameter Dictionary}

\begin{description}
\item[\texttt{FILPTS}] A list of filter points, each value separated from the other
         by a blank.  The points may be given on as many lines as
         necessary. When the values are given in an exponential format,
         positive exponents must have a \texttt{+} rather than a blank.
         Required. \textit{e.g.} \texttt{FILPTS .5 .5}

\item[\texttt{FNO}] The first \gls{shot} (or \gls{rp}) to apply the filter to.  Shot (\gls{rp})
         numbers must increase monotonically.
         \Gls{preset}=1

\item[\texttt{LNO}] The last \gls{shot} (\gls{rp}) number to apply the filter to.  \texttt{LNO} must be
         larger than \texttt{FNO} in each list and must increase list to list.
         Default=\texttt{FNO}

\item[\texttt{NSHIFT}] The number of samples to shift the data after convolution.  A
         negative shift moves the data to the left and \texttt{NSHIFT} data
         values are dropped from the beginning.  A positive shift moves
         the data to the right and \texttt{NSHIFT} zeroes are inserted at the
         beginning of data.  A zero phase filter should have a shift of
         \texttt{-NFPTS/2}.
         \Gls{preset}=0. \textit{e.g.} \texttt{NSHIFT -15}

\item[\texttt{END}] Terminates each parameter list.
\end{description}

\section{UMULT: User Controlled Multiplication with the Seismic Trace}
\label{cmd_umult}

Process \texttt{UMULT} multiplies user given points to every specified seismic
trace.  The user may specify the start time of the multiply.

A common use of process \texttt{UMULT} is to apply a different gain than
allowed in process gains.

Spatial interpolation is not available.

Each parameter list must be terminated with the word \texttt{END}.  The entire
set of \texttt{UMULT} parameters must be terminated by the word \texttt{END}.

\subsection{Parameter Dictionary}

\begin{description}
\item[\texttt{PTS}] A list of points to multiply the seismic trace by.  The values
        must be separated by a blank, tab, or new line.  When the
        values are given in an exponential format, positive exponents
        must have a \texttt{+} rather than a blank.
        Required. \textit{e.g.} \texttt{PTS 1 2 3 4 5 6 7 8 9 10 11}

\item[\texttt{FNO}] The first \gls{shot} (or \gls{rp}) to be multiplied.  Shot (\gls{rp}) numbers
        must increase monotonically.
        \Gls{preset} = 0

\item[\texttt{LNO}] The last \gls{shot} (\gls{rp}) number to multiply.  \texttt{LNO} must be larger
        than \texttt{FNO} in each list and must increase list to list.
        \Gls{preset} = 999999999

\item[\texttt{STIME}] The start time, in seconds, of the first point of each trace
        to be multiplied.  \textit{i.e.} the time of the first point in \texttt{PTS}.
        \Gls{preset} = the delay of the trace . \textit{e.g.} \texttt{STIME 1}

\item[\texttt{END}] Terminates each parameter list.
\end{description}

\section{VELAN: Constant Velocity or Semblance Velocity Analysis}
\label{cmd_velan}

A velocity analysis is an aid in determining the stacking velocities to
use in normal moveout.  There are currently 6 types of analysis
available in this package, the first 3 for reflectors and the last 3
for refractors.
\begin{enumerate}
\item constant velocity \glspl{gather}
\item constant velocity stack
\item semblance spectra
\item constant p \gls{not:tau}-p \glspl{gather}
\item tau-sum or slant stack
\item \gls{not:tau}-p semblance spectra
\end{enumerate}

\subsection{Constant Velocity Gathers}

A constant velocity \gls{gather} velocity analysis usually consists of all
the traces of a single \gls{rp} (reflection point) moved-out with a number of
different velocities that do not vary in time.

This method of analysis is useful for examining exactly how the data
aligns at different velocities.  The velocity is written into every
SEG-Y trace header in short integer word 46 (where 1 is the first).

Example: for 24 fold data a single \gls{rp} with vels 5000 200 7000 would
output the first 24 traces moved out with a velocity of 5000.  The next
24 traces would be moved-out with a velocity of 5500. The following 24
traces would be moved-out with a velocity of 6000, \textit{etc.}

\subsection{Constant Velocity Stack}

A constant velocity stack is a suite of stacked \glspl{gather}, usually several
adjacent \glspl{rp}, moved-out with velocities that do not vary with time.
This method of velocity analysis is useful because it shows the actual
stack of the data using the normal nmo and stack routines.  The velocity
is written into every SEG-Y trace header in short integer word 46
(where 1 is the first).

To obtain a constant velocity stack from this software package, process
stack must follow process velan and the \texttt{CVEL} option must be used. By
doing this, the constant velocity \glspl{gather} made by velan will be stacked.

Example:
\begin{verbatim}
  PROCS INPUT VELAN STACK OUTPUT END
  VELAN NRP 6 VELS 1800 50 2000 100 2500 END
  END
\end{verbatim}

will produce the following output:

\begin{verbatim}
     Traces   1-6, stack of the 6 rps at 1800 m/sec.
     Traces  7-12, stack of the 6 rps at 1850.
     Traces 13-18, stack of the 6 rps at 1900.
     Traces 19-24, stack of the 6 rps at 1950.
     Traces 25-30, stack of the 6 rps at 2000.
     Traces 31-36, stack of the 6 rps at 2100.
     Traces 37-42, stack of the 6 rps at 2200.
     Traces 43-48, stack of the 6 rps at 2300.
     Traces 49-54, stack of the 6 rps at 2400.
     Traces 55-60, stack of the 6 rps at 2500.
\end{verbatim}

The output is normally displayed with a space between the traces with
different velocities.

\subsection{Semblance Spectra}

A semblance velocity spectra is the printer plot of the semblance of
trace windows moved-out with constant velocities.  Semblance is the
ratio of the energy of the output trace (the stacked trace) and the mean
energy of the input traces.  The stacked trace is obtained by summing
all the traces of the \gls{rp} after moving them out. The move-out is a little
different from conventional \texttt{NMO} since the entire window receives the
same \texttt{NMO} as the center of the window.  The mean energy is simply the
sum of the energies of all the corresponding input windows.  Successive
windows are spaced half a window away, thus the windows overlap.

More than one \gls{rp} may be included in the spectra by using the parameter
\texttt{NRP}.  In this case, all the input \glspl{rp} (traces) are treated as one larger
\gls{rp}, as if smearing the subsurface.  This may increase the signal to
noise ratio in areas of negligible dip.

Velocity spectra are discussed in a paper by Terry Taner of
Seiscom-Delta in Geophysics, December 1969 \cite{Taner1969}.  Semblance is discussed in a
paper by Taner and Neidell of Seiscom in Geophysics, June 1971 \cite{Neidell1971}.

\subsection{Constant p Gathers}

A constant p \gls{gather} is analogous to the constant velocity \gls{gather} except
that the data is `moved-out' using the linear time shift equation of a
refracted ray ($t0=tx-x/v$ or $tau=t-px$ where $p=1/v$, the slowness).  The
parameter \texttt{REFRAC} must be used in order to obtain the refraction equation.

\subsection{Slant Stack}

A slant stack is obtained by stacking the constant p \glspl{gather}.   This is
analogous to constant velocity stack except that the `move-out' is for
refracted arrivals rather than reflected arrivals.  The \texttt{VELAN} parameter
must be used.

\subsection{Tau-p Semblance Spectra}

The \gls{not:tau}-p semblance spectra is analogous to the reflection semblance
spectra except that the linear refraction equation is used.

\subsection{Parameter Dictionary}

\subsubsection{Parameters Needed by Both Constant Velocity and Velocity Spectra}

\begin{description}
\item[\texttt{VELS}] The list of velocity-velocity increments to be included in the
          analysis.  The first and last entry must be velocities.
          \textit{e.g.} \texttt{VELS 1800 50 2200 100 3000} will produce an analysis with
          velocities 1800 1850 1900 1950 2000 2050 2100 2150 2200
          2300 2400 2500 2600 2700 2800 2900 3000.
          Required.  Up to 21 entries may be given

\item[\texttt{NRP}] The number of \glspl{rp} to be included in each analysis.  \Gls{preset}=1.

\item[\texttt{TYPE}] The type of velocity analysis to perform.  \Gls{preset}=\texttt{SPEC}
\begin{description}
    \item[\texttt{CVEL}]  constant velocity moveout will be done.
    \item[\texttt{SPEC}]  the velocity spectra velocity analysis will be done.
\end{description}

\item[\texttt{REFRAC}] Refraction event analysis (t0=tx-x/v) will be performed when
         \texttt{REFRAC} is set non-zero.
         \Gls{preset}=0.  \textit{e.g.} \texttt{REFRAC 1}

\item[\texttt{STAPER}] The number of traces to weight in order to form a spatial
         taper.  A spatial taper or window reduces the edge effects due
         to there not being an infinite number of traces in the
         analysis.  The taper is applied to \texttt{STAPER} traces at the
         beginning and the end of each \gls{rp} in the analysis.  The taper
         is a Bartlet window or linear ramp.  \textit{e.g.} staper 3 will weight
         the first and last trace by .25, the 2nd and second to last by
         .5, and the third and third to last by .75.
         \Gls{preset}=0 for normal incidence,
         \Gls{preset}=3 for refraction

\item[\texttt{TTAPER}] The length of a linear taper, in seconds, to apply to the ends
         of the data in order to prevent boundary problems in time.
         Each trace will be tapered from either the delay or mute time
         for \texttt{TTAPER} seconds.  Likewise, each trace will be tapered for
         \texttt{TTAPER} seconds at the end of data.  A linear taper is also
         called a Bartlett window.
         \Gls{preset}=0. For normal incidence data,
         \Gls{preset}=.2 For refraction (\gls{not:tau}-p) data
\end{description}


\subsubsection{CVEL Parameters}

\begin{description}
\item[\texttt{STRETC}] The amount of stretch (nmo), in seconds, permissible.  Data with \texttt{NMO} exceeding stretch will be muted.  \Gls{preset}=1.
\end{description}

\subsubsection{Velocity Spectra Parameters}

\begin{description}
\item[\texttt{WINLEN}] The window length, in seconds, of the window to use in the semblance spectra.  The window length should include a full wavelength.  \Gls{preset}=.100.  Example, \texttt{WINLEN .080}

\item[\texttt{VTUPLE}] The velocity (horizontal) scale to use on the printer plot of the velocity spectra.  The tuple is comprised of the minimum velocity to plot, the maximum velocity to plot, and the number of columns to use between the minimum and the maximum (inclusive).  \Gls{preset} = 0 0 0 example: \texttt{VTUPLE 5000 20000 101}

\item[\texttt{CHARS}] The character set to use on the line printer semblance spectra
         plot. The semblance values are normally divided into 10
         intervals, 0. -1., 1. -2., 2. -3., \ldots 9. -10.  The semblance
         value is a real number which is converted to integer by
         truncating.  This integer is then used as an index to the
         character array used in the plot.  The characters in the \texttt{CHARS}
         list must be separated by a blank.  A blank character in the
         \texttt{CHARS} list is represented by any two characters without a blank
         separator.  Up to 50 characters may be given.
         Preset= 0 1 2 3 4 5 6 7 8 9  \textit{e.g.}  \texttt{CHARS AA 1 2 3 4 5 6 7 8 9}

\item[\texttt{OPATH}] The pathname of an output file containing the semblance values
         in full floating point (before being truncated to integer).
         The output file format is \gls{ascii}, one semblance value per line
         and grouped by velocity, unless the last characters of
         the filename are \texttt{segy} or \texttt{mat} for SEG-Y or MATLAB.
  = \texttt{SEGY} These files are constructed as a \gls{cmp} \gls{gather} with each
 or \texttt{MAT}  velocity as a trace within the \gls{gather}.  The semblance values
         are resampled (spline interpolated) so that the semblance
         sample interval is the same as the time domain sample interval,
         and the semblance values are in host floating point, which
         allows seismic picking and plotting programs to function
         easily (\textit{e.g.} script \texttt{VPICK} for SIOSEIS/MATLAB picking).  The
         SEG-Y format contains the velocity in the SEG-Y header word 10
         (range) and the trace values are the modified SEG-Y format 5,
         host floating point.
         \Gls{preset} = none  \textit{e.g.}  \texttt{OPATH velan.1234}

\item[\texttt{END}] Terminates each parameter list.
\end{description}

\section{WBT / WBT2 / WBT3: Water Bottom Time Description}
\label{cmd_wbt}

Process \texttt{WBT} associates a water bottom time with every trace.  The time
is placed in the trace header and may be used by other processes
such as mute, nmo, or decon for 'hanging' windows from. 'Hanging'
windows means that the window times are added to the water bottom time,
thus the window times may be the same distance from the water bottom on
all traces.

Processes \texttt{WBT2} and \texttt{WBT3} are identical to \texttt{WBT} and enable three unique
\texttt{WBT} processes to be given in a single SIOSEIS job.

The water bottom times may be entered in one of several ways:
\begin{enumerate}
\item Listing rp-time pairs
\item Listing \gls{gmt} time-time pairs
\item Converting depths to time using the SEG-Y source depth entry.
\item Automatic picking using a sliding average amplitude window.
\item Automatic picking using the first absolute value amplitude that exceeds a threshold.
\item Automatic picking using the largest positive/negative/absolute value of each trace.
\end{enumerate}

The rp-time pairs method of entering the water bottom times
consists of entering a list of pairs of \gls{rp} numbers followed by the two
way travel time for the \gls{rp}.

The \gls{rp} numbers must strictly increase.  \Gls{rp} numbers may be assigned using
SIOSEIS processes geom or header.  The water bottom time specified will
be associated with all traces with the same \gls{rp} number (a single \gls{cdp} has
the same \gls{rp} number!).  Water bottom times not specified are calculated
by interpolation or held constant.

\textbf{NOTE}: The only \texttt{END} needed is the \texttt{END} to terminate the \texttt{WBT} parameters when
using rp-time pairs or \gls{gmt}-time pairs.

The \gls{gmt} time method consists of specifying the Julian day via the
parameter \texttt{DAY} and then a list of \gls{gmt} time - water bottom time pairs.
The parameter \texttt{DAY} must be given prior to the \gls{gmt} time and must be
given on the first pair.  Furthermore, the day must be given on day
changes.  The \gls{gmt} time is given in terms of decimal minutes of the 24
hour clock, thus 1532.75 represents 1532 and 45 seconds.  Water bottom
times not specified are calculated by interpolation or are held constant.

Example 1:
\begin{verbatim}
     wbt
          41 1.197
          43 1.199
     end
\end{verbatim}
\glspl{rp} less than 41 will receive a water bottom time of 1.197, \gls{rp} 41 a time
of 1.197, 42 a time of 1.198, 43 a time of 1.199, and \gls{rp}'s larger than
43 a time of 1.199.

Example 2:
\begin{verbatim}
     WBT
     DAY 265 1045 .21
          1100 .25
          2200 2.0
     DAY 266 0300 3.5
     END
\end{verbatim}

\subsection{Parameter Dictionary}

\begin{description}
\item[\texttt{PASS}] The passband of a 55 point time domain convolutional zero phase
         bandpass filter (same as process filter ftype 99) to apply
         before picking.  The filtered data are \textbf{NOT} passed to the next
         seismic process, thus the filter is applied ``offline''.  The
         low and high ``corners'' of the filter must be given.  Offline
         filtering is available on all automatic picking options (thres,
         solrat, peak, short/long).
         \Gls{preset} = none      \textit{e.g.}  \texttt{PASS 100 200}

\item[\texttt{LPRINT}] Print switch.
       =2, The water bottom time for every trace is printed.
       =4, When using the \texttt{SOLRAT} method, the short and long averages
         are printed on \textbf{EVERY} trace.  \textbf{THIS CAN CAUSE A LARGE VOLUME
         OF PRINT OUTPUT}.
         \Gls{preset} = 0     \textit{e.g.}   \texttt{LPRINT 4}

\item[\texttt{INDEX}] The index within the SEG-Y trace header where the water
         bottom time will be written.  Time is the floating point
         number of seconds and is type REAL.
         \Gls{preset} = 50    \textit{e.g.}     \texttt{INDEX 59}
\end{description}

\subsubsection{Converting SEG-Y water bottom depth (word 16) to time}

\begin{description}
\item[\texttt{VEL}] The velocity to convert the water depth (SEG-Y long word 16
         with scalar in short word 16) to time.  Useful with SeaBeam
         and Knudsen data.
         \Gls{preset} = 0     \textit{e.g.}   \texttt{VEL 1450}

\item[\texttt{TRACK}] The maximum deviation, in seconds, the water bottom time
         may have from the previous trace's water bottom time.  If the
         deviation exceeds \texttt{TRACK}, then the previous trace's water
         bottom time is used.  Done in time since most seismic plots
         are in time.
         \Gls{preset} = 99.    \textit{e.g.}  \texttt{TRACK .05}
\end{description}

\subsubsection{Listing rp-time pairs}

\begin{itemize}
\item List of \gls{rp} numbers and two way travel times.  The \gls{rp} numbers must increase.  \Gls{preset} = 0     \textit{e.g.}    \texttt{41 1.197}
\end{itemize}

\subsubsection{Listing gmt-time pairs}

\begin{description}
\item[\texttt{DAY}] The Julian day of the GMT-time pairs that follow.
       - List of \gls{gmt} times and two way travel times.  The \gls{gmt} times must
         increase unless the day parameter is used to set a new day.
         \Gls{preset} = 0     \textit{e.g.}    \texttt{1100 .25}
\end{description}

\subsubsection{Automatic Pick Using Average Amplitude Windows}

    The water bottom time is picked when the ratio of a short sliding
average amplitude window over a long average amplitude window exceeds
the user given ratio.  The picked time is put in trace header word 50.

\begin{description}
\item[\texttt{SOLRAT}] Short Over Long window RATio.
         \Gls{preset} = 0.     \textit{e.g.}  \texttt{SOLRAT} 1.5

\item[\texttt{SES}] Start and Stop times, in seconds, of the first Short window.
         The \texttt{SES} window should be prior to the water bottom, so that
         the first window  doesn't contain seismic data (lower
         amplitudes than the \texttt{SEL} window).
         \Gls{preset} = 0      \textit{e.g.}    \texttt{SES} 0 .1

\item[\texttt{SEL}] Start and Stop times, in seconds, of the Long window.  \texttt{SEL}
         should be below the anticipated water bottom, so that the
         window contains seismic data (higher amplitudes than the \texttt{SES}
         window).
         \Gls{preset} = 0      \textit{e.g.}    \texttt{SEL} 6 7

\item[\texttt{TRACK}] The maximum deviation, in seconds, the time of the pick
         may have from the previous good pick.  When the pick time is
         before the track, the pick is discarded and a the \texttt{SES} window
         is advanced one sample before trying again.  If the pick is
         greater than previous trace pick plus \texttt{TRACK}, the pick from
         the previous trace is used.
         \Gls{preset} = 99.    \textit{e.g.}  \texttt{TRACK .05}
\end{description}

\subsubsection{Automatic Pick Using First Amplitude Greater Than A Threshold}

    The water bottom time is picked when the first amplitude exceeds
a threshold.  The absolute value of the trace amplitude is compared
to the user given threshold.

\begin{description}
\item[\texttt{THRES}] The threshold (a value) which defines the water bottom.
         Usually the trace amplitudes in the water column are lower
         than the water bottom reflection by several orders of
         magnitude.  The trace is scanned from the first sample, so
         the direct arrival might exceed the threshold.  Process
         \texttt{PROUT} may be used to examine the trace amplitudes.
         \Gls{preset} = 0     \textit{e.g.} \texttt{THRES 16000}

\item[\texttt{TRACK}] The maximum deviation, in seconds, the time of the pick
         may have from the previous good pick.  When the time is
         outside (previous pick time +/- track), the previous
         pick time is used.
         \Gls{preset} = 99.    \textit{e.g.}  \texttt{TRACK .05}
\end{description}

\subsubsection{Parameters used by autopickers SOLRAT and THRES and VEL}

\begin{description}
\item[\texttt{PRESTK}] A \texttt{YES}/\texttt{NO} switch when set to \texttt{YES} means that the data are \gls{rp}
         sorted and that only the first trace of each \gls{rp} should be
         picked and that the water bottom time/depth of the first
         trace within the \gls{rp} should be used on all traces within
         the \gls{rp}.  In this case, a new \gls{rp} starts when the \gls{rp} number
         (SEG-Y header word 6) changes.
         \Gls{preset} = \texttt{NO}      \textit{e.g.}   \texttt{PRESTK YES}
\end{description}

\subsubsection{Automatic Pick Using the Peak Amplitude}

\begin{description}
\item[\texttt{PEAK}] Invokes the ``peak'' picker which uses the time associated with
         the ``largest'' amplitude of each trace.
       =\texttt{POS}, Use the largest positive value of each trace.
       =\texttt{NEG}, Use the largest negative value of each trace.
       =\texttt{ABS}, Use the largest absolute value of each trace.
        \Gls{preset} = none     \textit{e.g.}   \texttt{PEAK ABS}

\item[\texttt{SEPP}] Start and End time of the Peak Picker.  The window for
         \texttt{PEAK} to operate in.
         \Gls{preset} = entire trace.

\item[\texttt{TRACK}] The maximum deviation, in seconds, the time of the pick
         may have from the previous good pick.  When the time is
         outside (previous pick time +/- track), the previous
         pick time is used.
         \Gls{preset} = 99.    \textit{e.g.}  \texttt{TRACK .05}
\end{description}

\subsubsection[Automatic Peak/Trough Pick ]{Automatic peak/trough pick using a window based on the existing water bottom time.}

(implemented in  SIOSEIS ver 2014.4.2 (11 Jul. 2014))

\begin{description}
\item[\texttt{GUIDED}] Invokes the ``peak'' picker which uses the time associated with
         the ``largest'' amplitude of each trace.
       =\texttt{POS}, Use the largest positive value of each trace.
       =\texttt{NEG}, Use the largest negative value of each trace.
       =\texttt{ABS}, Use the largest absolute value of each trace.
        \Gls{preset} = none     \textit{e.g.}   \texttt{PEAK} ABS

\item[\texttt{SEG}] Start and End time of the \texttt{GUIDED} window.  The existing water bottom
         time in SEG-Y trace header word \texttt{INDEX} is added to \texttt{SEG} on each trace
         for the actual search window.  Remember that process \texttt{WBT} can be used
         to convert the SEG-Y water depth to time, and that there may be multiple
         process \texttt{WBT}s in the procs list.
         \Gls{preset} = none
	 \textit{e.g.} 
\begin{verbatim}
PROCS DISKIN WBT WBT2  ......
WBT  VEL 1500 END WBT2  GUIDED POS SEG -.02 .02 TRACK .005 END
\end{verbatim}

\item[\texttt{TRACK}] The maximum deviation, in seconds, the time of the pick
         may have from the previous good pick.  When the time is
         outside (previous pick time +/- track), the previous
         pick time is used.
         \Gls{preset} = 99.    \textit{e.g.}  \texttt{TRACK} .05
\end{description}

\subsubsection{Obsolete side-echo scanner}

\begin{description}
\item[\texttt{OFFSET}] The number of traces between the SeaBeam depth (center of the
         ship) and the reflection point.
         \Gls{preset} = 3

\item[\texttt{NSCAN}] The number of traces to scan for the ``closest'' SeaBeam depth.
         \texttt{NSCAN/2} traces will be scanned in both forward and aft
         directions.
         \Gls{preset} = 10
\end{description}

\section{WEIGHT: Trace Weighting}
\label{cmd_weight}

Process \texttt{WEIGHT} weights individual traces by multiplying the entire trace
by a constant (scalar multiply).
\begin{itemize}
\item A weight of 1. Results in no amplitude change on the trace.
\item A weight of 0. Results in the trace being killed.
\item A weight of -1. Results in the trace being reversed in polarity.
\end{itemize}

All traces have a default weight of 1.  This means that only those
traces specified by the user will be weighted.  Weights are not
spatially varied, thus only those \glspl{shot} actually specified will be
weighted.

Each parameter list must be terminated with the word \texttt{END}.  The entire
set of weight parameters must be terminated by the word \texttt{END}.

\subsection{Parameter Dictionary}
\begin{description}

\item[\texttt{FNO}] The first \gls{shot} (or \gls{rp}) to apply the weights to.  Shot (\gls{rp})
         numbers must increase monotonically.
         \Gls{preset} = 1

\item[\texttt{LNO}] The last \gls{shot} (\gls{rp}) number to apply the weights to.  \texttt{LNO} must be
         larger than \texttt{FNO} in each list and must increase list to list.
         Default = \texttt{FNO}  (reset to \texttt{FNO} on each list unless \texttt{LNO} is given).

\item[\texttt{XWP}] Range-weight-pairs.  A list of range and weight pairs.  \texttt{XWP}
         must be given with increasing ranges.  The program computes the
         absolute value of both user ranges and data ranges.
         \textit{e.g.} xwp 1000 3.0 2000 0. - Traces with ranges exactly equal to
         1000 will be multiplied by 3, and traces with a range of
         exactly 2000 will be multiplied 0. (\textit{i.e.} killed).  All other
         traces will not be multiplied.
         Default = all 1.

\item[\texttt{TWP}] Trace number-weight-pairs. A list of trace numbers (of a \gls{shot}
         or \gls{rp}) and weights (listed in pairs).  ONLY THOSE TRACES
         SPECIFIED WILL BE WEIGHTED.  Trace numbers must increase
         within each list.   \textit{e.g.}  twp 4 -1. 20 0. indicates that trace
         4 will be inverted in polarity and trace 20 will be killed.
         Default = all 1.

\item[\texttt{WEIGHT}] The multiplier for all traces of the specified \gls{shot}/\gls{rp} (fno).
         The default value is 1., which means that the weight is applied
         to \gls{shot} \texttt{FNO} to \texttt{LNO}, then reset to 1. (the default value).
         Default = 1.             \textit{e.g.}    \texttt{FNO 1234 WEIGHT 0 END}

\item[\texttt{W}] An abbreviation for weight.  Equivalent to weight.
         Default = 1.       \textit{e.g.} \texttt{FNO 1234 W -1 END}

\item[\texttt{HDR}] Specifies the index of the SEG-Y trace header floating point word
         as the trace scalar multiplier.  Each trace is independent.
         \textit{e.g.} hdr 55   causes every amplitude to be multiplied by the
         contents of the 55th floating point word in the SEG-Y trace header.

\item[\texttt{LHDR}] Specifies the index of the 32 bit integer SEG-Y trace header word
         as the trace scalar multiplier.  Each trace is independent.

\item[\texttt{IHDR}] Specifies the index of the 16 bit integer SEG-Y trace header word
         as the trace scalar multiplier.  Each trace is independent.

\item[\texttt{INVERSE}] A \texttt{YES}/\texttt{NO} switch indicating that the inverse or reciprocal of
          the given weight should be used.  \textit{i.e.} the multiplier becomes
          a divisor.
          \Gls{preset} = \texttt{NO}      \textit{e.g.}  \texttt{INVERSE} YES

\item[\texttt{TYPE}] The type of weight to apply.
         \Gls{preset} = not given (use \texttt{W} or \texttt{WEIGHT})     \textit{e.g.}  \texttt{TYPE SDEV}
\begin{description}
\item[\texttt{SDEV}] , The inverse of the standard deviation is used as a scalar multiplier for each trace.
\end{description}

\item[\texttt{LPRINT}] Programmer's debug switch.
\begin{description}
\item[8] The weight applied to each trace is printed.
\item[16] All the statisics are printed when \texttt{TYPE SDEV} is given.
\end{description}

\item[\texttt{END}] Terminates each parameter list.
\end{description}

\subsection{Notes}
\begin{enumerate}
\item In order to apply the same set of weights to all \glspl{shot} (\glspl{rp}), \texttt{LNO} must be set to a very large number. \textit{e.g.}  \texttt{LNO 32767}
\item a maximum of 100 \texttt{TWP} or \texttt{XWP} pairs may be given.
\item If multiple weights are give a a trace, they are multiplied together before the trace multiplication is performed.  \textit{e.g.} if \texttt{W 2} and \texttt{TYPE SDEV} results in a weight of 15, the total weigth applied is 30.
\end{enumerate}

\subsection{Example 1}
\begin{verbatim}
     PROCESS WEIGHT
          FNO 0 LNO 999999 XWP 1228 -1 1408 0 END
     END
\end{verbatim}
Will reverse the polarity of all traces having a range of 1228, and will
kill all traces with a range of 1408, on \glspl{shot} (\glspl{rp}) 1 through 32767.

\subsection{Example 2}
Weight by offset (ala \texttt{GECO})
\begin{verbatim}
PROCS  ....  HEADER WEIGHT .....
HEADER
   R60 = L10 + 6100    # 6100 is the far offset
   R59 = L10 / R60
   END
END
WEIGHT
   FNO 0 LNO 99999 HDR 59 END
END
\end{verbatim}

\subsection{Example 3}
Weight by inverse standard deviation.
\begin{verbatim}
WEIGHT
   FNO 0 LNO 999999 TYPE SDEV END    # Apply to all records 0-999999
END
\end{verbatim}

\section{XCORR: Cross-correlation}
\label{cmd_xcorr}

Process \texttt{XCORR} performs a cross-correlation between the seismic
traces and a ``pilot'' trace.  The first sample output is always lag zero.

When the pilot trace is in a different file from the data, use
parameters \texttt{PPATH}, \texttt{PSNO}, and \texttt{PTR}.  The pilot trace is read only once
when using parameter \texttt{PPATH}; parameters \texttt{PPATH}, \texttt{PSNO}, \texttt{PTR} may not be
given more than once in a SIOSEIS job.

When the pilot is in the same file as the data to be correlated,
the pilot trace should be the first trace of each \gls{shot} since \texttt{XCORR}
does not read the file; \texttt{XCORR} saves and uses the same pilot trace
until it receives a trace numbered \texttt{PILOT}.  \textit{e.g.} \texttt{PILOT 24} means that
data traces 1-23 do not have a pilot associated with them.  Therefore,
the first trace must be the pilot trace (\texttt{PILOT 1}) when using parameter
\texttt{PILOT}.

\subsection{Parameter Dictionary}
\begin{description}

\item[\texttt{NLAGS}] The number of correlation lags to perform and output.  This
         is the same as the number of samples to ouput. Normally the
         pilot and uncorrelated traces are very long and \texttt{NLAGS} is
         much smaller.
         REQUIRED.      \textit{e.g.} \texttt{NLAGS 500}

\item[\texttt{SETP}] Start and End Time of the Pilot trace to use in the
         correlation.  The pilot trace sent out is usually smaller
         than the recorded length, so \texttt{SETP} should be used to eliminate
         the data that is not part of the pilot.  Sometimes the pilot
         may not start at the first sample either, requiring the start
         time to be non-zero.
         \Gls{preset} = delay end-of-data.     \textit{e.g.} \texttt{SETP 0 1.}

\item[\texttt{SETD}] Start and End Time of the Data trace.
         \Gls{preset} = delay end-of-data.    \textit{e.g.}  \texttt{SETD 0 1.5}

\item[\texttt{PPATH}] Pilot PATHname.  The filename of the file that contains the
         pilot trace when the pilot is not trace 1 of each \gls{shot}.
         Used with \texttt{PSNO} and \texttt{PTR} to describe the location of the pilot
         trace.
         \Gls{preset} = none    \textit{e.g.}  \texttt{PPATH /data/vol3/henkart/cats/pilots}

\item[\texttt{PSNO}] Pilot Shot Number.  Used with \texttt{PPATH} and \texttt{PTR} to describe the
         location of the pilot trace.
         \Gls{preset} = none.    \textit{e.g.} \texttt{PSNO 101}

\item[\texttt{PTR}] Pilot TRace number.  Used with \texttt{PPATH} and \texttt{PSNO} to describe the
         location of the pilot trace.  The trace number within \texttt{PSNO}.
         \Gls{preset} = none     \textit{e.g.}  \texttt{PTR 3}

\item[\texttt{PILOT}] The trace number of the pilot trace to use in the
         cross-correlation.  A number other than 1 may result in
         traces before the pilot trace being correlated with an
         incorrect pilot.
         Preset = 1     \textit{e.g.} Don't use anything but 1

\item[\texttt{DOUBLE}] WHen given as \texttt{YES}, then the correlation is done in double
         precision,  Long correlations may need double precision
         in order that arithmetic overflow does not occur.
         \Gls{preset} = \texttt{NO}     \textit{e.g.} \texttt{DOUBLE YES}

\item[\texttt{CATS}] When given \texttt{YES}, the previous uncorrelated trace is used as
         the pilot.  \textit{e.g.} trace 1 is used to correlate trace 1 and 2,
         trace 2 is used to correlate trace 3, \textit{etc.}
         \Gls{preset} = \texttt{NO}   \textit{e.g.}  \texttt{CATS YES}

\item[\texttt{END}] Terminates each parameter list.
\end{description}

\subsection{Example}

A 1 second sweep frequency vibrator was recorded on a Geometrics
Strataview seismic recorder.  The data were recorded for 1.5 seconds.
The sample interval was .25 mils (.00025 seconds), so the longest
output possible is .5 seconds or \texttt{NLAGS 2000} (.5/.00025).
The Strataview records the pilot in trace 1, so the only \texttt{XCORR}
parameters needed were  \texttt{NLAGS 2000 SETP 0 1}.

\section{XSTAR: Convert EdgeTech XSTAR Data to SEG-Y Amplitude Data}
\label{cmd_xstar}

     Process \texttt{XSTAR} converts EdgeTech's chirp analytic data into a
more ``standard SEG-Y'' format and performs several signal processing
steps so that the data are more like conventional time series data.
Process \texttt{XSTAR} applies trace scaling and a complex modulus:
\begin{verbatim}
scalar = buf(51) / 32767. * 1.41E-14
trace(i) = scalar * SQRT( real * real + imag * imag).
\end{verbatim}
     As a result of the complex modulus, the output of process \texttt{XSTAR}
has half as many samples as the input AND twice the sample interval.
The complex modulus of the analytic data creates the envelope.

EdgeTech does not follow the SEG-Y standard, consequently the
data must be converted to SEG-Y.  GeoStar data are in PC byte order
and are missing the SEG-Y file header; GeoStar data must be converted
to using program \texttt{gstar2segy}.  Xstar data must be identified to SIOSEIS
in process \texttt{DISKIN} using parameter \texttt{FORMAT EDGETECH}.
(GeoStar data must not use it).

\textbf{Use \texttt{DISKIN} parameter     \texttt{FORMAT EDGETECH}}

     Parameter \texttt{TYPE} is required so that process \texttt{XSTAR} knows which
type of data it has.
\begin{description}
\item[0] GeoStar,
\item[1] 1 trace Xstar,
\item[2] 2 trace Xstar.
\item[3] 2 trace, where trace 1 is envelope and trace 2 is analytic.
\item[4] 2 trace Xstar without summing the traces.
\item[5] Xstar Version 5
\end{description}

     Of the two known \texttt{XSTAR} systems, one has two transducers and
the other one, as evidenced by the number of output traces they
produce.  Process \texttt{XSTAR} corrects most of the SEG-Y idiosyncracies
these systems produce (such as counting trace numbers from 0 and
setting the number of traces per ping to two when there are two
traces!).  The two traces of the dual transducer Xstar are
automatically summed in process Xstar unless parameters \texttt{FTR}/\texttt{LTR}
are given or \texttt{TYPE 4} is specified.

     There are many different pulses available on Edgetech
fish, but the pulse name is not recorded in the data.

     Towed fish data are hard to interpret if there is much
variation in the fish depth.  It is possible to remove this
variation and shift all the data to a datum.  If the fish
recorded the source and receiver depth/elevation in the SEG-Y
trace header, use process \texttt{SHIFT} parameters \texttt{DATUMV} and \texttt{DATUME}.

     If the fish did not have a depth sensor, several schemes
have been used to determine the fish depth.  See the SIOSEIS
examples for a method when in shallow water.  It is also possible
to do a datum correction if the water depth is known.  The
known water depth may be described via an \texttt{XYZ} file.  SIOSEIS
process \texttt{WBT} may be used to pick the water bottom return from
the chirp data and process \texttt{HEADER} can be used to save the pick
for the next chirp ping.  Process \texttt{XSTAR} can compute the depth
of the fish given the water depth at the fish and the fish
depth.  Parameter \texttt{WIREOUT} is used to locate the fish behind
the position (lat/long) in the SEG-Y header of each ping.  See
the 1999 Eel River example for a detailed discussion of a datum
correction given \texttt{XYZ} and \texttt{WIREOUT}.

Simple Example:
\begin{verbatim}
SIOSEIS << eof
PROCS DISKIN XSTAR FILTER DISKOA END
DISKIN
   FORMAT EDGETECH
   IPATH /home/vol3/henkart/eel/yr2000day196-1915z.xstar END
END
XSTAR
   TYPE 2 END
END
DISKOA
    OPATH /home/vol3/henkart/eel/yr2000day196-1915z.segy END
END
FILTER
   PASS 500 1000 END
END
END
eof
\end{verbatim}

\subsection{Required Parameters}
\begin{description}
    \item[\texttt{TYPE}] The type of Edgetech chirp used.
\begin{description}
\item[0] GeoStar.
\item[1] 1 trace Xstar
\item[2] 2 trace Xstar where the output is a single trace formed by summing the two input traces AFTER the envelope is formed.
\item[3] 2 trace Xstar where each output trace is just the envelope.  The two traces are NOT stacked, however the ``end-of-gather'' flag is set on trace two so that process stack will work.
\end{description}
\end{description}

\subsection{Optional Parameters}

\begin{description}
\item[\texttt{MKREAL}] A yes/no switch when set to \texttt{YES} indicates that the output
          trace should be the frequency corrected time domain trace.
          \Gls{preset} = \texttt{NO}.       \textit{e.g.} \texttt{YES}

\item[\texttt{DUMMIES}] A switch indicating how the program should treat missing
          pings or a missing ducer on the dual ducer xstar.
          Inserting dead traces for missing pings is useful to keep
          the horizontal scale constant when plotting.
          \Gls{preset} = 1.        \textit{e.g.} \texttt{DUMMIES 2 TYPE 2}
\begin{description}
    \item[0] missing pings are not output.
    \item[1] dead traces are created for missing pings.
    \item[2] Valid with the dual ducer (type 2) only.  When 1 ducer
          is missing, the single trace is scaled and output.  Missing
          pings are replaced by dead traces.
\end{description}

\item[\texttt{FTR}] /\texttt{LTR} - The first and last trace to use.  Only applicable with two
          trace Xstar data.  \texttt{FTR 1 LTR 1} indicates trace 2 will be
          ignored.  \texttt{FTR 2 LTR 2} indicates trace 1 will be ignored.

\textbf{It is suggested that the user verify that the two traces are similar
        before stacking the two traces.  \textit{i.e.} One trace might be
        noiser that the other and should not be used.}

\item[\texttt{WEIGHTS}] A list of weights to apply to the traces before summing
          (On Driscoll's XSTAR only since it has two traces!).
          \Gls{preset} = 1 1

\item[\texttt{XYZPATH}] The pathname of an \gls{ascii} file with longitude,
          latitude, and elevation of the fish.  SIOSEIS will
          find the closest XYZ position for every trace and
          inserts the fish depth into SEG-Y short integer header
          word 16 and the floating point two way travel time
          into SEG-Y floating point word 50 ( depth / 750. ).
          Fish depth = - fish elevation.  The longitude and
          the latitude must be in decimal degrees and the fish
          elevation must be negative meters.  \textit{e.g.}
\begin{verbatim}
          -124.438095 40.650543 -52.921194
          -124.437859 40.650543 -52.381282
          -124.437614 40.650543 -51.851186
\end{verbatim}

          The floating point two way travel time to the fish
          is needed because of the high sampling rate of the
          XSTAR fish.  One meter is several samples.

          \Gls{preset} = \texttt{' '}.    \textit{e.g.}  \texttt{XYZPATH Eureka.xyz}

\item[\texttt{BINPATH}] SIOSEIS converts the \gls{ascii} file into binary for internal
          use since \gls{ascii} I/O is quite time consuming.  The binary
          file may be saved and reused by giving this parameter.
          \Gls{preset} = \texttt{xyz.bin} \textit{e.g.} \texttt{BINPATH my\_xyz.bin}

\item[\texttt{WIREOUT}] The distance of the fish from the boat as measured by the
          amount of wire let out.  If the source (fish) depth is in
          the SEG-Y header from the previous ping, process \texttt{XSTAR} will
          calculate the water depth at the fish rather than the boat
          when an XYZ file is given.

\item[\texttt{LPRINT}] SIOSEIS debug print switch.  A bit switch.
\begin{description}
    \item[4] The fish lat/long are printed if \texttt{XYZPATH} is given.
    \item[8] The XYZ file lat/long/depth used is printed.
\end{description}

\item[\texttt{DELTAD}] A warning is printed when the closest XY is greater than
         \texttt{DELTAD} meters away.
         \Gls{preset} = .01    \textit{e.g.}   \texttt{DELTAD .0004}
\end{description}

\chapter{Diagnostic Commands}

\section{DEBUG}
\label{cmd_debug}

If the word \texttt{DEBUG} is placed in or before the
procs list, SIOSEIS will print the name of the process prior to executing
each process.  This diagnostic tool is especially useful if SIOSEIS bombs
and you have no idea of where it bombs. For example:

\begin{verbatim}
SIOSEIS << eof
     debug procs syn diskoa end
about to enter edit of   PROCS
  syn
   about to enter edit of   SYN
     ntrcs 1 FNO 1 LNO 2 secs 1 values 1.1 2.2 3.3 4.4 -5.5 end
  end
  diskoa
   about to enter edit of   DISKOA
      opath data  end
  end
  end
 ****    0 ERRORS IN THIS JOB   ****
   ABOUT TO ENTER   SYN
   ABOUT TO ENTER   DISKOA
   ABOUT TO ENTER   SYN
   ABOUT TO ENTER   DISKOA
   END OF SIOSEIS RUN
\end{verbatim}

\section{ECHO/NOECHO}
\label{cmd_echo}

SIOSEIS release 1991.0 will contain a mechanism for selectively turning off and
on the output printing.  Sometimes a process' parameter list is very long and
clutters up the output print file.  Placing \texttt{NOECHO} prior to the
process' parameters will stop the echoing of the parameters.  Placing
\texttt{ECHO} after the \texttt{END} terminating the process' parameters turns
the printing on.

% \section{EDIT}

\section{LPRINT}
\label{cmd_lprint}
Nearly every process will print out programmer information in both the edit and
execute phase.  Some processes will print information useful to the user
though.  The value associated with lprint indicates the type of information to
be printed.  The value is a bit switch where bit 1 ($2^{0}$) indicates that the
edit parameters are to be printed.  Bit 2 ($2^{1}$) indicates that the
execution module will print some information.

Example:
\begin{verbatim}
SIOSEIS << eof
PROCS DISKIN PROUT END
DISKIN
LPRINT 3
IPATH data END

data
3  0  0  1  -12345  0  0  366  0  2500  1  0  60  0  0
0.  0  0.  1  -1.000000  0  0  1  1  0.  0.  0  1  0.  0  0  0  0.
Version    2.100000
end
PROUT
     FNO 1 LNO 999 FTR 1 LTR 999 end
END
END
****    0 ERRORS IN THIS JOB   ****
data
1  0  60  0  0  0.  0
0.  1  -1.000000  0  0  1  0.  0.
0  1  0.  0  0  0  0.
binary hdr sort=  1
shot  1   trace   1   rp   0   trace   0
no=  1   itrno=  1   fno=  0   lno=  0   ftr=  -12345   ltr=  0   nextno=  0
nexttr=  -12345   iptype=  1   idtype=  1   jsort=  0
SHOT     1 TRACE     1 RP     0 TRACE     0
shot  2   trace   1   rp   0   trace   0
no=  2   itrno=  1   fno=  0   lno=  0   ftr=  -12345   ltr=  0   nextno=  0
nexttr=  -12345   iptype=  1   idtype=  1   jsort=  0
SHOT     2 TRACE     1 RP     0 TRACE     0
END OF SIOSEIS RUN
\end{verbatim}

\section{OVERRIDE}
\label{cmd_override}

Overrides severe warnings. A \textbf{SEVERE WARNING} is a SIOSEIS
\textbf{ERROR} that can be overriden (ignored) by using \texttt{OVERRIDE}.
\textit{e.g.} \texttt{OVERRIDE PROCS DISKIN AGC PLOT END}

\section{REALTIME}
\label{cmd_realtime}

Process \texttt{REALTIME} is designed to work with process \texttt{DISKIN} for
reading SEG-Y files as the data are acquired.  \texttt{REALTIME} signals
\texttt{DISKIN} to check for an increased file size.  If the file remains the
same size for 60 seconds, \texttt{DISKIN} stops in an orderly manner.

\chapter{SIOSEIS's Use of the SEG-Y Trace Header}
\label{c_segy_header}
The SEG standards may be found at
\url{https://seg.org/Publications/SEG-Technical-Standards}

\emph{Q: Which SEG-Y standard is described here? \cite{SEG_Y_r0},
\cite{SEG_Y_r1}, or \cite{SEG_Y_r2}?}

SIOSEIS uses the Society of Exploration Geophysists' SEG-Y tape format
internally to each seismic process as a method of associating the geophysical
variables with the data trace.  This could also be called a dynamic data base.
A trace is defined to be the concatenation of a trace header and the data
trace.  A seismic process within SIOSEIS may modify the trace header as well as
the data trace.

Each seismic process examines every trace header for the trace id flag and will
not process the trace if it is dead.  Processes \texttt{WEIGHT}, \texttt{MUTE},
and \texttt{GATHER} can set this flag to indicate that the trace is dead.

Processes that have time oriented parameters, such as \texttt{DECON} and
\texttt{NMO}, allow the water bottom time to be added to the time parameters.
This allows the water bottom to become a datum.  (This is called `hanging from
the water bottom').  The water bottom time is set into the header by process
\texttt{WBT} which requires that the \gls{shot} geometry information to be in the
header.

Process \texttt{GEOM} uses the \gls{shot} number and trace number from the header to
calculate and set the shot-receiver distance and the \gls{rp} number of the trace.

Process \texttt{GATHER} collects or sorts the traces so that all the traces
with the same \gls{rp} number (in the header) are physically adjacent to each other.
Gather further sorts each \gls{rp} by range (the shot-receiver distance in the
header).  Gather sets a flag in the trace header of the last trace within
the \gls{gather}.

Process \texttt{MUTE} not only mutes the data, but puts the mute times into the
trace header.  If the entire trace is muted, the trace id flag in the header is
set to indicate a dead trace.

Process \texttt{NMO} corrects the data to normal incidence times, using the
shot-receiver distance from the header.  Data that has been muted is not
moved out.  \texttt{NMO} can mute the data also.

Process \texttt{STACK} sums all the traces, regardless of \gls{rp} number, until the
end of gather flag in the trace header is found.  Stack also examines the mute
times in the header so that each stacked data sample can be divided by the
number of live samples contributing to the stack.

Also see IRIS.HEADER

\section[Trace Header Layout]{The Trace Header Layout as Used by Scripps Institution of Oceanography}

The trace header is 240 bytes long (120 \texttt{INTEGER*2} or 60 \texttt{INTEGER*4})

\begin{verbatim}
Byte        16BIT 32BIT 64BIT pointer    DESCRIPTION OF USE
            WORD  WORD  WORD   name
            INDEX INDEX INDEX
            ----- ----- ----- -------

    1-4             1         llseqptr   trace sequence number in the line
    5-8             2         lrseqptr   trace sequence number in the file.
   9-12             3     3   lshotptr * \gls{shot} number or stacked trace number
                                         "Original field record number"
  13-16             4     4   lshtrptr * trace number within the \gls{shot}
  17-20             5     5   lespnptr   "Energy source point number -- used
                                         when more than one record occurs at
                                         the same effective surface location".
                                         SIOSEIS puts the SEG-D FFID (file
                                         number) here.
  21-24             6     6   lrpnptr    \gls{rp} or cdp number
  25-28             7     7   lrptrptr   trace number within the cdp
  29-30     15            75  itridptr * trace id:  1= live, 2=dead, 28=metadata
  31-32     16                ivstkptr   Number of traces vertically stacked
  33-34     17            77  ifoldptr   cdp fold (coverage)
  35-36     18                iuseptr    Data use:  1=production, 2=test
  37-40             10    10  ldisptr    source to receiver distance (range)
  41-44             11                   Receiver group elevation w.r.t.
                                         sea level (depth is negative)
  45-48             12                   Surface elevation at source.
  49-52             13                   Source depth below surface (a positive number)
  53-56             14                   Datum elevation at receiver group.
  57-60             15                   Datum elevation at source
  61-64             16    16  lwbdptr    water depth at the source
  65-68             17                   water depth at the receiver group.
  69-70     35                           Scalar to be applied to elevations
                                         and depths in bytes 41-68.
                                         If positive use as a multiplier.
                                         If negative, use as a divisor.
  71-72     36                           Scalar to be applied to coordinates
                                         in bytes 73-88.
                                         If positive use as a multiplier.
                                         If negative, use as a divisor.
  73-76             19    19  lsxcoptr   longitude in seconds of arc.
                                         Source X coordinate.
                                         See short word 36 (bytes 31-32) - scalar
  77-80             20    20  lsycoptr   latitude in seconds of arc.
                                         Source Y coordinate.
                                         See short word 36 (bytes 31-32) - scalar
  81-84             21    21  lrxcoptr   Receiver longitude or X coordinate
  85-88             22    22  lrycoptr   Receiver latitude or Y coordinate
  89-90     45                           Coordinate units; 1 = length (meters or feet)
                                         2 = seconds of arc, 3 = decimal degrees,
                                         4 = degrees, minutes, seconds (DDDMMSS.ss)
  91-92     46            106 icvelptr   Velocity analysis velocity.
                                         (SEG-Y weathering velocity)
  93-94
  95-96     48                           Feathering angle in decidegrees.
                                         (SEG-Y uphole time at source)
  97-98     49                           Cross-line offset in meters.
                                         (SEG-Y upholes time at group)
105-106     53                ilagaptr   Lag time A in ms. before time 0
107-108     54      27        ilagbptr   Lag time B in ms. before time 0
                                         Used by UTIG as upper word of delay,
                                         so the delay is bytes 107-110
109-110     55            115 idelmptr   deep water delay in ms. (or meters)
                                         Rev 1 uses bytes 215-216 for a scalar
111-112     56            116 istmptr    start mute time in ms.
113-114     57            117 iendmptr   end mute time in ms.
115-116     58            118 isampptr * "Number of data samples in this trace".
                                         excludes header - unsigned int
117-118     59            119 isiptr   * "Sample interval in us for this trace".
125-126     63                           course.  SEG-Y correlation switch.
127-128     64                           speed (tenths of knots). SEG-Y start sweep freq.
157-158     79            139 iyrptr     year data was recorded.
159-160     80            140 idayptr    day of year
161-162     81            141 ihrptr     hour of day
163-164     82            142 iminptr    minute of hour
165-166     83            143 isecptr    second of minute
167-168     84            144 igmtptr    1=local, 2=gmt
                                         sometimes used for milliseconds of \gls{shot}

   The SEGY standard designates BYTES 181-240 as UNDEFINED.

181-184             46    46  ldelsptr   SIOSEIS:  deep water delay in seconds (prior to rev 2013.3)
                                         KEL:  Frequency channel code and Hours of ping
185-188
189-192
193-196             49    49  lsisptr    SIOSEIS: sample interval in seconds
                                         KEL: Rx_Gain and ProcessingGain
197-200             50    50  lwbtsptr   SIOSEIS: water bottom time in seconds.
                                         KEL: Sensitivity
201-204             51    51  lgatptr    SIOSEIS: <0, indicates the end of a \gls{gather}
                                                  >0, the number of traces stacked
                                         EdgeTech: Trace Scalar.
                                         KEL: Primary channel parameter setting
205-208
209-212
*OLD*  213-214     107     54    54  isbptr     SeaBeam/Hydrosweep water depth
213-214
215-216   108       54                   Rev 1 scalar for deep water delay. (post rev 2013.3)
215-218             55    55  iscalar    Trace scale factor (multiplier) (EdgeTech)

\end{verbatim}
* \enquote{Strongly recommended that this information always be recorded}.
  SIOSEIS ALWAYS honors this recommendation.  SIOSEIS does not use the
  SEG-Y recommendation of recording word one, the \enquote{Trace sequence
  number within line \ldots}
Quotes (``'') are supplied around the exact wording from the SEG-Y standard.
\emph{Q: Which SEG-Y standard \cite{SEG_Y_r0}, \cite{SEG_Y_r1}, or \cite{SEG_Y_r2}?}

\lstset{language=[77]Fortran}
\begin{lstlisting}[caption={SEGYPTR COMMON Block}]
      COMMON /segyptr/ llsegptr, lrseqptr, lshotptr, lshtrptr, lrpnptr,
     *                 lrptrptr, itridptr, ldisptr,  lwbdptr,  lsxcoptr,
     *                 lrxcoptr, idelmptr, istmptr,  iendmptr, isampptr,
     *                 isiptr,   iyrptr,   idayptr,  ihrptr,   iminptr,
     *                 isecptr,  igmtptr,  ldelsptr, lsmusptr, lemusptr,
     *                 lsisptr,  lwbtsptr, lgatptr,  lssmsptr, lesmsptr,
     *                 lsbptr,   ifoldptr, icvleptr, lespnptr, ilagaptr,
     *                 ilagbptr
\end{lstlisting}

\section{SEG-Y Binary Tape Header}

The SEG-Y binary tape header is 400 bytes or 200 \texttt{INTEGER*2}

\begin{verbatim}
16 Bit
word
index
------
 7             number of traces per ensemble.
 9             sample interval in microseconds of the first trace
11             number of samples of the first trace
13        *    segy format type
               =1, ibm floating point.
               =2, 32 bit integer.
               =3, 16 bit integer.
               =4, 16 bit UTIG floating point.
               =5, 32 bit IEEE floating point.
14             Ensemble fold.  Not used on \gls{shot} \glspl{gather}.  Same as word 7
               on CMP \glspl{gather}.  CDP fold on stacked data.
15             THE TYPE (SORT) OF DATA
               =0 or 1, the data are sorted by \glspl{shot}
               =2, the data are sorted by \gls{rp} (cdp \glspl{gather})
31             THE DOMAIN OF THE DATA.
               = 0 OR 1, Time
               = 2, frequency-wavenumber domain in rectangular coordinates
               = 3, frequency-wavenumber domain in polar   coordinates
               = 4, frequency domain in rectangular coordinates
               = 5, frequency domain in polar coordinates
               = 6, depth domain
               = 7, \gls{not:tau}-p or slant stack domain
               = 8, F-K "user friendly" polar
               = 9, complex time domain or Analytic trace
32             The number of wavenumbers of the data set when the data are
               in the F-K domain (needed by fkmigr).
33             = The tx sample interval in us.
34             = The tx time delay in ms.
36             = The number of traces in the tx domain.

151            = SEG Y Format Revision Number.
152            = Fixed length trace flag.
               =1, All traces in the file have the same number of
                 samples, every ensemble has the same number of
                 traces, and the ensemble numbers are strictly
                 monotontically increasing.  This allows random
                 access to all traces.
153            = The Number of 3200-byte Textual Header Extension records
                 following the Binary Header.
\end{verbatim}
* \enquote{Strongly recommended that this information always be recorded}.
  SIOSEIS \textbf{ALWAYS} honors this recommendation for the format code.
  SIOSEIS does not use the SEG-Y recommendation of recording words
  that are in the trace header also. \textit{E.g.} SIOSEIS permits the
  sample interval and number of samples to vary from trace to trace.

\lstset{language=[ANSI]C}
\begin{lstlisting}[caption={segy\_header.h}]
/*  The SEG-Y trace header  */

#define   HEADER_LENGTH  240

struct SEGY_TRACE_HEADER {
	long		seq_num;	/* bytes 0-3, trace sequence number in the line */
	long		seq_reel;	/* bytes 4-7, trace sequence number in the reel */
	long		shot_num;	/* bytes 8-11, shot number or stacked trace number
						   "Original field record number"  */
	long		shot_tr;	/* bytes 12-15, trace number within the shot */
	long		espn;	/* bytes 16-19, "Energy source point number -- used
						    when more than one record occurs at
						    the same effective surface location". */
	long		rp_num;	/* bytes 20-23,  rp or cdp number  */
	long		rp_tr;	/* bytes 24-27,  trace number within the cdp */
	short	trc_id;	/* bytes 28-29,  trace id:  1= live, 2=dead  */
	short	num_vstk;	/* bytes 30-31,  Number of traces vertically stacked */
	short	cdp_fold;	/* bytes 32-33,  cdp fold (coverage)  */
	short	use;		/* bytes 34-35,  Data use:  1=production, 2=test  */
	long		range;	/* bytes 36-39,  source to receiver distance (range) */
	long		grp_elev;	/* bytes 40-43,  Receiver group elevation w.r.t.
					                  sea level (depth is negative)  */
	long		src_elev;	/* bytes 44-47,  Source elevation  */
	long		src_depth; /* bytes 48-51,  Source depth below surface.
							(depth is a positive number!) */
	long		grp_datum; /* bytes 51-55,  Datum elevation at receiver group.  */
	long		src_datum; /* bytes 56-59,  Datum elevation at source */
	long		src_wbd;	/* bytes 60-63,  water depth at the source */
	long		grp_wbd;	/* bytes 64-67,  water depth at the receiver group. */
	short	elev_scalar; /* bytes 68-69, Scalar to be applied to elevations
						and depths in bytes 41-68.
						If positive use as a multiplier.
						If negative, use as a divisor.  */
	short	coord_scalar; /* bytes 70-71, Scalar to be applied to
						coordinates in bytes 72-87.
						If positive use as a multiplier.
						If negative, use as a divisor.  */
	long		src_long;	/* bytes 72-75, longitude in seconds of arc.
						Source X coordinate */
	long		src_lat;	/* bytes 76-79, latitude in seconds of arc.
                                         Source Y coordinate   */
	long 	grp_long;	/* bytes 80-83, Receiver longitude or X coordinate */
	long		grp_lat;	/* bytes 84-87, Receiver latitude or Y coordinate */
	short	coord_units; /* bytes 88-89, = 2, coordinate units = seconds of arc */
	short	wvel;	/* bytes 90-91, weathering or water velocity */
	short	sbvel;	/* bytes 92-93, subweathering velocity  */
	short	src_up_vel; /* bytes 94-95, uphole time at source  */
	short	grp_up_vel; /* bytes 96-97, uphole time at group  */
	short	src_static; /* bytes 98-99, Source static correction  */
	short	grp_static; /* bytes 100-101, Group static correction  */
	short	tot_static; /* bytes 102-103, Total static applied */
	short	laga;	/* bytes 104-105, Lag time A in ms. before time 0 */
/*****	short	lagb;	/* bytes 106-107, Lag time B in ms. before time 0 */
	long		delay_mils; /* bytes 106-109, deep water delay in ms. (or meters)  */
	short	smute_mils; /* bytes 110-111, start mute time in ms. */
	short	emute_mils; /* bytes 112-113, end mutes time in ms.  */
	short	nsamps;	/* bytes 114-115, "Number of data samples in this
						trace" - excludes header */
	short	si_micros;	/* bytes 116-117, Sample interval in us for this trace  */
	short	other_1[19]; /* bytes 118-155, Other short integer stuff */
	short	year;	/* bytes 156-157, year data was recorded. */
	short	day_of_yr; /* bytes 158-159, recording day of year */
	short	hour;	/* bytes 160-161, recording hour of day  */
	short	min;		/* bytes 162-163, recording minute of hour  */
	short	sec;		/* bytes 164-165, recording second of minute */
	short	mils;		/* bytes 166-167, recording millisecond  */
					/* OFFICIAL SEGY says: "time basis code"  */
	short	tr_weight; /* bytes 168-169, Trace weighting factor  */
	short	other_2[5]; /* bytes 170-179, Other short integer stuff */
	float	delay;	/* bytes 180-183, deep water delay in seconds (or meters) */
	float	smute_sec; /* bytes 184-187, start mute time in seconds */
	float	emute_sec;	/* bytes 188-191, end mute time in seconds  */
	float	si_secs;	/* bytes 192-195, sample interval in seconds */
	float	wbt_secs;	/* bytes 196-199, water bottom time in seconds */
	long		end_of_rp; /* bytes 200-203, <0, indicates the end of a gather
							>0, the number of traces stacked
							Also EdgeTech's Trace Scalar.  */
	float	dummy1;	/* bytes 204-207 */
	float	dummy2;	/* bytes 208-211 */
	float	dummy3;	/* bytes 212-215 */
	float	dummy4;	/* bytes 216-219 */
	float	dummy5;	/* bytes 220-223 */
	float	dummy6;	/* bytes 224-227 */
	float	dummy7;	/* bytes 228-231 */
	float	dummy8;	/* bytes 232-235 */
	float	dummy9;	/* bytes 236-239 */
};
\end{lstlisting}
