\chapter{Introduction}

SIOSEIS is a software package for enhancing and manipulating marine
seismic reflection and refraction data, sponsored by the National 
Science Foundation (NSF) and the Scripps Industrial Associates.

See \url{http://sioseis.ucsd.edu} for details.

\section{Project Goals}

Kurt Schwehr forked the 2018.2.2 version to git from the
point release tar.  The goals of the fork are:

\begin{itemize}
\item Modernize the code.  e.g. K\&R C to C99
\item Add tests and setup CI
\item Modernize the build system
\item Remove dead code
\item Remove version commit messages from the code
\item Copy the documentation from the web to git
\item Provide a place to track bugs and hopefully take pull requests
\end{itemize}

\chapter{License}

SIOSEIS is released under a 3-Clause BSD license; see \texttt{LICENSE}.

\chapter{Development Environment}
%  1. Define the development environment necessary to implement the program.
%  2. The definition shall include:
%    * the operating system the program will be developed in,
%    * applicable hardware configuration requirements,
%    * required development tools,
%    * third party libraries that are to be embedded in the program,
%    * the directory structure for source code and libraries, and
%    * the instructions for compiling and linking the program.

\section{Operating Environment}
Any operating system which supports the following tools is suitable.

\section{Requirements}
The following software and libraries are required for building SIOSEIS:

\begin{itemize}
    \item A C compiler which supports both K\&R C and the C99 standard
    \item A Fortran compiler which supports both FORTRAN 77 and the Fortran 2008 standard
\end{itemize}

\section{Optional Tools}
The following tools are optional:

\begin{itemize}
    \item The CMake build automation system (version 3.14 or later), available at \url{https://cmake.org/}.
    \item Doxygen, available at \url{http://www.doxygen.nl/}
    \item LaTeX, preferably the TeXLive distribution, available at \url{https://www.tug.org/texlive/}. Used for generating PDF documentation.
    \item GraphViz, available at \url{https://www.graphviz.org/}. Used for generating call/caller graphs and software dependency diagrams.
\end{itemize}

\chapter{Build Process}

The build system will hopefully be replaced soon. At present, the primary build
process uses \texttt{make} and a CMake configuration is currently under
development. CMake successfully builds and packages SIOSEIS but should be
considered experimental.

\section{make}

To build all SIOSEIS executables, use \texttt{make all}.

The available \texttt{make} targets are:
\begin{itemize}
\item \texttt{clean}   - remove all objects and executables
\item \texttt{all}     - build everything
\item \texttt{tar}     - create a release source tar using VERSION
\item \texttt{sioseis} - build just \texttt{sioseis}
\item \texttt{sio2sun} - build just \texttt{sio2sun}
\item \texttt{sio2hp}  - build just \texttt{sio2hp}
\item \texttt{lsd}     - build just \texttt{lsd}
\item \texttt{lsh}     - build just \texttt{lsd}
\item \texttt{dutil}   - build just \texttt{dutil}
\end{itemize}

By default, the makefile uses \texttt{FC=gfortran} and \texttt{CC=gcc}.
Many compiler flags are currently commented out in an effort to discover what
compiles on ``every'' machine (the least common denominator.)

\section{CMake}

The build process follows the basic CMake model:
\begin{enumerate}
\item Retrieve/unpack the source code. Let's call the top-most directory containing the \texttt{CMakeLists.txt} file the \textit{base directory}.
\item Open a terminal session in the base directory
\item \texttt{mkdir build}
\item \texttt{cd build}
\item \texttt{cmake ..}
\item Note that you may need to explicitly specify the C and Fortran compilers
      if they are not properly detected by CMake by using (\textit{e.g.})
      \texttt{FC=gfortran CC=gcc cmake ..}
\item \texttt{make}
\item (optional) To build the documentation, run \texttt{make docs}
\item For extra fancy documentation, run \texttt{cd docs/latex \&\& make} and
      you should get \texttt{<base directory>/build/doc/latex/refman.pdf}
      (what you are probably reading right now)
\end{enumerate}

If everything works, you should find a \texttt{sioseis} executable in the
\texttt{build} directory.

\subsection{Packaging}

Once the software has been built with \texttt{make}, executable installers and
archives may be created by running \texttt{cpack}. This should generate
archives in \texttt{.zip}, \texttt{.tar.gz}, and \texttt{.tar.bz2} formats as
well as a \texttt{.deb} Debian installation package if Debian packaging tools
are available (this presupposes you are on a Debian-compatible linux system
such as Debian, Ubuntu, or Linux Mint.) By default this packages the
\texttt{sioseis}, \texttt{sio2sun}, \texttt{sio2hp}, \texttt{lsd},
\texttt{lsh}, \texttt{dutil}, \texttt{jul2cal}, and \texttt{cal2jul} utilities.
Review the \texttt{CMakeLists.txt} file for details on what is packaged and
where files are installed. Note that this packaging configuration is specific
to Debian-based Linux systems but could be extended to produce RPM packages for
Red Hat-based systems, or packages and installers for OSX and Windows systems
with minor modification to the \texttt{CMakeLists.txt} configuration file.

% \subsection{Windows}

% The build process on Windows is slightly different:

% \begin{enumerate}
% \item Retrieve/unpack the source code. Again, let's call the top-most directory
%       containing the \texttt{CMakeLists.txt} file the \textit{base directory}.
% \item Open a terminal session in the base directory
% \item \texttt{mkdir build}
% \item \texttt{cd build}
% \item \texttt{cmake -A "x64" ..}
% \item Run \texttt{sioseis.sln}. It doesn't look like it will do anything but
%       it should launch Visual Studio and open up a project/solution where you
%       can build SIOSEIS.
% \end{enumerate}

% If everything works, you should find a \texttt{sioseis.exe} executable
% somewhere in/under the \texttt{build} directory, probably in a folder named
% \texttt{Debug} or \texttt{Release}. Either should work; \texttt{Release} should
% be faster than \texttt{Debug} but will give you less info if it crashes.

% \section{Packaging}

% For convenience of deployment, SIOSEIS may be packaged using CPack
% (see \url{https://cmake.org/cmake/help/latest/module/CPack.html}). On
% Debian-like systems (Debian, Ubuntu, Linux Mint, \textit{etc.}, CPack will
% generate \texttt{.zip}, \texttt{.tar.bz2}, and \texttt{.tar.gz} archives along
% with a \texttt{.deb} installer. There is support in \texttt{CMakeLists.txt} for
% creating \texttt{.rpm} (Red Hat Package Manager) installers for RPM-based
% systems. Similarly, there is support for NSIS (Nullsoft Scriptable Install
% System - see \url{https://nsis.sourceforge.io/Main_Page}) on Windows;
% unfortunately neither of these have been tested and are therefore commented
% out. However there is no known reason why installers for other systems should
% not be possible; only a single binary and a few text documents are installed.

% % The build process is as follows on Microsoft Windows using Visual Studio and MSVC as the C++ compiler:
% % \begin{enumerate}
% % \item Install Visual Studio and CMake
% % \item Copy the SIOSEIS project tree to a local directory (unpack from zip file, retrieve from source control with \texttt{git clone}, Team Foundation Server, or Visual Studio)
% % \item Unpack the required software dependencies within the \texttt{ext\_lib} directory beneath the project root and compile any which require compilation (typically SIOSEIS)
% % \item Create a directory named \texttt{build} beneath the local project root
% % \item Run \texttt{cmake -A x64 ..} from a command line within the \texttt{build} directory. CMake should warn if any of the required dependencies could not be found (SIOSEIS, csv-parser). Additional options to cmake may be needed if dependencies are not found in the expected locations in \texttt{ext\_lib}
% % \item If CMake is successful, the file \texttt{sioseis.sln} should be generated; open this file with Visual Studio, build the \texttt{BUILD\_ALL} project to compile the library and the test applications.
% % \item To run the unit tests, build the \texttt{ALL\_TESTS} project or run \texttt{ctest -C Debug} from within the \texttt{build} directory.
% % \item To build the documentation, build the \texttt{doc\_doxygen} project (requires Doxygen and LaTeX)
% % \item To package the software for application development (libraries, C++ headers, test applications and reference data only; no source code), run \texttt{cpack} from the \texttt{build} directory.
% % \end{enumerate}
% % 
% % All compiler and linker options are configured by CMake as defined within the file \texttt{CMakeLists.txt}.
% % 
% % The SIOSEIS project directory is organized as follows:
% % 
% % \VerbatimInput{sioseis_directory_tree.txt}
% % 
% % \clearpage
% % 
% % where:
% % 
% % \begin{itemize}
% % \item \texttt{build} is an initially empty directory used by CMake to build the project.
% % \item \texttt{cmake} contains cmake utilities which find and configure dependencies (XLNT, csv-parser, ChemStation \texttt{usradd}) and assist with test coverage analysis.
% % \item \texttt{contrib} contains auxiliary scripts and data used for maintaining the code base.
% % \item \texttt{doxygen} contains Doxygen configuration files for customizing PDF output.
% % \item \texttt{ext\_lib} contains documentation explaining how to configure software dependencies used by SIOSEIS.
% % \item \texttt{img} contains images and logos to be used in documentation and installers.
% % \item \texttt{RefData} contains reference data files used by unit test to verify the library.
% % \item \texttt{StaticVenting/include} contains C++ header files with the extension \texttt{.hpp}.
% % \item \texttt{StaticVenting/src} contains C++ source code with the extension \texttt{.cpp}.
% % \item \texttt{Test/src} contains C++ source files for unit tests with the extension \texttt{.cpp}.
% % \item \texttt{userdoc} contains static documentation files, bibliographic information, \textit{etc.} to include in the generated documentation.
% % \end{itemize}

% \chapter{Program Structure}

\chapter{Coding Conventions}

\section{Language Standard}

C99 and Fortran 2008, with all vendor extensions explicitly documented.
Existing code is typically in FORTRAN 77 or K\&R C, however new code should
adhere to the more recent standards.

% \section{Numeric Precision}

% Floating-point calculations are expected to be performed using \texttt{double}
% floating point values rather than single-precision \texttt{float} types.  

% \section{Inline Documentation}

% Doxygen is used for generating API documentation; all program elements (code,
% types, data structures) should have associated Doxygen-compatible comments
% associated providing a description, references/citations, and equations as
% applicable. Function arguments should be documented with data flow direction
% (in, out, both). Variables containing physical quantities should be labeled
% with units of measure or noted as being dimensionless. Unless otherwise
% justified, all quantities should be provided in SI units.

% \section{External Dependencies}

% Code from external developers incorporated in the project should be kept
% separate from internal code, ideally imported and built via CMake's
% \texttt{ExternalProject} or \texttt{FetchContent} facilities. External code
% should not be changed unless absolutely necessary and any local changes should
% be communicated back to the original maintainers to minimize the number of
% customizations required.

% \section{Dynamic Memory Use}

% Memory usage should be periodically checked with a tool such as
% \texttt{valgrind}.

% \section{Code Format and Style}

% Coding style is maintained by the \texttt{uncrustify} tool using the formatting
% rules listed in \texttt{uncrustify.atsc.cfg}. This configuration file is stored
% in the \texttt{contrib/uncrustify} directory under the main project directory
% and is retained in version control as part of the project.
% See \texttt{contrib/ftmcpp.zsh} for an example of how to use
% \texttt{uncrustify}.

\chapter{Known Issues}

As of 2010, tape I/O probably will not work on non-Sun computers
(\texttt{mtio.h} and other issues)

